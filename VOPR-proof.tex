% !TEX root =main.tex






\begin{proof}

Before proving Theorem \ref{theorem::VOPR}, we present Lemma \ref{theorem::evaluation-of-random-poly} and Theorem \ref{theorem:coef-poly-prod} that will be used in the proof of Theorem \ref{theorem::VOPR}. 
% !TEX root =main.tex
Informally, Lemma \ref{theorem::evaluation-of-random-poly} states that the evaluation of a random polynomial at a fixed value results in a uniformly random value. %It will be used in {\vopr}'s proof to show that during the verification (in {\vopr}) a malicious party cannot learn anything about its counter party's input.
%\noindent\textbf{Remark 3.} In terms of assumption on the collusion between dealers and assistant clients,  parties, the main difference between using OPE in  and  is that if we will have use \cite{} then even if all but one dealers collude with each other, with all but one assistant clients,  with the rest of authorizer clients and client $B$, they cannot learn honest parties' set elements. However, we use the OPE in \cite{} the assumption is that no dealers collude with any assistant clients. 


%\begin{lemma}\label{theorem::evaluation-of-random-poly}
%Let $(x_{\st i}, y_{\st i})$ be arbitrary elements of a finite field $\mathbb{F}_{\st p}$, where $p$ is a security parameter and sufficiently large prime number.  The probability that the evaluation of a random polynomial $\bm\mu(x)$ at $x_{\st i}$ equals $ y_{\st i}$ is negligible in the security parameter. More formally, for $x_{\st i}, y_{\st i} \in \mathbb{F}_{\st p}$ and $\bm\mu(x)\stackrel{\st\$}\leftarrow \mathbb{F}_{\st p}[X]$, $1\leq deg(\mu)\leq d$, we have: $Pr[\bm\mu(x_{\st i})=y_{\st i}]=\epsilon(p)$. 
%\end{lemma}


\begin{lemma}\label{theorem::evaluation-of-random-poly}
Let $x_{\st i}$ be an element of a finite field $\mathbb{F}_{\st p}$, picked uniformly at random and $\bm\mu(x)$ be a random polynomial of constant degree $d$ (where $d=const(p)$) and defined over $\mathbb{F}_{\st p}[X]$. 
%
 Then, the evaluation of $\bm\mu(x)$ at $x_{\st i}$ is distributed uniformly at random over the elements of the  field, i.e., $Pr[\bm\mu(x_{\st i})=y]=\frac{1}{p}$, where $y\in \mathbb{F}_{\st p}$. 


%equals $ y_{\st i}$ is negligible in the security parameter. More formally, for $x_{\st i}, y_{\st i} \in \mathbb{F}_{\st p}$ and $\bm\mu(x)\stackrel{\st\$}\leftarrow \mathbb{F}_{\st p}[X]$, $1\leq deg(\mu)\leq d$, we have: $Pr[\bm\mu(x_{\st i})=y_{\st i}]=\epsilon(p)$. 
\end{lemma}



\begin{proof} Let $\bm\mu(x)=a_{\st 0}+\sum\limits^{\st d}_{\st j=1} a_{\st j}x^{\st j}$, where the  coefficients  are distributed uniformly at random over the field. 



Then, for any choice of $x$ and random coefficients   $a_{\st 1},...,a_{\st d}$, it holds that:
%
$$Pr[\bm\mu(x)=y]=Pr[\sum\limits_{\st i=0}^{\st d} a_{\st j}\cdot x^{\st j} = y] = Pr[a_{\st 0} = y-\sum\limits_{\st j=1}^{\st d} a_{\st j}\cdot x^{\st j}] = \frac{1}{p}$$

  $\forall y\in \mathbb{F}_{\st p}$, as $a_{\st 0}$ has been picked uniformly at random from $\mathbb{F}_{\st p}$. 
%
\end{proof} 



Informally, Theorem \ref{theorem:coef-poly-prod} states that the product of two arbitrary polynomials (in coefficient form) is a polynomial whose roots are the union of the two original polynomials.  Below, we formally state it. The theorem has been taken from \cite{AbadiMZ21}. 


\begin{theorem}\label{theorem:coef-poly-prod}
Let $\mathbf{p}$ and   $\mathbf{q}$ be two arbitrary non-constant polynomials of degree $d$ and $d'$ respectively, such that  $\mathbf{p} , \mathbf{q}   \in \mathbb{F}_{\st p}[X]$ and they are in coefficient form. Then, the product of the two polynomials is a polynomial whose roots include precisely the two polynomials' roots. 
\end{theorem}


%\begin{proof}
%Let $P=\{p_{\st 1},...,p_{\st t}\}$ and $Q=\{q_{\st 1},...,q_{\st t'}\}$ be the roots of polynomials $\mathbf{p}$ and   $\mathbf{q}$  respectively.  By the Polynomial Remainder Theorem,  polynomials $\mathbf{p}$ and $\mathbf{q}$  can be written as $\mathbf{p}(x)=\mathbf{g}(x)\cdot\prod\limits_{\st i=1}^{\st t}(x-p_{\st i})$ and $\mathbf{q}(x)=\mathbf{g}'(x)\cdot\prod\limits_{\st i=1}^{\st t'}(x-q_{\st i})$ respectively, where $\mathbf{g}(x)$ has degree $d-t$ and $\mathbf{g}'(x)$ has degree $d'-t'$. Let the product of the two polynomials be $\mathbf{r}(x)=\mathbf{p}(x)\cdot \mathbf{q}(x)$. For every $p_{\st i}\in P$, it holds  that $\mathbf{r}(p_{\st i})=0$. Because (a) there exists no non-constant polynomial in $\mathbb{F}[X]$ that has a multiplicative inverse (so it could cancel out factor $(x-p_{\st i})$ of $\mathbf{p}(x)$) and (b) $p_{\st i}$ is a root of $\mathbf{p}(x)$. The same argument  can be used to show for every $q_{\st i}\in Q$, it holds  that $\mathbf{r}(q_{\st i})=0$. Thus, $\mathbf{r}(x)$ preserves  roots of  both  $\mathbf{p}$ and $\mathbf{q}$. Moreover, $\mathbf{r}$ does not have any other roots (than $P$ and $Q$). In particular, if $\mathbf{r}(\alpha)=0$, then $\mathbf{p}(\alpha)\cdot \mathbf{q}(\alpha)=0$. Since there is no non-trivial divisors of zero in $\mathbb{F}[X]$  (as it is an integral domain), it must hold that either $\mathbf{p}(\alpha)=0$ or $\mathbf{q}(\alpha)=0$. Hence, $\alpha\in P$ or $\alpha\in Q$.  %\hfill\(\Box\)
%\end{proof}\hfill\(\Box\)








We refer readers to Appendix \ref{sec::proof-of-poly-union} for the proof of Theorem \ref{theorem:coef-poly-prod}. Next, we  prove the main theorem, i.e., Theorem \ref{theorem::VOPR}, by considering the case where each party is corrupt, in turn. 


% In each case, we invoke the simulator with the corresponding party’s input and output. 

\

\noindent\textbf{Case 1: Corrupt sender.} In the real execution, the sender's view is defined as follows: 


$$ \mathsf{View}_{\st S}^{\st \vopr} \Big((\bm\psi, \bm{\alpha}), \bm\beta\Big)=\{\bm\psi, \bm{\alpha}, r_{\st S},  \bm\beta(z), \bm\theta(z), \mathsf{View}^{\st \ole^{\st +}}_{\st S}, \bot \}$$
%
where $r_{\st S}$ is the outcome of internal random coins of the sender and $\mathsf{View}^{\st \ole^{\st +}}_{\st S}$ refers to the sender's real-model view during the execution of  $\ole^{\st +}$. The simulator $\mathsf{Sim}^{\st \vopr}_{\st S}$, which receives $\bm\psi$ and $\bm \alpha$, works as follows. 
%
\begin{enumerate}
\item generates an empty view. It appends to the view polynomials $(\bm\psi$, $\bm{\alpha})$ and coins $r'_{\st S}$ chosen uniformly at random. 
%
\item computes polynomial $\bm\beta=\bm\beta_{\st 1} \cdot \bm\beta_{\st 2}$, where $\bm\beta_{\st 1}$ is a random polynomial of degree $1$ and $\bm\beta_{\st 2}$ is an arbitrary polynomial of degree $e'-1$. Next,  it constructs polynomial $\bm\theta$ as follows: $\bm\theta=\bm\psi\cdot \bm\beta+\bm \alpha$.

\item picks value $z\stackrel{\st\$}\leftarrow \mathbb{F}_{p}$. Then, it evaluates polynomials $\bm\beta$ and $\bm\theta$  at point $z$. This results in values $\bm\beta_{\st z}$ and $\bm\theta_{\st z}$ respectively. It appends these two values to the view. 
\item extracts the sender-side simulation of $\ole^{\st +}$ from  $\ole^{\st +}$'s simulator. Let $\mathsf{Sim}^{\st \ole^{\st +}}_{\st S}$ be this simulation. Note, the latter simulation is guaranteed to exist, as $\ole^{\st +}$ has been proven secure (in \cite{GhoshN19}). It appends $\mathsf{Sim}^{\st \ole^{\st +}}_{\st S}$ and $\bot$ to its view. 
\end{enumerate}


Now, we are ready to show that the two views are computationally indistinguishable. The sender's inputs are identical in both models, so they have identical distributions. Since the real-model semi-honest adversary samples its randomness according to the protocol's description, the random coins in both models have identical distributions.  Next, we explain why values $\bm\beta(z)$ in the real model and $\bm\beta_{\st z}$ in the ideal model are (computationally) indistinguishable. In the real model, $\bm\beta(z)$ is the evaluation of polynomial $\bm\beta=\bm\beta_{\st 1}\cdot \bm\beta_{\st 2}$ at random point $z$, where $\bm\beta_{\st 1}$ is a random polynomial. We know that $\bm\beta(z)=\bm\beta_{\st 1}(z)\cdot \bm\beta_{\st 2}(z)$, for any (non-zero) $z$.  Moreover, by Lemma \ref{theorem::evaluation-of-random-poly}, we know that $\bm\beta_{\st 1}(z)$ is a uniformly random value. Therefore, $\bm\beta(z)=\bm\beta_{\st 1}(z)\cdot \bm\beta_{\st 2}(z)$ is a uniformly random value as well. In the ideal world, polynomial $\bm\beta$ has the same structure as $\bm\beta$ has (i.e., $\bm\beta=\bm\beta_{\st 1}\cdot \bm\beta_{\st 2}$, where $\bm\beta_{\st 1}$ is a random polynomial). That means $\bm\beta_{\st z}$ is a uniformly random value too. Thus,  $\bm\beta(z)$ and $\bm\beta_{\st z}$ are computationally indistinguishable. Next, we turn our attention to values $\bm\theta(z)$ in the real model and $\bm\theta_{\st z}$ in the ideal model. We know that $\bm\theta(z)$ is a function of $\bm\beta_{\st 1}(z)$, as polynomial $\bm\theta$ has been defined as $\bm\theta=\bm\psi\cdot \bm(\bm\beta_{\st 1}\cdot \bm\beta_{\st 2})+\bm\alpha$. Similarly, $\bm\theta_{\st z}$ is a function of  $\bm\beta_{\st z}$. As we have already discussed,  $\bm\beta(z)$ and $\bm\beta_{\st z}$ are computationally indistinguishable, so are their functions $\bm\theta(z)$ and $\bm\theta_{\st z}$. Moreover, as  $\ole^{\st +}$ has been proven secure,  $\mathsf{View}^{\st \ole^{\st +}}_{\st S}$ and  $\mathsf{Sim}^{\st \ole^{\st +}}_{\st S}$ are computationally indistinguishable. It is also clear that $\bot$ is identical in both models. We conclude that the two views are computationally indistinguishable.


\


\noindent\textbf{Case 2: Corrupt receiver.}  Let $\mathsf{Sim}^{\st\vopr}_{\st R}$ be the simulator, in this case, which uses a subroutine adversary, $\mathcal{A}_{\st R}$. $\mathsf{Sim}^{\st \vopr}_{\st R}$ works as follows. 
%
\begin{enumerate}
%
\item simulates ${\ole^{\st +}}$ and receives $\mathcal{A}_{\st R}$'s input coefficients $b_{\st j}$ for all $j$, $0\leq j \leq e'$, as we are in $f_{\st \ole^{\st +}}$-hybrid model.
%
\item reconstructs polynomial $\bm \beta$, given the above coefficients. 
%
\item simulates the honest sender's inputs as follows. 
%%%%%%%%%
It picks two random polynomials: ${\bm\psi}=\sum\limits^{\st e}_{\st i=0}{g}_{\st i}\cdot x^{\st i}$ and  ${\bm\alpha}=\sum\limits^{\st e+e'}_{\st j=0}{a}_{\st j}\cdot x^{\st j}$, such that ${g}_{\st i}\stackrel{\st \$}\leftarrow \mathbb{F}_{\st p}$ and  every $a_{\st j}$ has the  form: $a_{\st j}=\sum\limits^{\substack{\st k=e'\\ \st t=e}}_{\st t,k=0} a_{\st t,k}$,  where $t+k=j$ and $a_{\st t,k}\stackrel{\st \$}\leftarrow \mathbb{F}_{\st p}$. 
%%%%%%%
\item sends to ${\ole^{\st +}}$'s functionality values $g_{\st i}$ and $a_{\st i,j}$ and receives   $c_{\st i,j}$ from this functionality (for all $i,j$).
%
\item sends all ${c}_{\st i,j}$ to TTP and receives polynomial ${\bm\theta}$. 
%
\item picks a random value $ z$ from $\mathbb{F}_{\st p}$. Then, it computes $ \psi_{\st  z} = {\bm\psi}(  z)$ and $\alpha_{\st  z}= {\bm\alpha}(z)$. 
%
\item sends $ z$ and all ${c}_{\st i,j}$ to $\mathcal{A}_{\st R}$ which sends back $\theta_{ z}$ and $\beta_{\st z}$ to the simulator. 
%
\item sends  $ \psi_{\st  z}$ and $\alpha_{\st z}$ to $\mathcal{A}_{\st R}$. 
%
\item checks if the following relation hold:
%
\begin{equation}\label{equ::beta}
 \beta_{\st \bar z}={\bm\beta}( z) \hspace{6mm} \wedge \hspace{6mm} \theta_{\st  z}={\bm\theta}( z) \hspace{6mm}\wedge\hspace{6mm} {\bm\theta}( z)=  \psi_{\st  z}\cdot \beta_{\st  z}+\alpha_{\st  z}
 \end{equation}
 %
% \begin{equation}\label{equ::theta}
% \bar\theta_{\st \bar z}=\bar{\bm\theta}(\bar z) 
% \end{equation}
% %
%\begin{equation}\label{equ::theta-equals=psi}
%\bar{\bm\theta}(\bar z)= \bar \psi_{\st \bar z}\cdot \bar\beta_{\st \bar z}+\bar\alpha_{\st  \bar z}
%  \end{equation}
%%
% \begin{equation}\label{equ::theta-evl-equals=psi}
% \bar\theta_{\st \bar z}= \bar \psi_{\st \bar z}\cdot \bar\beta_{\st \bar z}+\bar\alpha_{\st  \bar z}
 %\end{equation}
%





%%%%%%%%%%%%%%%


%
% $\bar{\bm\theta}(\bar z) =\bar\theta_{\st \bar z} = \bar \psi_{\st \bar z}\cdot \bar\beta_{\st \bar z}+\bar\alpha_{\st  \bar z}$. 
 %
 If Relation \ref{equ::beta} does not hold, it aborts (i.e., sends abort signal $\Lambda$ to the sender) and still proceeds to the next step. 
 
 \item outputs whatever $\mathcal{A}_{\st R}$ outputs. 
%
\end{enumerate}

We first focus on the adversary's output. Both values of $z$ in the real and ideal models have been picked uniformly at random from $\mathbb{F}_{\st p}$; therefore, they have identical distributions. In the real model, values  $\psi_{\st z}$ and $\alpha_{\st z}$ are the result of the evaluations of two random polynomials at (random) point $z$. In the ideal model, values $\psi_{\st  z}$ and $\alpha_{\st z}$ are also the result of the evaluations of two random polynomials (i.e., ${\bm\psi}$ and ${\bm\alpha}$) at point $ z$.  By Lemma \ref{theorem::evaluation-of-random-poly}, we know that the evaluation of a random polynomial at an arbitrary value  
 yields a uniformly random value in $\mathbb{F}_{\st p}$. Therefore, the distribution of pair $(\psi_{\st z}, \alpha_{\st z})$ in the real model is identical to that of pair $( \psi_{\st  z}, \alpha_{\st  z})$ in the ideal model. Moreover, the final result (i.e., values ${c}_{\st i,j}$) in the real model has the same distribution as the final result (i.e., values ${c}_{\st i,j}$)  in the ideal model, as they are the outputs of the ideal calls to $f_{\st \ole^{\st +}}$, as we are in the $f_{\st \ole^{\st +}}$-hybrid model. 

Next, we turn our attention to the sender's output. We will show that the output distributions of the honest sender in the ideal and real models are statistically close. 
%
%Note that the messages distributions the sender receive from the ideal call to $f_{\st \text{OLE}^{\st +}}$ are identical in both models. 
%
Our focus will be on the probability that it aborts in each model, as it does not receive any other output. In the ideal model, $\mathsf{Sim}^{\st \vopr}_{\st R}$ is given the honestly generated result polynomial ${\bm \theta}$ (computed by TTP) and the adversary's input polynomial ${\bm \beta}$. $\mathsf{Sim}^{\st \vopr}_{\st R}$ aborts with a probability of 1 if Relation \ref{equ::beta} does not hold. However, in the real model, the honest sender (in addition to its inputs) is given only $\beta_{\st  z}$ and $\theta_{\st  z}$ and is not given polynomials ${\bm\beta}$ and  ${\bm\theta}$; it wants to check if the following equation holds, $\theta_{\st z} =  \psi_{\st z}\cdot  \beta_{\st z}+ \alpha_{\st z}$. Note, polynomial $\bm\theta=\bm\psi\cdot \bm\beta+\bm\alpha$ (resulted from ${c}_{\st i,j}$) is well-structured, as it satisfies the following three conditions, regardless of the adversary's input $\bm\beta$ to $\ole^{\st +}$, (i) $deg(\bm\theta)=Max \Big(deg(\bm\beta)+deg(\bm\psi), deg(\bm\alpha) \Big)$, as $\mathbb{F}_{\st p}[X]$ is an integral domain and ($\bm\psi,\bm\alpha$) are random polynomials, (ii)  the roots of the product polynomial $\bm\nu=\bm\psi\cdot \bm\beta$ contains exactly both polynomials' roots, by Theorem \ref{theorem:coef-poly-prod}, and (iii)  the roots of $\bm\nu+\bm\alpha$ is the intersection of the roots of $\bm\nu$ and $\bm\alpha$, as shown in \cite{DBLP:conf/crypto/KissnerS05}. Furthermore, polynomial $\bm \theta$ reveals no information (about  $\bm \psi$ and $\bm\alpha$ except their degrees) to the adversary and the pair $(\psi_{\st z}, \alpha_{\st z})$ is given to the adversary after it sends the pair $(\theta_{\st z}, \beta_{\st z})$ to the sender. 
%
There are exactly four cases where pair $(\theta_{\st z}, \beta_{\st z})$ can be constructed by the real-model adversary. Below, we state each case and the probability that the adversary is detected in that case during the verification, i.e., $\theta_{\st z}\stackrel{\st ?}=\psi{\st z}\cdot \beta_{\st z}+\alpha_{\st z}$. 
%
\begin{enumerate}
%
\item $\theta_{\st z}= \bm\theta(z) \wedge  \beta_{\st  z}={\bm\beta}( z)$. This is a trivial non-interesting case, as the adversary has behaved honestly, so it can always pass the verification. %, i.e., $\theta_{\st z}\stackrel{\st ?}=\psi{\st z}\cdot \beta_{\st z}+\alpha_{\st z}$.
%
\item $\theta_{\st z}\neq \bm\theta(z) \wedge  \beta_{\st  z}={\bm\beta}( z)$. In this case, the adversary is detected with a probability of 1. 
%
\item $\theta_{\st z}= \bm\theta(z) \wedge  \beta_{\st  z}\neq {\bm\beta}( z)$.  In this case, the adversary is also detected with a probability of 1.
%
\item $\theta_{\st z}\neq \bm\theta(z) \wedge  \beta_{\st  z} \neq {\bm\beta}( z)$. In this case, the adversary is detected with an overwhelming probability, i.e., $1-\frac{1}{2^{\st 2\lambda}}$. 
%
\end{enumerate}

As we illustrated above, in the real model, the lowest probability that the honest sender would abort in case of adversarial behaviour is $1-\frac{1}{2^{\st 2\lambda}}$. Thus, the honest sender's output distributions in the ideal and real models are statistically close, i.e., $1$ \text{vs} $1-\frac{1}{2^{\st 2\lambda}}$. 

We conclude that the distribution of the joint outputs of the honest sender and adversary in the real and ideal models are computationally indistinguishable. 
  \hfill\(\Box\)\end{proof}
  
  
  
  
  
  
  
  
  
  
  
  