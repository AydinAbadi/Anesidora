% !TEX root =main.tex

\vspace{-1.5mm}

\subsection{Commitment Scheme}\label{subsec:short-commit}
\vspace{-1mm}

A commitment scheme involves a  \emph{sender} and a \emph{receiver}. It includes  two phases, \emph{commit} and  \emph{open}. In the \emph{commit} phase, the sender  commits to a message $x$ as $\mathtt{Com}(x,r)=\mathtt{Com}_{\scriptscriptstyle x}$, that involves a secret value,  $r$. In the \emph{open} phase, the sender sends the opening $\ddot{x}:=(x,r)$ to the receiver which verifies its correctness: $\mathtt{Ver}(\mathtt{Com}_{\scriptscriptstyle x},\ddot{x})\stackrel{\scriptscriptstyle ?}=1$ and accepts if the output is $1$. A commitment scheme must satisfy: (a) \textit{hiding}: it is infeasible for an adversary to learn any information about the committed  message $x$, until the commitment ${com}$ is opened, and (b) \textit{binding}: it is infeasible for a malicious sender to open a commitment ${com}$ to different values than that was  used in the commit phase. We refer readers to Appendix \ref{subsec:commit} for further details. 




