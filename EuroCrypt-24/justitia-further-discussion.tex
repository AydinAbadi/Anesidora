% !TEX root =main.tex


%%%%%%%%%%%%%%%%%
\section{Further Discussion on  \withFai}\label{sec::Discussion-justitia}


\subsection{Input and Output Privacy}

Intuitively, all parties locally blind/encrypt all (set element related) messages before they send them the smart contract \scf. Also, the resulting polynomial that \scf computes is in a blinded form. Thus, (i) no party (including the protocol's participants) can learn other parties input set elements and (ii) the non-participants of the protocol cannot learn anything about the protocol's output, not even the intersection cardinality. 


\subsection{Strawman Approaches}


\subsubsection{Relying on a Server-aide PSI.} One may be tempted to replace $\withFai$ with a scheme in which all clients send their encrypted sets to a server (potentially semi-honest and plays \aud's role) which computes the result in a privacy-preserving manner.  We highlight that the main difference is that in this (hypothetical) scheme the server is \emph{always involved};  whereas, in our protocol, \aud remains offline as long as the clients behave honestly and it is invoked only when the contract detects misbehaviours.  


\subsubsection{Charging the Buyer a Flat Fee.}
One may want to add a straightforward payment mechanism to an existing multi-party PSI in such a way that a buyer always must pay a fixed amount, e.g., depending on the total number of clients and the minimum size of the sets. However, this approach is problematic, because it:

\begin{enumerate}
\item forces the buyer to always pay even if some malicious clients misbehave during the protocol execution and affect the results' correctness which would allow malicious clients to learn the result at the buyer's expense without letting it learn the correct result, as there exists no fair multi-party PSI in the literature to ensure either all parties learn a correct result or neither does. 

\item forces the buyer to always pay independent of the exact size of the intersection. If it has to pay more than the amount it would have paid for the exact cardinality of the intersection then the buyer would be discouraged to participate in the protocol in the first place. On the other hand, if the buyer has to pay less than what it would have paid for the size of the intersection, then other clients would be discouraged to participate in the protocol. 

\end{enumerate}



% !TEX root =main.tex


\subsection{Error Probability}\label{sec::error-prob}


Recall that in \fpsi, in step \ref{JUS::check-non-zero-coeff},  each client $C$ needs to ensure polynomials $\bm\omega^{\st (C,D)}\cdot \bm\pi^{\st  {  {(C)}}}$ and  $\bm\rho^{\st (C,D)}$ do not contain any zero coefficient. This check ensures that client $C$ (the receiver in \vopr) does not insert any zero values to  \vopr (in particular to $\ole^{\st +}$ which is a subroutine of \vopr). If a zero value is inserted in \vopr, then an honest receiver will learn only a random value and more importantly cannot pass \vopr's verification phase. 
%
Nevertheless,  this check can be removed from step \ref{JUS::check-non-zero-coeff}, if we allow \fpsi to output an error with a small probability.\footnote{By error we mean even if all parties are honest,  \vopr halts when an honest party inserts $0$ to it.} In the remainder of this section, we show that this probability is negligible. 


First, we focus on the product $\bm\omega^{\st (C,D)}\cdot \bm\pi^{\st  {  {(C)}}}$. We know that $\bm\pi^{\st  {  {(C)}}}$ is of the form $\prod\limits^{\st d}_{\st i=1} (x-s'_{\st i})$, where $s'_{\st i}$ is either a set element $s_{\st i}$ or a random value.  Thus, all of its coefficients are non-zero. Also, the probability that at least one of the coefficients  of $d$-degree random polynomial $\bm\omega^{\st (C,D)}$ equals $0$ is at most $\frac{d+1}{p}$.  Below, we state it formally.


\begin{theorem}\label{theorem::zero-coeff-in-ran-poly}
Let $\bm\delta=\sum\limits_{\st t=0}^{\st d}u_{\st  t} \cdot x^{\st t}$, where  $u_{\st  t}\stackrel{\st\$}\leftarrow\mathbb{F}_{\st p}$, for all $t, 0\leq t\leq d$. Then, the probability that at least one of the coefficients equals $0$ is at most $\frac{d+1}{p}$, i.e., 

$$Pr[\exists u_{\st t}, u_{\st t}=0]\leq \frac{d+1}{p}$$
\end{theorem}


\begin{proof}
The proof is straightforward. Since $\bm\delta$'s coefficients are picked uniformly at random from  $\mathbb{F}_{\st p}$, the probability that $u_{\st t}$ equals $0$ is $\frac{1}{p}$. Since $\bm\delta$ is of degree $d$, due to the union bound, the probability that at least one of $u_{\st t}$s equals $0$ is at most $\frac{d+1}{p}$. 
%
 \end{proof}



Next, we show that the probability that the polynomial  $\bm\omega^{\st (C,D)}\cdot \bm\pi^{\st  {  {(C)}}}$ has at least one zero coefficient is negligible in the security parameter; we assume polynomial $\bm\omega^{\st (C,D)}$  has no zero coefficient. 


%For the sake of simplicity,  we set $\bm{p}_1=\bm\omega^{\st (C,D)}=a_{\st 0}\cdot a_{\st 1}\cdot x+...+a_{\st d}\cdot x^{\st d}$ and $\bm{p}_2= \bm\pi^{\st  {  {(C)}}}$. 

\begin{theorem}\label{theorem::zero-coeff-in-product}
Let  $\bm\alpha=\sum\limits_{\st j=0}^{\st m}a_{\st  i} \cdot x^{\st i}$ and  $\bm\beta=\sum\limits_{\st j=0}^{\st n}b_{\st  j} \cdot x^{\st j}$,  where $a_{\st i}\stackrel{\st\$}\leftarrow \mathbb{F}_{\st p}$ and $a_{\st i}, b_{\st j}\neq0$, for all $i,j, 0\leq i \leq m$ and $0\leq j \leq n$. Also, let $\bm\gamma=\bm\alpha\cdot\bm\beta=\sum\limits_{\st j=0}^{\st m+n}c_{\st  j} \cdot x^{\st j}$. Then, the probability that at lest one of the coefficients of polynomial $\bm\gamma$ equals $0$ is at most $\frac{m+n+1}{p}$, i.e., 
%
$$Pr[\exists c_{\st j}, c_{\st j}=0]\leq \frac{m+n+1}{p}$$
%
\end{theorem}



\begin{proof}
Each coefficient $c_{\st  k}$ of $\bm\gamma$ can be defined as $c_{\st  k}=\sum\limits_{\substack{\st j=0\\ \st i=0}}^{\substack{\st i=m\\ \st j=n}}a_{\st i}\cdot b_{\st j}$, where $i+j=k$. We can rewrite $c_{\st  k}$ as  $c_{\st  k}=a_{\st w}\cdot b_{\st z}+ \sum\limits_{\substack{\st j=0, j\neq z\\ \st i=0, i\neq w}}^{\substack{\st i=m\\ \st j=n}}a_{\st i}\cdot b_{\st j}$, where $w+z=i+j=k$. 
%
We consider two cases for each $c_{\st  k}$:

\begin{itemize}

\item[$\bullet$]  {Case 1}: $\sum\limits_{\substack{\st j=0, j\neq z\\ \st i=0, i\neq w}}^{\substack{\st i=m\\ \st j=n}}a_{\st i}\cdot b_{\st j}=0$.  This is a trivial case, because with the  probability of $1$ it holds that $c_{\st  k}=a_{\st w}\cdot b_{\st z}\neq 0$, as by  definition $a_{\st w}, b_{\st z}\neq0$ and $\mathbb{F}_{\st p}$ is an integral domain. 

%In this case, since $a_{\st w}$ has been picked uniformly at random, the probability that $c_{\st  k}=a_{\st w}\cdot b_{\st z}=0$ is $\frac{1}{p}$.

%
% This is a trivial case, because with the  probability of $1$ it holds that $c_{\st  k}=a_{\st w}\cdot b_{\st z}\neq 0$. 


%In this case, because $a_{\st w}$ has been picked uniformly at random, with the  probability of $\frac{1}{p}$ it holds that $c_{\st  k}=a_{\st w}\cdot b_{\st z}=0$. 


\item[$\bullet$]  {Case 2}: $q=\sum\limits_{\substack{\st j=0, j\neq z\\ \st i=0, i\neq w}}^{\substack{\st i=m\\ \st j=n}}a_{\st i}\cdot b_{\st j}\neq 0$. In this case, for  event $c_{\st  k}=a_{\st w}\cdot b_{\st z}+q=0$ to occure, $a_{\st w}\cdot b_{\st z}$ must equal the additive inverse of $q$. Since $a_{\st w}$ has been picked uniformly at random, the probability that such an even occurs is $\frac{1}{p}$.
\end{itemize}

The above analysis is for a single $c_{\st  k}$. Thus, due to the union bound, the probability that at least one of the coefficients $c_{\st  k}$ equals $0$ is at most $\sum\limits^{\st m+n}_{\st j=0}\frac{1}{p}= \frac{m+n+1}{p}$.
%
 \end{proof}




Next we turn out attention to $\bm\rho^{\st (C,D)}$. Due to Theorem \ref{theorem::zero-coeff-in-ran-poly}, the probability that at least one of the coefficients of $\bm\rho^{\st (C,D)}$ equals $0$ is at most $\frac{d+1}{p}$. 
%
Hence, due to Theorems \ref{theorem::zero-coeff-in-ran-poly}, \ref{theorem::zero-coeff-in-product}, and union bound, the probability that at least one of the coefficients in $\bm\omega^{\st (C,D)}\cdot \bm\pi^{\st  {  {(C)}}}$ and  $\bm\rho^{\st (C,D)}$ equals $0$ is at most $\frac{3d+2}{p}$, which is negligible in  $p$. 


















 
% !TEX root =main.tex


%%%%%%%%%%%%%%%%%
 %%%%%%%%%%%%%
\subsection{Main Challenges that \withFai Overcomes}\label{sec::Justitia-challenges}

 We needed to address several key challenges, to design an efficient scheme that realises \p. Below, we outline these challenges.
 

 \subsubsection{Keeping Overall Complexities Low.}
 
 In general, in multi-party PSIs, each client needs to send messages to the rest of the clients and/or engage in secure computation with them, e.g., in \cite{DBLP:conf/scn/InbarOP18,DBLP:conf/ccs/KolesnikovMPRT17}, which would result in communication and/or computation quadratic with the number of clients. To address this challenge, we  (a) allow one of the clients as a dealer to interact with the rest of the clients,\footnote{This approach has similarity with the non-secure PSIs in \cite{GhoshN19}.} and   (b) we use a smart contract, which acts as a bulletin board to which most messages are sent and also performs lightweight computation on the clients' messages. The combination of these approaches will keep the overall communication and computation linear with the number of clients (and sets' cardinality). 
 
 

 
 \subsubsection{Randomising Input Polynomials.}  In multi-party PSIs that use the polynomial representation, it is essential that a client's input polynomial be randomised by another client \cite{AbadiMZ21}. To do that securely and efficiently, we require the dealer and each client together to engage in an instance of \vopr, which we developed in Section \ref{sec::subroutines}. 
 

 
 \subsubsection{Preserving the Privacy of Outgoing Messages.} Although the use of public smart contracts (e.g., Ethereum) will help keep overall complexity low, it introduces another challenge; namely, if clients do not protect the privacy of the messages they send to the smart contracts, then other clients (e.g., dealer) and non-participants of PSI (i.e., the public) can learn the clients' set elements and/or the intersection. To efficiently protect the privacy of each client's messages (sent to the contracts) from the dealer, we require the clients (except the dealer) to engage in \zspaa which lets each of them generate a pseudorandom polynomial with which it can blind its message. To protect the privacy of the intersection from the public, we require all clients to run a coin-tossing protocol to agree on a blinding polynomial, with which the final result that encodes the intersection on the smart contract will be blinded.  
 
 

 \subsubsection{Ensuring the Correctness of Subroutine Protocols' Outputs.} 
 
 In general, any MPC that must remain secure against an active adversary is equipped with a verification mechanism that ensures an adversary is detected if it affects messages' integrity, during the protocol's execution. This is the case for the subroutine protocols that we use, i.e., \vopr and \zspaa. However, this type of check itself is not always sufficient. Because in certain cases, the output of an MPC may be fed as input to another MPC and we need to ensure that the \emph{intact} output of the first MPC is fed to the second one. This is the case in our PSI's subroutines too. To address this challenge, we use unforgeable polynomials; specifically, the output of \vopr is an unforgeable polynomial (that encodes the actual output). If the adversary tampers with the \vopr's output and uses it later, then a verifier can detect it. We will have the same integrity guarantee for the output of \zspaa for free. Because (i) \vopr is called before \zspaa, and (ii) if clients use intact outputs of \zspaa, then the final result (i.e., the sum of all clients' messages) will not contain any output of \zspaa, as they will cancel out each other. Thus, by checking the correctness of the final result, one can ensure the correctness of the outputs of \vopr and \zspaa, in one go. 
 %%%%%%%%%%%%%%





 


% !TEX root =main.tex


\subsubsection{Confidentiality of polynomial $\zeta$.} 

The dealer is the only party who picks and knows the secret polynomial $\zeta$. 
This polynomial remains secret until all parties provide their messages to the smart contract. Then, the dealer in step \ref{f-psi::D-gen-switching-poly} on page~\pageref{f-psi::D-gen-switching-poly} sends $\zeta$ to the smart contract. After this point it is not secret, however, the know of $\zeta$ would not benefit malicious parties at this point, as they have all provided their inputs to the smart contract. 

% !TEX root =main.tex



\subsection{Concrete Parameters}\label{sec::conc-parameters}

\subsubsection{Hash Table Parameters.}


As detailed in Appendix \ref{Preliminary-Hash-Table}, a hash table is characterised by several key parameters:

\begin{itemize}

\item[$\bullet$] $h$: the number of bins.

\item[$\bullet$] $d$: the maximum size (or capacity) of each bin.

\item[$\bullet$] $c$: the maximum number of elements that are mapped to the hash table.

\item[$\bullet$] $pr$: the probability that the number of elements mapped to a bin does not exceed a predefined capacity.
 
 \end{itemize}
 
 It is important to note that in the context of PSI, the parameter $c$ represents the maximum of the sizes of all sets involved. 
 
 The literature, as evidenced in sources such as \cite{Feather2020-full,DBLP:conf/ccs/KolesnikovMPRT17,DBLP:conf/uss/Pinkas0SZ15}) has extensively examined the specific parameters of a hash table, even within the context of PSI. For instance, as illustrated in Section 6 and Appendix J.1 of \cite{Feather2020-full}, when $c$ falls within the range of  $[2^{\st 10},\ 2^{\st 20}]$ and $pr$ is set to $2^{\st -40}$, then $d$ is $100$. Furthermore, we have $h\approx\frac{4c}{d}$. To provide a concrete value, when $c=2^{\st 20}$ and $d=100$, we would have $h=41943$. 




\subsubsection{Field Size.} 
%
In this paper, all arithmetic operations are defined over a finite field $\mathbb{F}_{\st p}$, where $\log_{\st 2}(p)=\lambda$ represents the security parameter. The outputs of $\mathtt {PRF}(.)$ and  $\mathtt {PRP}(.) $ are also of size $\lambda$. Concrete value of $\lambda$ can be set based on the maximum bit size of set elements. For instance, one can choose $\lambda = 60$, when the maximum bit size of set elements is slightly less than $60$ or  $\lambda = 128$ if it is slightly less than $128$. 





