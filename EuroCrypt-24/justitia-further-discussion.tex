% !TEX root =main.tex


%%%%%%%%%%%%%%%%%
\section{Further Discussion on  \withFai}\label{sec::Discussion-justitia}


\subsection{Input and Output Privacy}

Intuitively, all parties locally blind/encrypt all (set element related) messages before they send them the smart contract \scf. Also, the resulting polynomial that \scf computes is in a blinded form. Thus, (i) no party (including the protocol's participants) can learn other parties input set elements and (ii) the non-participants of the protocol cannot learn anything about the protocol's output, not even the intersection cardinality. 


\subsection{Strawman Approaches}


\subsubsection{Relying on a Server-aide PSI.} One may be tempted to replace $\withFai$ with a scheme in which all clients send their encrypted sets to a server (potentially semi-honest and plays \aud's role) which computes the result in a privacy-preserving manner.  We highlight that the main difference is that in this (hypothetical) scheme the server is \emph{always involved};  whereas, in our protocol, \aud remains offline as long as the clients behave honestly and it is invoked only when the contract detects misbehaviours.  


\subsubsection{Charging the Buyer a Flat Fee.}
One may want to add a straightforward payment mechanism to an existing multi-party PSI in such a way that a buyer always must pay a fixed amount, e.g., depending on the total number of clients and the minimum size of the sets. However, this approach is problematic, because it:

\begin{enumerate}
\item forces the buyer to always pay even if some malicious clients misbehave during the protocol execution and affect the results' correctness which would allow malicious clients to learn the result at the buyer's expense without letting it learn the correct result, as there exists no fair multi-party PSI in the literature to ensure either all parties learn a correct result or neither does. 

\item forces the buyer to always pay independent of the exact size of the intersection. If it has to pay more than the amount it would have paid for the exact cardinality of the intersection then the buyer would be discouraged to participate in the protocol in the first place. On the other hand, if the buyer has to pay less than what it would have paid for the size of the intersection, then other clients would be discouraged to participate in the protocol. 

\end{enumerate}



% !TEX root =main.tex


\section{Error Probability}\label{sec::error-prob}


Recall that in \fpsi, in step \ref{JUS::check-non-zero-coeff},  each client $C$ needs to ensure polynomials $\bm\omega^{\st (C,D)}\cdot \bm\pi^{\st  {  {(C)}}}$ and  $\bm\rho^{\st (C,D)}$ do not contain any zero coefficient. This check ensures that client $C$ (the receiver in \vopr) does not insert any zero values to  \vopr (in particular to $\ole^{\st +}$ which is a subroutine of \vopr). If a zero value is inserted in \vopr, then an honest receiver will learn only a random value and more importantly cannot pass \vopr's verification phase. 
%
Nevertheless,  this check can be removed from step \ref{JUS::check-non-zero-coeff}, if we allow \fpsi to output an error with a small probability.\footnote{By error we mean even if all parties are honest,  \vopr halts when an honest party inserts $0$ to it.} In the remainder of this section, we show that this probability is negligible. 


First, we focus on the product $\bm\omega^{\st (C,D)}\cdot \bm\pi^{\st  {  {(C)}}}$. We know that $\bm\pi^{\st  {  {(C)}}}$ is of the form $\prod\limits^{\st d}_{\st i=1} (x-s'_{\st i})$, where $s'_{\st i}$ is either a set element $s_{\st i}$ or a random value.  Thus, all of its coefficients are non-zero. Also, the probability that at least one of the coefficients  of $d$-degree random polynomial $\bm\omega^{\st (C,D)}$ equals $0$ is at most $\frac{d+1}{p}$.  Below, we state it formally.


\begin{theorem}\label{theorem::zero-coeff-in-ran-poly}
Let $\bm\delta=\sum\limits_{\st t=0}^{\st d}u_{\st  t} \cdot x^{\st t}$, where  $u_{\st  t}\stackrel{\st\$}\leftarrow\mathbb{F}_{\st p}$, for all $t, 0\leq t\leq d$. Then, the probability that at least one of the coefficients equals $0$ is at most $\frac{d+1}{p}$, i.e., 

$$Pr[\exists u_{\st t}, u_{\st t}=0]\leq \frac{d+1}{p}$$
\end{theorem}


\begin{proof}
The proof is straightforward. Since $\bm\delta$'s coefficients are picked uniformly at random from  $\mathbb{F}_{\st p}$, the probability that $u_{\st t}$ equals $0$ is $\frac{1}{p}$. Since $\bm\delta$ is of degree $d$, due to the union bound, the probability that at least one of $c_{\st t}$s equals $0$ is at most $\frac{d+1}{p}$. 
 \hfill\(\Box\)\end{proof}



Next, we show that the probability that the polynomial  $\bm\omega^{\st (C,D)}\cdot \bm\pi^{\st  {  {(C)}}}$ has at least one zero coefficient is negligible in the security parameter; we assume polynomial $\bm\omega^{\st (C,D)}$  has no zero coefficient. 


%For the sake of simplicity,  we set $\bm{p}_1=\bm\omega^{\st (C,D)}=a_{\st 0}\cdot a_{\st 1}\cdot x+...+a_{\st d}\cdot x^{\st d}$ and $\bm{p}_2= \bm\pi^{\st  {  {(C)}}}$. 

\begin{theorem}\label{theorem::zero-coeff-in-product}
Let  $\bm\alpha=\sum\limits_{\st j=0}^{\st m}a_{\st  i} \cdot x^{\st i}$ and  $\bm\beta=\sum\limits_{\st j=0}^{\st n}b_{\st  j} \cdot x^{\st j}$,  where $a_{\st i}\stackrel{\st\$}\leftarrow \mathbb{F}_{\st p}$ and $a_{\st i}, b_{\st j}\neq0$, for all $i,j, 0\leq i \leq m$ and $0\leq j \leq n$. Also, let $\bm\gamma=\bm\alpha\cdot\bm\beta=\sum\limits_{\st j=0}^{\st m+n}c_{\st  j} \cdot x^{\st j}$. Then, the probability that at lest one of the coefficients of polynomial $\bm\gamma$ equals $0$ is at most $\frac{m+n+1}{p}$, i.e., 
%
$$Pr[\exists c_{\st j}, c_{\st j}=0]\leq \frac{m+n+1}{p}$$
%
\end{theorem}



\begin{proof}
Each coefficient $c_{\st  k}$ of $\bm\gamma$ can be defined as $c_{\st  k}=\sum\limits_{\substack{\st j=0\\ \st i=0}}^{\substack{\st i=m\\ \st j=n}}a_{\st i}\cdot b_{\st j}$, where $i+j=k$. We can rewrite $c_{\st  k}$ as  $c_{\st  k}=a_{\st w}\cdot b_{\st z}+ \sum\limits_{\substack{\st j=0, j\neq z\\ \st i=0, i\neq w}}^{\substack{\st i=m\\ \st j=n}}a_{\st i}\cdot b_{\st j}$, where $w+z=i+j=k$. 
%
We consider two cases for each $c_{\st  k}$:

\begin{itemize}

\item[$\bullet$]  {Case 1}: $\sum\limits_{\substack{\st j=0, j\neq z\\ \st i=0, i\neq w}}^{\substack{\st i=m\\ \st j=n}}a_{\st i}\cdot b_{\st j}=0$.  This is a trivial case, because with the  probability of $1$ it holds that $c_{\st  k}=a_{\st w}\cdot b_{\st z}\neq 0$, as by  definition $a_{\st w}, b_{\st z}\neq0$ and $\mathbb{F}_{\st p}$ is an integral domain. 

%In this case, since $a_{\st w}$ has been picked uniformly at random, the probability that $c_{\st  k}=a_{\st w}\cdot b_{\st z}=0$ is $\frac{1}{p}$.

%
% This is a trivial case, because with the  probability of $1$ it holds that $c_{\st  k}=a_{\st w}\cdot b_{\st z}\neq 0$. 


%In this case, because $a_{\st w}$ has been picked uniformly at random, with the  probability of $\frac{1}{p}$ it holds that $c_{\st  k}=a_{\st w}\cdot b_{\st z}=0$. 


\item[$\bullet$]  {Case 2}: $q=\sum\limits_{\substack{\st j=0, j\neq z\\ \st i=0, i\neq w}}^{\substack{\st i=m\\ \st j=n}}a_{\st i}\cdot b_{\st j}\neq 0$. In this case, for  event $c_{\st  k}=a_{\st w}\cdot b_{\st z}+q=0$ to occure, $a_{\st w}\cdot b_{\st z}$ must equal the additive inverse of $q$. Since $a_{\st w}$ has been picked uniformly at random, the probability that such an even occurs is $\frac{1}{p}$.
\end{itemize}

The above analysis is for a single $c_{\st  k}$. Thus, due to the union bound, the probability that at least one of the coefficients $c_{\st  k}$ equals $0$ is at most $\sum\limits^{\st m+n}_{\st j=0}\frac{1}{p}= \frac{m+n+1}{p}$.
%
 \hfill\(\Box\)\end{proof}




Next we turn out attention to $\bm\rho^{\st (C,D)}$. Due to Theorem \ref{theorem::zero-coeff-in-ran-poly}, the probability that at least one of the coefficients of $\bm\rho^{\st (C,D)}$ equals $0$ is at most $\frac{d+1}{p}$, . 

%\begin{theorem}\label{theorem::zero-coeff-in-ran-poly}
%Let $\bm\delta=\sum\limits_{\st t=0}^{\st d}u_{\st  t} \cdot x^{\st t}$, where  $u_{\st  t}\stackrel{\st\$}\leftarrow\mathbb{F}_{\st p}$, for all $t, 0\leq t\leq d$. Then, the probability that at least one of coefficients equals $0$ is at most $\frac{d}{p}$, i.e., 
%$Pr[\exists d_{\st j}, d_{\st j}=0]\leq \frac{d+1}{p}$.
%\end{theorem}
%
%
%\begin{proof}
%The proof is straightforward. Since $\bm\delta$'s coefficients are picked uniformly at random from  $\mathbb{F}_{\st p}$, the probability that $c_{\st j}$ equals $0$ is $\frac{1}{p}$. Since $\bm\delta$ is of degree $d$, due to the union bound, the probability that at least one of $c_{\st j}$s equals $0$ is at most $\frac{d+1}{p}$. 
% \hfill\(\Box\)\end{proof}

Hence, due to Theorems \ref{theorem::zero-coeff-in-ran-poly}, \ref{theorem::zero-coeff-in-product}, and union bound, the probability that at least one of the coefficients in $\bm\omega^{\st (C,D)}\cdot \bm\pi^{\st  {  {(C)}}}$ and  $\bm\rho^{\st (C,D)}$ equals $0$ is at most $\frac{3d+2}{p}$, which is negligible is the security parameter $p$. 


















 
% !TEX root =main.tex


%%%%%%%%%%%%%%%%%
 %%%%%%%%%%%%%
\subsection{Main Challenges that \withFai Overcomes}\label{sec::Justitia-challenges}

To design an efficient scheme that realises \p,  we had to address several key challenges. Below, we outline these challenges.



 \subsubsection{Keeping Overall Complexities Low.}
 
 In general, in multi-party PSIs, each client must exchange messages with the other clients and potentially engage in secure computations with them, as seen in \cite{DBLP:conf/scn/InbarOP18,DBLP:conf/ccs/KolesnikovMPRT17}. This can lead to communication and computational costs that grow quadratically with the number of clients.
 
 To tackle this challenge, we employed two strategies: (a) allowing one of the clients to act as a dealer, interacting with the remaining clients\footnote{This approach has similarity with the non-secure PSIs in \cite{GhoshN19}.}, and (b) implementing a smart contract that serves as a bulletin board for receiving most messages and conducting lightweight computations on the clients' messages. The combination of these approaches ensures that the overall communication and computation remain linear in relation to the number of clients (and the cardinality of sets).



 
 
 \subsubsection{Randomising Input Polynomials.}  In multi-party PSIs that utilise the polynomial representation, it is crucial for a client's input polynomial to undergo randomisation by another client \cite{AbadiMZ21}. To achieve this securely and efficiently, we required the dealer and each client to jointly participate in an instance of  \vopr, a protocol we developed in Section \ref{sec::subroutines}. 
 

 
 \subsubsection{Preserving the Privacy of Outgoing Messages.} 
 
 
 While the utilisation of public smart contracts, such as Ethereum, helps maintain overall complexity low, it introduces another challenge. Specifically, if clients fail to safeguard the privacy of the messages they transmit to the smart contracts, then both other clients (e.g., the dealer) and individuals who are not participants in PSI (i.e., the public) can gain access to the clients' set elements and/or the intersection.
 
 To ensure the efficient protection of each client's messages sent to the contracts from the dealer, we necessitate that the clients, excluding the dealer, participate in \zspaa. This protocol allows each client to create a pseudorandom polynomial, which they can employ to obscure their messages. To safeguard the privacy of the intersection from the public, we require that  all clients to run a coin-tossing protocol to reach a consensus on a blinding polynomial.  This blinding polynomial will be used to obscure the final result that encodes the intersection on the smart contract.  
 
 

 \subsubsection{Ensuring the Correctness of Subroutine Protocols' Outputs.} 
 
 
 Typically, any MPC protocol designed to withstand active adversaries incorporates a verification mechanism to detect any tampering with message integrity during the protocol's execution. This applies to the subroutine protocols we utilise, namely \vopr and \zspaa. 
 
 
  However, relying solely on this type of check is not always adequate. There are situations where the output of one MPC serves as input to another MPC, and it becomes essential to guarantee that the unaltered output of the first MPC is securely passed to the second one. This holds true for our PSI's subroutines as well. 
  
  To address this challenge, we employ unforgeable polynomials. Specifically, the output of \vopr is an unforgeable polynomial that encodes the actual output. If the adversary tampers with the \vopr's output and later uses it, then a verifier can detect this tampering. 
 
 We obtain the same integrity guarantee for the output of \zspaa without any additional effort. This is because (i) \vopr is called before \zspaa, and (ii) if clients use the unaltered outputs of \zspaa, then the final result (i.e., the sum of all clients' messages) will not contain any output of \zspaa, as they will cancel each other out. Thus, by verifying the correctness of the final result, one can ensure the correctness of the outputs of \vopr and \zspaa, in a single step. 
 %%%%%%%%%%%%%%





 


% !TEX root =main.tex


\subsubsection{Confidentiality of polynomial $\zeta$.} 

 The dealer is the sole entity responsible for selecting and possessing knowledge of the secret polynomial $\zeta$.  This polynomial remains confidential until all parties submit their messages to the smart contract. Subsequently, in step \ref{f-psi::D-gen-switching-poly} on page~\pageref{f-psi::D-gen-switching-poly}, the dealer transmits $\zeta$ to the smart contract. After this juncture, $\zeta$ is no longer a secret. However, malicious parties gain no advantage from knowing $\zeta$, as they have already submitted their inputs to the smart contract.




% !TEX root =main.tex



\subsection{Concrete Parameters}\label{sec::conc-parameters}

\subsubsection{Hash Table Parameters.}

As stated in Appendix \ref{Preliminary-Hash-Table}, a hash table has the following main parameters: (1) $h$: the number of bins, (2) $d$: the bin’s maximum size (or capacity), (3) $c$: the maximum number of elements that are mapped to the hash table, and (4) $pr$: the probability that the number of elements mapped to a bin does not exceed a predefined capacity. Note that in the context of PSI, $c$ is the maximum of the sizes of all sets.

The literature (e.g., in \cite{Feather2020-full,DBLP:conf/ccs/KolesnikovMPRT17,DBLP:conf/uss/Pinkas0SZ15}) has already studied the concrete parameters of a hash table, even in the context of PSI. For instance, as demonstrated in Section 6 and Appendix J.1 in \cite{Feather2020-full}, when $c$ is in the range $[2^{\st 10},\ 2^{\st 20}]$ and $pr=2^{\st -40}$, then $d=100$. Furthermore, we have $h\approx\frac{4c}{d}$. To provide a concrete value, we would have $h=41943$, when $c=2^{\st 20}$ and $d=100$. 

\subsubsection{Field Size.} 
%
In this paper, all arithmetic operations are defined over a finite field $\mathbb{F}_{\st p}$, where $\log_{\st 2}(p)=\lambda$ represents the security parameter. The outputs of $\mathtt {PRF}(.)$ and  $\mathtt {PRP}(.) $ are also of size $\lambda$. Concrete value of $\lambda$ can be set based on the maximum bit size of set elements. For instance, one can choose $\lambda = 60$ and $\lambda = 128$, when the maximum bit size of set elements is slightly less than $60$ and $128$ respectively. 





