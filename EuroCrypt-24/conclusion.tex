% !TEX root =main.tex

\vspace{-4mm}
\section{Conclusion and Future Direction}\label{sec::concl}
\vspace{-2mm}


PSI stands as a crucial protocol with numerous applications. In this paper, we introduced \withFai, the first multi-party fair PSI, guaranteeing that either all parties obtain the result or, in case of an unfair protocol termination, honest parties receive financial compensation. We then enhanced it into \withRew, the first PSI that ensures honest parties who contribute their private sets receive rewards proportionate to the number of elements they disclose. 

Since the concept of rewarding participants in MPC could potentially boost MPC's real-world adoption, a compelling open question arises: \textit{How can we generalise the idea of rewarding participants in MPC?}


%PSI is a crucial protocol with numerous real-world applications. In this work, we proposed, \withFai, the first multi-party fair PSI that ensures that either all parties get the result or if the protocol aborts in an unfair manner, then honest parties will receive financial compensation. We then upgraded it to \withRew, the first PSI ensuring that honest parties who contribute their private sets receive a reward proportionate to the number of elements they reveal. Since an MPC that rewards participants for contributing their private inputs would help increase its real-world adoption, an interesting open question is: 
%%
%%\begin{center}
%%
% \emph{How can we generalise the idea of rewarding participants to MPC?}
 %
 %\end{center}