% !TEX root =main.tex


\vspace{-3.3mm}
\subsection{Zero-sum Pseudorandom Values Agreement Protocol (\zspa)}
\vspace{-1mm}

The \zspa  allows $m$ parties (the majority of which maybe malicious) to efficiently agree on (a set of vectors, where each $i$-th vector has) $m$ pseudorandom values such that their sum equals zero. 

At a high level, \zspa operates as follows. Initially, the parties sign a smart contract and register record their respective  addresses within it. Subsequently, they execute a  coin-tossing protocol to as \ct to reach a consensus on a key $k$.  Following this, one of the parties generates $m-1$ pseudorandom values $z_{\scriptscriptstyle i, j}$ (where $1\leq j\leq m-1$) using key $k$ and $\mathtt{PRF}$. The last value in this set is determined as the additive inverse of the sum of the previously generated values, expressed as
 $z_{\scriptscriptstyle i, m}=-\sum\limits^{\scriptscriptstyle m-1}_{\scriptscriptstyle j=1}z_{\scriptscriptstyle i, j}$ (akin to the standard XOR-based secret sharing \cite{Schneier0078909}). 
%
%Next, it commits to each value, where it uses $k_{\scriptscriptstyle 2}$ to generate the randomness of each commitment. 

Next, it constructs a Merkle tree  atop the pseudorandom values and retains solely the root of the tree $g$ and the hash value $q$ of the key within the smart contract. Subsequently, each party, utilising the key, independently verifies if the values on the contract have been constructed correctly. If this verification passes, the party sends a signed message labeled as ``approved'' to the contract. Note that the contract exclusively accepts messages from registered parties. 

Hence, the functionality that \zspa computes is defined as $f^{\st \zspa}\underbrace{(\bot,..., \bot)}_{\st m}\rightarrow \underbrace{((k, g, q),..., (k, g,q))}_{\st m}$, where $g$ is the root of the Markle tree built on the pseudorandom values $z_{\st i, j}$, $q$ is the hash value of the key used to generate the pseudorandom values, and $m\geq 2$. Figure \ref{fig:ZSPA} presents \zspa in detail.  



\zspa will be used in \withFai to enable clients $\{A_{\st 1},...,A_{\st m}\}$ to provably agree on sets of pseudorandom values. Each client uses the elements within these sets as coefficients to generate a pseudorandom polynomial. When all clients' pseudorandom polynomials are combined, the result is guaranteed to be zero, due to the property of \zspa.  Each of these polynomials will be used by a client to blind the messages it sends to the smart contract, to protect the privacy of the plaintext message (from \aud, D, and the public). To compute the sum of the plaintext messages, one can easily sum the blinded messages, which removes the blinding polynomials. 



%\zspa will be used in \withFai to let clients $\{A_{\st 1},...,A_{\st m}\}$ provably agree on sets of pseudorandom values, where each set represents a pseudorandom polynomial (as the set's elements are considered the polynomial's coefficients). Due to \zspa's property, the sum of these polynomials is zero.  Each of these polynomials will be used by a client to blind the messages it sends to the smart contract, to protect the privacy of the plaintext message (from \aud, D, and the public). To compute the sum of the plaintext messages, one can easily sum the blinded messages, which removes the blinding polynomials. 

\vspace{-1.5mm}
% !TEX root =main.tex




\begin{figure}[ht]%[!htbp]
\setlength{\fboxsep}{1pt}
\begin{center}
\scalebox{.85}{
    \begin{tcolorbox}[enhanced,width=5.5in, 
    drop fuzzy shadow southwest,
    colframe=black,colback=white]


\small{

\begin{enumerate}[leftmargin=1mm]
\item[$\bullet$]    {Parties.} A set of clients $\{    A_{\st 1},...,  A_{\st m}\}$.
%
\item[$\bullet$]    {Input.}  $m$: the total number of participants, $adr$: a deployed smart contract's address, and $b$: the total number of vectors. Let $b'=b-1$. 
%
\item[$\bullet$]   {Output.}  $k$: a secret key that generates $b$ vectors $[z_{\scriptscriptstyle 0,1},...,z_{\scriptscriptstyle 0,m}],...,[z_{\scriptscriptstyle b',1},...,z_{\scriptscriptstyle b',m}]$ of pseudorandom values, $h$: hash of the key,  $g$: a Merkle tree's root, and a vector of signed messages. 


%, such that the sum of each vector's elements equals zero: $\sum\limits^{\scriptscriptstyle m}_{\scriptscriptstyle j=1}z_{\scriptscriptstyle i,j}=0$. 


\item {\textbf{Coin-tossing.} $\ct (in_{\st 1},..., in_{\st m})\rightarrow k$}. 

All participants run a coin-tossing protocol to agree on $\mathtt{PRF}$'s key, $k$.
\item\label{ZSPA:val-gen}  {\textbf{Encoding.} $\mathtt{Encode}(k, m)\rightarrow (g,q)$}.

 One of the parties takes the following steps:  
\begin{enumerate}

\item for every $i$ (where $0\leq i \leq b'$), generates $m$ pseudorandom values as follows. 
%
 $$\forall j, 1\leq j \leq m-1: z_{\scriptscriptstyle i,j}=\mathtt{PRF}(k,i||j), \hspace{5mm} z_{\scriptscriptstyle i,m}=-\sum\limits^{\scriptscriptstyle m-1}_{\scriptscriptstyle j=1}z_{\scriptscriptstyle i,j}$$
%
\vs
\item   constructs a Merkel tree on top of all pseudorandom values,  $\mkgen(z_{\scriptscriptstyle 0,1},...,z_{\scriptscriptstyle b',m})\rightarrow g$. 

\item sends the Merkel tree's root: $g$,   and the key's hash: $q=\mathtt {H}(k)$ to $adr$. 

\end{enumerate}

\item\label{ZSPA:verify}{\textbf{Verification.} $\mathtt{Verify}(k, g, q, m)\rightarrow (a, s)$}. 

Each party checks if, all $z_{\scriptscriptstyle i,j}$ values, the root $g$, and key's hash $q$ have been correctly generated, by retaking  step \ref{ZSPA:val-gen}. If the checks pass, it sets $a=1$,  sets $s$ to a singed ``approved'' message, and sends $s$ to $adr$. Otherwise, it aborts by returning $a=0$ and $s = \bot$. 


 \end{enumerate}
}
 \end{tcolorbox}
 }
\end{center}
\vs
\vs
\caption{Zero-sum Pseudorandom Values Agreement (\zspa) } 
\label{fig:ZSPA}
\end{figure}



%\vspace{-2mm}
\begin{theorem}\label{theorem::ZSPA-comp-correctness}
Let $f^{\st \zspa}$ be the functionality defined above. If \ct is secure against a malicious adversary and the correctness of $\mathtt{PRF}$, $\mathtt{H}$, and Merkle tree hold, then \zspa,  in Figure \ref{fig:ZSPA}, securely computes $f^{\st \zspa}$ in the presence of $m-1 $ malicious adversaries. 
\end{theorem}


We refer readers to Appendix \ref{sec::proof-of-zspa} for the proof of Theorem \ref{theorem::ZSPA-comp-correctness}. 


%Informally, there are four main security requirements that ZSPA must meet: (a) privacy, (b)  non-refutability, (c) indistinguishability, and (d) result correctness. Privacy here means given the state of the contract, an external party cannot learn any information about any of the (pseudorandom) values:  $z_{\scriptscriptstyle j}$; while non-refutability means that if a party sends ``approved" then in future cannot deny the knowledge of the values whose representation is stored in the contract. Furthermore, indistinguishability means that every $z_{\scriptscriptstyle j}$ ($1\leq j \leq m$) should be indistinguishable from a truly random value and result correctness means that a malicious result generator cannot convince other parties to accept an invalid final result, i.e., the root constructed on the invalid leaf node(s). In Figure \ref{fig:ZSPA}, we provide ZSPA that efficiently generates $b$ vectors where each vector elements is sum to zero. 






%\begin{figure}[ht]
%\setlength{\fboxsep}{0.7pt}
%\begin{center}
%\begin{boxedminipage}{12.3cm}

%
%\begin{figure}[ht]%[!htbp]
%\setlength{\fboxsep}{1pt}
%\begin{center}
%    \begin{tcolorbox}[enhanced,width=5.5in, 
%    drop fuzzy shadow southwest,
%    colframe=black,colback=white]
%
%
%\small{
%
%\begin{enumerate}
%\item[$\bullet$]  \textit{Parties.} $\{\resizeT {\textit A}_{\resizeS {\textit  1}},..., \resizeT {\textit A}_{\resizeS {\textit  m}}\}$
%\item[$\bullet$]  \textit{Input.}  $m$: the total number of participants and a deployed smart contract's address. 
%\item[$\bullet$] \textit{Output.}  $k$: a secret key that generates $b+1$ vectors $[z_{\scriptscriptstyle 0,1},...,z_{\scriptscriptstyle 1,m}],...,[z_{\scriptscriptstyle b,1},...,z_{\scriptscriptstyle b,m}]$ of pseudorandom values, $h$: hash of the key,  $g$: a Merkle tree's root, and a vector of signed messages. 
%
%
%%, such that the sum of each vector's elements equals zero: $\sum\limits^{\scriptscriptstyle m}_{\scriptscriptstyle j=1}z_{\scriptscriptstyle i,j}=0$. 
%
%
%\item All participants run a coin tossing protocol to agree on a key $k$  of $\mathtt{PRF}$.
%\item\label{ZSPA:val-gen} One of the parties:  
%\begin{enumerate}
%
%\item for every $i$ (where $0\leq i \leq b$), generates $m$ pseudorandom values as follows. 
%%
% $$\forall j, 1\leq j \leq m-1: z_{\scriptscriptstyle i,j}=\mathtt{PRF}(k,i||j), \hspace{5mm} z_{\scriptscriptstyle i,m}=-\sum\limits^{\scriptscriptstyle m-1}_{\scriptscriptstyle j=1}z_{\scriptscriptstyle i,j}$$
%%
%\item   constructs a Merkel tree on top of all pseudorandom values,  $\mathtt{MT.genTree}(z_{\scriptscriptstyle 0,1},...,z_{\scriptscriptstyle b,m})\rightarrow g$. 
%
%\item  sends the Merkel tree's root: $g$,   and the key's hash: $q=\mathtt {H}(k)$ to the smart contract. 
%
%\end{enumerate}
%
%\item\label{ZSPA:verify} The rest of parties (given $k_{\scriptscriptstyle 1}, k_{\scriptscriptstyle 2}$) check if, all $z_{\scriptscriptstyle i,j}$ values, the root $g$ and key's hash have been correctly generated (by redoing  step \ref{ZSPA:val-gen}). If the checks pass, each party sends a singed ``approved'' message to the  contract. Otherwise, it aborts. 
%
%
% \end{enumerate}
%}
% \end{tcolorbox}
%\end{center}
%\caption{Zero-sum Pseudorandom Values Agreement (ZSPA) Protocol} 
%\label{fig:ZSPA}
%\end{figure}
%




%%%%%%%%%%%%%%%%%%%%%%%%%%%%%%%%%%%%%%%
%\begin{figure}[ht]
%\setlength{\fboxsep}{0.7pt}
%\begin{center}
%\begin{boxedminipage}{12.3cm}
%
%\small{
%
%\begin{enumerate}
%\item[$\bullet$]  \textit{Parties:} $\{\resizeT {\textit A}_{\resizeS {\textit  1}},..., \resizeT {\textit A}_{\resizeS {\textit  m}}\}$
%\item[$\bullet$]  \textit{Public Parameters and Functions:} A pseudorandom function: $\mathtt{PRF}$, a deployed smart contract, and total number of participants: $m$. 
%\item[$\bullet$] \textit{Output}:  All parties agree on $b+1$ vectors $[z_{\scriptscriptstyle 0,1},...,z_{\scriptscriptstyle 1,m}],...,[z_{\scriptscriptstyle b,1},...,z_{\scriptscriptstyle b,m}]$, of pseudorandom values, such that the sum of each vector's elements equals zero: $\sum\limits^{\scriptscriptstyle m}_{\scriptscriptstyle j=1}z_{\scriptscriptstyle i,j}=0$
%
%
%\item All participants run a coin tossing protocol to agree on two keys $k_{\scriptscriptstyle 1}$ and $k_{\scriptscriptstyle 2}$ of $\mathtt{PRF}$.
%\item\label{ZSPA:val-gen} One of the parties:  
%\begin{enumerate}
%
%\item For every $i$, computes $m$ pseudorandom values: $\forall j, 1\leq j \leq m-1: z_{\scriptscriptstyle i,j}=\mathtt{PRF}(k_{\scriptscriptstyle 1},i||j)$ and sets $z_{\scriptscriptstyle i,m}=-\sum\limits^{\scriptscriptstyle m-1}_{\scriptscriptstyle j=1}z_{\scriptscriptstyle i,j}$, where $0\leq i \leq b$
%
%\item   commits to every $z_{\scriptscriptstyle i,j}$  as follows: $\mathtt{a}_{\scriptscriptstyle i,j}=\mathtt{Com}(z_{\scriptscriptstyle i,j}, q_{\scriptscriptstyle i,j})$, where the randomness of the commitment is computed as: $ q_{\scriptscriptstyle i,j}=\mathtt{PRF}(k_{\scriptscriptstyle 2},i||j)$ and  $1\leq j \leq m$.
%
%\item   constructs a Merkel tree on top of the committed values: $\mathtt{MT}(\mathtt{a}_{\scriptscriptstyle 0,1},...,\mathtt{a}_{\scriptscriptstyle b,m})\rightarrow g$ 
%
%\item  sends the Merkel tree's root: $g$,   and the keys' hashes: $\mathtt {H}(k_{\scriptscriptstyle 1})$ and $ \mathtt {H}(k_{\scriptscriptstyle 2})$, to the contract. 
%
%\end{enumerate}
%
%\item\label{ZSPA:verify} The rest of parties (given $k_{\scriptscriptstyle 1}, k_{\scriptscriptstyle 2}$) check if, all $z_{\scriptscriptstyle i,j}$ values, the root $g$ and keys' hashes have been correctly generated (by redoing  step \ref{ZSPA:val-gen}). If passed, each party sends a singed ``approved'' message to the  contract. Otherwise, it aborts. 
%
%
% \end{enumerate}
%}
%\end{boxedminipage}
%\end{center}
%\caption{Zero-sum Pseudorandom Values Agreement ($\mathtt{ZSPA}$) Protocol} 
%\label{fig:ZSPA}
%\end{figure}


