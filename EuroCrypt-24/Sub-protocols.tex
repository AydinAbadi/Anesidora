% !TEX root =main.tex


\vspace{-4.5mm}



\section{Other Subroutines Used in \withFai}\label{sec::subroutines}
\vspace{-1.4mm}

In this section, we present three subroutines and a primitive that we have developed and used in the instantiation of \p, known as \withFai. 


\vspace{-3mm}
\subsection{Verifiable Oblivious Polynomial Randomisation (\vopr)}\label{sec::vopr}
\vspace{-1.2mm}

%In this section, we present ``Verifiable Oblivious Polynomial Randomisation'' (VOPR) protocol. 

In the \vopr, two type of parties are involved, (i) a sender, potentially a passive adversary, and (ii) a receiver, potentially an active adversary. The protocol enables the receiver, with input polynomial $\bm\beta$ (of degree $e'$), and the sender, with input random polynomials $\bm\psi$ (of degree $e$) and  $\bm{\alpha}$ (of degree $e+e'$),   to compute: $\bm\theta=\bm\psi\cdot \bm\beta+\bm\alpha$, satisfying the following conditions: (a) the receiver learns only $\bm\theta$ and gains no knowledge of the sender's input even when setting $\bm \beta=0$, (b) the sender gains no knowledge, and (c) protocol detects the receiver's misbehavior.  Thus, the functionality that  \vopr computes is defined as $f^{\st {\vopr}}( (\bm\psi, \bm{\alpha}), \bm\beta)\rightarrow(\bot, \bm\psi\cdot \bm\beta+\bm\alpha)$. 

We will use {\vopr} in \withFai for two main purposes:  (a) to enable a party to re-randomise its counterpart's polynomial (representing its set) and (b) to impose a MAC-like structure on the randomised polynomial.  This structure will allow a verifier to detect any modifications to \vopr's output. 

Now, we will outline how we design \vopr without using any zero-knowledge proofs.\footnote{Previously, Ghosh \textit{et al.}  \cite{GhoshN19} designed a protocol called Oblivious Polynomial Addition (OPA) to meet similar security requirements that we laid out above. But, as shown in \cite{AbadiMZ21}, OPA  is susceptible to several serious attacks. } In the setup phase, both parties represent their input polynomials in the regular coefficient form; therefore, the sender's polynomials are defined as $\bm\psi=\sum\limits^{\st e}_{\st i=0}g_{\st i}\cdot x^{\st i}$ and  $\bm\alpha=\sum\limits^{\st e+e'}_{\st j=0}a_{\st j}\cdot x^{\st j}$ and the receiver's polynomial is defined as $\bm\beta=\sum\limits^{\st e'}_{\st i=0}b_{\st i}\cdot x^{\st i}$, where $b_{\st i}\neq 0$. However, the sender computes each coefficient $a_{\st j}$ (of polynomial $\bm \alpha$) as follows,  $a_{\st j}=\sum\limits^{\substack{\st k=e'\\ \st t=e}}_{\st t,k=0} a_{\st t,k}$,  where  $t+k=j$ and each $a_{\st t,k}$ is a random value. For instance, if $e=4$ and $e'=3$, then $a_{\st 3}=a_{\st \st 0,3}+a_{\st 3,0}+a_{\st 1,2}+a_{\st 2,1}$. Shortly, we explain why polynomial $\bm\alpha$ is constructed this way. 



In the computation phase,  to compute polynomial $\bm\theta$, the two parties interactively multiply and add the related coefficients in a secure way using $\ole^{\st +}$. Specifically,
%
%For simplicity, let $i=0$. 
%
for every $j$  (where $0\leq j\leq e'$) the sender sends $g_{\st i}$ and $a_{\st i,j}$ to an instance of  $\ole^{\st +}$, while the receiver sends $b_{\st j}$ to the same instance,  which returns $c_{\st i,j}=g_{\st i}\cdot b_{\st j}+ a_{\st i,j}$ to the receiver. This process is repeated for every $i$, where $0 \leq i \leq e$. Then, the receiver uses $c_{\st i,j}$ values to construct the resulting polynomial, $\bm\theta=\bm\psi\cdot \bm\beta+\bm\alpha$.  

The sender imposes the above structure on (the coefficients of)  $\bm\alpha$ during the setup to enable the parties to securely compute $\bm\theta$ via  $\ole^{\st +}$.  This structure serves two key purposes: (1) allowing the sender  to blind each product $g_{\st i}\cdot b_{\st j}$  with  random value $a_{\st i,j}$ which is a component of $\bm\alpha$'s coefficient and (2) enabling the receiver to construct a result polynomial in the form $\bm\theta=\bm\psi\cdot \bm\beta+\bm\alpha$. 


To verify the correctness of the result, the sender selects a random value, $z$, and transmits it to the receiver.   Subsequently, the receiver computes  $\bm\theta(z)$ and $\bm\beta(z)$ and sends these two values  to the sender. The sender computes  $\bm\psi(z)$ and $\bm\alpha(z)$ and proceeds to validate the equation $\bm\theta({ z})=\bm\psi({ z})\cdot \bm\beta({ z})+\bm\alpha({ z})$. If the check is successful, the sender accepts the result. Figure \ref{fig:VOPR} describes \vopr in detail. 


\vspace{-6.3mm}
% !TEX root =main.tex



%%%%%%%%
\begin{figure}[!htb]%[!htbp]
\setlength{\fboxsep}{.8pt}
\begin{center}
\scalebox{.85}{
    \begin{tcolorbox}[enhanced,width=5.5in, 
    drop fuzzy shadow southwest,
    colframe=black,colback=white]
%%%%%%%%

\svs
\begin{enumerate}[leftmargin=1mm]
\small{
\item[$\bullet$] \textit{Input.}
\begin{enumerate}[leftmargin=3mm]
\item[$\bullet$]  \textit{Public Parameters}: upper bound on input polynomials' degree: $e$ and $e'$. %, where $e\geq e'$.
%\item[$\bullet$]  \textit{Sender}: picks an upper bound on input polynomials's degree:  $d$ and $d'$. It sends them to the receiver.
\item[$\bullet$]  \textit{Sender Input}:  random polynomials: $\bm\psi=\sum\limits^{\st e}_{\st i=0}g_{\st i}\cdot x^{\st i}$ and  $\bm\alpha=\sum\limits^{\st e+e'}_{\st j=0}a_{\st j}\cdot x^{\st j}$, where $g_{\st i}\stackrel{\st \$}\leftarrow \mathbb{F}_p$.  Each $a_{\st j}$ has the  form: $a_{\st j}=\sum\limits^{\substack{\st k=e'\\ \st t=e}}_{\st t,k=0} a_{\st t,k}$,  such that $t+k=j$ and $a_{\st t,k}\stackrel{\st \$}\leftarrow \mathbb{F}_p$.

\item[$\bullet$] \textit{Receiver Input}:  polynomial $\bm\beta=\bm\beta_{\st 1}\cdot \bm\beta_{\st 2}=\sum\limits^{\st e'}_{\st i=0}b_{\st i}\cdot x^{\st i}$, where $\bm\beta_{\st 1}$ is a random polynomial of degree $1$ and $\bm\beta_{\st 2}$ is an arbitrary polynomial of degree $e'-1$.


\end{enumerate}
\item[$\bullet$] \textit{Output.} The receiver gets $\bm\theta=\bm\psi\cdot \bm\beta+\bm\alpha$.
\item \textbf{Computation:}

\begin{enumerate} [leftmargin=3mm]

\item Sender and receiver together for every $j$, $0\leq j\leq e'$,  invoke $e+1$ instances of $\ole^{\st +}$. In particular, $\forall j, 0\leq j\leq e'$: sender sends $g_{\st i}$ and $a_{\st i,j}$ while the receiver sends $b_{\st j}$ to $\ole^{\st +}$ that returns: $c_{\st i,j}=g_{\st i}\cdot b_{\st j}+ a_{\st i,j}$ to the receiver ($\forall i, 0\leq i\leq e$). 


 \item The receiver sums component-wise values $c_{\st i,j}$  that results in polynomial:
 %
 \vspace{-2mm}
 %
  $$\bm\theta=\bm\psi\cdot \bm\beta+\bm\alpha=\sum\limits^{\substack{\st i=e\\ \st j=e'}}_{\st i,j=0}c_{\st i, j}\cdot x^{\st i+j}$$ 
 %
  \vspace{-2mm}
  %
 




% \item The receiver sums component-wise values $c_{\st i,j}$  that results polynomial $\bm\theta=\bm\psi\cdot \bm\beta+\bm\alpha=\sum\limits^{\st e+e'}_{\st j=0}c_{\st j}\cdot x^{\st j}$, where  each $c_{\st j}$ has   form: $c_{\st j}=\sum\limits^{\substack{\st k=e'\\ \st t=e}}_{\st  t,k=0} c_{\st t,k}$, such that $ t+k=j$.
%\item Sender: $\forall j, 1\leq j\leq 2d+1$, computes $a_{\st j}=a(x_{\st j})$ and $r_{\st j}=r(x_{\st j})$. Then, it  inserts $(a_{\st j}, r_{\st j})$ into  $\mathcal{F}_{\st OLE^{\st +}}$
%\item\label{computing-receiver} Receiver:  $\forall j, 1 \leq j\leq 2d+1$, computes $b_{\st j}=b(x_{\st j})$. Then, it  inserts $b_{\st j}$ into  $\mathcal{F}_{\st OLE^{\st +}}$ and receives $s_{\st j}=a_{\st j}+b_{\st j}\cdot r_{\st j}$. It interpolates a polynomial $s(x)$ using pairs $s_{\st j},x_{\st j}$. 
\end{enumerate}
\vs
\item \label{Verification} \textbf{Verification:}
\begin{enumerate}[leftmargin=3mm]

\item \label{picking-random-x}Sender: picks a random value  $z$ and sends it to the receiver. 


\item\label{receiver-OLE-invocation} Receiver: sends $\theta_{\st z}=\bm\theta(z)$ and $\beta_{\st z}=\bm\beta(z)$ to the sender.

\item\label{receiver-OLE-invocation} Sender:  computes $\psi_{\st z}=\bm\psi(z)$ and $\alpha_{\st z}=\bm\alpha(z)$ and checks   if equation  $\theta_{\st z}=\psi{\st z}\cdot \beta_{\st z}+\alpha_{\st z}$ holds. If the equation holds, it concludes that the computation was performed correctly. Otherwise, it aborts. 
%
\vs
\end{enumerate}
}
 \end{enumerate}
 \end{tcolorbox}
 }
\end{center}
\vs
\vs
\caption{Verifiable Oblivious Polynomial Randomization ({\vopr}) } 
\label{fig:VOPR}
\end{figure}
 %%%%%%%

Note that \vopr requires the sender to insert non-zero coefficients, i.e., $b_{\st i}\neq 0$ for all $i,0 \leq i \leq e'$. If the sender includes a zero-coefficient, then it will only obtain a random value (due to  $\ole^{\st +}$), making it fail to pass \vopr's verification phase. However, this requirement does not affect the correctness of \withFai, as we will explain in Section \ref{Fair-PSI-Protocol} and Appendix \ref{sec::error-prob}.  

\vspace{-2mm}



%\vspace{-1mm}
\begin{theorem}\label{theorem::VOPR}
%
Let $f^{\st \vopr}$ be the functionality defined above. If the enhanced \ole (i.e., $\ole^{\st +}$) is secure against malicious (or active) adversaries, then the  \vopr, presented in Figure \ref{fig:VOPR}, securely computes $f^{\st \vopr}$ in the presence of (i) semi-honest sender and honest receiver or (ii) malicious receiver and honest sender. 
%
\end{theorem}

\vspace{-1mm}
We refer readers to Appendix \ref{sec::proof-of-vopr} for the proof of Theorem \ref{theorem::VOPR}. 
\vspace{-1mm}

% !TEX root =main.tex

\subsection{Zero-sum Pseudorandom Values Agreement Protocol (\zspa)}

The \zspa  allows $m$ parties (the majority of which is potentially malicious) to efficiently agree on (a set of vectors, where each $i$-th vector has) $m$ pseudorandom values such that their sum equals zero. At a high level, the parties first sign a smart contract, register their accounts/addresses in it, and then run a  coin-tossing protocol \ct to agree on a key: $k$.  Next, one of the parties generates $m-1$ pseudorandom values $z_{\scriptscriptstyle i, j}$ (where $1\leq j\leq m-1$) using key $k$ and $\mathtt{PRF}$. It sets the last value as the additive inverse of the sum of the values generated, i.e. $z_{\scriptscriptstyle i, m}=-\sum\limits^{\scriptscriptstyle m-1}_{\scriptscriptstyle j=1}z_{\scriptscriptstyle i, j}$ (similar to the standard XOR-based secret sharing \cite{Schneier0078909}). 
%
%Next, it commits to each value, where it uses $k_{\scriptscriptstyle 2}$ to generate the randomness of each commitment. 
%
Then, it constructs a Merkel tree on top of the pseudorandom values and stores only the tree's root $g$ and the key's hash value $q$ in the smart contract.  Then, each party (using the key) locally checks if the values (on the contract) have been constructed correctly; if so, then it sends a (signed) ``approved" message to the contract which only accepts messages from registered parties. Hence, the functionality that \zspa computes is defined as $f^{\st \zspa}\underbrace{(\bot,..., \bot)}_{\st m}\rightarrow \underbrace{((k, g, q),..., (k, g,q))}_{\st m}$, where $g$ is the Markle tree's root built on the pseudorandom values $z_{\st i, j}$, $q$ is the hash value of the key used to generate the pseudorandom values, and $m\geq 2$. Figure \ref{fig:ZSPA} presents \zspa in detail.  


Briefly, \zspa will be used in \withFai to allow clients $\{A_{\st 1},...,A_{\st m}\}$ to provably agree on a set of pseudorandom values, where each set represents a pseudorandom polynomial (as the elements of the set are considered the polynomial's coefficients). Due to \zspa's property, the sum of these polynomials is zero.  Each of these polynomials will be used by a client to blind/encrypt the messages it sends to the smart contract, to protect the privacy of the plaintext message (from \aud, D, and the public). To compute the sum of the plaintext messages, one can easily sum the blinded messages, which removes the blinding polynomials. 

% !TEX root =main.tex




\begin{figure}[ht]%[!htbp]
\setlength{\fboxsep}{1pt}
\begin{center}
\scalebox{.85}{
    \begin{tcolorbox}[enhanced,width=5.5in, 
    drop fuzzy shadow southwest,
    colframe=black,colback=white]


\small{

\begin{enumerate}[leftmargin=1mm]
\item[$\bullet$]    {Parties.} A set of clients $\{    A_{\st 1},...,  A_{\st m}\}$.
%
\item[$\bullet$]    {Input.}  $m$: the total number of participants, $adr$: a deployed smart contract's address, and $b$: the total number of vectors. Let $b'=b-1$. 
%
\item[$\bullet$]   {Output.}  $k$: a secret key that generates $b$ vectors $[z_{\scriptscriptstyle 0,1},...,z_{\scriptscriptstyle 0,m}],...,[z_{\scriptscriptstyle b',1},...,z_{\scriptscriptstyle b',m}]$ of pseudorandom values, $h$: hash of the key,  $g$: a Merkle tree's root, and a vector of signed messages. 


%, such that the sum of each vector's elements equals zero: $\sum\limits^{\scriptscriptstyle m}_{\scriptscriptstyle j=1}z_{\scriptscriptstyle i,j}=0$. 


\item {\textbf{Coin-tossing.} $\ct (in_{\st 1},..., in_{\st m})\rightarrow k$}. 

All participants run a coin-tossing protocol to agree on $\mathtt{PRF}$'s key, $k$.
\item\label{ZSPA:val-gen}  {\textbf{Encoding.} $\mathtt{Encode}(k, m)\rightarrow (g,q)$}.

 One of the parties takes the following steps:  
\begin{enumerate}

\item for every $i$ (where $0\leq i \leq b'$), generates $m$ pseudorandom values as follows. 
%
 $$\forall j, 1\leq j \leq m-1: z_{\scriptscriptstyle i,j}=\mathtt{PRF}(k,i||j), \hspace{5mm} z_{\scriptscriptstyle i,m}=-\sum\limits^{\scriptscriptstyle m-1}_{\scriptscriptstyle j=1}z_{\scriptscriptstyle i,j}$$
%
\vs
\item   constructs a Merkel tree on top of all pseudorandom values,  $\mkgen(z_{\scriptscriptstyle 0,1},...,z_{\scriptscriptstyle b',m})\rightarrow g$. 

\item sends the Merkel tree's root: $g$,   and the key's hash: $q=\mathtt {H}(k)$ to $adr$. 

\end{enumerate}

\item\label{ZSPA:verify}{\textbf{Verification.} $\mathtt{Verify}(k, g, q, m)\rightarrow (a, s)$}. 

Each party checks if, all $z_{\scriptscriptstyle i,j}$ values, the root $g$, and key's hash $q$ have been correctly generated, by retaking  step \ref{ZSPA:val-gen}. If the checks pass, it sets $a=1$,  sets $s$ to a singed ``approved'' message, and sends $s$ to $adr$. Otherwise, it aborts by returning $a=0$ and $s = \bot$. 


 \end{enumerate}
}
 \end{tcolorbox}
 }
\end{center}
\vs
\vs
\caption{Zero-sum Pseudorandom Values Agreement (\zspa) } 
\label{fig:ZSPA}
\end{figure}



\begin{theorem}\label{theorem::ZSPA-comp-correctness}
Let $f^{\st \zspa}$ be the functionality defined above. If \ct is secure against a malicious adversary and the correctness of $\mathtt{PRF}$, $\mathtt{H}$, and Merkle tree holds, then \zspa,  in Figure \ref{fig:ZSPA}, securely computes $f^{\st \zspa}$ in the presence of $m-1 $ malicious  adversaries. 
\end{theorem}


\begin{proof}
For the sake of simplicity, we will assume the sender, which generates the result, sends the result directly to the rest of the parties, i.e., receivers, instead of sending it to a smart contract. We first consider the case in which the sender is corrupt. 

\

\noindent\textbf{Case 1: Corrupt sender.}  Let $\mathsf{Sim}^{\st \zspa}_{\st S}$ be the simulator using a subroutine adversary, $\mathcal{A}_{\st S}$. $\mathsf{Sim}^{\st \zspa}_{\st S}$ works as follows. 
%
\begin{enumerate}
%
\item simulates  \ct  and receives the output value $k$ from $f_{\st \ct}$, as we are in $f_{\st \ct}$-hybrid model.
%
\item sends $k$ to TTP and receives back from it $m$ pairs, where each pair is of the form $( g,  q)$. 
%
\item sends $ k$ to $\mathcal{A}_{\st S}$ and receives back from it $m$ pairs  where each pair is of the form $( g',  q')$. 
%
\item checks whether the following equations hold (for each pair): $ g= g' \hspace{2mm} \wedge  \hspace{2mm}  q= q'$. If the two equations do not hold, then it aborts (i.e., sends abort signal $\Lambda$ to the receiver) and proceeds to the next step.
%
\item outputs whatever $\mathcal{A}_{\st S}$ outputs.
%
 \end{enumerate}
 
 We first focus on the adversary’s output. In the real model, the only messages that the adversary receives are those messages it receives as the result of the ideal call to $f_{\st \ct}$. These messages have identical distribution to the distribution of the messages in the ideal model, as the \ct is secure. Now, we move on to the receiver’s output. We will show that the output distributions of the honest receiver in the ideal and real models are computationally indistinguishable. In the real model,  each element of pair $(g, p)$ is the output of a deterministic function on the output of $f_{\st \ct}$. We know the output of $f_{\st \ct}$ in the real and ideal models have an identical distribution, and so do the evaluations of deterministic functions (i.e., Merkle tree, $\mathtt{H}$, and $\mathtt{PRF}$) on them, as long as these three functions' correctness holds. Therefore, each pair $(g,q)$ in the real model has an identical distribution to pair $(g,  q)$ in the ideal model.  For the same reasons, the honest receiver in the real model aborts with the same probability as  $\mathsf{Sim}^{\st \zspa}_{\st S}$ does in the ideal model.  We conclude that the distributions of the joint outputs of the adversary and honest receiver in the real and ideal models are  (computationally) indistinguishable. 

\


\noindent\textbf{Case 2: Corrupt receiver.}   Let $\mathsf{Sim}^{\st \zspa}_{\st R}$ be the simulator that uses subroutine adversary $\mathcal{A}_{\st R}$. $\mathsf{Sim}^{\st \zspa}_{\st R}$ works as follows. 

\begin{enumerate}
%
\item simulates   \ct  and receives the output value $ k$ from $f_{\st \ct}$.
%
\item sends $ k$ to TTP and receives back $m$ pairs of the form $( g,  q)$ from TTP. 
%
\item sends $( k,  g,  q)$ to $\mathcal{A}_{\st R}$ and outputs whatever  $\mathcal{A}_{\st R}$ outputs. 
%
 \end{enumerate}
 
 
In the real model, the adversary receives two sets of messages, the first set includes the transcripts (including $ k$) it receives when it makes an ideal call to $f_{\st \ct}$ and the second set includes pair $(g, q)$. As we already discussed above (because we are in the  $f_{\st \ct}$-hybrid model) the distributions of the messages it receives from $f_{\st \ct}$ in the real and ideal models are identical. Moreover, the distribution of $f_{\st \ct}$'s output (i.e., $\bar k$ and $k$) in both models is identical; therefore, the honest sender's output distribution in both models is identical. As we already discussed,  the evaluations of deterministic functions (i.e., Merkle tree, $\mathtt{H}$, and $\mathtt{PRF}$) on $f_{\st \ct}$'s outputs have an identical distribution. Therefore, each pair $(g, q)$ in the real model has an identical distribution to the pair $(g, q)$ in the ideal model.  Hence, the distribution of the joint outputs of the adversary and honest receiver in the real and ideal models is indistinguishable.
%
  \hfill\(\Box\)\end{proof}

In addition to the security guarantee (i.e., computation's correctness against malicious sender or receiver) stated by Theorem \ref{theorem::ZSPA-comp-correctness}, \zspa offers  (a) privacy against the public, and (b)  non-refutability. Informally, privacy here means that given the state of the contract (i.e., $g$ and  $q$), an external party cannot learn any information about any of the pseudorandom values,  $z_{\scriptscriptstyle j}$; while non-refutability means that if a party sends ``approved" then in future cannot deny the knowledge of the values whose representation is stored in the contract. %Furthermore, indistinguishability means that every $z_{\scriptscriptstyle j}$ ($1\leq j \leq m$) should be indistinguishable from a truly random value. 




\begin{theorem}
If  $\mathtt{H}$ is preimage resistance, $\mathtt{PRF}$ is secure, the signature scheme used in the smart contract is secure (i.e., existentially unforgeable under chosen message attacks), and the blockchain is secure (i.e., offers persistence and liveness properties \cite{GarayKL15}) then \zspa offers (i) privacy against the public and (ii) non-refutability. 
\end{theorem}
 
 

\begin{proof}
First, we focus on privacy. Since key $k$, for $\mathtt{PRF}$, has been picked uniformly at random and $\mathtt{H}$ is preimage resistance, the probability that given $g$ the adversary can find $k$ is negligible in the security parameter, i.e., $\negl(\lambda)$. Furthermore, because $\mathtt{PRF}$ is secure (i.e., its outputs are indistinguishable from random values) and  $\mathtt{H}$ is preimage resistance, given the Merkle tree's root $g$, the probability that the adversary can find a leaf node, which is the output of $\mathtt{PRF}$, is $\negl(\lambda)$ too. 

Now we move on the non-refutability. Due to the persistency property of the blockchain, once a transaction/message goes more than $v$ blocks deep into the blockchain of one honest player (where $v$ is a security parameter), then it will be included in every honest player's blockchain with overwhelming probability, and it will be assigned a permanent
position in the blockchain (so it will not be modified with an overwhelming probability). Also, due to the liveness property,   all transactions originating from honest parties will eventually end up at a depth of more than $v$ blocks in an honest player's blockchain; therefore, the adversary cannot
perform a selective denial of service attack against honest account holders.  Moreover, due to the security of the digital signature (i.e., existentially unforgeable under chosen message attacks), one cannot deny sending the messages it sent to the blockchain and smart contract. 
%
\hfill\(\Box\)
\end{proof}



%
%
%\begin{theorem}
%If  $\mathtt{H}$ is preimage resistance, $\mathtt{PRF}$ is secure, the signature scheme used in the smart contract is secure (i.e., existentially unforgeable under chosen message attacks), and the blockchain is secure (i.e., offers liveness property and the hash power of the adversary is lower than those of honest miners) then \zspa offers (i) privacy against the public and (ii) non-refutability. 
%\end{theorem}
% 
% 
%
%\begin{proof}
%First, we focus on privacy. Since key $k$, for $\mathtt{PRF}$, has been picked uniformly at random and $\mathtt{H}$ is preimage resistance, the probability that given $g$ the adversary can find $k$ is negligible in the security parameter, i.e., $\negl(\lambda)$. Furthermore, because $\mathtt{PRF}$ is secure (i.e., its outputs are indistinguishable from random values) and  $\mathtt{H}$ is preimage resistance, given the Merkle tree's root $g$, the probability that the adversary can find a leaf node, which is the output of $\mathtt{PRF}$, is $\negl(\lambda)$ too. 
%  \hfill\(\Box\)\end{proof}




%Informally, there are four main security requirements that ZSPA must meet: (a) privacy, (b)  non-refutability, (c) indistinguishability, and (d) result correctness. Privacy here means given the state of the contract, an external party cannot learn any information about any of the (pseudorandom) values:  $z_{\scriptscriptstyle j}$; while non-refutability means that if a party sends ``approved" then in future cannot deny the knowledge of the values whose representation is stored in the contract. Furthermore, indistinguishability means that every $z_{\scriptscriptstyle j}$ ($1\leq j \leq m$) should be indistinguishable from a truly random value and result correctness means that a malicious result generator cannot convince other parties to accept an invalid final result, i.e., the root constructed on the invalid leaf node(s). In Figure \ref{fig:ZSPA}, we provide ZSPA that efficiently generates $b$ vectors where each vector elements is sum to zero. 






%\begin{figure}[ht]
%\setlength{\fboxsep}{0.7pt}
%\begin{center}
%\begin{boxedminipage}{12.3cm}

%
%\begin{figure}[ht]%[!htbp]
%\setlength{\fboxsep}{1pt}
%\begin{center}
%    \begin{tcolorbox}[enhanced,width=5.5in, 
%    drop fuzzy shadow southwest,
%    colframe=black,colback=white]
%
%
%\small{
%
%\begin{enumerate}
%\item[$\bullet$]  \textit{Parties.} $\{\resizeT {\textit A}_{\resizeS {\textit  1}},..., \resizeT {\textit A}_{\resizeS {\textit  m}}\}$
%\item[$\bullet$]  \textit{Input.}  $m$: the total number of participants and a deployed smart contract's address. 
%\item[$\bullet$] \textit{Output.}  $k$: a secret key that generates $b+1$ vectors $[z_{\scriptscriptstyle 0,1},...,z_{\scriptscriptstyle 1,m}],...,[z_{\scriptscriptstyle b,1},...,z_{\scriptscriptstyle b,m}]$ of pseudorandom values, $h$: hash of the key,  $g$: a Merkle tree's root, and a vector of signed messages. 
%
%
%%, such that the sum of each vector's elements equals zero: $\sum\limits^{\scriptscriptstyle m}_{\scriptscriptstyle j=1}z_{\scriptscriptstyle i,j}=0$. 
%
%
%\item All participants run a coin tossing protocol to agree on a key $k$  of $\mathtt{PRF}$.
%\item\label{ZSPA:val-gen} One of the parties:  
%\begin{enumerate}
%
%\item for every $i$ (where $0\leq i \leq b$), generates $m$ pseudorandom values as follows. 
%%
% $$\forall j, 1\leq j \leq m-1: z_{\scriptscriptstyle i,j}=\mathtt{PRF}(k,i||j), \hspace{5mm} z_{\scriptscriptstyle i,m}=-\sum\limits^{\scriptscriptstyle m-1}_{\scriptscriptstyle j=1}z_{\scriptscriptstyle i,j}$$
%%
%\item   constructs a Merkel tree on top of all pseudorandom values,  $\mathtt{MT.genTree}(z_{\scriptscriptstyle 0,1},...,z_{\scriptscriptstyle b,m})\rightarrow g$. 
%
%\item  sends the Merkel tree's root: $g$,   and the key's hash: $q=\mathtt {H}(k)$ to the smart contract. 
%
%\end{enumerate}
%
%\item\label{ZSPA:verify} The rest of parties (given $k_{\scriptscriptstyle 1}, k_{\scriptscriptstyle 2}$) check if, all $z_{\scriptscriptstyle i,j}$ values, the root $g$ and key's hash have been correctly generated (by redoing  step \ref{ZSPA:val-gen}). If the checks pass, each party sends a singed ``approved'' message to the  contract. Otherwise, it aborts. 
%
%
% \end{enumerate}
%}
% \end{tcolorbox}
%\end{center}
%\caption{Zero-sum Pseudorandom Values Agreement (ZSPA) Protocol} 
%\label{fig:ZSPA}
%\end{figure}
%




%%%%%%%%%%%%%%%%%%%%%%%%%%%%%%%%%%%%%%%
%\begin{figure}[ht]
%\setlength{\fboxsep}{0.7pt}
%\begin{center}
%\begin{boxedminipage}{12.3cm}
%
%\small{
%
%\begin{enumerate}
%\item[$\bullet$]  \textit{Parties:} $\{\resizeT {\textit A}_{\resizeS {\textit  1}},..., \resizeT {\textit A}_{\resizeS {\textit  m}}\}$
%\item[$\bullet$]  \textit{Public Parameters and Functions:} A pseudorandom function: $\mathtt{PRF}$, a deployed smart contract, and total number of participants: $m$. 
%\item[$\bullet$] \textit{Output}:  All parties agree on $b+1$ vectors $[z_{\scriptscriptstyle 0,1},...,z_{\scriptscriptstyle 1,m}],...,[z_{\scriptscriptstyle b,1},...,z_{\scriptscriptstyle b,m}]$, of pseudorandom values, such that the sum of each vector's elements equals zero: $\sum\limits^{\scriptscriptstyle m}_{\scriptscriptstyle j=1}z_{\scriptscriptstyle i,j}=0$
%
%
%\item All participants run a coin tossing protocol to agree on two keys $k_{\scriptscriptstyle 1}$ and $k_{\scriptscriptstyle 2}$ of $\mathtt{PRF}$.
%\item\label{ZSPA:val-gen} One of the parties:  
%\begin{enumerate}
%
%\item For every $i$, computes $m$ pseudorandom values: $\forall j, 1\leq j \leq m-1: z_{\scriptscriptstyle i,j}=\mathtt{PRF}(k_{\scriptscriptstyle 1},i||j)$ and sets $z_{\scriptscriptstyle i,m}=-\sum\limits^{\scriptscriptstyle m-1}_{\scriptscriptstyle j=1}z_{\scriptscriptstyle i,j}$, where $0\leq i \leq b$
%
%\item   commits to every $z_{\scriptscriptstyle i,j}$  as follows: $\mathtt{a}_{\scriptscriptstyle i,j}=\mathtt{Com}(z_{\scriptscriptstyle i,j}, q_{\scriptscriptstyle i,j})$, where the randomness of the commitment is computed as: $ q_{\scriptscriptstyle i,j}=\mathtt{PRF}(k_{\scriptscriptstyle 2},i||j)$ and  $1\leq j \leq m$.
%
%\item   constructs a Merkel tree on top of the committed values: $\mathtt{MT}(\mathtt{a}_{\scriptscriptstyle 0,1},...,\mathtt{a}_{\scriptscriptstyle b,m})\rightarrow g$ 
%
%\item  sends the Merkel tree's root: $g$,   and the keys' hashes: $\mathtt {H}(k_{\scriptscriptstyle 1})$ and $ \mathtt {H}(k_{\scriptscriptstyle 2})$, to the contract. 
%
%\end{enumerate}
%
%\item\label{ZSPA:verify} The rest of parties (given $k_{\scriptscriptstyle 1}, k_{\scriptscriptstyle 2}$) check if, all $z_{\scriptscriptstyle i,j}$ values, the root $g$ and keys' hashes have been correctly generated (by redoing  step \ref{ZSPA:val-gen}). If passed, each party sends a singed ``approved'' message to the  contract. Otherwise, it aborts. 
%
%
% \end{enumerate}
%}
%\end{boxedminipage}
%\end{center}
%\caption{Zero-sum Pseudorandom Values Agreement ($\mathtt{ZSPA}$) Protocol} 
%\label{fig:ZSPA}
%\end{figure}




%% !TEX root =main.tex




\begin{figure}[ht]%[!htbp]
\setlength{\fboxsep}{1pt}
\begin{center}
\scalebox{.85}{
    \begin{tcolorbox}[enhanced,width=5.5in, 
    drop fuzzy shadow southwest,
    colframe=black,colback=white]


{\small{

%\underline{$\mathtt{Audit}( \vv{{k}},  q, \bm\zeta, \bar d, g, \vv v)\rightarrow (L, \vv{{\mu}})$}
\begin{enumerate}
%\item[$\bullet$] Parties: clients: $\{  {   A}_{    {    1}},...,   {   A}_{    {    m}}\}$, the dealer and  an Arbiter.


\item[$\bullet$]    {Parties.} A set of clients $\{ A_{\st 1},...,  A_{\st m}\}$ and an external auditor, \aud. 

\item[$\bullet$]    {Input.}  $m$: the total number of participants (excluding the auditor), $\bm\zeta$: a random polynomial of degree $1$, $b$: the total number of vectors, and $adr$: a deployed smart contract's address. Let $b'=b-1$.





%\item[$\bullet$]   {Input.} $\vv{{k}}=[k_{\st 1},..., k_{\st m}]$,    $q$: a  hash value, $\bm\zeta$: a random polynoimal of degree $1$, $\bar d$: a polynoimal's degree,   $g$: a root of Merkle tree, and $\vv v$: binary vector of size $m$. 


\item[$\bullet$]  {Output of  each} $  A_{\st j}$.   $k$: a secret key that generates $b$ vectors $[z_{\scriptscriptstyle 0,1},...,z_{\scriptscriptstyle 0,m}],...,[z_{\scriptscriptstyle b',1},...,z_{\scriptscriptstyle b', m}]$ of pseudorandom values, $h$: hash of the key,  $g$: a Merkle tree's root, and a vector of signed messages. 



\item[$\bullet$]    {Output of \aud.} $L$: a list of misbehaving parties' indices, and  $\vv{{\mu}}$: a vector of random polynomials.
%
\item\label{ZSPA::ZSPA-invocation} {\textbf{\zspa invocation.}  $\zspa(\bot,..., \bot)\rightarrow \Big((k, g, q),..., (k, g,q )\Big)$}. 

All parties in $\{A_{\st 1},...,  A_{\st m}\}$ call the same instance of \zspa, which results in  $(k, g, q), ..., (k, g, q)$. 
%

\item\label{ZSPA-A::Auditor-computation}  {\textbf{Auditor computation.} $\mathtt{Audit}( \vv{{k}},  q, \bm\zeta, b, g)\rightarrow (L, \vv{{\mu}})$}. 

\aud\ takes the below steps. Note,  each $k_{\st j}\in \vv{{k}}$ is given by  $  A_{\st j}$. An honest party's input, $k_{\st j}$,  equals $k$, where $1\leq j \leq m$. 


\begin{enumerate}
%
\item runs the checks in the verification phase (i.e., Phase \ref{ZSPA:verify}) of \zspa for every $j$, i.e., $\mathtt{Verify}(k_{\st j}, g, q, m)\rightarrow (a_{\st j}, s)$.
\item appends $j$ to $L$, if any checks fails, i.e., if $a_{\st j}=0$. In this case, it skips the next two steps for the current $j$. 



%
%
%\item  Checks whether equation $\mathtt{H}(k_{\st j})=q$ holds  for every $j$, $1\leq j \leq m$.   
%%
%\begin{itemize}
%%
%\item[$\bullet$] if any $j$-th check fails,  it adds $j$ to $L$.
%%
%\item[$\bullet$]  if $L$ contains all $j\in[1,m]$, it returns $L$ and aborts. 
%%
%\end{itemize}
%%
%\item\label{zero-sum-arbiter-verification} Verifies the Merkle tree's root, $g$, by checking if the tree (corresponding to  $g$) has been correctly constructed on the correct leaf nodes. In particular, it takes the following steps. 
%
%\begin{enumerate}
%
%\item regenerates the tree's leaf nodes (similar to step \ref{ZSPA:val-gen} in Fig. \ref{fig:ZSPA}) as follows. Let $k$ be a key that passed the above check.  For every $i$ (where $0\leq i \leq \bar d$), it recomputes $m$ pseudorandom values: 
%%
%$$\forall j, 1\leq j \leq m-1: z_{\st i,j}=\mathtt{PRF}(k,i||j), \hspace{4mm} z_{\st i,m}=-\sum\limits^{\st m-1}_{\st j=1}z_{\st i,j}$$
%%
%\item   constructs a Merkel tree on top of all pseudorandom values generated in the previous step, i.e., $\mathtt{MT.genTree}(z_{\st 0,1},...,z_{\st \bar d,m})\rightarrow g'$. 
%%
%\item checks if $g=g'$. If the equation does not hold, then it adds to $L$ every index $j$ whose value in $\vv v$ is $1$, i.e., $\vv v[j]=1$; in this case, it returns $L$ and aborts.
%%
%\end{enumerate}
%

\item\label{ZSPA-A::gen-z} For every $i$ (where $0\leq i \leq b'$), it recomputes $m$ pseudorandom values: 
%
$\forall j, 1\leq j \leq m-1: z_{\st i,j}=\mathtt{PRF}(k,i||j), \hspace{4mm} z_{\st i,m}=-\sum\limits^{\st m-1}_{\st j=1}z_{\st i,j}$.
%
 \item generates polynomial $\bm\mu^{\st (j)}$ as follows: 
  %
   $\bm\mu^{\st (j)} = \bm\zeta\cdot \bm\xi^{\st (j)}-\bm\tau^{\st (j)}$, 
   %
    where $\bm\xi^{\st (j)}$ is a random polynomial of degree $b'-1$ and $\bm\tau^{\st (j)}=\sum\limits^{\st b'}_{\st i=0}z_{\st i,j}\cdot x^{\st i}$. By the end of this step, a vector $\vv{{\mu}}$ containing at most $m$ polynomials is generated. 
%
 \item returns   list $L$ and $\vv{{\mu}}$.
 
\end{enumerate}
 \end{enumerate}
}}
 \end{tcolorbox}
 }
\end{center}
\caption{\zspa with an external auditor (\zspaa)} 
\label{fig:arbiter}
\end{figure}



%%%%%%%%%%%%%%%%%%%%%%%%%%%%%%%%%%%%%%%%%%%%%%
%\begin{figure}[ht]%[!htbp]
%\setlength{\fboxsep}{1pt}
%\begin{center}
%    \begin{tcolorbox}[enhanced,width=5.5in, 
%    drop fuzzy shadow southwest,
%    colframe=black,colback=white]
%
%
%{\small{
%
%\underline{$\mathtt{Audit}( \vv{{k}},  q, \bm\zeta, \bar d, g, \vv v)\rightarrow (L, \vv{{\mu}})$}
%\begin{enumerate}
%%\item[$\bullet$] Parties: clients: $\{  {   A}_{    {    1}},...,   {   A}_{    {    m}}\}$, the dealer and  an Arbiter.
%\item[$\bullet$]   {Input.} $\vv{{k}}=[k_{\st 1},...,k_{\st m}]$,    $q$: a  hash value, $\bm\zeta$: a random polynoimal of degree $1$, $\bar d$: a polynoimal's degree,   $g$: a root of Merkle tree, and $\vv v$: binary vector of size $m$. 
%
%
%\item[$\bullet$]    {Output.} A list of rejected values' indices: $L$, a vector of random polynomials: $\vv{{\mu}}$.
%%
%\item  Checks whether equation $\mathtt{H}(k_{\st j})=q$ holds  for every $j$, $1\leq j \leq m$.   
%%
%\begin{itemize}
%%
%\item[$\bullet$] if any $j$-th check fails,  it adds $j$ to $L$.
%%
%\item[$\bullet$]  if $L$ contains all $j\in[1,m]$, it returns $L$ and aborts. 
%%
%\end{itemize}
%%
%\item\label{zero-sum-arbiter-verification} Verifies the Merkle tree's root, $g$, by checking if the tree (corresponding to  $g$) has been correctly constructed on the correct leaf nodes. In particular, it takes the following steps. 
%
%\begin{enumerate}
%
%\item regenerates the tree's leaf nodes (similar to step \ref{ZSPA:val-gen} in Fig. \ref{fig:ZSPA}) as follows. Let $k$ be a key that passed the above check.  For every $i$ (where $0\leq i \leq \bar d$), it recomputes $m$ pseudorandom values: 
%%
%$$\forall j, 1\leq j \leq m-1: z_{\st i,j}=\mathtt{PRF}(k,i||j), \hspace{4mm} z_{\st i,m}=-\sum\limits^{\st m-1}_{\st j=1}z_{\st i,j}$$
%%
%\item   constructs a Merkel tree on top of all pseudorandom values generated in the previous step, i.e., $\mathtt{MT.genTree}(z_{\st 0,1},...,z_{\st \bar d,m})\rightarrow g'$. 
%%
%\item checks if $g=g'$. If the equation does not hold, then it adds to $L$ every index $j$ whose value in $\vv v$ is $1$, i.e., $\vv v[j]=1$; in this case, it returns $L$ and aborts.
%%
%\end{enumerate}
%%
% \item Generates polynomial $\bm\mu^{\st (j)}$, for every $j$ such that $j\in[1,m]$ and $j \notin L$,  as follows:
%  %
%   $$\bm\mu^{\st (j)} = \bm\zeta\cdot \bm\xi^{\st (j)}-\bm\tau^{\st (j)}$$
%   %
%    where $\bm\xi^{\st (j)}$ is a random polynomial of degree $\bar d-1$ and $\bm\tau^{\st (j)}=\sum\limits^{\st \bar d}_{\st i=0}z_{\st i,j}\cdot x^{\st i}$. By the end of this step, a vector $\vv{{\mu}}$ containing at most $m$ polynomials is generated. 
%%
% \item Returns   list $L$ and $\vv{{\mu}}$.
% 
%
% \end{enumerate}
%}}
% \end{tcolorbox}
%\end{center}
%\caption{$\text{Audit}$ Algorithm} 
%\label{fig:arbiter}
%\end{figure}


% !TEX root =main.tex




\vs




\subsection{\zspa's Extension: \zspa with an External Auditor (\zspaa)}


In this section, we present an extension of \zspa, called \zspaa which lets a (trusted) third-party auditor, \aud, help identify misbehaving clients in the \zspa and generate a vector of random polynomials. Informally, \zspaa requires that misbehaving parties are always detected, except with a negligible probability. \aud of this protocol will be invoked by \withFai when \withFai's smart contract detects that a combination of the messages sent by the clients is not well-formed. Later, in \withFai's proof, we will show that even a \emph{semi-honest} \aud who observes all messages that clients send to \withFai's smart contracts, cannot learn anything about their set elements. We present \zspaa in Figure \ref{fig:arbiter}. 


\vs


% !TEX root =main.tex




\begin{figure}[ht]%[!htbp]
\setlength{\fboxsep}{1pt}
\begin{center}
\scalebox{.85}{
    \begin{tcolorbox}[enhanced,width=5.5in, 
    drop fuzzy shadow southwest,
    colframe=black,colback=white]


{\small{

%\underline{$\mathtt{Audit}( \vv{{k}},  q, \bm\zeta, \bar d, g, \vv v)\rightarrow (L, \vv{{\mu}})$}
\begin{enumerate}
%\item[$\bullet$] Parties: clients: $\{  {   A}_{    {    1}},...,   {   A}_{    {    m}}\}$, the dealer and  an Arbiter.


\item[$\bullet$]    {Parties.} A set of clients $\{ A_{\st 1},...,  A_{\st m}\}$ and an external auditor, \aud. 

\item[$\bullet$]    {Input.}  $m$: the total number of participants (excluding the auditor), $\bm\zeta$: a random polynomial of degree $1$, $b$: the total number of vectors, and $adr$: a deployed smart contract's address. Let $b'=b-1$.





%\item[$\bullet$]   {Input.} $\vv{{k}}=[k_{\st 1},..., k_{\st m}]$,    $q$: a  hash value, $\bm\zeta$: a random polynoimal of degree $1$, $\bar d$: a polynoimal's degree,   $g$: a root of Merkle tree, and $\vv v$: binary vector of size $m$. 


\item[$\bullet$]  {Output of  each} $  A_{\st j}$.   $k$: a secret key that generates $b$ vectors $[z_{\scriptscriptstyle 0,1},...,z_{\scriptscriptstyle 0,m}],...,[z_{\scriptscriptstyle b',1},...,z_{\scriptscriptstyle b', m}]$ of pseudorandom values, $h$: hash of the key,  $g$: a Merkle tree's root, and a vector of signed messages. 



\item[$\bullet$]    {Output of \aud.} $L$: a list of misbehaving parties' indices, and  $\vv{{\mu}}$: a vector of random polynomials.
%
\item\label{ZSPA::ZSPA-invocation} {\textbf{\zspa invocation.}  $\zspa(\bot,..., \bot)\rightarrow \Big((k, g, q),..., (k, g,q )\Big)$}. 

All parties in $\{A_{\st 1},...,  A_{\st m}\}$ call the same instance of \zspa, which results in  $(k, g, q), ..., (k, g, q)$. 
%

\item\label{ZSPA-A::Auditor-computation}  {\textbf{Auditor computation.} $\mathtt{Audit}( \vv{{k}},  q, \bm\zeta, b, g)\rightarrow (L, \vv{{\mu}})$}. 

\aud\ takes the below steps. Note,  each $k_{\st j}\in \vv{{k}}$ is given by  $  A_{\st j}$. An honest party's input, $k_{\st j}$,  equals $k$, where $1\leq j \leq m$. 


\begin{enumerate}
%
\item runs the checks in the verification phase (i.e., Phase \ref{ZSPA:verify}) of \zspa for every $j$, i.e., $\mathtt{Verify}(k_{\st j}, g, q, m)\rightarrow (a_{\st j}, s)$.
\item appends $j$ to $L$, if any checks fails, i.e., if $a_{\st j}=0$. In this case, it skips the next two steps for the current $j$. 



%
%
%\item  Checks whether equation $\mathtt{H}(k_{\st j})=q$ holds  for every $j$, $1\leq j \leq m$.   
%%
%\begin{itemize}
%%
%\item[$\bullet$] if any $j$-th check fails,  it adds $j$ to $L$.
%%
%\item[$\bullet$]  if $L$ contains all $j\in[1,m]$, it returns $L$ and aborts. 
%%
%\end{itemize}
%%
%\item\label{zero-sum-arbiter-verification} Verifies the Merkle tree's root, $g$, by checking if the tree (corresponding to  $g$) has been correctly constructed on the correct leaf nodes. In particular, it takes the following steps. 
%
%\begin{enumerate}
%
%\item regenerates the tree's leaf nodes (similar to step \ref{ZSPA:val-gen} in Fig. \ref{fig:ZSPA}) as follows. Let $k$ be a key that passed the above check.  For every $i$ (where $0\leq i \leq \bar d$), it recomputes $m$ pseudorandom values: 
%%
%$$\forall j, 1\leq j \leq m-1: z_{\st i,j}=\mathtt{PRF}(k,i||j), \hspace{4mm} z_{\st i,m}=-\sum\limits^{\st m-1}_{\st j=1}z_{\st i,j}$$
%%
%\item   constructs a Merkel tree on top of all pseudorandom values generated in the previous step, i.e., $\mathtt{MT.genTree}(z_{\st 0,1},...,z_{\st \bar d,m})\rightarrow g'$. 
%%
%\item checks if $g=g'$. If the equation does not hold, then it adds to $L$ every index $j$ whose value in $\vv v$ is $1$, i.e., $\vv v[j]=1$; in this case, it returns $L$ and aborts.
%%
%\end{enumerate}
%

\item\label{ZSPA-A::gen-z} For every $i$ (where $0\leq i \leq b'$), it recomputes $m$ pseudorandom values: 
%
$\forall j, 1\leq j \leq m-1: z_{\st i,j}=\mathtt{PRF}(k,i||j), \hspace{4mm} z_{\st i,m}=-\sum\limits^{\st m-1}_{\st j=1}z_{\st i,j}$.
%
 \item generates polynomial $\bm\mu^{\st (j)}$ as follows: 
  %
   $\bm\mu^{\st (j)} = \bm\zeta\cdot \bm\xi^{\st (j)}-\bm\tau^{\st (j)}$, 
   %
    where $\bm\xi^{\st (j)}$ is a random polynomial of degree $b'-1$ and $\bm\tau^{\st (j)}=\sum\limits^{\st b'}_{\st i=0}z_{\st i,j}\cdot x^{\st i}$. By the end of this step, a vector $\vv{{\mu}}$ containing at most $m$ polynomials is generated. 
%
 \item returns   list $L$ and $\vv{{\mu}}$.
 
\end{enumerate}
 \end{enumerate}
}}
 \end{tcolorbox}
 }
\end{center}
\caption{\zspa with an external auditor (\zspaa)} 
\label{fig:arbiter}
\end{figure}



%%%%%%%%%%%%%%%%%%%%%%%%%%%%%%%%%%%%%%%%%%%%%%
%\begin{figure}[ht]%[!htbp]
%\setlength{\fboxsep}{1pt}
%\begin{center}
%    \begin{tcolorbox}[enhanced,width=5.5in, 
%    drop fuzzy shadow southwest,
%    colframe=black,colback=white]
%
%
%{\small{
%
%\underline{$\mathtt{Audit}( \vv{{k}},  q, \bm\zeta, \bar d, g, \vv v)\rightarrow (L, \vv{{\mu}})$}
%\begin{enumerate}
%%\item[$\bullet$] Parties: clients: $\{  {   A}_{    {    1}},...,   {   A}_{    {    m}}\}$, the dealer and  an Arbiter.
%\item[$\bullet$]   {Input.} $\vv{{k}}=[k_{\st 1},...,k_{\st m}]$,    $q$: a  hash value, $\bm\zeta$: a random polynoimal of degree $1$, $\bar d$: a polynoimal's degree,   $g$: a root of Merkle tree, and $\vv v$: binary vector of size $m$. 
%
%
%\item[$\bullet$]    {Output.} A list of rejected values' indices: $L$, a vector of random polynomials: $\vv{{\mu}}$.
%%
%\item  Checks whether equation $\mathtt{H}(k_{\st j})=q$ holds  for every $j$, $1\leq j \leq m$.   
%%
%\begin{itemize}
%%
%\item[$\bullet$] if any $j$-th check fails,  it adds $j$ to $L$.
%%
%\item[$\bullet$]  if $L$ contains all $j\in[1,m]$, it returns $L$ and aborts. 
%%
%\end{itemize}
%%
%\item\label{zero-sum-arbiter-verification} Verifies the Merkle tree's root, $g$, by checking if the tree (corresponding to  $g$) has been correctly constructed on the correct leaf nodes. In particular, it takes the following steps. 
%
%\begin{enumerate}
%
%\item regenerates the tree's leaf nodes (similar to step \ref{ZSPA:val-gen} in Fig. \ref{fig:ZSPA}) as follows. Let $k$ be a key that passed the above check.  For every $i$ (where $0\leq i \leq \bar d$), it recomputes $m$ pseudorandom values: 
%%
%$$\forall j, 1\leq j \leq m-1: z_{\st i,j}=\mathtt{PRF}(k,i||j), \hspace{4mm} z_{\st i,m}=-\sum\limits^{\st m-1}_{\st j=1}z_{\st i,j}$$
%%
%\item   constructs a Merkel tree on top of all pseudorandom values generated in the previous step, i.e., $\mathtt{MT.genTree}(z_{\st 0,1},...,z_{\st \bar d,m})\rightarrow g'$. 
%%
%\item checks if $g=g'$. If the equation does not hold, then it adds to $L$ every index $j$ whose value in $\vv v$ is $1$, i.e., $\vv v[j]=1$; in this case, it returns $L$ and aborts.
%%
%\end{enumerate}
%%
% \item Generates polynomial $\bm\mu^{\st (j)}$, for every $j$ such that $j\in[1,m]$ and $j \notin L$,  as follows:
%  %
%   $$\bm\mu^{\st (j)} = \bm\zeta\cdot \bm\xi^{\st (j)}-\bm\tau^{\st (j)}$$
%   %
%    where $\bm\xi^{\st (j)}$ is a random polynomial of degree $\bar d-1$ and $\bm\tau^{\st (j)}=\sum\limits^{\st \bar d}_{\st i=0}z_{\st i,j}\cdot x^{\st i}$. By the end of this step, a vector $\vv{{\mu}}$ containing at most $m$ polynomials is generated. 
%%
% \item Returns   list $L$ and $\vv{{\mu}}$.
% 
%
% \end{enumerate}
%}}
% \end{tcolorbox}
%\end{center}
%\caption{$\text{Audit}$ Algorithm} 
%\label{fig:arbiter}
%\end{figure}





\begin{theorem}\label{theorem::ZSPA-A}
If \zspa is secure, $\mathtt{H}$ is second-preimage resistant, and the correctness of $\mathtt{PRF}$, $\mathtt{H}$, and Merkle tree holds,  then \zspaa securely computes $f^{\st \zspaa}$ in the presence of $m-1 $ malicious adversaries.% or (ii) a semi-honest auditor. 
\end{theorem}

\svs

We refer readers to Appendix \ref{sec::proof-of-zspaa} for the proof of Theorem \ref{theorem::ZSPA-A}. 

%As we stated previously, the ZSPA-A protocol will be invoked as a subroutine in the fair PSI protocol. As part of proving Theorem \ref{theorem::ZSPA-A}, we would like to show that the semi-honest auditor's view can be simulated (so it cannot learn the parties' set elements), even if it has access to those transcripts of the fair PSI protocol sent to the smart contract; because such an approach offers a stronger security guarantee than proving the ZSPA-A protocol in isolation.  Therefore, we will present the proof of Theorem \ref{theorem::ZSPA-A} after we present the fair PSI protocol. 





%\begin{theorem}\label{theorem::ZSPA-comp-correctness}
%If the coin-tossing protocol is secure against a malicious adversary, then the ZSPA protocol,  in Figure \ref{fig:ZSPA}, securely computes $f^{\st \text {ZSPA}}$ in the presence of a malicious adversary. 
%\end{theorem}


%\begin{figure}%[ht]
%\setlength{\fboxsep}{0.7pt}
%\begin{center}
%\begin{boxedminipage}{12.3cm}
%
%\small{
%
%\begin{enumerate}
%\item[$\bullet$] Parties: clients: $\{\resizeT {\textit A}_{\resizeS {\textit  1}},..., \resizeT {\textit A}_{\resizeS {\textit  m}}\}$, the dealer and  an Arbiter.
%\item[$\bullet$] Input: Empty malicious clients list: $L$ and a deployed smart contract's address. 
%\item[$\bullet$] Output: Misbehaving clients list: $L$
%\item Every client sends to the Arbiter  two keys: $k_{\scriptscriptstyle 1}, k_{\scriptscriptstyle 2}$, used to generate the zero-sum values and their commitments. 
%%
%\item  The Arbiter checks if the clients  provided correct keys, by ensuring that the keys' hashes matches the ones stored in the contract. It appends the IDs of those  provided inconsistent keys to $L$. If all clients provided inconsistent keys it aborts. Otherwise, it proceed to the next step where it uses correct keys: $k_{\scriptscriptstyle 1}, k_{\scriptscriptstyle 2}$. 
%%
%\item\label{zero-sum-arbiter-verification} The Arbiter (given correct keys) regenerate the  zero-sum values $z_{\scriptscriptstyle i, j}$ and verify the correctness of their commitments and the Merkel tree root contracted on top of the commitments, i.e. takes the same step as step \ref{ZSPA:verify} in Fig \ref{fig:ZSPA}.   It aborts if any of the   checks is rejected, and appends to $L$ the IDs of the clients which sent the ``approved'' message to the contract. 
%%
% \item The Arbiter for each client $\resizeT {\textit C}$, who provided correct keys,  generates polynomial $\bm\mu^{\resizeS {\textit {(C)}}}$, for each bin, as follows:
%  %
%   $$\bm\mu^{\resizeS {\textit {(C)}}} = \bm\zeta\cdot \bm\xi^{\resizeS {\textit {(C)}}}-\bm\tau^{\resizeS {\textit {(C)}}}$$
%   %
%    where $\bm\xi^{\resizeS {\textit {(C)}}}$ is a random polynomial of degree $3d+1$ and $\bm\tau^{\resizeS {\textit {(C)}}}=\sum\limits^{\st 3d+2}_{\st i=0}z_{\st i,c}\cdot x^{\st i}$. By the end of this step, a vector $\vv{\bm{\mu}}$ containing polynomial $\bm\mu^{\resizeS {\textit {(C)}}}$ for every bin of client $\resizeT {\textit C}$ that is not in list $L$. 
%    %
%     \item returns   list $L$ and $\vv{\bm{\mu}}$.
 
%
%
% \item The dealer, for each client $\resizeT {\textit C}\in \{\resizeT {\textit A}_{\resizeS {\textit  1}},..., \resizeT {\textit A}_{\resizeS {\textit  m}}\}$,  sends to the Arbiter a blind polynomial of the form: $\bm\zeta\cdot \bm\eta^{\resizeS {\textit {(D,C)}}}-(\bm\gamma^{\resizeS {\textit {(D,C)}}}+\bm\delta^{\resizeS {\textit {(D,C)}}})$, where $\bm\eta^{\resizeS {\textit {(D,C)}}}$ is a fresh random degree $3d+1$ polynomial. The blind polynomial will allow the arbiter to obliviously verify the correctness of the message each client sent to the  contract. 
% 
% \item The Arbiter for each client $\resizeT {\textit C}$ who provided correct keys: 
% 
% \begin{enumerate}
% \item adds together the blind polynomial above and the blind polynomial $\bm\nu^{\resizeS {\textit {(C)}}}$ the client sent to the contract (in step \ref{blindPoly-C-sends-to-contract} in the PSI protocol). Then, it removes the client's zero-sum pseudorandom values from the result. In particular, it computes:    
%\begin{equation*}
%\begin{split}
% \bm\iota^{\resizeS {\textit {(C)}}}&=\bm\zeta\cdot \bm\eta^{\resizeS {\textit {(D,C)}}}-(\bm\gamma^{\resizeS {\textit {(D,C)}}}+\bm\delta^{\resizeS {\textit {(D,C)}}})+\bm\nu^{\resizeS {\textit {(C)}}}-\sum\limits^{\scriptscriptstyle 3d+1}_{\scriptscriptstyle i=0}z_{\scriptscriptstyle i,c}\cdot x^{\scriptscriptstyle i} \\ &=\bm\zeta\cdot(\bm\eta^{\resizeS {\textit {(D,C)}}} + \bm\omega^{\resizeS {\textit {(D,C)}}}\cdot \bm\omega^{\resizeS {\textit {(C,D)}}}\cdot \bm\pi^{\resizeS {\textit {(C)}}}+\bm\rho^{\resizeS {\textit {(D,C)}}}\cdot \bm\rho^{\resizeS {\textit {(C,D)}}}\cdot \bm\pi^{\resizeS {\textit {(D)}}})
% \end{split}
%\end{equation*}
%  \item checks if $\bm\zeta$ can divide $\bm\iota^{\resizeS {\textit {(C)}}}$. If can not, it appends the client's ID to $L$.
%  \end{enumerate}
  %$deg(\eta^{\resizeS {\textit {D,C}}})=3d+1$
% \end{enumerate}
%}
%\end{boxedminipage}
%\end{center}
%\caption{$\mathtt{Arbiter}$ Protocol} 
%\label{fig:arbiter}
%\end{figure}








