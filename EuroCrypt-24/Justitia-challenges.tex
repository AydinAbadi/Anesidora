% !TEX root =main.tex


%%%%%%%%%%%%%%%%%
 %%%%%%%%%%%%%
\subsection{Main Challenges that \withFai Overcomes}\label{sec::Justitia-challenges}

To design an efficient scheme that realises \p,  we had to address several key challenges. Below, we outline these challenges.



 \subsubsection{Keeping Overall Complexities Low.}
 
 In general, in multi-party PSIs, each client must exchange messages with the other clients and potentially engage in secure computations with them, as seen in \cite{DBLP:conf/scn/InbarOP18,DBLP:conf/ccs/KolesnikovMPRT17}. This can lead to communication and computational costs that grow quadratically with the number of clients.
 
 To tackle this challenge, we employed two strategies: (a) allowing one of the clients to act as a dealer, interacting with the remaining clients\footnote{This approach has similarity with the non-secure PSIs in \cite{GhoshN19}.}, and (b) implementing a smart contract that serves as a bulletin board for receiving most messages and conducting lightweight computations on the clients' messages. The combination of these approaches ensures that the overall communication and computation remain linear in relation to the number of clients (and the cardinality of sets).



 
 
 \subsubsection{Randomising Input Polynomials.}  In multi-party PSIs that utilise the polynomial representation, it is crucial for a client's input polynomial to undergo randomisation by another client \cite{AbadiMZ21}. To achieve this securely and efficiently, we required the dealer and each client to jointly participate in an instance of  \vopr, a protocol we developed in Section \ref{sec::subroutines}. 
 

 
 \subsubsection{Preserving the Privacy of Outgoing Messages.} 
 
 
 While the utilisation of public smart contracts, such as Ethereum, helps maintain overall complexity low, it introduces another challenge. Specifically, if clients fail to safeguard the privacy of the messages they transmit to the smart contracts, then both other clients (e.g., the dealer) and individuals who are not participants in PSI (i.e., the public) can gain access to the clients' set elements and/or the intersection.
 
 To ensure the efficient protection of each client's messages sent to the contracts from the dealer, we necessitate that the clients, excluding the dealer, participate in \zspaa. This protocol allows each client to create a pseudorandom polynomial, which they can employ to obscure their messages. To safeguard the privacy of the intersection from the public, we require that  all clients to run a coin-tossing protocol to reach a consensus on a blinding polynomial.  This blinding polynomial will be used to obscure the final result that encodes the intersection on the smart contract.  
 
 

 \subsubsection{Ensuring the Correctness of Subroutine Protocols' Outputs.} 
 
 
 Typically, any MPC protocol designed to withstand active adversaries incorporates a verification mechanism to detect any tampering with message integrity during the protocol's execution. This applies to the subroutine protocols we utilise, namely \vopr and \zspaa. 
 
 
  However, relying solely on this type of check is not always adequate. There are situations where the output of one MPC serves as input to another MPC, and it becomes essential to guarantee that the unaltered output of the first MPC is securely passed to the second one. This holds true for our PSI's subroutines as well. 
  
  To address this challenge, we employ unforgeable polynomials. Specifically, the output of \vopr is an unforgeable polynomial that encodes the actual output. If the adversary tampers with the \vopr's output and later uses it, then a verifier can detect this tampering. 
 
 We obtain the same integrity guarantee for the output of \zspaa without any additional effort. This is because (i) \vopr is called before \zspaa, and (ii) if clients use the unaltered outputs of \zspaa, then the final result (i.e., the sum of all clients' messages) will not contain any output of \zspaa, as they will cancel each other out. Thus, by verifying the correctness of the final result, one can ensure the correctness of the outputs of \vopr and \zspaa, in a single step. 
 %%%%%%%%%%%%%%





 
