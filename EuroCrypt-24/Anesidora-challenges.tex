% !TEX root =main.tex


%%%%%%%%%%%%%%%%%
 %%%%%%%%%%%%%



\section{Main Challenges that \withRew Overcomes}\label{sec::Annesidora-challenges}


\subsection{Rewarding Clients Proportionate to the Intersection Size.}
In PSIs, the main private information about the clients which is revealed to a result recipient is the private set elements that the clients have in common. Thus, honest clients must receive a reward proportionate to the intersection cardinality, from a buyer. To receive the reward, the clients need to reach a consensus on the intersection cardinality. The naive way to do that is to let every client find the intersection and declare it to the smart contract. Under the assumption that the majority of clients are honest, then the smart contract can reward the honest result recipient (from the buyer's deposit). But, the honest majority assumption is strong in the context of multi-party PSI. 

Moreover, this approach requires all clients to extract the intersection, which would increase the overall costs.  Some clients may not even be interested in or available to do so. This task could also be conducted by a single entity, such as the dealer; but this approach would introduce a single point of failure and all clients have to depend on this entity.  
%
To address these challenges, we allow any two clients to become extractors.  Each of them finds and sends to the contract the (encrypted) elements in the intersection. The contract compensates them if it determines their honesty. This approach allows us to avoid (i) relying on the assumption of an honest majority, (ii) requiring all clients to find the intersection, and (iii) depending on a single trusted or semi-honest party to perform the task. 






\subsection{Dealing with Extractors' Collusion.}
%


Introducing two extractors brings about another challenge: the possibility of collusion between them (as well as with the buyer) to present a consistent but erroneous result. This kind of behaviour cannot be discerned by a verifier unless the verifier consistently undertakes the delegated task themselves, which would defeat the purpose of delegation. To efficiently confront this challenge, we implement counter-collusion smart contracts (as detailed in Section \ref{Counter-Collusion-Smart-Contracts}) that creates distrust among the extractors and provide incentives for them to behave honestly. 













 
