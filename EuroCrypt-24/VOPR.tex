% !TEX root =main.tex



%%%%%%%%
\begin{figure}[!ht]%[!htbp]
\setlength{\fboxsep}{1pt}
\begin{center}
    \begin{tcolorbox}[enhanced,width=4.75in,height=111mm,right=.5mm,
    drop fuzzy shadow southwest,
    colframe=black,colback=white]
%%%%%%%%
\vspace{-2.2mm}
\begin{enumerate}[leftmargin=-.01mm]
\small{
\item[$\bullet$] \textit{Input.}
\begin{enumerate}
\item[$\bullet$]  \textit{Public Parameters}: upper bound on input polynomials' degree: $e$ and $e'$. %, where $e\geq e'$.
%\item[$\bullet$]  \textit{Sender}: picks an upper bound on input polynomials's degree:  $d$ and $d'$. It sends them to the receiver.

\vspace{-1mm}
\item[$\bullet$]  \textit{Sender Input}:  random polynomials: $\bm\psi=\sum\limits^{\st e}_{\st i=0}g_{\st i}\cdot x^{\st i}$ and  $\bm\alpha=\sum\limits^{\st e+e'}_{\st j=0}a_{\st j}\cdot x^{\st j}$, where $g_{\st i}\stackrel{\st \$}\leftarrow \mathbb{F}_p$.  Each $a_{\st j}$ has the  form: $a_{\st j}=\sum\limits^{\substack{\st k=e'\\ \st t=e}}_{\st t,k=0} a_{\st t,k}$,  such that $t+k=j$ and $a_{\st t,k}\stackrel{\st \$}\leftarrow \mathbb{F}_p$.

\vspace{-1mm}
\item[$\bullet$] \textit{Receiver Input}:  polynomial $\bm\beta=\bm\beta_{\st 1}\cdot \bm\beta_{\st 2}=\sum\limits^{\st e'}_{\st i=0}b_{\st i}\cdot x^{\st i}$, where $\bm\beta_{\st 1}$ is a random polynomial of degree $1$, $\bm\beta_{\st 2}$ is an arbitrary polynomial of degree $e'-1$, and $b_{\st i}\neq 0$.


\end{enumerate}
\item[$\bullet$] \textit{Output.} The receiver gets $\bm\theta=\bm\psi\cdot \bm\beta+\bm\alpha$.
\item \textbf{Computation:}

\begin{enumerate} 

\item Sender and receiver together for every $j$, $0\leq j\leq e'$,  invoke $e+1$ instances of $\ole^{\st +}$. In particular, $\forall j, 0\leq j\leq e'$: sender sends $g_{\st i}$ and $a_{\st i,j}$ while the receiver sends $b_{\st j}$ to $\ole^{\st +}$ that returns: $c_{\st i,j}=g_{\st i}\cdot b_{\st j}+ a_{\st i,j}$ to the receiver ($\forall i, 0\leq i\leq e$). 


 \item The receiver sums component-wise values $c_{\st i,j}$  that results in polynomial:
 %
 \vspace{-3mm}
 %
  $$\bm\theta=\bm\psi\cdot \bm\beta+\bm\alpha=\sum\limits^{\substack{\st i=e\\ \st j=e'}}_{\st i,j=0}c_{\st i, j}\cdot x^{\st i+j}$$ 
 %
  \vspace{-3mm}
  %
 




% \item The receiver sums component-wise values $c_{\st i,j}$  that results polynomial $\bm\theta=\bm\psi\cdot \bm\beta+\bm\alpha=\sum\limits^{\st e+e'}_{\st j=0}c_{\st j}\cdot x^{\st j}$, where  each $c_{\st j}$ has   form: $c_{\st j}=\sum\limits^{\substack{\st k=e'\\ \st t=e}}_{\st  t,k=0} c_{\st t,k}$, such that $ t+k=j$.
%\item Sender: $\forall j, 1\leq j\leq 2d+1$, computes $a_{\st j}=a(x_{\st j})$ and $r_{\st j}=r(x_{\st j})$. Then, it  inserts $(a_{\st j}, r_{\st j})$ into  $\mathcal{F}_{\st OLE^{\st +}}$
%\item\label{computing-receiver} Receiver:  $\forall j, 1 \leq j\leq 2d+1$, computes $b_{\st j}=b(x_{\st j})$. Then, it  inserts $b_{\st j}$ into  $\mathcal{F}_{\st OLE^{\st +}}$ and receives $s_{\st j}=a_{\st j}+b_{\st j}\cdot r_{\st j}$. It interpolates a polynomial $s(x)$ using pairs $s_{\st j},x_{\st j}$. 
\end{enumerate}
\vspace{-1.6mm}
\item \label{Verification} \textbf{Verification:}
\begin{enumerate}%[leftmargin=1mm]

\item \label{picking-random-x}Sender: picks a random (non-zero) value  $z$ and sends it to the receiver. 

\item\label{receiver-OLE-invocation} Receiver: sends $\theta_{\st z}=\bm\theta(z)$ and $\beta_{\st z}=\bm\beta(z)$ to the sender.

\item\label{receiver-OLE-invocation} Sender:  computes $\psi_{\st z}=\bm\psi(z)$ and $\alpha_{\st z}=\bm\alpha(z)$ and checks   if equation  $\theta_{\st z}=\psi{\st z}\cdot \beta_{\st z}+\alpha_{\st z}$ holds. If the equation holds, it concludes that the computation was performed correctly. Otherwise, it aborts. 
%
\end{enumerate}
}
 \end{enumerate}
 \end{tcolorbox}
\end{center}
\vspace{-5.5mm}
\caption{Verifiable Oblivious Polynomial Randomisation (\vopr)} 
\label{fig:VOPR}
\vspace{-5.5mm}
\end{figure}
 %%%%%%%
