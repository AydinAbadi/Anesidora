% !TEX root =main.tex



\subsection{Concrete Parameters}\label{sec::conc-parameters}

\subsubsection{Hash Table Parameters.}


As detailed in Appendix \ref{Preliminary-Hash-Table}, a hash table is characterised by several key parameters:

\begin{itemize}

\item[$\bullet$] $h$: the number of bins.

\item[$\bullet$] $d$: the maximum size (or capacity) of each bin.

\item[$\bullet$] $c$: the maximum number of elements that are mapped to the hash table.

\item[$\bullet$] $pr$: the probability that the number of elements mapped to a bin does not exceed a predefined capacity.
 
 \end{itemize}
 
 It is important to note that in the context of PSI, the parameter $c$ represents the maximum of the sizes of all sets involved. 
 
 The literature, as evidenced in sources such as \cite{Feather2020-full,DBLP:conf/ccs/KolesnikovMPRT17,DBLP:conf/uss/Pinkas0SZ15}) has extensively examined the specific parameters of a hash table, even within the context of PSI. For instance, as illustrated in Section 6 and Appendix J.1 of \cite{Feather2020-full}, when $c$ falls within the range of  $[2^{\st 10},\ 2^{\st 20}]$ and $pr$ is set to $2^{\st -40}$, then $d$ is $100$. Furthermore, we have $h\approx\frac{4c}{d}$. To provide a concrete value, when $c=2^{\st 20}$ and $d=100$, we would have $h=41943$. 




\subsubsection{Field Size.} 
%
In this paper, all arithmetic operations are defined over a finite field $\mathbb{F}_{\st p}$, where $\log_{\st 2}(p)=\lambda$ represents the security parameter. The outputs of $\mathtt {PRF}(.)$ and  $\mathtt {PRP}(.) $ are also of size $\lambda$. Concrete value of $\lambda$ can be set based on the maximum bit size of set elements. For instance, one can choose $\lambda = 60$, when the maximum bit size of set elements is slightly less than $60$ or  $\lambda = 128$ if it is slightly less than $128$. 




