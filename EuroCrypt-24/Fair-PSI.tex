% !TEX root =main.tex

\vspace{-3.6mm}

\section{Definition of \p}\label{sec::F-PSI-model}%\label{Fair-PSI-Protocol}


\vspace{-1.4mm}


Now, we present the concept of multi-party PSI with Fair Compensation  (\p) which ensures that  either all clients receive the result or honest parties are financially compensated in the event of an unfair protocol abortion, wherein only dishonest parties gain access to the result.  
%
In a  $\mathcal{PSI}^{\st \mathcal{FC}}$, three types of parties are involved; namely, (1) a set of clients $\{A_{\st 1},...,A_{\st m}\}$ potentially \emph{malicious} (i.e., active adversaries), where  all but one may collude with each other, (2) a non-colluding dealer, $D$, potentially semi-honest (i.e., a passive adversary), and (3) an auditor, \aud, potentially semi-honest, where all parties except \aud have input set. For simplicity, we assume that one can determine whether a given address belongs to \aud. The fundamental functionality computed by any  multi-party PSI can be defined as  $f^{\st\text{PSI}} (S_{\st 1},..., S_{\st m})\rightarrow(\underbrace{S_{\st\cap},..., S_{\st\cap}}_{\st m})$, where $S_{\st\cap}= S_{\st 1} \cap S_{\st 2}, ...,\cap\  S_{\st m}$.  

At a high level, the standard simulation-based paradigm ensures only privacy and output correctness \cite{DBLP:books/cu/Goldreich2004}. To enhance the standard simulation-based model and accommodate additional security requirements (such as fairness, compensation, or transactions' correctness) researchers often follow these steps: (i) define a set of predicates and (ii) define a wrapper that encapsulates and  parameterises the original paradigm with the defined predicates. This approach can be observed in \cite{KiayiasZZ16,BadertscherMTZ17,BadertscherGKRZ19}. Consequently, to formally define a \p, we parameterise $f^{\st\text{PSI}}$ with four predicates,  $Q:=(\qinit, \qdel, \qUnFAbt, \qFAbt)$, which ensure that certain financial conditions are met. We borrow three of these predicates (i.e., $\qinit, \qdel, \qUnFAbt$) from \cite{KiayiasZZ16}. Nevertheless, we will (i) introduce an additional predicate  \qFAbt and (ii) provide more formal accurate definitions of these predicates. Below, we provide a brief description of these predicates.% (in Appendix \ref{sec::Formal-Definitions-of-F-PSI-Predicates}). 






   %  that are invoked after the functionality $f^{\st \text{PSI}}$ is executed. 
%   We borrow  $\qinit, \qdel$ and $\qUnFAbt$ from \cite{KiayiasZZ16}; nevertheless, we will (i) introduce an additional predicate  \qFAbt and (ii) provide more formal accurate definitions of these predicates. 

\vspace{-1.7mm}
\begin{itemize}
    \item[$\bullet$] \underline{\qinit:  Initiation predicate}. It defines the conditions under which a protocol realising \p should begin execution, i.e., when all set owners have sufficient deposit. 
 
  \item[$\bullet$] \underline{\qdel: Delivery predicate}. It defines the circumstances in which parties receive their output, specifically when honest parties receive their deposit back.
 
  \item[$\bullet$] \underline{\qUnFAbt: UnFair-Abort predicate}. It specifies the conditions under which the simulator can force parties to abort if the adversary learns the output, i.e., when an honest party receives its deposit back along with a predefined amount of compensation.
 
 
 
 \item[$\bullet$] \underline{\qFAbt: Fair-Abort predicate}. It specifies the conditions under which the simulator can compel parties to abort if the adversary receives no output, i.e., when honest parties receive their deposits back. 
 
   \end{itemize}
   \vspace{-2.8mm}
%
%We observed that the latter predicate should have been defined in the generic framework in \cite{KiayiasZZ16} too; as the framework should have also captured the cases where an adversary may abort without learning any output after the onset of the protocol.  

By requiring any protocol that realises \p, to implement a wrapped version of $f^{\st\text{PSI}}$ that includes $Q$, we will ensure that an honest set owner only aborts in an unfair manner if \qUnFAbt returns  $1$, it only aborts in a fair manner if \qFAbt returns  $1$, and outputs a valid value if \qdel returns $1$. We refer readers to Appendix \ref{sec::Formal-Definitions-of-F-PSI-Predicates} for formal definitions of these predicates.   Next, we present a formal definition of \p. 
 

% 
% 
% \begin{definition}
% %
%  [\qinit: Initiation predicate] Let $\mathcal{G}$ be a stable ledger, $adr_{\st sc}$ be smart contract $sc$'s address, $Adr$ be a set of $m+1$ distinct addresses, and $\xc$ be a fixed amount of coins. Then, predicate $\qinit(\mathcal{G}, adr_{\st sc}, m+1, Adr, \xc)$ returns $1$ if every address in $Adr$ has at least $\xc$ coins in $sc$; otherwise, it returns $0$. 
% %
% \end{definition}
%
% 
% 
%    \begin{definition}  [\qdel:
%    %
%    Delivery predicate] Let $pram:=(\mathcal{G}, adr_{\st sc}, \xc)$ be the parameters defined above, and   $adr_{\st i}\in Adr$ be the address of an honest party. 
%    %
%%    Let also $G$ be a compensation function that takes as input  two parameters $(deps, m')$, where $deps$ is the amount of coins  that all $m+1$ parties  deposit; it returns the amount of compensation each honest party must receive, i.e., $G(deps, m')\rightarrow c'$. 
%    %
%    Then, predicate $\qdel(pram, adr_{\st i})$ returns $1$ if $adr_{\st i}$ has sent $\xc$ amount to $sc$ and received  $\xc$ amount from it; thus,  its balance in $sc$ is $0$. Otherwise, it returns $0$. 
% %
%  \end{definition}
% 
% 
% 
%   \begin{definition}  [\qUnFAbt: UnFair-Abort predicate]
%   %
% Let $pram:=(\mathcal{G}, adr_{\st sc}, \xc)$ be the parameters defined above, and $Adr'\subset Adr$ be a set containing honest parties' addresses, $m' = |Adr'|$,  and   $adr_{\st i}\in Adr'$. Let also $G$ be a compensation function that takes as input  three parameters $(\depsc, adr_{\st i}, m')$, where $\depsc$ is the amount of coins  that all $m+1$ parties  deposit. It returns the amount of compensation each honest party must receive, i.e., $G(\depsc, ard_{\st i}, m')\rightarrow \xci$. Then, predicate $\qUnFAbt$ is defined as $\qUnFAbt(pram, G, \depsc, m', adr_{\st i})\rightarrow (a,b)$, where $a=1$ if $adr_{\st i}$ is an honest party's address and $adr_{\st i}$ has sent $\xc$ amount to $sc$ and received  $\xc+\xci$  from it, and $b=1$ if $adr_{\st i}$ is \aud's address and $adr_{\st i}$ received $\xci$  from $sc$. Otherwise, $a=b=0$. 
%  %
%  \end{definition}
%  
%  
%\begin{definition}  [\qFAbt: Fair-Abort predicate]
%   %
% Let $pram:=(\mathcal{G}, adr_{\st sc}, \xc)$ be the parameters defined above, and $Adr'\subset Adr$ be a set containing honest parties' addresses, $m' = |Adr'|$,     $adr_{\st i}\in Adr'$, and  $adr_{\st j}$ be \aud's address. Let $G$ be the compensation function, defined above and let $G(deps, ard_{\st j}, m')\rightarrow \xc_{\st j}$ be the compensation that the auditor must receive.  Then, predicate $\qFAbt(pram, G, \depsc, m', adr_{\st i}, adr_{\st j})$ returns $1$, if $adr_{\st i}$ (s.t. $adr_{\st i}\neq adr_{\st j}$) has sent $\xc$ amount to $sc$ and received  $\xc$  from it, and $adr_{\st j}$ received $\xc_{\st j}$  from $sc$. Otherwise, it returns $0$. 
%  %
% \end{definition}
  
  
  
 
%Note that we have upgraded the simulation-based definition of secure computation (i.e., Definition \ref{def::MPC-active-adv}) to define the security requirements of \p, by incorporating the above predicates into the definition. 
 
 \vspace{-1mm}
 
\begin{definition}[\p]\label{def::PSI-Q-fair}
Let $f^{\st \text{PSI}}$ be the multi-party PSI functionality defined above. Then, protocol $\Gamma$ realises  $f^{\st \text{PSI}}$ with $Q$-fairness in the presence of $m-1$ static active-adversary clients (i.e., $A_{\st j}$s) or a static passive dealer $D$ or passive auditor $Aud$, if for every PPT adversary $\mathcal{A}$ for the real model, there exists a PPT adversary (or simulator) $\mathsf{Sim}$ for the ideal model, such that for every $I\in \{A_{\st 1},...,A_{\st m}, D, Aud\}$, it holds that: 
%\begin{equation*}
$\{\mathsf {Ideal}^{\st \mathcal{W}(f^{\st \text{PSI}},Q)}_{\st \mathsf{Sim}(z), I}(S_{\st 1},..., S_{\st m+1})\}_{\st S_{\st 1},..., S_{\st m+1},z}\\\stackrel{c}{\equiv} \{\mathsf{Real}_{\st \mathcal{A}(z), I}^{\st \Gamma}(S_{\st 1},..., S_{\st m+1}) \}_{\st S_{\st 1},..., S_{\st m+1}, z}$,  
%\end{equation*}
where  $z$ is an auxiliary input given to $\mathcal{A}$ and  $\mathcal{W}(f^{\st \text{PSI}},Q)$ is a functionality that wraps $f^{\st \text{PSI}}$ with predicates $Q:=(\qinit, \qdel,$ $\qUnFAbt, \qFAbt)$. 
  \end{definition}
 
%   \begin{definition}  [$Q^{\st \text{Del}}$:
%   %
%    Delivery predicate] Let $pram:=(\mathcal{G}, adr_{\st sc}, c)$ be the parameters defined above, and   $adr_{\st i}\in Adr$ be the address of an honest party. Let also $G$ be a compensation function that takes as input  two parameters $(deps, m')$, where $deps$ is the amount of coins  that all $m+1$ parties  deposit; it returns the amount of compensation each honest party must receive, i.e., $G(deps, m')\rightarrow c'$. Then, predicate $Q^{\st \text{Del}}(pram, G, deps, m', adr_{\st i})$ returns $1$ if $adr_{\st i}$ has sent $c$ amount to $sc$ and received  $c+c'$  from it. Otherwise, it returns $0$. 
% %
%  \end{definition}
 
 
 
%  \begin{definition}  [$Q^{\st \text{Abt}}$: Abort predicate]
% Let $pram:=(\mathcal{G}, adr_{\st sc}, c)$ be the parameters defined above, and $Adr'\subset Adr$ be a set containing honest parties' addresses, $m' = |Adr'|$,  and   $adr_{\st i}\in Adr'$. Let also $G$ be a compensation function that takes as input  three parameters $(deps, adr_{\st i}, m')$, where $deps$ is the amount of coins  that all $m+1$ parties  deposit, $adr_{\st i}$ is an hones party's address, and $m' = |Adr'|$; it returns the amount of compensation each honest party must receive, i.e., $G(deps, ard_{\st i}, m')\rightarrow c_{\st i}$. Then, predicate $Q^{\st \text{Abt}}(pram, G, deps, m', adr_{\st i})$ returns $1$ if $adr_{\st i}$ has sent $c$ amount to $sc$ and received  $c+c'$  from it. Otherwise, it returns $0$. 
%  
%  \end{definition}
 
 %Recall, the standard simulation-based model (presented in Section \ref{}) can adequately capture the security definition of secure multi-party computation and accordingly regular PSI; however, it is not 
 

 
\vspace{-3mm}
  
% !TEX root =main.tex


\vspace{-3mm}



\section{Other Subroutines Used in \withFai}\label{sec::subroutines}
\vspace{-1mm}

In this section, we present three subroutines and a primitive that we developed and are used in the instantiation of \p, i.e., \withFai. 


\vspace{-3mm}
\subsection{Verifiable Oblivious Polynomial Randomisation (\vopr)}\label{sec::vopr}
\vspace{-1mm}

%In this section, we present ``Verifiable Oblivious Polynomial Randomisation'' (VOPR) protocol. 

In the \vopr, two parties are involved, (i) a sender which is potentially a passive adversary and (ii) a receiver that is potentially an active adversary. The protocol allows the receiver with input polynomial $\bm\beta$ (of degree $e'$) and the sender with input random polynomials $\bm\psi$ (of degree $e$) and  $\bm{\alpha}$ (of degree $e+e'$)   to compute: $\bm\theta=\bm\psi\cdot \bm\beta+\bm\alpha$, such that (a) the receiver learns only $\bm\theta$ and nothing about the sender's input even if it sets $\bm \beta=0$, (b) the sender learns nothing, and (c) the receiver's misbehaviour is detected in the protocol. Thus, the functionality that  \vopr computes is defined as $f^{\st {\vopr}}( (\bm\psi, \bm{\alpha}), \bm\beta)\rightarrow(\bot, \bm\psi\cdot \bm\beta+\bm\alpha)$. 
%
We will use {\vopr} in \withFai for two main reasons:  (a) to let a party re-randomise its counterparty's polynomial (representing its set) and (b) to impose a MAC-like structure to the randomised polynomial; such a structure will allow a verifier to detect if \vopr's output has been modified. 

Now, we outline how we design \vopr without using any (expensive) zero-knowledge proofs.\footnote{Previously, Ghosh \textit{et al.}  \cite{GhoshN19} designed a protocol called Oblivious Polynomial Addition (OPA) to meet similar security requirements that we laid out above. But, as shown in \cite{AbadiMZ21}, OPA  is susceptible to several serious attacks. } In the setup phase, both parties represent their input polynomials in the regular coefficient form; therefore, the sender's polynomials are defined as $\bm\psi=\sum\limits^{\st e}_{\st i=0}g_{\st i}\cdot x^{\st i}$ and  $\bm\alpha=\sum\limits^{\st e+e'}_{\st j=0}a_{\st j}\cdot x^{\st j}$ and the receiver's polynomial is defined as $\bm\beta=\sum\limits^{\st e'}_{\st i=0}b_{\st i}\cdot x^{\st i}$, where $b_{\st i}\neq 0$. However, the sender computes each coefficient $a_{\st j}$ (of polynomial $\bm \alpha$) as follows,  $a_{\st j}=\sum\limits^{\substack{\st k=e'\\ \st t=e}}_{\st t,k=0} a_{\st t,k}$,  where  $t+k=j$ and each $a_{\st t,k}$ is a random value. For instance, if $e=4$ and $e'=3$, then $a_{\st 3}=a_{\st \st 0,3}+a_{\st 3,0}+a_{\st 1,2}+a_{\st 2,1}$. Shortly, we explain why polynomial $\bm\alpha$ is constructed this way. 



In the computation phase,  to compute polynomial $\bm\theta$, the two parties interactively multiply and add the related coefficients in a secure way using $\ole^{\st +}$ (presented in Section \ref{sec::OLE-plus}). Specifically,
%
%For simplicity, let $i=0$. 
%
for every $j$  (where $0\leq j\leq e'$) the sender sends $g_{\st i}$ and $a_{\st i,j}$ to an instance of  $\ole^{\st +}$, while the receiver sends $b_{\st j}$ to the same instance,  which returns $c_{\st i,j}=g_{\st i}\cdot b_{\st j}+ a_{\st i,j}$ to the receiver. This process is repeated for every $i$, where $0 \leq i \leq e$. Then, the receiver uses $c_{\st i,j}$ values to construct the resulting polynomial, $\bm\theta=\bm\psi\cdot \bm\beta+\bm\alpha$.  


The reason that the sender imposes the above structure to (the coefficients of)  $\bm\alpha$ in the setup, is to let the parties securely compute $\bm\theta$ via  $\ole^{\st +}$. Specifically, by imposing this structure (1) the sender  can blind each product $g_{\st i}\cdot b_{\st j}$  with  random value $a_{\st i,j}$ which is a component of $\bm\alpha$'s coefficient and (2) the receiver can construct a result polynomial of the form $\bm\theta=\bm\psi\cdot \bm\beta+\bm\alpha$. 


To check the result's correctness, the sender picks and sends a random value $z$ to the receiver which computes  $\bm\theta(z)$ and $\bm\beta(z)$ and sends these two values  to the sender. The sender computes  $\bm\psi(z)$ and $\bm\alpha(z)$ and then checks if equation  $\bm\theta({ z})=\bm\psi({ z})\cdot \bm\beta({ z})+\bm\alpha({ z})$ holds. It accepts the result if the check passes.   

Figure \ref{fig:VOPR} describes \vopr in detail. Note, \vopr requires the sender to insert non-zero coefficients, i.e., $b_{\st i}\neq 0$ for all $i,0 \leq i \leq e'$. If the   sender inserts a zero-coefficient, then it will learn only a random value (due to  $\ole^{\st +}$), accordingly it cannot pass \vopr's verification phase. However, such a requirement will not affect Justitia's correctness, as we will discuss in Section \ref{Fair-PSI-Protocol} and Appendix \ref{sec::error-prob}.  

\vspace{-2mm}

% !TEX root =main.tex



%%%%%%%%
\begin{figure}[!htb]%[!htbp]
\setlength{\fboxsep}{.8pt}
\begin{center}
\scalebox{.85}{
    \begin{tcolorbox}[enhanced,width=5.5in, 
    drop fuzzy shadow southwest,
    colframe=black,colback=white]
%%%%%%%%

\svs
\begin{enumerate}[leftmargin=1mm]
\small{
\item[$\bullet$] \textit{Input.}
\begin{enumerate}[leftmargin=3mm]
\item[$\bullet$]  \textit{Public Parameters}: upper bound on input polynomials' degree: $e$ and $e'$. %, where $e\geq e'$.
%\item[$\bullet$]  \textit{Sender}: picks an upper bound on input polynomials's degree:  $d$ and $d'$. It sends them to the receiver.
\item[$\bullet$]  \textit{Sender Input}:  random polynomials: $\bm\psi=\sum\limits^{\st e}_{\st i=0}g_{\st i}\cdot x^{\st i}$ and  $\bm\alpha=\sum\limits^{\st e+e'}_{\st j=0}a_{\st j}\cdot x^{\st j}$, where $g_{\st i}\stackrel{\st \$}\leftarrow \mathbb{F}_p$.  Each $a_{\st j}$ has the  form: $a_{\st j}=\sum\limits^{\substack{\st k=e'\\ \st t=e}}_{\st t,k=0} a_{\st t,k}$,  such that $t+k=j$ and $a_{\st t,k}\stackrel{\st \$}\leftarrow \mathbb{F}_p$.

\item[$\bullet$] \textit{Receiver Input}:  polynomial $\bm\beta=\bm\beta_{\st 1}\cdot \bm\beta_{\st 2}=\sum\limits^{\st e'}_{\st i=0}b_{\st i}\cdot x^{\st i}$, where $\bm\beta_{\st 1}$ is a random polynomial of degree $1$ and $\bm\beta_{\st 2}$ is an arbitrary polynomial of degree $e'-1$.


\end{enumerate}
\item[$\bullet$] \textit{Output.} The receiver gets $\bm\theta=\bm\psi\cdot \bm\beta+\bm\alpha$.
\item \textbf{Computation:}

\begin{enumerate} [leftmargin=3mm]

\item Sender and receiver together for every $j$, $0\leq j\leq e'$,  invoke $e+1$ instances of $\ole^{\st +}$. In particular, $\forall j, 0\leq j\leq e'$: sender sends $g_{\st i}$ and $a_{\st i,j}$ while the receiver sends $b_{\st j}$ to $\ole^{\st +}$ that returns: $c_{\st i,j}=g_{\st i}\cdot b_{\st j}+ a_{\st i,j}$ to the receiver ($\forall i, 0\leq i\leq e$). 


 \item The receiver sums component-wise values $c_{\st i,j}$  that results in polynomial:
 %
 \vspace{-2mm}
 %
  $$\bm\theta=\bm\psi\cdot \bm\beta+\bm\alpha=\sum\limits^{\substack{\st i=e\\ \st j=e'}}_{\st i,j=0}c_{\st i, j}\cdot x^{\st i+j}$$ 
 %
  \vspace{-2mm}
  %
 




% \item The receiver sums component-wise values $c_{\st i,j}$  that results polynomial $\bm\theta=\bm\psi\cdot \bm\beta+\bm\alpha=\sum\limits^{\st e+e'}_{\st j=0}c_{\st j}\cdot x^{\st j}$, where  each $c_{\st j}$ has   form: $c_{\st j}=\sum\limits^{\substack{\st k=e'\\ \st t=e}}_{\st  t,k=0} c_{\st t,k}$, such that $ t+k=j$.
%\item Sender: $\forall j, 1\leq j\leq 2d+1$, computes $a_{\st j}=a(x_{\st j})$ and $r_{\st j}=r(x_{\st j})$. Then, it  inserts $(a_{\st j}, r_{\st j})$ into  $\mathcal{F}_{\st OLE^{\st +}}$
%\item\label{computing-receiver} Receiver:  $\forall j, 1 \leq j\leq 2d+1$, computes $b_{\st j}=b(x_{\st j})$. Then, it  inserts $b_{\st j}$ into  $\mathcal{F}_{\st OLE^{\st +}}$ and receives $s_{\st j}=a_{\st j}+b_{\st j}\cdot r_{\st j}$. It interpolates a polynomial $s(x)$ using pairs $s_{\st j},x_{\st j}$. 
\end{enumerate}
\vs
\item \label{Verification} \textbf{Verification:}
\begin{enumerate}[leftmargin=3mm]

\item \label{picking-random-x}Sender: picks a random value  $z$ and sends it to the receiver. 


\item\label{receiver-OLE-invocation} Receiver: sends $\theta_{\st z}=\bm\theta(z)$ and $\beta_{\st z}=\bm\beta(z)$ to the sender.

\item\label{receiver-OLE-invocation} Sender:  computes $\psi_{\st z}=\bm\psi(z)$ and $\alpha_{\st z}=\bm\alpha(z)$ and checks   if equation  $\theta_{\st z}=\psi{\st z}\cdot \beta_{\st z}+\alpha_{\st z}$ holds. If the equation holds, it concludes that the computation was performed correctly. Otherwise, it aborts. 
%
\vs
\end{enumerate}
}
 \end{enumerate}
 \end{tcolorbox}
 }
\end{center}
\vs
\vs
\caption{Verifiable Oblivious Polynomial Randomization ({\vopr}) } 
\label{fig:VOPR}
\end{figure}
 %%%%%%%

%\vspace{-1mm}
\begin{theorem}\label{theorem::VOPR}
%
Let $f^{\st \vopr}$ be the functionality defined above. If the enhanced \ole (i.e., $\ole^{\st +}$) is secure against malicious (or active) adversaries, then the  Verifiable Oblivious Polynomial Randomisation (\vopr), presented in Figure \ref{fig:VOPR}, securely computes $f^{\st \vopr}$ in the presence of (i) semi-honest sender and honest receiver or (ii) malicious receiver and honest sender. 
%
\end{theorem}

\vspace{-1mm}
We refer readers to Appendix \ref{sec::proof-of-vopr} for the proof of Theorem \ref{theorem::VOPR}. 


% !TEX root =main.tex

\subsection{Zero-sum Pseudorandom Values Agreement Protocol (\zspa)}

The \zspa  allows $m$ parties (the majority of which is potentially malicious) to efficiently agree on (a set of vectors, where each $i$-th vector has) $m$ pseudorandom values such that their sum equals zero. At a high level, the parties first sign a smart contract, register their accounts/addresses in it, and then run a  coin-tossing protocol \ct to agree on a key: $k$.  Next, one of the parties generates $m-1$ pseudorandom values $z_{\scriptscriptstyle i, j}$ (where $1\leq j\leq m-1$) using key $k$ and $\mathtt{PRF}$. It sets the last value as the additive inverse of the sum of the values generated, i.e. $z_{\scriptscriptstyle i, m}=-\sum\limits^{\scriptscriptstyle m-1}_{\scriptscriptstyle j=1}z_{\scriptscriptstyle i, j}$ (similar to the standard XOR-based secret sharing \cite{Schneier0078909}). 
%
%Next, it commits to each value, where it uses $k_{\scriptscriptstyle 2}$ to generate the randomness of each commitment. 
%
Then, it constructs a Merkel tree on top of the pseudorandom values and stores only the tree's root $g$ and the key's hash value $q$ in the smart contract.  Then, each party (using the key) locally checks if the values (on the contract) have been constructed correctly; if so, then it sends a (signed) ``approved" message to the contract which only accepts messages from registered parties. Hence, the functionality that \zspa computes is defined as $f^{\st \zspa}\underbrace{(\bot,..., \bot)}_{\st m}\rightarrow \underbrace{((k, g, q),..., (k, g,q))}_{\st m}$, where $g$ is the Markle tree's root built on the pseudorandom values $z_{\st i, j}$, $q$ is the hash value of the key used to generate the pseudorandom values, and $m\geq 2$. Figure \ref{fig:ZSPA} presents \zspa in detail.  


Briefly, \zspa will be used in \withFai to allow clients $\{A_{\st 1},...,A_{\st m}\}$ to provably agree on a set of pseudorandom values, where each set represents a pseudorandom polynomial (as the elements of the set are considered the polynomial's coefficients). Due to \zspa's property, the sum of these polynomials is zero.  Each of these polynomials will be used by a client to blind/encrypt the messages it sends to the smart contract, to protect the privacy of the plaintext message (from \aud, D, and the public). To compute the sum of the plaintext messages, one can easily sum the blinded messages, which removes the blinding polynomials. 

% !TEX root =main.tex




\begin{figure}[ht]%[!htbp]
\setlength{\fboxsep}{1pt}
\begin{center}
\scalebox{.85}{
    \begin{tcolorbox}[enhanced,width=5.5in, 
    drop fuzzy shadow southwest,
    colframe=black,colback=white]


\small{

\begin{enumerate}[leftmargin=1mm]
\item[$\bullet$]    {Parties.} A set of clients $\{    A_{\st 1},...,  A_{\st m}\}$.
%
\item[$\bullet$]    {Input.}  $m$: the total number of participants, $adr$: a deployed smart contract's address, and $b$: the total number of vectors. Let $b'=b-1$. 
%
\item[$\bullet$]   {Output.}  $k$: a secret key that generates $b$ vectors $[z_{\scriptscriptstyle 0,1},...,z_{\scriptscriptstyle 0,m}],...,[z_{\scriptscriptstyle b',1},...,z_{\scriptscriptstyle b',m}]$ of pseudorandom values, $h$: hash of the key,  $g$: a Merkle tree's root, and a vector of signed messages. 


%, such that the sum of each vector's elements equals zero: $\sum\limits^{\scriptscriptstyle m}_{\scriptscriptstyle j=1}z_{\scriptscriptstyle i,j}=0$. 


\item {\textbf{Coin-tossing.} $\ct (in_{\st 1},..., in_{\st m})\rightarrow k$}. 

All participants run a coin-tossing protocol to agree on $\mathtt{PRF}$'s key, $k$.
\item\label{ZSPA:val-gen}  {\textbf{Encoding.} $\mathtt{Encode}(k, m)\rightarrow (g,q)$}.

 One of the parties takes the following steps:  
\begin{enumerate}

\item for every $i$ (where $0\leq i \leq b'$), generates $m$ pseudorandom values as follows. 
%
 $$\forall j, 1\leq j \leq m-1: z_{\scriptscriptstyle i,j}=\mathtt{PRF}(k,i||j), \hspace{5mm} z_{\scriptscriptstyle i,m}=-\sum\limits^{\scriptscriptstyle m-1}_{\scriptscriptstyle j=1}z_{\scriptscriptstyle i,j}$$
%
\vs
\item   constructs a Merkel tree on top of all pseudorandom values,  $\mkgen(z_{\scriptscriptstyle 0,1},...,z_{\scriptscriptstyle b',m})\rightarrow g$. 

\item sends the Merkel tree's root: $g$,   and the key's hash: $q=\mathtt {H}(k)$ to $adr$. 

\end{enumerate}

\item\label{ZSPA:verify}{\textbf{Verification.} $\mathtt{Verify}(k, g, q, m)\rightarrow (a, s)$}. 

Each party checks if, all $z_{\scriptscriptstyle i,j}$ values, the root $g$, and key's hash $q$ have been correctly generated, by retaking  step \ref{ZSPA:val-gen}. If the checks pass, it sets $a=1$,  sets $s$ to a singed ``approved'' message, and sends $s$ to $adr$. Otherwise, it aborts by returning $a=0$ and $s = \bot$. 


 \end{enumerate}
}
 \end{tcolorbox}
 }
\end{center}
\vs
\vs
\caption{Zero-sum Pseudorandom Values Agreement (\zspa) } 
\label{fig:ZSPA}
\end{figure}



\begin{theorem}\label{theorem::ZSPA-comp-correctness}
Let $f^{\st \zspa}$ be the functionality defined above. If \ct is secure against a malicious adversary and the correctness of $\mathtt{PRF}$, $\mathtt{H}$, and Merkle tree holds, then \zspa,  in Figure \ref{fig:ZSPA}, securely computes $f^{\st \zspa}$ in the presence of $m-1 $ malicious  adversaries. 
\end{theorem}


\begin{proof}
For the sake of simplicity, we will assume the sender, which generates the result, sends the result directly to the rest of the parties, i.e., receivers, instead of sending it to a smart contract. We first consider the case in which the sender is corrupt. 

\

\noindent\textbf{Case 1: Corrupt sender.}  Let $\mathsf{Sim}^{\st \zspa}_{\st S}$ be the simulator using a subroutine adversary, $\mathcal{A}_{\st S}$. $\mathsf{Sim}^{\st \zspa}_{\st S}$ works as follows. 
%
\begin{enumerate}
%
\item simulates  \ct  and receives the output value $k$ from $f_{\st \ct}$, as we are in $f_{\st \ct}$-hybrid model.
%
\item sends $k$ to TTP and receives back from it $m$ pairs, where each pair is of the form $( g,  q)$. 
%
\item sends $ k$ to $\mathcal{A}_{\st S}$ and receives back from it $m$ pairs  where each pair is of the form $( g',  q')$. 
%
\item checks whether the following equations hold (for each pair): $ g= g' \hspace{2mm} \wedge  \hspace{2mm}  q= q'$. If the two equations do not hold, then it aborts (i.e., sends abort signal $\Lambda$ to the receiver) and proceeds to the next step.
%
\item outputs whatever $\mathcal{A}_{\st S}$ outputs.
%
 \end{enumerate}
 
 We first focus on the adversary’s output. In the real model, the only messages that the adversary receives are those messages it receives as the result of the ideal call to $f_{\st \ct}$. These messages have identical distribution to the distribution of the messages in the ideal model, as the \ct is secure. Now, we move on to the receiver’s output. We will show that the output distributions of the honest receiver in the ideal and real models are computationally indistinguishable. In the real model,  each element of pair $(g, p)$ is the output of a deterministic function on the output of $f_{\st \ct}$. We know the output of $f_{\st \ct}$ in the real and ideal models have an identical distribution, and so do the evaluations of deterministic functions (i.e., Merkle tree, $\mathtt{H}$, and $\mathtt{PRF}$) on them, as long as these three functions' correctness holds. Therefore, each pair $(g,q)$ in the real model has an identical distribution to pair $(g,  q)$ in the ideal model.  For the same reasons, the honest receiver in the real model aborts with the same probability as  $\mathsf{Sim}^{\st \zspa}_{\st S}$ does in the ideal model.  We conclude that the distributions of the joint outputs of the adversary and honest receiver in the real and ideal models are  (computationally) indistinguishable. 

\


\noindent\textbf{Case 2: Corrupt receiver.}   Let $\mathsf{Sim}^{\st \zspa}_{\st R}$ be the simulator that uses subroutine adversary $\mathcal{A}_{\st R}$. $\mathsf{Sim}^{\st \zspa}_{\st R}$ works as follows. 

\begin{enumerate}
%
\item simulates   \ct  and receives the output value $ k$ from $f_{\st \ct}$.
%
\item sends $ k$ to TTP and receives back $m$ pairs of the form $( g,  q)$ from TTP. 
%
\item sends $( k,  g,  q)$ to $\mathcal{A}_{\st R}$ and outputs whatever  $\mathcal{A}_{\st R}$ outputs. 
%
 \end{enumerate}
 
 
In the real model, the adversary receives two sets of messages, the first set includes the transcripts (including $ k$) it receives when it makes an ideal call to $f_{\st \ct}$ and the second set includes pair $(g, q)$. As we already discussed above (because we are in the  $f_{\st \ct}$-hybrid model) the distributions of the messages it receives from $f_{\st \ct}$ in the real and ideal models are identical. Moreover, the distribution of $f_{\st \ct}$'s output (i.e., $\bar k$ and $k$) in both models is identical; therefore, the honest sender's output distribution in both models is identical. As we already discussed,  the evaluations of deterministic functions (i.e., Merkle tree, $\mathtt{H}$, and $\mathtt{PRF}$) on $f_{\st \ct}$'s outputs have an identical distribution. Therefore, each pair $(g, q)$ in the real model has an identical distribution to the pair $(g, q)$ in the ideal model.  Hence, the distribution of the joint outputs of the adversary and honest receiver in the real and ideal models is indistinguishable.
%
  \hfill\(\Box\)\end{proof}

In addition to the security guarantee (i.e., computation's correctness against malicious sender or receiver) stated by Theorem \ref{theorem::ZSPA-comp-correctness}, \zspa offers  (a) privacy against the public, and (b)  non-refutability. Informally, privacy here means that given the state of the contract (i.e., $g$ and  $q$), an external party cannot learn any information about any of the pseudorandom values,  $z_{\scriptscriptstyle j}$; while non-refutability means that if a party sends ``approved" then in future cannot deny the knowledge of the values whose representation is stored in the contract. %Furthermore, indistinguishability means that every $z_{\scriptscriptstyle j}$ ($1\leq j \leq m$) should be indistinguishable from a truly random value. 




\begin{theorem}
If  $\mathtt{H}$ is preimage resistance, $\mathtt{PRF}$ is secure, the signature scheme used in the smart contract is secure (i.e., existentially unforgeable under chosen message attacks), and the blockchain is secure (i.e., offers persistence and liveness properties \cite{GarayKL15}) then \zspa offers (i) privacy against the public and (ii) non-refutability. 
\end{theorem}
 
 

\begin{proof}
First, we focus on privacy. Since key $k$, for $\mathtt{PRF}$, has been picked uniformly at random and $\mathtt{H}$ is preimage resistance, the probability that given $g$ the adversary can find $k$ is negligible in the security parameter, i.e., $\negl(\lambda)$. Furthermore, because $\mathtt{PRF}$ is secure (i.e., its outputs are indistinguishable from random values) and  $\mathtt{H}$ is preimage resistance, given the Merkle tree's root $g$, the probability that the adversary can find a leaf node, which is the output of $\mathtt{PRF}$, is $\negl(\lambda)$ too. 

Now we move on the non-refutability. Due to the persistency property of the blockchain, once a transaction/message goes more than $v$ blocks deep into the blockchain of one honest player (where $v$ is a security parameter), then it will be included in every honest player's blockchain with overwhelming probability, and it will be assigned a permanent
position in the blockchain (so it will not be modified with an overwhelming probability). Also, due to the liveness property,   all transactions originating from honest parties will eventually end up at a depth of more than $v$ blocks in an honest player's blockchain; therefore, the adversary cannot
perform a selective denial of service attack against honest account holders.  Moreover, due to the security of the digital signature (i.e., existentially unforgeable under chosen message attacks), one cannot deny sending the messages it sent to the blockchain and smart contract. 
%
\hfill\(\Box\)
\end{proof}



%
%
%\begin{theorem}
%If  $\mathtt{H}$ is preimage resistance, $\mathtt{PRF}$ is secure, the signature scheme used in the smart contract is secure (i.e., existentially unforgeable under chosen message attacks), and the blockchain is secure (i.e., offers liveness property and the hash power of the adversary is lower than those of honest miners) then \zspa offers (i) privacy against the public and (ii) non-refutability. 
%\end{theorem}
% 
% 
%
%\begin{proof}
%First, we focus on privacy. Since key $k$, for $\mathtt{PRF}$, has been picked uniformly at random and $\mathtt{H}$ is preimage resistance, the probability that given $g$ the adversary can find $k$ is negligible in the security parameter, i.e., $\negl(\lambda)$. Furthermore, because $\mathtt{PRF}$ is secure (i.e., its outputs are indistinguishable from random values) and  $\mathtt{H}$ is preimage resistance, given the Merkle tree's root $g$, the probability that the adversary can find a leaf node, which is the output of $\mathtt{PRF}$, is $\negl(\lambda)$ too. 
%  \hfill\(\Box\)\end{proof}




%Informally, there are four main security requirements that ZSPA must meet: (a) privacy, (b)  non-refutability, (c) indistinguishability, and (d) result correctness. Privacy here means given the state of the contract, an external party cannot learn any information about any of the (pseudorandom) values:  $z_{\scriptscriptstyle j}$; while non-refutability means that if a party sends ``approved" then in future cannot deny the knowledge of the values whose representation is stored in the contract. Furthermore, indistinguishability means that every $z_{\scriptscriptstyle j}$ ($1\leq j \leq m$) should be indistinguishable from a truly random value and result correctness means that a malicious result generator cannot convince other parties to accept an invalid final result, i.e., the root constructed on the invalid leaf node(s). In Figure \ref{fig:ZSPA}, we provide ZSPA that efficiently generates $b$ vectors where each vector elements is sum to zero. 






%\begin{figure}[ht]
%\setlength{\fboxsep}{0.7pt}
%\begin{center}
%\begin{boxedminipage}{12.3cm}

%
%\begin{figure}[ht]%[!htbp]
%\setlength{\fboxsep}{1pt}
%\begin{center}
%    \begin{tcolorbox}[enhanced,width=5.5in, 
%    drop fuzzy shadow southwest,
%    colframe=black,colback=white]
%
%
%\small{
%
%\begin{enumerate}
%\item[$\bullet$]  \textit{Parties.} $\{\resizeT {\textit A}_{\resizeS {\textit  1}},..., \resizeT {\textit A}_{\resizeS {\textit  m}}\}$
%\item[$\bullet$]  \textit{Input.}  $m$: the total number of participants and a deployed smart contract's address. 
%\item[$\bullet$] \textit{Output.}  $k$: a secret key that generates $b+1$ vectors $[z_{\scriptscriptstyle 0,1},...,z_{\scriptscriptstyle 1,m}],...,[z_{\scriptscriptstyle b,1},...,z_{\scriptscriptstyle b,m}]$ of pseudorandom values, $h$: hash of the key,  $g$: a Merkle tree's root, and a vector of signed messages. 
%
%
%%, such that the sum of each vector's elements equals zero: $\sum\limits^{\scriptscriptstyle m}_{\scriptscriptstyle j=1}z_{\scriptscriptstyle i,j}=0$. 
%
%
%\item All participants run a coin tossing protocol to agree on a key $k$  of $\mathtt{PRF}$.
%\item\label{ZSPA:val-gen} One of the parties:  
%\begin{enumerate}
%
%\item for every $i$ (where $0\leq i \leq b$), generates $m$ pseudorandom values as follows. 
%%
% $$\forall j, 1\leq j \leq m-1: z_{\scriptscriptstyle i,j}=\mathtt{PRF}(k,i||j), \hspace{5mm} z_{\scriptscriptstyle i,m}=-\sum\limits^{\scriptscriptstyle m-1}_{\scriptscriptstyle j=1}z_{\scriptscriptstyle i,j}$$
%%
%\item   constructs a Merkel tree on top of all pseudorandom values,  $\mathtt{MT.genTree}(z_{\scriptscriptstyle 0,1},...,z_{\scriptscriptstyle b,m})\rightarrow g$. 
%
%\item  sends the Merkel tree's root: $g$,   and the key's hash: $q=\mathtt {H}(k)$ to the smart contract. 
%
%\end{enumerate}
%
%\item\label{ZSPA:verify} The rest of parties (given $k_{\scriptscriptstyle 1}, k_{\scriptscriptstyle 2}$) check if, all $z_{\scriptscriptstyle i,j}$ values, the root $g$ and key's hash have been correctly generated (by redoing  step \ref{ZSPA:val-gen}). If the checks pass, each party sends a singed ``approved'' message to the  contract. Otherwise, it aborts. 
%
%
% \end{enumerate}
%}
% \end{tcolorbox}
%\end{center}
%\caption{Zero-sum Pseudorandom Values Agreement (ZSPA) Protocol} 
%\label{fig:ZSPA}
%\end{figure}
%




%%%%%%%%%%%%%%%%%%%%%%%%%%%%%%%%%%%%%%%
%\begin{figure}[ht]
%\setlength{\fboxsep}{0.7pt}
%\begin{center}
%\begin{boxedminipage}{12.3cm}
%
%\small{
%
%\begin{enumerate}
%\item[$\bullet$]  \textit{Parties:} $\{\resizeT {\textit A}_{\resizeS {\textit  1}},..., \resizeT {\textit A}_{\resizeS {\textit  m}}\}$
%\item[$\bullet$]  \textit{Public Parameters and Functions:} A pseudorandom function: $\mathtt{PRF}$, a deployed smart contract, and total number of participants: $m$. 
%\item[$\bullet$] \textit{Output}:  All parties agree on $b+1$ vectors $[z_{\scriptscriptstyle 0,1},...,z_{\scriptscriptstyle 1,m}],...,[z_{\scriptscriptstyle b,1},...,z_{\scriptscriptstyle b,m}]$, of pseudorandom values, such that the sum of each vector's elements equals zero: $\sum\limits^{\scriptscriptstyle m}_{\scriptscriptstyle j=1}z_{\scriptscriptstyle i,j}=0$
%
%
%\item All participants run a coin tossing protocol to agree on two keys $k_{\scriptscriptstyle 1}$ and $k_{\scriptscriptstyle 2}$ of $\mathtt{PRF}$.
%\item\label{ZSPA:val-gen} One of the parties:  
%\begin{enumerate}
%
%\item For every $i$, computes $m$ pseudorandom values: $\forall j, 1\leq j \leq m-1: z_{\scriptscriptstyle i,j}=\mathtt{PRF}(k_{\scriptscriptstyle 1},i||j)$ and sets $z_{\scriptscriptstyle i,m}=-\sum\limits^{\scriptscriptstyle m-1}_{\scriptscriptstyle j=1}z_{\scriptscriptstyle i,j}$, where $0\leq i \leq b$
%
%\item   commits to every $z_{\scriptscriptstyle i,j}$  as follows: $\mathtt{a}_{\scriptscriptstyle i,j}=\mathtt{Com}(z_{\scriptscriptstyle i,j}, q_{\scriptscriptstyle i,j})$, where the randomness of the commitment is computed as: $ q_{\scriptscriptstyle i,j}=\mathtt{PRF}(k_{\scriptscriptstyle 2},i||j)$ and  $1\leq j \leq m$.
%
%\item   constructs a Merkel tree on top of the committed values: $\mathtt{MT}(\mathtt{a}_{\scriptscriptstyle 0,1},...,\mathtt{a}_{\scriptscriptstyle b,m})\rightarrow g$ 
%
%\item  sends the Merkel tree's root: $g$,   and the keys' hashes: $\mathtt {H}(k_{\scriptscriptstyle 1})$ and $ \mathtt {H}(k_{\scriptscriptstyle 2})$, to the contract. 
%
%\end{enumerate}
%
%\item\label{ZSPA:verify} The rest of parties (given $k_{\scriptscriptstyle 1}, k_{\scriptscriptstyle 2}$) check if, all $z_{\scriptscriptstyle i,j}$ values, the root $g$ and keys' hashes have been correctly generated (by redoing  step \ref{ZSPA:val-gen}). If passed, each party sends a singed ``approved'' message to the  contract. Otherwise, it aborts. 
%
%
% \end{enumerate}
%}
%\end{boxedminipage}
%\end{center}
%\caption{Zero-sum Pseudorandom Values Agreement ($\mathtt{ZSPA}$) Protocol} 
%\label{fig:ZSPA}
%\end{figure}




%% !TEX root =main.tex




\begin{figure}[ht]%[!htbp]
\setlength{\fboxsep}{1pt}
\begin{center}
\scalebox{.85}{
    \begin{tcolorbox}[enhanced,width=5.5in, 
    drop fuzzy shadow southwest,
    colframe=black,colback=white]


{\small{

%\underline{$\mathtt{Audit}( \vv{{k}},  q, \bm\zeta, \bar d, g, \vv v)\rightarrow (L, \vv{{\mu}})$}
\begin{enumerate}
%\item[$\bullet$] Parties: clients: $\{  {   A}_{    {    1}},...,   {   A}_{    {    m}}\}$, the dealer and  an Arbiter.


\item[$\bullet$]    {Parties.} A set of clients $\{ A_{\st 1},...,  A_{\st m}\}$ and an external auditor, \aud. 

\item[$\bullet$]    {Input.}  $m$: the total number of participants (excluding the auditor), $\bm\zeta$: a random polynomial of degree $1$, $b$: the total number of vectors, and $adr$: a deployed smart contract's address. Let $b'=b-1$.





%\item[$\bullet$]   {Input.} $\vv{{k}}=[k_{\st 1},..., k_{\st m}]$,    $q$: a  hash value, $\bm\zeta$: a random polynoimal of degree $1$, $\bar d$: a polynoimal's degree,   $g$: a root of Merkle tree, and $\vv v$: binary vector of size $m$. 


\item[$\bullet$]  {Output of  each} $  A_{\st j}$.   $k$: a secret key that generates $b$ vectors $[z_{\scriptscriptstyle 0,1},...,z_{\scriptscriptstyle 0,m}],...,[z_{\scriptscriptstyle b',1},...,z_{\scriptscriptstyle b', m}]$ of pseudorandom values, $h$: hash of the key,  $g$: a Merkle tree's root, and a vector of signed messages. 



\item[$\bullet$]    {Output of \aud.} $L$: a list of misbehaving parties' indices, and  $\vv{{\mu}}$: a vector of random polynomials.
%
\item\label{ZSPA::ZSPA-invocation} {\textbf{\zspa invocation.}  $\zspa(\bot,..., \bot)\rightarrow \Big((k, g, q),..., (k, g,q )\Big)$}. 

All parties in $\{A_{\st 1},...,  A_{\st m}\}$ call the same instance of \zspa, which results in  $(k, g, q), ..., (k, g, q)$. 
%

\item\label{ZSPA-A::Auditor-computation}  {\textbf{Auditor computation.} $\mathtt{Audit}( \vv{{k}},  q, \bm\zeta, b, g)\rightarrow (L, \vv{{\mu}})$}. 

\aud\ takes the below steps. Note,  each $k_{\st j}\in \vv{{k}}$ is given by  $  A_{\st j}$. An honest party's input, $k_{\st j}$,  equals $k$, where $1\leq j \leq m$. 


\begin{enumerate}
%
\item runs the checks in the verification phase (i.e., Phase \ref{ZSPA:verify}) of \zspa for every $j$, i.e., $\mathtt{Verify}(k_{\st j}, g, q, m)\rightarrow (a_{\st j}, s)$.
\item appends $j$ to $L$, if any checks fails, i.e., if $a_{\st j}=0$. In this case, it skips the next two steps for the current $j$. 



%
%
%\item  Checks whether equation $\mathtt{H}(k_{\st j})=q$ holds  for every $j$, $1\leq j \leq m$.   
%%
%\begin{itemize}
%%
%\item[$\bullet$] if any $j$-th check fails,  it adds $j$ to $L$.
%%
%\item[$\bullet$]  if $L$ contains all $j\in[1,m]$, it returns $L$ and aborts. 
%%
%\end{itemize}
%%
%\item\label{zero-sum-arbiter-verification} Verifies the Merkle tree's root, $g$, by checking if the tree (corresponding to  $g$) has been correctly constructed on the correct leaf nodes. In particular, it takes the following steps. 
%
%\begin{enumerate}
%
%\item regenerates the tree's leaf nodes (similar to step \ref{ZSPA:val-gen} in Fig. \ref{fig:ZSPA}) as follows. Let $k$ be a key that passed the above check.  For every $i$ (where $0\leq i \leq \bar d$), it recomputes $m$ pseudorandom values: 
%%
%$$\forall j, 1\leq j \leq m-1: z_{\st i,j}=\mathtt{PRF}(k,i||j), \hspace{4mm} z_{\st i,m}=-\sum\limits^{\st m-1}_{\st j=1}z_{\st i,j}$$
%%
%\item   constructs a Merkel tree on top of all pseudorandom values generated in the previous step, i.e., $\mathtt{MT.genTree}(z_{\st 0,1},...,z_{\st \bar d,m})\rightarrow g'$. 
%%
%\item checks if $g=g'$. If the equation does not hold, then it adds to $L$ every index $j$ whose value in $\vv v$ is $1$, i.e., $\vv v[j]=1$; in this case, it returns $L$ and aborts.
%%
%\end{enumerate}
%

\item\label{ZSPA-A::gen-z} For every $i$ (where $0\leq i \leq b'$), it recomputes $m$ pseudorandom values: 
%
$\forall j, 1\leq j \leq m-1: z_{\st i,j}=\mathtt{PRF}(k,i||j), \hspace{4mm} z_{\st i,m}=-\sum\limits^{\st m-1}_{\st j=1}z_{\st i,j}$.
%
 \item generates polynomial $\bm\mu^{\st (j)}$ as follows: 
  %
   $\bm\mu^{\st (j)} = \bm\zeta\cdot \bm\xi^{\st (j)}-\bm\tau^{\st (j)}$, 
   %
    where $\bm\xi^{\st (j)}$ is a random polynomial of degree $b'-1$ and $\bm\tau^{\st (j)}=\sum\limits^{\st b'}_{\st i=0}z_{\st i,j}\cdot x^{\st i}$. By the end of this step, a vector $\vv{{\mu}}$ containing at most $m$ polynomials is generated. 
%
 \item returns   list $L$ and $\vv{{\mu}}$.
 
\end{enumerate}
 \end{enumerate}
}}
 \end{tcolorbox}
 }
\end{center}
\caption{\zspa with an external auditor (\zspaa)} 
\label{fig:arbiter}
\end{figure}



%%%%%%%%%%%%%%%%%%%%%%%%%%%%%%%%%%%%%%%%%%%%%%
%\begin{figure}[ht]%[!htbp]
%\setlength{\fboxsep}{1pt}
%\begin{center}
%    \begin{tcolorbox}[enhanced,width=5.5in, 
%    drop fuzzy shadow southwest,
%    colframe=black,colback=white]
%
%
%{\small{
%
%\underline{$\mathtt{Audit}( \vv{{k}},  q, \bm\zeta, \bar d, g, \vv v)\rightarrow (L, \vv{{\mu}})$}
%\begin{enumerate}
%%\item[$\bullet$] Parties: clients: $\{  {   A}_{    {    1}},...,   {   A}_{    {    m}}\}$, the dealer and  an Arbiter.
%\item[$\bullet$]   {Input.} $\vv{{k}}=[k_{\st 1},...,k_{\st m}]$,    $q$: a  hash value, $\bm\zeta$: a random polynoimal of degree $1$, $\bar d$: a polynoimal's degree,   $g$: a root of Merkle tree, and $\vv v$: binary vector of size $m$. 
%
%
%\item[$\bullet$]    {Output.} A list of rejected values' indices: $L$, a vector of random polynomials: $\vv{{\mu}}$.
%%
%\item  Checks whether equation $\mathtt{H}(k_{\st j})=q$ holds  for every $j$, $1\leq j \leq m$.   
%%
%\begin{itemize}
%%
%\item[$\bullet$] if any $j$-th check fails,  it adds $j$ to $L$.
%%
%\item[$\bullet$]  if $L$ contains all $j\in[1,m]$, it returns $L$ and aborts. 
%%
%\end{itemize}
%%
%\item\label{zero-sum-arbiter-verification} Verifies the Merkle tree's root, $g$, by checking if the tree (corresponding to  $g$) has been correctly constructed on the correct leaf nodes. In particular, it takes the following steps. 
%
%\begin{enumerate}
%
%\item regenerates the tree's leaf nodes (similar to step \ref{ZSPA:val-gen} in Fig. \ref{fig:ZSPA}) as follows. Let $k$ be a key that passed the above check.  For every $i$ (where $0\leq i \leq \bar d$), it recomputes $m$ pseudorandom values: 
%%
%$$\forall j, 1\leq j \leq m-1: z_{\st i,j}=\mathtt{PRF}(k,i||j), \hspace{4mm} z_{\st i,m}=-\sum\limits^{\st m-1}_{\st j=1}z_{\st i,j}$$
%%
%\item   constructs a Merkel tree on top of all pseudorandom values generated in the previous step, i.e., $\mathtt{MT.genTree}(z_{\st 0,1},...,z_{\st \bar d,m})\rightarrow g'$. 
%%
%\item checks if $g=g'$. If the equation does not hold, then it adds to $L$ every index $j$ whose value in $\vv v$ is $1$, i.e., $\vv v[j]=1$; in this case, it returns $L$ and aborts.
%%
%\end{enumerate}
%%
% \item Generates polynomial $\bm\mu^{\st (j)}$, for every $j$ such that $j\in[1,m]$ and $j \notin L$,  as follows:
%  %
%   $$\bm\mu^{\st (j)} = \bm\zeta\cdot \bm\xi^{\st (j)}-\bm\tau^{\st (j)}$$
%   %
%    where $\bm\xi^{\st (j)}$ is a random polynomial of degree $\bar d-1$ and $\bm\tau^{\st (j)}=\sum\limits^{\st \bar d}_{\st i=0}z_{\st i,j}\cdot x^{\st i}$. By the end of this step, a vector $\vv{{\mu}}$ containing at most $m$ polynomials is generated. 
%%
% \item Returns   list $L$ and $\vv{{\mu}}$.
% 
%
% \end{enumerate}
%}}
% \end{tcolorbox}
%\end{center}
%\caption{$\text{Audit}$ Algorithm} 
%\label{fig:arbiter}
%\end{figure}


% !TEX root =main.tex




\vs




\subsection{\zspa's Extension: \zspa with an External Auditor (\zspaa)}


In this section, we present an extension of \zspa, called \zspaa which lets a (trusted) third-party auditor, \aud, help identify misbehaving clients in the \zspa and generate a vector of random polynomials. Informally, \zspaa requires that misbehaving parties are always detected, except with a negligible probability. \aud of this protocol will be invoked by \withFai when \withFai's smart contract detects that a combination of the messages sent by the clients is not well-formed. Later, in \withFai's proof, we will show that even a \emph{semi-honest} \aud who observes all messages that clients send to \withFai's smart contracts, cannot learn anything about their set elements. We present \zspaa in Figure \ref{fig:arbiter}. 


\vs


% !TEX root =main.tex




\begin{figure}[ht]%[!htbp]
\setlength{\fboxsep}{1pt}
\begin{center}
\scalebox{.85}{
    \begin{tcolorbox}[enhanced,width=5.5in, 
    drop fuzzy shadow southwest,
    colframe=black,colback=white]


{\small{

%\underline{$\mathtt{Audit}( \vv{{k}},  q, \bm\zeta, \bar d, g, \vv v)\rightarrow (L, \vv{{\mu}})$}
\begin{enumerate}
%\item[$\bullet$] Parties: clients: $\{  {   A}_{    {    1}},...,   {   A}_{    {    m}}\}$, the dealer and  an Arbiter.


\item[$\bullet$]    {Parties.} A set of clients $\{ A_{\st 1},...,  A_{\st m}\}$ and an external auditor, \aud. 

\item[$\bullet$]    {Input.}  $m$: the total number of participants (excluding the auditor), $\bm\zeta$: a random polynomial of degree $1$, $b$: the total number of vectors, and $adr$: a deployed smart contract's address. Let $b'=b-1$.





%\item[$\bullet$]   {Input.} $\vv{{k}}=[k_{\st 1},..., k_{\st m}]$,    $q$: a  hash value, $\bm\zeta$: a random polynoimal of degree $1$, $\bar d$: a polynoimal's degree,   $g$: a root of Merkle tree, and $\vv v$: binary vector of size $m$. 


\item[$\bullet$]  {Output of  each} $  A_{\st j}$.   $k$: a secret key that generates $b$ vectors $[z_{\scriptscriptstyle 0,1},...,z_{\scriptscriptstyle 0,m}],...,[z_{\scriptscriptstyle b',1},...,z_{\scriptscriptstyle b', m}]$ of pseudorandom values, $h$: hash of the key,  $g$: a Merkle tree's root, and a vector of signed messages. 



\item[$\bullet$]    {Output of \aud.} $L$: a list of misbehaving parties' indices, and  $\vv{{\mu}}$: a vector of random polynomials.
%
\item\label{ZSPA::ZSPA-invocation} {\textbf{\zspa invocation.}  $\zspa(\bot,..., \bot)\rightarrow \Big((k, g, q),..., (k, g,q )\Big)$}. 

All parties in $\{A_{\st 1},...,  A_{\st m}\}$ call the same instance of \zspa, which results in  $(k, g, q), ..., (k, g, q)$. 
%

\item\label{ZSPA-A::Auditor-computation}  {\textbf{Auditor computation.} $\mathtt{Audit}( \vv{{k}},  q, \bm\zeta, b, g)\rightarrow (L, \vv{{\mu}})$}. 

\aud\ takes the below steps. Note,  each $k_{\st j}\in \vv{{k}}$ is given by  $  A_{\st j}$. An honest party's input, $k_{\st j}$,  equals $k$, where $1\leq j \leq m$. 


\begin{enumerate}
%
\item runs the checks in the verification phase (i.e., Phase \ref{ZSPA:verify}) of \zspa for every $j$, i.e., $\mathtt{Verify}(k_{\st j}, g, q, m)\rightarrow (a_{\st j}, s)$.
\item appends $j$ to $L$, if any checks fails, i.e., if $a_{\st j}=0$. In this case, it skips the next two steps for the current $j$. 



%
%
%\item  Checks whether equation $\mathtt{H}(k_{\st j})=q$ holds  for every $j$, $1\leq j \leq m$.   
%%
%\begin{itemize}
%%
%\item[$\bullet$] if any $j$-th check fails,  it adds $j$ to $L$.
%%
%\item[$\bullet$]  if $L$ contains all $j\in[1,m]$, it returns $L$ and aborts. 
%%
%\end{itemize}
%%
%\item\label{zero-sum-arbiter-verification} Verifies the Merkle tree's root, $g$, by checking if the tree (corresponding to  $g$) has been correctly constructed on the correct leaf nodes. In particular, it takes the following steps. 
%
%\begin{enumerate}
%
%\item regenerates the tree's leaf nodes (similar to step \ref{ZSPA:val-gen} in Fig. \ref{fig:ZSPA}) as follows. Let $k$ be a key that passed the above check.  For every $i$ (where $0\leq i \leq \bar d$), it recomputes $m$ pseudorandom values: 
%%
%$$\forall j, 1\leq j \leq m-1: z_{\st i,j}=\mathtt{PRF}(k,i||j), \hspace{4mm} z_{\st i,m}=-\sum\limits^{\st m-1}_{\st j=1}z_{\st i,j}$$
%%
%\item   constructs a Merkel tree on top of all pseudorandom values generated in the previous step, i.e., $\mathtt{MT.genTree}(z_{\st 0,1},...,z_{\st \bar d,m})\rightarrow g'$. 
%%
%\item checks if $g=g'$. If the equation does not hold, then it adds to $L$ every index $j$ whose value in $\vv v$ is $1$, i.e., $\vv v[j]=1$; in this case, it returns $L$ and aborts.
%%
%\end{enumerate}
%

\item\label{ZSPA-A::gen-z} For every $i$ (where $0\leq i \leq b'$), it recomputes $m$ pseudorandom values: 
%
$\forall j, 1\leq j \leq m-1: z_{\st i,j}=\mathtt{PRF}(k,i||j), \hspace{4mm} z_{\st i,m}=-\sum\limits^{\st m-1}_{\st j=1}z_{\st i,j}$.
%
 \item generates polynomial $\bm\mu^{\st (j)}$ as follows: 
  %
   $\bm\mu^{\st (j)} = \bm\zeta\cdot \bm\xi^{\st (j)}-\bm\tau^{\st (j)}$, 
   %
    where $\bm\xi^{\st (j)}$ is a random polynomial of degree $b'-1$ and $\bm\tau^{\st (j)}=\sum\limits^{\st b'}_{\st i=0}z_{\st i,j}\cdot x^{\st i}$. By the end of this step, a vector $\vv{{\mu}}$ containing at most $m$ polynomials is generated. 
%
 \item returns   list $L$ and $\vv{{\mu}}$.
 
\end{enumerate}
 \end{enumerate}
}}
 \end{tcolorbox}
 }
\end{center}
\caption{\zspa with an external auditor (\zspaa)} 
\label{fig:arbiter}
\end{figure}



%%%%%%%%%%%%%%%%%%%%%%%%%%%%%%%%%%%%%%%%%%%%%%
%\begin{figure}[ht]%[!htbp]
%\setlength{\fboxsep}{1pt}
%\begin{center}
%    \begin{tcolorbox}[enhanced,width=5.5in, 
%    drop fuzzy shadow southwest,
%    colframe=black,colback=white]
%
%
%{\small{
%
%\underline{$\mathtt{Audit}( \vv{{k}},  q, \bm\zeta, \bar d, g, \vv v)\rightarrow (L, \vv{{\mu}})$}
%\begin{enumerate}
%%\item[$\bullet$] Parties: clients: $\{  {   A}_{    {    1}},...,   {   A}_{    {    m}}\}$, the dealer and  an Arbiter.
%\item[$\bullet$]   {Input.} $\vv{{k}}=[k_{\st 1},...,k_{\st m}]$,    $q$: a  hash value, $\bm\zeta$: a random polynoimal of degree $1$, $\bar d$: a polynoimal's degree,   $g$: a root of Merkle tree, and $\vv v$: binary vector of size $m$. 
%
%
%\item[$\bullet$]    {Output.} A list of rejected values' indices: $L$, a vector of random polynomials: $\vv{{\mu}}$.
%%
%\item  Checks whether equation $\mathtt{H}(k_{\st j})=q$ holds  for every $j$, $1\leq j \leq m$.   
%%
%\begin{itemize}
%%
%\item[$\bullet$] if any $j$-th check fails,  it adds $j$ to $L$.
%%
%\item[$\bullet$]  if $L$ contains all $j\in[1,m]$, it returns $L$ and aborts. 
%%
%\end{itemize}
%%
%\item\label{zero-sum-arbiter-verification} Verifies the Merkle tree's root, $g$, by checking if the tree (corresponding to  $g$) has been correctly constructed on the correct leaf nodes. In particular, it takes the following steps. 
%
%\begin{enumerate}
%
%\item regenerates the tree's leaf nodes (similar to step \ref{ZSPA:val-gen} in Fig. \ref{fig:ZSPA}) as follows. Let $k$ be a key that passed the above check.  For every $i$ (where $0\leq i \leq \bar d$), it recomputes $m$ pseudorandom values: 
%%
%$$\forall j, 1\leq j \leq m-1: z_{\st i,j}=\mathtt{PRF}(k,i||j), \hspace{4mm} z_{\st i,m}=-\sum\limits^{\st m-1}_{\st j=1}z_{\st i,j}$$
%%
%\item   constructs a Merkel tree on top of all pseudorandom values generated in the previous step, i.e., $\mathtt{MT.genTree}(z_{\st 0,1},...,z_{\st \bar d,m})\rightarrow g'$. 
%%
%\item checks if $g=g'$. If the equation does not hold, then it adds to $L$ every index $j$ whose value in $\vv v$ is $1$, i.e., $\vv v[j]=1$; in this case, it returns $L$ and aborts.
%%
%\end{enumerate}
%%
% \item Generates polynomial $\bm\mu^{\st (j)}$, for every $j$ such that $j\in[1,m]$ and $j \notin L$,  as follows:
%  %
%   $$\bm\mu^{\st (j)} = \bm\zeta\cdot \bm\xi^{\st (j)}-\bm\tau^{\st (j)}$$
%   %
%    where $\bm\xi^{\st (j)}$ is a random polynomial of degree $\bar d-1$ and $\bm\tau^{\st (j)}=\sum\limits^{\st \bar d}_{\st i=0}z_{\st i,j}\cdot x^{\st i}$. By the end of this step, a vector $\vv{{\mu}}$ containing at most $m$ polynomials is generated. 
%%
% \item Returns   list $L$ and $\vv{{\mu}}$.
% 
%
% \end{enumerate}
%}}
% \end{tcolorbox}
%\end{center}
%\caption{$\text{Audit}$ Algorithm} 
%\label{fig:arbiter}
%\end{figure}





\begin{theorem}\label{theorem::ZSPA-A}
If \zspa is secure, $\mathtt{H}$ is second-preimage resistant, and the correctness of $\mathtt{PRF}$, $\mathtt{H}$, and Merkle tree holds,  then \zspaa securely computes $f^{\st \zspaa}$ in the presence of $m-1 $ malicious adversaries.% or (ii) a semi-honest auditor. 
\end{theorem}

\svs

We refer readers to Appendix \ref{sec::proof-of-zspaa} for the proof of Theorem \ref{theorem::ZSPA-A}. 

%As we stated previously, the ZSPA-A protocol will be invoked as a subroutine in the fair PSI protocol. As part of proving Theorem \ref{theorem::ZSPA-A}, we would like to show that the semi-honest auditor's view can be simulated (so it cannot learn the parties' set elements), even if it has access to those transcripts of the fair PSI protocol sent to the smart contract; because such an approach offers a stronger security guarantee than proving the ZSPA-A protocol in isolation.  Therefore, we will present the proof of Theorem \ref{theorem::ZSPA-A} after we present the fair PSI protocol. 





%\begin{theorem}\label{theorem::ZSPA-comp-correctness}
%If the coin-tossing protocol is secure against a malicious adversary, then the ZSPA protocol,  in Figure \ref{fig:ZSPA}, securely computes $f^{\st \text {ZSPA}}$ in the presence of a malicious adversary. 
%\end{theorem}


%\begin{figure}%[ht]
%\setlength{\fboxsep}{0.7pt}
%\begin{center}
%\begin{boxedminipage}{12.3cm}
%
%\small{
%
%\begin{enumerate}
%\item[$\bullet$] Parties: clients: $\{\resizeT {\textit A}_{\resizeS {\textit  1}},..., \resizeT {\textit A}_{\resizeS {\textit  m}}\}$, the dealer and  an Arbiter.
%\item[$\bullet$] Input: Empty malicious clients list: $L$ and a deployed smart contract's address. 
%\item[$\bullet$] Output: Misbehaving clients list: $L$
%\item Every client sends to the Arbiter  two keys: $k_{\scriptscriptstyle 1}, k_{\scriptscriptstyle 2}$, used to generate the zero-sum values and their commitments. 
%%
%\item  The Arbiter checks if the clients  provided correct keys, by ensuring that the keys' hashes matches the ones stored in the contract. It appends the IDs of those  provided inconsistent keys to $L$. If all clients provided inconsistent keys it aborts. Otherwise, it proceed to the next step where it uses correct keys: $k_{\scriptscriptstyle 1}, k_{\scriptscriptstyle 2}$. 
%%
%\item\label{zero-sum-arbiter-verification} The Arbiter (given correct keys) regenerate the  zero-sum values $z_{\scriptscriptstyle i, j}$ and verify the correctness of their commitments and the Merkel tree root contracted on top of the commitments, i.e. takes the same step as step \ref{ZSPA:verify} in Fig \ref{fig:ZSPA}.   It aborts if any of the   checks is rejected, and appends to $L$ the IDs of the clients which sent the ``approved'' message to the contract. 
%%
% \item The Arbiter for each client $\resizeT {\textit C}$, who provided correct keys,  generates polynomial $\bm\mu^{\resizeS {\textit {(C)}}}$, for each bin, as follows:
%  %
%   $$\bm\mu^{\resizeS {\textit {(C)}}} = \bm\zeta\cdot \bm\xi^{\resizeS {\textit {(C)}}}-\bm\tau^{\resizeS {\textit {(C)}}}$$
%   %
%    where $\bm\xi^{\resizeS {\textit {(C)}}}$ is a random polynomial of degree $3d+1$ and $\bm\tau^{\resizeS {\textit {(C)}}}=\sum\limits^{\st 3d+2}_{\st i=0}z_{\st i,c}\cdot x^{\st i}$. By the end of this step, a vector $\vv{\bm{\mu}}$ containing polynomial $\bm\mu^{\resizeS {\textit {(C)}}}$ for every bin of client $\resizeT {\textit C}$ that is not in list $L$. 
%    %
%     \item returns   list $L$ and $\vv{\bm{\mu}}$.
 
%
%
% \item The dealer, for each client $\resizeT {\textit C}\in \{\resizeT {\textit A}_{\resizeS {\textit  1}},..., \resizeT {\textit A}_{\resizeS {\textit  m}}\}$,  sends to the Arbiter a blind polynomial of the form: $\bm\zeta\cdot \bm\eta^{\resizeS {\textit {(D,C)}}}-(\bm\gamma^{\resizeS {\textit {(D,C)}}}+\bm\delta^{\resizeS {\textit {(D,C)}}})$, where $\bm\eta^{\resizeS {\textit {(D,C)}}}$ is a fresh random degree $3d+1$ polynomial. The blind polynomial will allow the arbiter to obliviously verify the correctness of the message each client sent to the  contract. 
% 
% \item The Arbiter for each client $\resizeT {\textit C}$ who provided correct keys: 
% 
% \begin{enumerate}
% \item adds together the blind polynomial above and the blind polynomial $\bm\nu^{\resizeS {\textit {(C)}}}$ the client sent to the contract (in step \ref{blindPoly-C-sends-to-contract} in the PSI protocol). Then, it removes the client's zero-sum pseudorandom values from the result. In particular, it computes:    
%\begin{equation*}
%\begin{split}
% \bm\iota^{\resizeS {\textit {(C)}}}&=\bm\zeta\cdot \bm\eta^{\resizeS {\textit {(D,C)}}}-(\bm\gamma^{\resizeS {\textit {(D,C)}}}+\bm\delta^{\resizeS {\textit {(D,C)}}})+\bm\nu^{\resizeS {\textit {(C)}}}-\sum\limits^{\scriptscriptstyle 3d+1}_{\scriptscriptstyle i=0}z_{\scriptscriptstyle i,c}\cdot x^{\scriptscriptstyle i} \\ &=\bm\zeta\cdot(\bm\eta^{\resizeS {\textit {(D,C)}}} + \bm\omega^{\resizeS {\textit {(D,C)}}}\cdot \bm\omega^{\resizeS {\textit {(C,D)}}}\cdot \bm\pi^{\resizeS {\textit {(C)}}}+\bm\rho^{\resizeS {\textit {(D,C)}}}\cdot \bm\rho^{\resizeS {\textit {(C,D)}}}\cdot \bm\pi^{\resizeS {\textit {(D)}}})
% \end{split}
%\end{equation*}
%  \item checks if $\bm\zeta$ can divide $\bm\iota^{\resizeS {\textit {(C)}}}$. If can not, it appends the client's ID to $L$.
%  \end{enumerate}
  %$deg(\eta^{\resizeS {\textit {D,C}}})=3d+1$
% \end{enumerate}
%}
%\end{boxedminipage}
%\end{center}
%\caption{$\mathtt{Arbiter}$ Protocol} 
%\label{fig:arbiter}
%\end{figure}









  
% !TEX root =main.tex




\vs 




\subsection{Unforgeable Polynomials}


In this section, we introduce the notion of ``unforgeable polynomials''. Informally, an unforgeable polynomial has a secret factor. To ensure that an unforgeable polynomial has not been tampered with, a verifier can check whether the polynomial is divisible by the secret factor. 


To turn an arbitrary polynomial $\bm\pi$ of degree $d$ into an unforgeable polynomial $\bm\theta$, one can (i) pick three secret random polynomials $(\bm\zeta, \bm\omega, \bm \gamma)$ and (ii) compute $\bm\theta=\bm\zeta\cdot \bm\omega\cdot\bm \pi + \bm \gamma \bmod p$, where  $deg(\bm\zeta)= 1, deg(\bm\omega)=d,$ and   $deg(\bm\gamma)= 2d+1$. 
%
To verify whether $\bm\theta$ has been tampered with, a verifier (given $\bm\theta, \bm \gamma$, and $\bm\zeta$) can check if $\bm\theta-\bm \gamma$ is divisible by $\bm\zeta$. The security of \emph{unforgeable polynomial} states that an adversary (who does not know the three secret random polynomials) cannot tamper with an unforgeable polynomial without being detected, except with a negligible probability, in the security parameter. Below, we formally state it. 




\begin{theorem}[Unforgeable Polynomial]\label{proof::unforgeable-poly}
%Let polynomials $\zeta$ and $\gamma$ be two secret uniformly random polynomials (i.e., $\zeta, \gamma\stackrel{\st\$}\leftarrow \mathbb F_{\st p}[x]$),   $GCD(\zeta, \gamma)=1$, polynomial $\pi$ be an arbitrary polynomial,   $deg(\zeta)= 1, deg(\gamma)= d+1$,  $deg(\pi)=d$, and $p$ be a $\lambda$-bit prime number. Also, let polynomial $\theta$ be defined as  $\theta=\zeta\cdot \pi+ \gamma \bmod p$. Given $(\theta,\pi)$, the probability that a PPT adversary (which does not know $\zeta$ and $\gamma$) can forge $\theta$ to an arbitrary polynomial $\theta'$ such that  $\theta'\neq \theta$, $deg(\theta')\leq poly(\lambda)$, and $\zeta$ divides $\theta'-\gamma$ is negligible in the security parameter, i.e.,
%
Let polynomials $\bm\zeta$, $\bm\omega$, and $\bm\gamma$ be three secret uniformly random polynomials (i.e., $\bm\zeta,\bm\omega, \bm\gamma\stackrel{\st\$}\leftarrow \mathbb F_{\st p}[x]$),   $GCD(\bm\zeta, \bm\gamma)=1$, polynomial $\bm\pi$ be an arbitrary polynomial,   $deg(\bm\zeta)= 1, deg(\bm\omega)=d,  deg(\bm\gamma)= 2d+1$,  $deg(\bm\pi)=d$, and $p$ be a $\lambda$-bit prime number. Also, let polynomial $\bm\theta$ be defined as  $\bm\theta=\bm\zeta\cdot \bm\omega\cdot\bm \pi+\bm \gamma \bmod p$. Given $(\bm\theta,\bm\pi)$, the probability that an adversary (which does not know $\bm\zeta, \bm\omega$, and $\bm\gamma$) can forge $\bm\theta$ to an arbitrary polynomial $\bm\delta$ such that  $\bm\delta\neq \bm\theta$, $deg(\bm\delta)= const(\lambda)$, and $\bm\zeta$ divides $\bm\delta-\bm\gamma$ is negligible in the security parameter $\lambda$, i.e., 
%
$Pr[ \bm\zeta \ | \ (\bm\delta-\bm\gamma) ]\leq \negl(\lambda)$.
%
\end{theorem}

\vs
\svs
\begin{proof}

Let $\bm\tau=\bm\delta-\bm\gamma$ and $\bm\zeta=a\cdot x+b$. Since $\bm\gamma$ is a random polynomial of degree $2d+1$ and unknown to the adversary, given $(\bm\theta, \bm\pi)$,  the adversary cannot learn anything about the factor $\bm\zeta$; as from its point of view every polynomial of degree $1$ in $\mathbb{F}_{\st p}[X]$ is equally likely to be $\bm\zeta$. Moreover,  polynomial $\bm\tau$ has at most $Max\big(deg(\bm\delta), 2d+1\big)$ irreducible non-constant factors.  For $\bm\zeta $ to divide $\bm\tau$,  one of the factors of $\bm\tau$ must be equal to $\bm\zeta$. We  also know that $\bm\zeta$ has been picked uniformly at random (i.e., $a,b
\stackrel{\st \$}\leftarrow \mathbb F_{\st p}$) and by definition $GCD(\bm\zeta, \bm\gamma)=1$. Thus, the probability that $\bm\zeta $ divides $\bm\tau$ is negligible in the security parameter, $\lambda$. Specifically, $Pr[ \bm\zeta \ | \ (\bm\delta-\bm\gamma)]\leq \frac{Max\big(deg(\bm\delta), 2d+1\big)} {2^{\st 2\lambda}}=\negl(\lambda)$. 
\hfill\(\Box\)\end{proof} 

%$Max\big(deg(\theta'), d+1\big)$
 An interesting feature of an unforgeable polynomial is that the verifier can perform the check without needing to know the original polynomial $\bm\pi$. Another feature of the unforgeable polynomial is that it supports \emph{linear combination} and accordingly \emph{batch verification}. Specifically, to turn $n$ arbitrary polynomials $[\bm\pi_{\st 1},..., \bm\pi_{\st n}]$ into unforgeable polynomials, one can construct  $\bm\theta_{\st i}=\bm\zeta\cdot \bm\omega_{\st i}\cdot \bm\pi_{\st i}+ \bm\gamma_{\st i} \bmod p$, where $\forall i, 1\leq i\leq n$.  
 
 

 
To check whether all polynomials $[\bm\theta_{\st 1},..., \bm\theta_{\st n}]$ are intact, a verifier can (i) compute their sum $\bm \chi=\sum\limits_{\st i=1}^{\st n}\bm\theta_{\st i}$ and (ii) check whether $\bm \chi- \sum\limits_{\st i=1}^{\st n}\bm\gamma_{\st i} $ is divisible by $\bm \zeta$.  Informally, the security of \emph{unforgeable polynomials' linear combination} states that an adversary (who does not know the three secret random polynomials for each $\bm\theta_{\st i}$) cannot tamper with any subset of the unforgeable polynomials without being detected, except with a negligible probability. We formally state it, below. 
 
 

%%%%%%%%%%%%%%%%%%%%%%%%%%%%%%%

\begin{theorem}[Unforgeable Polynomials' Linear Combination]\label{Unforgeable-Polynomials-Linear-Combination}
%
 Let polynomial $\bm\zeta$ be a secret polynomial picked uniformly at random; also, let   $\vv{\bm\omega}=[\bm\omega_{\st 1},..., \bm\omega_{\st n}]$ and $\vv{\bm\gamma}=[\bm\gamma_{\st 1},..., \bm\gamma_{\st n}]$ be two vectors of secret uniformly random polynomials (i.e., ${\bm\zeta}, \bm\omega_{\st i}, \bm\gamma_{\st i} \stackrel{\st\$}\leftarrow \mathbb F_{\st p}[x]$), $GCD(\bm\zeta,  \bm\gamma_{\st i})=1$,  $\vv{\bm\pi}=[\bm\pi_{\st 1},..., \bm\pi_{\st n}]$ be a vector of arbitrary polynomials,   $deg(\bm\zeta)= 1, deg(\bm\omega_{\st i})=d,  deg(\bm\gamma_{\st i})= 2d+1$,  $deg(\bm\pi_{\st i})=d$,  $p$ be a $\lambda$-bit prime number, and $1\leq i \leq n$. Moreover, let polynomial $\bm\theta_{\st i}$ be defined as  $\bm\theta_{\st i}=\bm\zeta\cdot \bm\omega_{\st i}\cdot \bm\pi_{\st i}+ \bm\gamma_{\st i} \bmod p$, and $\vv{\bm\theta} = [\bm\theta_{\st 1},..., \bm\theta_{\st n}]$.  Given $(\vv{\bm\theta}, \vv{\bm\pi})$, the probability that an adversary (which does not know $\bm\zeta, \vv{\bm\omega}$, and $\vv{\bm\gamma}$) can forge $t$ polynomials, without loss of generality, say $\bm\theta_{\st 1},..., \bm\theta_{\st t} \in \vv{\bm\theta}$ to arbitrary polynomials $\bm\delta_{\st 1},..., \bm\delta_{\st t}$ such that   $\sum\limits_{\st j=1}^{\st t}\bm\delta_{\st j}\neq \sum\limits_{\st j=1}^{\st t}\bm\theta_{\st j}$, $deg(\bm\delta_{\st j})= const(\lambda)$, and $\bm\zeta$ divides $(\sum\limits_{\st j=1}^{\st t}\bm\delta_{\st j} + \sum\limits_{\st j=t+1}^{\st n}\bm\theta_{\st j} - \sum\limits_{\st j=1}^{\st n}\bm\gamma_{\st j} )$ is negligible in the security parameter $\lambda$, i.e.,  
%
$Pr[ \bm\zeta \ | \ (\sum\limits_{\st j=1}^{\st t}\bm\delta_{\st j} + \sum\limits_{\st j=t+1}^{\st n}\bm\theta_{\st j} - \sum\limits_{\st j=1}^{\st n}\bm\gamma_{\st j} ) ]\leq \negl(\lambda)$.
%
\end{theorem}

\vs
\svs

%%%%
\begin{proof}  
This proof is a generalisation of that of Theorem \ref{proof::unforgeable-poly}.  
Let $\bm\tau_{\st j}=\bm\delta_{\st j}-\bm\gamma_{\st j}$ and $\bm\zeta=a\cdot x+b$. Since  every $\bm\gamma_{\st j}$ is a random polynomial of degree $2d+1$ and unknown to the adversary, given $(\vv{\bm\theta}, \vv{\bm\pi})$,  the adversary cannot learn anything about the factor $\bm\zeta$. Each polynomial $\bm\tau_{\st j}$ has at most $Max\big(deg(\bm\delta_{\st j}), 2d+1\big)$ irreducible non-constant factors. 
%
%In order for $\bm\zeta$ to divide polynomial $\sum\limits_{\st j=1}^{\st t}\bm\delta_{\st j} + \sum\limits_{\st j=t+1}^{\st n}\bm\theta_{\st j} - \sum\limits_{\st j=1}^{\st n}\bm\gamma_{\st j}$  one of the factors of every $\bm\tau_{\st j}$ needs to equal $\bm\zeta$, where $1 \leq j \leq t$. 
%
We  know that $\bm\zeta$ has been picked uniformly at random (i.e., $a,b
\stackrel{\st \$}\leftarrow \mathbb F_{\st p}$), by definition $GCD(\bm\zeta, \bm\gamma_{\st j})=1$, and $\bm\zeta$  does divide every $\bm\theta_{\st j}$. Therefore, the probability that $\bm\zeta$ divides $\sum\limits_{\st j=1}^{\st t}\bm\delta_{\st j} + \sum\limits_{\st j=t+1}^{\st n}\bm\theta_{\st j} - \sum\limits_{\st j=1}^{\st n}\bm\gamma_{\st j}$ equals the probability that $\bm\zeta$ equals to one of the factors of  every $\bm\tau_{\st j}$, that is negligible in the security parameter. Concretely,
%
$$Pr[ \bm\zeta \ | \ (\sum\limits_{\st j=1}^{\st t}\bm\delta_{\st j} + \sum\limits_{\st j=t+1}^{\st n}\bm\theta_{\st j} - \sum\limits_{\st j=1}^{\st n}\bm\gamma_{\st j} ) ]\leq  \frac{\prod \limits^{\st t}_{\st j=1}Max\big(deg(\bm\delta_{\st j}), 2d+1\big)} {2^{\st 2\lambda t}}=\negl(\lambda)$$
%
\hfill\(\Box\)
\end{proof} 

\svs

Briefly, in \withFai, we will use unforgeable polynomials (and their linear combinations) to allow a smart contract to efficiently check whether the polynomials that the clients send to it are intact, i.e., they are \vopr's outputs.










 \vspace{-2.3mm}
\section{\withFai: Concrete Construction of \p}

\vspace{-.3mm}
 
% %%%%%%%%%%%%%
%\subsection{Main Challenges to Overcome}
%
% We need to address several key challenges, to design an efficient scheme that realises \p. Below, we outline these challenges.
% 
%  \vspace{-1mm}
% \subsubsection{Keeping Overall Complexities Low.}
% 
% In general, in multi-party PSIs, each client needs to send messages to the rest of the clients and/or engage in secure computation with them, e.g., in \cite{DBLP:conf/scn/InbarOP18,DBLP:conf/ccs/KolesnikovMPRT17}, which would result in communication and/or computation quadratic with the number of clients. To address this challenge, we  (a) allow one of the clients as a dealer to interact with the rest of the clients,\footnote{This approach has similarity with the non-secure PSIs in \cite{GhoshN19}.} and   (b) we use a smart contract, which acts as a bulletin board to which most messages are sent and also performs lightweight computation on the clients' messages. The combination of these approaches will keep the overall communication and computation linear with the number of clients (and sets' cardinality). 
% 
% 
% \vspace{-2mm}
% 
% \subsubsection{Randomising Input Polynomials.}  In multi-party PSIs that use the polynomial representation, it is essential that a client's input polynomial be randomised by another client \cite{AbadiMZ21}. To do that securely and efficiently, we require the dealer and each client together to engage in an instance of \vopr, which we developed in Section \ref{sec::subroutines}. 
% 
%  \vspace{-2mm}
% 
% \subsubsection{Preserving the Privacy of Outgoing Messages.} Although the use of public smart contracts (e.g., Ethereum) will help keep overall complexity low, it introduces another challenge; namely, if clients do not protect the privacy of the messages they send to the smart contracts, then other clients (e.g., dealer) and non-participants of PSI (i.e., the public) can learn the clients' set elements and/or the intersection. To efficiently protect the privacy of each client's messages (sent to the contracts) from the dealer, we require the clients (except the dealer) to engage in \zspaa which lets each of them generate a pseudorandom polynomial with which it can blind its message. To protect the privacy of the intersection from the public, we require all clients to run a coin-tossing protocol to agree on a blinding polynomial, with which the final result that encodes the intersection on the smart contract will be blinded.  
% 
% 
%  \vspace{-2mm}
% \subsubsection{Ensuring the Correctness of Subroutine Protocols' Outputs.} 
% 
% In general, any MPC that must remain secure against an active adversary is equipped with a verification mechanism that ensures an adversary is detected if it affects messages' integrity, during the protocol's execution. This is the case for the subroutine protocols that we use, i.e., \vopr and \zspaa. However, this type of check itself is not always sufficient. Because in certain cases, the output of an MPC may be fed as input to another MPC and we need to ensure that the \emph{intact} output of the first MPC is fed to the second one. This is the case in our PSI's subroutines too. To address this challenge, we use unforgeable polynomials; specifically, the output of \vopr is an unforgeable polynomial (that encodes the actual output). If the adversary tampers with the \vopr's output and uses it later, then a verifier can detect it. We will have the same integrity guarantee for the output of \zspaa for free. Because (i) \vopr is called before \zspaa, and (ii) if clients use intact outputs of \zspaa, then the final result (i.e., the sum of all clients' messages) will not contain any output of \zspaa, as they will cancel out each other. Thus, by checking the correctness of the final result, one can ensure the correctness of the outputs of \vopr and \zspaa, in one go. 
% %%%%%%%%%%%%%%
 
 

\vspace{-2mm}
  
\subsection{An Overview of \withFai (\fpsi)}\label{Fair-PSI-Protocol}
%This section presents \fpsi, a protocol that realises \p. 

%\subsubsection{An Overview.} 

\vspace{-1mm}

At a very high level, \withFai (\fpsi) operates as follows. Initially, each client encodes its set elements into a polynomial. Subsequently, all clients collectively sign a smart contract \scf and make a predefined coin deposit into it. One of the clients, designated as the dealer, $D$, takes on the responsibility of randomising the polynomials of the remaining clients and applying a specific structure to their respective polynomials. The clients also randomise $D$'s polynomials. Following this, all clients transmit their randomised polynomials to \scf.  

 \scf consolidates all the polynomials and verifies whether the resultant polynomial still adheres to the structure imposed by $D$. If it determines that the resultant polynomial lacks the prescribed structure, it then triggers an auditor, \aud, to identify clients who may have acted improperly and impose penalties on them. However, if the resultant polynomial indeed maintains the required structure,  \scf produces an encoded polynomial and reimburses the deposits made by the clients. In this scenario, all clients can utilise the encoded polynomial (provided by \scf) to determine the intersection. Figure \ref{fig:parties-interactions-in-Jus} in Appendix \ref{sec::Justitia-workflow} outlines the interaction among the parties. 





%%%%%%%%%%%%


One of the novelties of \fpsi is its lightweight verification mechanism, which enables a smart contract  \scf to efficiently  validate the accuracy of clients' messages while preserving the confidentiality of the clients' set elements. To achieve this, $D$ randomises each client's polynomials and constructs unforgeable polynomials on the randomised polynomials.  If any client alters an unforgeable polynomial it receives and subsequently sends the modified polynomial to \scf,  then  \scf can detect this manipulation by verifying if the sum of all clients' (unforgeable) polynomials is divisible by a specific polynomial of degree $1$.  This verification process is considered lightweight for several reasons: (i) it does not use any public key cryptography, which can often be computationally expensive, (ii) it only requires polynomial division, and (iii) it can perform batch verification by aggregating all clients' randomised polynomials and then verifying the correctness of the result. 


%%%%%%%%%%%%




%
%One of the novelties of F-PS is a lightweight verification mechanism which allows a smart contract to efficiently verify the correctness of the clients' messages without being able to learn/reveal the clients' set elements. To achieve this, the dealer during randomising other clients' polynomials, imposes a MAC-like structure on the randomised polynomials, such that if a client (who receives its randomised polynomial) tampers with it, then the smart contract would detect it. To do the verification, the smart contract needs to only check whether the sum of all clients' randomised polynomials is divisible by a polynomial of degree $1$.  The verification is lightweight because: (i) it does not rely on any public key cryptography (i.e., zero-knowledge proofs), (ii) it needs to perform only polynomial division, and (iii) it can perform batch verification, i.e., instead of individually checking each client's randomised polynomial, it sums all clients randomised polynomials (related to a hash table's bin) and then checks the result's correctness.


% his own
%outsourced dataset and having any knowledge of the other client’s dataset 
%
%
% mainly stems from our observation (stated  in Theorem \ref{proof::unforgeable-poly}) which leads to an  efficient verification mechanism carried out by the contract. 


Now, we will delve into a more detailed description of \fpsi. Initially, all clients sign and deploy \scf, each contributing a predetermined deposit amount to it. Subsequently, they collectively execute a coin-tossing protocol \ct to reach a consensus on a key, $mk$. This key will be used to generate a series of blinding polynomials, concealing the final result from the public. 
%
Following this, each client independently maps its set elements to a hash table and represents the contents of each bin (in the hash table) as a polynomial, $\bm\pi$. 

Afterward, for each bin, the following steps are undertaken.  All clients, excluding $D$, participate in \zspaa to reach an agreement on a set of pseudorandom blinding factors, ensuring that their sum equals zero. 
%
Next, $D$ takes on the task of randomising each client's polynomial $\bm\pi$ and constructing an unforgeable polynomial on it. To achieve this, $D$ and every other client, $C$, participate in \vopr. This process results in
$C$ receiving a polynomial. Following this, $D$ and each $C$ invoke \vopr once more, this time to randomise $D$'s polynomial, resulting in $C$ obtaining another unforgeable polynomial. 
%
Note that the output of \vopr does not reveal any information about any client's original polynomial $\bm\pi$. This is because $D$  has previously concealed this polynomial with another secret random polynomial during the execution of
  \vopr. 
  
  Each $C$ adds the two polynomials together, applies blinding (using the output of  \zspaa), and then forwards the result to \scf. 
%
 Once all clients have transmitted their input polynomials to \scf, $D$ sends a \textbf{switching polynomial}  to \scf.  This switching polynomial allows \scf to obliviously alter the secret blinding polynomials previously employed by $D$ during the execution of \vopr. This alteration ensures that each client's original polynomial $\bm \pi$ is blinded with a different blinding polynomial, which can be independently generated and removed by the clients themselves using the key $mk$.  %The switching polynomial is constructed in a way that does not affect the verification of unforgeable polynomials. 

Following this, $D$ transmits a secret polynomial $\bm\zeta$ to \scf. This polynomial enables \scf to validate the correctness of unforgeable polynomials. Subsequently, \scf\ sum all clients' polynomials together and checks if $\bm\zeta$ can divide the sum. \scf approves the clients' inputs if the polynomial can divide the sum; otherwise, it invokes \aud to identify misbehaving parties.  


In case of misbehavior, all honest parties receive a refund of their deposits, while the deposits of the misbehaving parties are redistributed among the honest ones. If all clients behave honestly,  then each client can independently ascertain the intersection. To achieve this, they use $mk$ to remove the blinding polynomial from the sum (that the contract generated), evaluate the unblinded polynomial at each of their set elements, and consider an element to be part of the intersection if the evaluation yields zero. 
%The efficiency of the verification in our protocol  mainly stems from our  observation that if an adversary who know only $xx$ modified the polynomial of the form $xx$ then $\zeta$ will not divide result polynomial after unblinding will not divide with a high probability.  

\vspace{-3.5mm}
\subsection{Detailed Description of \fpsi.}
\vspace{-1mm}

 Now, we will provide an in-depth explanation of how \fpsi  operates. (For a description of the main notations used, please refer to Table \ref{table:notation-table} on page \pageref{table:notation-table}). 

\vspace{-2.5mm}
\begin{enumerate}[leftmargin=.55cm]

%\item[$\bullet$] \textbf{Input:} a pseudorandom function: $\mathtt{PRF}$, a hash table's parameters (i.e., the  total number of bins: $h$ and a bin's capacity: $d$), and clients' sets: $S^{\st (I)}$, where $I\in \bar{P}$.

%\item[$\bullet$] \textbf{Output:}  for every bin of the hash table, it outputs a polynomial: $\phi$, whose roots are  encrypted sets elements (of the bin) in the intersection.
\item\label{gen-FPSI-cont} All clients in $\cl=\{ A_{\st 1},...,   A_{\st m},  D\}$ sign a smart contract: \scf and deploy it to a blockchain. Each client obtains the address of the deployed contract. Also, all clients participate in \ct to agree on a secret master key, $mk$.

\item \label{encode-encrypt} Each client in $\cl$  builds a  hash table,  $\mathtt{HT}$, and inserts the set elements into  it, i.e.,  $\forall i: \mathtt{H}( s_{\st i})={indx}$, then $ s_{\st i}\rightarrow \mathtt{HT}_{\st indx}$. It pads every bin with random dummy elements to $d$ elements (if needed). Then,  for every bin, it constructs a polynomial whose roots are the bin's content: $\bm\pi=\prod\limits^{\st d}_{\st i=1} (x-s'_{\st i})$, where $s'_{\st i}$ is either $s_{\st i}$ or a random value. 
%
\item \label{ZSPA} Every client $ C$ in $\cl\setminus D$, for every bin, agree on $b=3d+3$ vectors of pseudorandom blinding factors: $z_{\st i,j}$, such that the sum of each vector elements is zero, i.e., $\sum\limits^{\st m}_{\st j=1}z_{\st i,j}=0$, where $0\leq i\leq b-1$. To do that, they participate in step \ref{ZSPA::ZSPA-invocation} of \zspaa. By the end of this step, for each bin, they agree on a secret key $k$ (that will be used to generate the zero-sum values) as well as two values stored in $\mathcal{SC}_{\fpsi}$, i.e., $q$: the key's hash value and $g$: a Merkle tree's root. After time $t_{\st 1}$,  $D$ ensures that all other clients have agreed on the vectors (i.e., all provided ``approved''  to the contract). If the check fails, it halts. 
%
\item\label{F-PSI::each-client-deposit} Each client in $\cl$ deposits $\yc+\chc$ amount to $\mathcal{SC}_{\fpsi}$. After time $t_{\st 2}$, $\mathcal{SC}_{\fpsi}$ checks if in total $(\yc+\chc)\cdot (m+1)$ amount has been deposited. If the check fails, it refunds the clients' deposit and halts. 







\item\label{JUS::check-non-zero-coeff}  $D$ picks a  random polynomial $\bm\zeta \stackrel{\st\$}\leftarrow \mathbb{F}_{\st p}[X]$ of degree $1$, for each bin.  
It, for each client $C$, allocates to each bin two random polynomials: $\bm\omega^{\st(D,C)}, \bm\rho^{\st (D,C)}\stackrel{\st\$}\leftarrow \mathbb{F}_{\st p}[X]$ of degree $d$, and  two  random polynomials: $\bm\gamma^{\st (D,C)}, \bm\delta^{\st (D,C)} \leftarrow \mathbb{F}_{\st p}[X]$ of degree $3d+1$. Also, each client $C$, for each bin, picks two  random polynomials: $\bm\omega^{\st (C,D)}, \bm\rho^{\st (C,D)}\stackrel{\st\$}\leftarrow \mathbb{F}_{\st p}[X]$ of degree $d$, and checks polynomials $\bm\omega^{\st (C,D)}\cdot \bm\pi^{\st  {  {(C)}}}$ and  $\bm\rho^{\st (C,D)}$ do not contain zero coefficients. Appendix \ref{sec::error-prob}, on page \pageref{sec::error-prob}, provides further discussion about this check. %It also evaluates each polynomial at every element of $\vv{\bm{x}}$ that results in  $\omega^{  {  {D,C}}}_{\st i}$ and $\rho^{  {  {D,C}}}_{\st i}$.




\item\label{e-psi::D-randomises}  $D$ randomises other clients' polynomials. To do so, for every bin, it invokes an instance of {\vopr} (presented in Fig. \ref{fig:VOPR}) with  each client $  C$; where  $D$ sends $\bm\zeta \cdot \bm\omega^{\st  {  {(D,C)}}}$ and $\bm\gamma^{\st  {  {(D,C)}}}$, while client $ C$ sends $\bm\omega^{\st  {  {(C,D)}}}\cdot \bm\pi^{\st  {  {(C)}}}$ to {\vopr}. Each client $C$, for every bin, receives a blinded polynomial of the following form: 

 \vspace{-3mm}
$$\bm\theta^{  {  {(C)}}}_{\st 1}=\bm\zeta \cdot \bm\omega^{\st  {  {(D,C)}}}\cdot \bm\omega^{\st  {  {(C,D)}}}\cdot \bm\pi^{\st  {  {(C)}}}+\bm\gamma^{\st  {  {(D,C)}}}$$

 \vspace{-1mm}

 from {\vopr}. If any party aborts, the deposit would be refunded to all parties.

\item\label{e-psi::C-randomises} Each client $    {  C}$ randomises  $ {D}$'s polynomial. To do that, each client $    {  C}$, for each bin,  invokes an instance of {\vopr} with   $ {D}$,    where each client $    {  C}$  sends $\bm\rho^{\st  {  {(C,D)}}}$, while  ${D}$  sends $\bm\zeta\cdot\bm \rho^{\st  {  {(D,C)}}}\cdot \bm\pi^{  {  {(D)}}}$ and $\bm\delta^{\st  {  {(D,C)}}}$ to {\vopr}. Every client   $    {  C}$, for each bin,  receives a blinded polynomial of the following form: 
%

 \vspace{-2.5mm}

$$\bm\theta^{  {  {(C)}}}_{\st 2}=\bm\zeta \cdot \bm\rho^{\st  {  {(D,C)}}}\cdot \bm\rho^{\st  {  {(C,D)}}}\cdot \bm\pi^{\st  {  {(D)}}}+\bm\delta^{\st  {  {(D,C)}}}$$

 \vspace{-1mm}
 
 from {\vopr}. If a party aborts, all parties get their deposit back. 


\item\label{blindPoly-C-sends-to-contract} Each client $ C$, for every bin, blinds the sum of polynomials $\bm\theta^{\st  {  {(C)}}}_{\st 1}$ and $\bm\theta^{\st  {  {(C)}}}_{\st 2}$  using the blinding factors: $z_{\st i,c}$, generated in step \ref{ZSPA}. Specifically, it computes the following blinded polynomial (for every bin):  
%
$\bm\nu^{ \st {  {(C)}} }= \bm\theta^{ \st {  {(C)}}}_{\st 1}+\bm\theta^{\st  {  {(C)}}}_{\st 2}+\bm \tau^{\st  {  {(C)}} }$, 
%
where $\bm\tau^{\st  {  {(C)}}}=\sum\limits^{\st 3d+2}_{\st i=0}z_{\st i,c}\cdot x^{\st i}$. Next, it sends  all $\bm\nu^{\st  {  {(C)}} }$ to $\mathcal{SC}_{\fpsi}$. If any party aborts, the deposit would be refunded to all parties.


%\item Client $    {  D}$ ensures all clients have sent their inputs to $\mathcal{SC}_{  {   {F-PSI}}}$. In the case where $m'$ parties do not provide their inputs, client $    {  D}$ aborts. In this case, the rest (including the dealer) get their deposit back. Also,  the deposit of the parties who did not send  inputs would be evenly distributed among the rest. The total amount each party above receives is: $y+\frac{m'\cdot y}{m-m'}$




%%
%\item Client $    {  D}$ and each client $    {  C}$ collaboratively, for each bin, generate a polynomial that will be used to (obliviously) check if  $    {  C}$ misbehaved during the computation of each $\bm\nu^{  {  {(C)}} }$. To do so, for every bin, client $    {  D}$ invokes an instance of $\mathtt{VOPR}$ with  each client $    {  C}$, where  client $    {  D}$ sends: $\bm\zeta$, while client $    {  C}$ sends $\bm\xi^{  {  {(C)}}}$ and $-\bm\tau^{  {  {(C)}}}$   to $\mathtt{VOPR}$, where $\bm\xi^{  {  {(C)}}}$ is a random polynomial of degree $3d+1$. Client $    {  D}$ for each  $    {  C}$'s bin recives the following polynomial: 
%%
%$$\bm\mu^{  {  {(D,C)}}} = \bm\zeta\cdot \bm\xi^{  {  {(C)}}}-\bm\tau^{  {  {(C)}}}$$
%%




\item\label{f-psi::D-gen-random-poly} ${D}$ ensures all clients sent their inputs to $\mathcal{SC}_{\fpsi}$. If the check fails, it halts and the deposit is  refunded to all parties. Otherwise, it allocates a pseudorandom polynomial $\bm\gamma'$ of degree $3d$, to each bin. To do so, it uses $mk$ to derive a key for each bin: $k_{\st  { {indx}}}=\mathtt{PRF}(mk, {    {   {indx}}})$ and  uses $k_{\st  { {indx}}}$ to generate $3d+1$ pseudorandom coefficients $g_{\st  { {j,indx}}}=\mathtt{PRF}(k_{\st  { {indx}}}, j)$ where $ 0\leq j \leq 3d$. Also, for each bin, it allocates a  random polynomial  $\bm\omega'^{\st  {  {(D)}}}$ of degree $d$. 

\item\label{f-psi::D-gen-switching-poly}  $ {D}$,  for every bin, computes a switching polynomial of the form:  
%
 \vspace{-4.1mm}

$$\bm\nu^{\st  {  {(D)}}}=\bm\zeta \cdot  \bm\omega'^{\st  {  {(D)}}}\cdot \bm\pi^{\st  {  {(D)}} }-\sum\limits^{\st  {   A}_{\st  {   m}}}_{\st   {  {C }= }   {   A}_{\st  {  1}}}(\bm\gamma^{\st  {  {(D,C)}}} + \bm\delta^{\st  {  {(D,C)}}}) + \bm\zeta \cdot \bm\gamma'$$ 

 \vspace{-2.2mm}
 
It sends to $\mathcal{SC}_{\fpsi}$  polynomials $\bm\nu^{\st  {  {(D)}}}$ and $\bm\zeta$, for each bin.

 \item\label{compute-res-poly}  $\mathcal{SC}_{\fpsi}$ takes the following steps:
 \begin{enumerate}[leftmargin=.7mm]
 \item for every bin, sums all related polynomials  provided by all clients in $\bar{P}$:
 %
 $\bm\phi= \bm\nu^{\st  {  {(D)}} }+\sum\limits^{\st  {   A}_{\st  {   m}}}_{\st   {  {C }= }   {   A}_{\st  {  1}}}\bm\nu^{\st  {  {(C)}} }
 = \\= \bm\zeta\cdot \bigg(\bm\omega'^{\st  {  {(D)}}}\cdot \bm\pi^{\st  {  {(D)}} } +\sum\limits^{\st  {   A}_{  {   m}}}_{\st  {  {C }= }   {   A}_{\st  {  1}}}(\bm\omega^{\st  {  {(D,C)}}} \cdot \bm\omega^{\st  {  {(C,D)}}}\cdot \bm\pi^{\st  {  (C)}}) +\bm\pi^{\st  {  {(D)}}}\cdot\sum\limits^{\st  {   A}_{ \st {   m}}}_{\st  {C= }   {   A}_{\st  {  1}}}(\bm\rho^{\st  {  {(D,C)}}} \cdot \bm\rho^{\st  {  {(C,D)}}}) + \bm\gamma'\bigg)$

 
 
% \begin{equation*}
%\begin{split}
% \bm\phi&= \bm\nu^{\st  {  {(D)}} }+\sum\limits^{\st  {   A}_{\st  {   m}}}_{\st   {  {C }= }   {   A}_{\st  {  1}}}\bm\nu^{\st  {  {(C)}} }\\
% &= \bm\zeta\cdot \bigg(\bm\omega'^{\st  {  {(D)}}}\cdot \bm\pi^{\st  {  {(D)}} } +\sum\limits^{\st  {   A}_{  {   m}}}_{\st  {  {C }= }   {   A}_{\st  {  1}}}(\bm\omega^{\st  {  {(D,C)}}} \cdot \bm\omega^{\st  {  {(C,D)}}}\cdot \bm\pi^{\st  {  (C)}}) +\bm\pi^{\st  {  {(D)}}}\cdot\sum\limits^{\st  {   A}_{ \st {   m}}}_{\st  {C= }   {   A}_{\st  {  1}}}(\bm\rho^{\st  {  {(D,C)}}} \cdot \bm\rho^{\st  {  {(C,D)}}}) + \bm\gamma'\bigg)
%  \end{split}
%\end{equation*}
% \item\label{F-PSI:detect-misbehaving-party} ensures that, for every bin, $\bm\zeta$ divides $\bm\phi$. Otherwise, it aborts and Arbiter protocol (presented in Fig. \ref{fig:arbiter}) is invoked to find misbehaving parties.
 
  \item\label{F-PSI:detect-misbehaving-party} checks whether, for every bin, $\bm\zeta$ divides $\bm\phi$. If the check passes, it sets $Flag=True$. Otherwise, it sets $Flag=False$. 
  
   %aborts and Arbiter protocol (presented in Fig. \ref{fig:arbiter}) is invoked to find misbehaving parties.
 
 
% \item if the check passes (i.e., $Flag=True$), each party gets back its deposit (i.e., $y$ amount).
 \end{enumerate}
 
\item\label{F-PSI::flag-is-true} If $Flag=True$, then the following steps are taken:

\begin{enumerate}[leftmargin=.7mm]
 \item $\mathcal{SC}_{\fpsi}$ sends back each party's deposit, i.e., $\yc+\chc$ amount.
 
  \item each client (given $\bm\zeta$ and $mk$) finds the elements in the intersection as follows. 
  \begin{enumerate}
  \item derives a bin's pseudorandom polynomial, $\bm\gamma'$, from $mk$. 
  
  \item removes the blinding polynomial from each bin's polynomial: 
  \vspace{-.5mm}
  $$\bm\phi'=\bm\phi-\bm\zeta\cdot \bm\gamma'$$
  
  \item\label{F-PSI::find-intersection} evaluates each bin's unblinded polynomial at every element $s_{\st i}$ belonging to that bin and considers the element in the intersection if the evaluation is zero: i.e., $\bm\phi'(s_{\st i})=0$.
 
 \end{enumerate}
 
 
 \end{enumerate}
 
 \item\label{F-PSI::flag-is-false} If $Flag=False$, then the following steps are taken.
 
 

 
 \begin{enumerate}[leftmargin=.7mm]
 

 \item\label{auditor}  \aud asks every client $    {  C}$ to send to it the  $\mathtt{PRF}$'s key (generated in step \ref{ZSPA}), for every bin. It inserts the keys to $\vv k$.  It generates a list $\bar L$ initially empty. Then, for every bin,  \aud takes step \ref{ZSPA-A::Auditor-computation} of \zspaa, i.e., invokes  $\mathtt{Audit}(m, \vv{{k}},  q, \bm\zeta, 3d+3, g)\rightarrow (L, \vv{{\mu}})$.  Every time it invokes $\mathtt{Audit}$, it appends the elements of returned $L$ to $\bar L$.  \aud for each bin sends  $ \vv{{\mu}}$ to $\mathcal{SC}_{\fpsi}$. It also sends  to $\mathcal{SC}_{\fpsi}$ the list $\bar L$ of all misbehaving clients detected so far. The above check allows \aud to identify clients who have misbehaved in \zspaa. 
 

 
 \item to  help identify further  misbehaving clients, $D$ takes the following steps,  for each bin of client $    {  C}$ whose ID is not in $\bar L$.   
 \begin{enumerate}
 \item\label{gen-unmaking-poly} computes polynomial $\bm\chi^{  {  {(D, C)}}}$ as follows. 
 %
 \vspace{-.1mm}
 %
 $$\bm\chi^{ \st {  {(D, C)}}}=\bm\zeta\cdot \bm\eta^{ \st {  {(D,C)}}}-(\bm\gamma^{ \st {  {(D,C)}}}+\bm\delta^{ \st {  {(D,C)}}})$$
 
 \vspace{-.3mm}
 
  where $\bm\eta^{ \st {  {(D,C)}}}$ is a fresh random polynomial of degree $3d+1$. 
  
  \item\label{send-unmaking-poly} sends  polynomial $\bm\chi^{ \st {  {(D, C)}}}$ to  $\mathcal{SC}_{\fpsi}$. 
  

 \end{enumerate}
  Note, if $\bar L$ contains all clients' IDs, then $D$ does not need to take the above steps \ref{gen-unmaking-poly} and \ref{send-unmaking-poly}. 
 %%%%%%%%%%%%%%%%%%%%%%
 
 \item  $\mathcal{SC}_{\fpsi}$,   takes the following steps to check if the client misbehaved,  for each bin of client $    {  C}$ whose ID is not in $\bar L$.
 
 %for each client $    {  C}$'s bin, takes the following steps to check if the client misbehaved.
 
  \begin{enumerate}
  
 \item computes  polynomial $\bm\iota^{\st  {  {(C)}}}$ as follows: 
 
 $\bm\iota^{ \st {  {(C)}}}=\bm\chi^{\st  {  {(D, C)}}}+\bm\nu^{\st  {  {(C)}}} +\bm\mu^{ \st {  {(C)}}}  
 =\\ = \bm\zeta\cdot(\bm\eta^{ \st {  {(D,C)}}} + \bm\omega^{ \st {  {(D,C)}}}\cdot \bm\omega^{ \st {  {(C,D)}}}\cdot \bm\pi^{ \st {  {(C)}}}+\bm\rho^{ \st {  {(D,C)}}}\cdot \bm\rho^{ \st {  {(C,D)}}}\cdot \bm\pi^{ \st {  {(D)}}}+\bm\xi^{ \st {  {(C)}}})$\\ 
%
%  \begin{equation*}
%\begin{split}
% \bm\iota^{ \st {  {(C)}}}&=\bm\chi^{\st  {  {(D, C)}}}+\bm\nu^{\st  {  {(C)}}} +\bm\mu^{ \st {  {(C)}}} \\ 
% &=\bm\zeta\cdot(\bm\eta^{ \st {  {(D,C)}}} + \bm\omega^{ \st {  {(D,C)}}}\cdot \bm\omega^{ \st {  {(C,D)}}}\cdot \bm\pi^{ \st {  {(C)}}}+\bm\rho^{ \st {  {(D,C)}}}\cdot \bm\rho^{ \st {  {(C,D)}}}\cdot \bm\pi^{ \st {  {(D)}}}+\bm\xi^{ \st {  {(C)}}})
% \end{split}
%\end{equation*}
%
 where $\bm\mu^{ \st {  {(C)}}} \in \vv{\mu}$ was sent to $\mathcal{SC}_{\fpsi}$  by \aud in step \ref{auditor}.   
  \item checks if $\bm\zeta$  divides $\bm\iota^{ \st {  {(C)}}}$. If the check fails, it appends the client's ID to  a list $ L'$.
  %
  \end{enumerate}
   If $\bar L$ contains all clients' IDs, then $\mathcal{SC}_{\fpsi}$ does not take the above two steps. 

 %
   \item  $\mathcal{SC}_{\fpsi}$  refunds the honest parties' deposit. Also, it retrieves the total amount of  $\chc$ from the deposit of dishonest clients (i.e., those clients whose IDs are in $\bar L$ or $L'$) and sends it to \aud.  It also splits the remaining deposit of the misbehaving parties among the honest ones. Thus, each honest client  receives $\yc+\chc+\frac{m'\cdot (\yc+\chc)-\chc}{m-m'}$ amount in total, where $m'$ is the total number of misbehaving parties.
 
 
%  \item  $\mathcal{SC}_{  {   {F-PSI}}}$  refunds the honest parties' deposit and splits the deposit of the misbehaving parties (i.e., those clients whose IDs are in $\bar L$ or $L'$)  among the honest ones. Thus, each honest party would receive $y+\frac{m'\cdot y}{m-m'}$ amount in total, where $m'$ is the total number of misbehaving parties.
 %%%%%%%%%%%%%%%%%%%%%
  \end{enumerate}
  
% \item If  $Flag=False$,  then $\mathcal{SC}_{  {   {F-PSI}}}$,  for each client $    {  C}$'s bin, takes the following steps:
% 
%  \begin{enumerate}
%  
% \item computes the following polynomial: 
% 
%  \begin{equation*}
%\begin{split}
% \bm\iota^{  {  {(C)}}}&=\bm\chi^{  {  {(D, C)}}}+\bm\nu^{  {  {(C)}}} \\ 
% &=\bm\zeta\cdot(\bm\eta^{  {  {(D,C)}}} + \bm\omega^{  {  {(D,C)}}}\cdot \bm\omega^{  {  {(C,D)}}}\cdot \bm\pi^{  {  {(C)}}}+\bm\rho^{  {  {(D,C)}}}\cdot \bm\rho^{  {  {(C,D)}}}\cdot \bm\pi^{  {  {(D)}}}+\bm\xi^{  {  {(C)}}})
% \end{split}
%\end{equation*}
% 
%  \item checks if $\bm\zeta$  divides $\bm\iota^{  {  {(C)}}}$. If does not, it appends the client's ID to a  list, $L$.
%  
%  \end{enumerate}
 
% \begin{equation*}
%\begin{split}
% \bm\iota^{  {  {(C)}}}&=\bm\zeta\cdot \bm\eta^{  {  {(D,C)}}}-(\bm\gamma^{  {  {(D,C)}}}+\bm\delta^{  {  {(D,C)}}})+\bm\nu^{  {  {(C)}}}-\sum\limits^{\st 3d+1}_{\st i=0}z_{\st i,c}\cdot x^{\st i} \\ &=\bm\zeta\cdot(\bm\eta^{  {  {(D,C)}}} + \bm\omega^{  {  {(D,C)}}}\cdot \bm\omega^{  {  {(C,D)}}}\cdot \bm\pi^{  {  {(C)}}}+\bm\rho^{  {  {(D,C)}}}\cdot \bm\rho^{  {  {(C,D)}}}\cdot \bm\pi^{  {  {(D)}}})
% \end{split}
%\end{equation*}
%  \item checks if $\bm\zeta$ can divide $\bm\iota^{  {  {(C)}}}$. If can not, it appends the client's ID to $L$.
% 
 

 
 
 
% \item Each client (given $\bm\zeta$ and $k_{\st 1}$), finds the elements in the intersection as follows. First, it derives a bin's pseudorandom polynomial: $\bm\gamma'$ from $k_{\st 1}$.  Next, it removes the blinding polynomial from each bin's polynomial: $\bm\phi'=\bm\phi-\bm\zeta\cdot \bm\gamma'$. Then, it evaluates each bin's unblinded polynomial at every  element belonging to that bin and considers the element in the intersection if the evaluation is zero: i.e. $\bm\phi'(s^{  {  {(I)}}}_{\st i})=0$
 
  \end{enumerate}
 
% the result: $cccc$ by locally evaluating the result polynomial: $\phi(x)$, at every  encrypted element, $e^{  {  {(I)}}}_{\st i}$, it has and considering the elements in the intersection if the following equation holds.  $\forall i, 1\leq i\leq d: \phi(e^{  {  {(I)}} }_{\st i})-\zeta(e^{  {  {(I)}}}_{\st i})\cdot \gamma'(e^{  {  {(I)}}}_{\st i})=0$.
 
 
 
%\begin{remark} After the Arbiter detects  misbehaving parties,  in step \ref{F-PSI:detect-misbehaving-party},  it sends their ID's to $\mathcal{SC}_{  {   {F-PSI}}}$ which refunds the honest parties' deposit and splits the misbehaving parties' deposit among the honest ones. Thus, each honest party would receive: $y+\frac{m'\cdot y}{m-m'}$ amount in total, where $m'$ is the total number of misbehaving parties.
% \end{remark}
 
 

%\begin{remark}

We refer readers to Appendix \ref{sec::Discussion-justitia} for a more in-depth discussion of $\withFai$ which encompasses (i) the inadequacy of strawman approaches such as using a server-aided PSI or charging the buyer a flat fee, (ii) instructions on how to eliminate the check in step \ref{JUS::check-non-zero-coeff}, (iii) an outline of the primary challenges we addressed during $\withFai$ design, and (iv) choice of concrete (security) parameters.  

Next, we present a theorem that formally states the security of \fpsi. 
 
 \vspace{-2mm}
 
 \begin{theorem}\label{theorem::F-PSI-security}
Let polynomials $\bm\zeta$, $\bm\omega$, and $\bm\gamma$ be three secret uniformly random polynomials. If  $\bm\theta=\bm\zeta\cdot \bm\omega\cdot\bm \pi+\bm \gamma \bmod p$ is an unforgeable polynomial (w.r.t. Theorem \ref{proof::unforgeable-poly}), \zspaa, \vopr,  $\mathtt{PRF}$, and smart contracts are secure, then \fpsi securely realises  $f^{\st \text{PSI}}$ with $Q$-fairness (w.r.t. Definition \ref{def::PSI-Q-fair}) in the presence of $m-1$  active-adversary clients (i.e., $A_{\st j}$s) or a passive dealer client, passive auditor, or passive public. 
 \end{theorem}
 

 \vspace{-2.3mm}

We refer readers to Appendix \ref{sec::F-PSI-proof} for the proof of Theorem \ref{theorem::F-PSI-security}. 



 
 