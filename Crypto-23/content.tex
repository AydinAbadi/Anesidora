% !TEX root =main.tex







\section{Introduction}


Secure Multi-Party Computation (MPC)  allows multiple mutually distrustful parties to jointly compute a certain functionality on their private inputs without revealing anything beyond the result. Private Set Intersection (PSI) is a subclass of MPC that aims to efficiently achieve the same security property as MPC does. %lets parties compute the intersection of their private sets without  revealing anything beyond the sets' intersection. 
%
PSI has numerous applications. For instance, it has  been used  in Vertical Federated Learning (VFL) \cite{LuD20}, COVID-19 contact tracing schemes  \cite{DBLP:conf/asiacrypt/DuongPT20},  remote diagnostics \cite{BrickellPSW07}, and finding leaked credentials \cite{ThomasPYRKIBPPB19}. %Apple's child safety solution to combat ``Child Sexual Abuse Material''  \cite{Apple-PSI}. %PSI has been considered by the ``Financial Action Task Force'' (FATF) as one of the vital tools for enabling collaborative analytics between financial institutions to  strengthen ``Anti-Money Laundering'' (AML)  
%and ``Countering the Financing of Terrorism'' (CFT) compliance
 %\cite{FATF}. 



There exist two facts about PSIs: (i) a non-empty result always reveals something about the parties' private input sets (i.e., the set elements that are in the intersection), and (ii) various variants of PSIs do not output the result to all parties, even in those PSIs that do,  not all of the parties are necessarily interested in it.  Given these facts, one may ask a natural question:  



%\begin{center}
%\emph{Why do the parties that do not receive the result or are not interested in it participate in a PSI which would ultimately reveal some information about their private inputs?}
%\end{center}

\begin{center}
\emph{How can we incentivise the parties that do not receive the result or are not interested in it to participate in a PSI which would ultimately reveal some information about their private inputs?}
\end{center}



To date, the literature has not answered the above question. The literature has assumed that all parties will participate in a PSI for free and bear the privacy cost (in addition to computation and computation overheads imposed by the PSI).  %not be to help the result recipient compute the intersection.  
%
In this work, for the first time, we answer the above question. We present a multi-party PSI, called ``\withRew'', that allows a buyer who initiates the PSI computation (and is interested in the result) to pay other parties proportionate to the number of elements it learns about other parties' private inputs.  \withRew is efficient and mainly relies on symmetric key primitives.  Its computation and communication complexities are linear with the number of parties and set cardinality. \withRew remains secure even if the majority of parties are corrupt by active adversaries which may collude with each other. 




%Smart-PSI is  mainly based on symmetric key primitives, modified F-PSI as well as a game theory based approach. The latter approach is leveraged to create tension, betrayal and distrust    between the clients (who  want to collude with the buyer to increase their shares) and buyer (who wants to pay less). This is the first time a game theory based approach utilised in a PSI protocol.



%In this paper,  we provide the first efficient \emph{multi-party fair} PSI protocol (F-PSI).  It allows either all clients to get the result or if the protocol aborts in an unfair manner (where only dishonest parties learn the result), then honest parties will be financially compensated. The protocol is mainly based on symmetric key primitives and a smart contract. This is the first time a smart contract is used in a PSI protocol.



%Moreover, we provide another PSI protocol (E-PSI)  that  allows a buyer  who initiates the PSI computation (and interested in the result) to pay other parties in a fair manner, where the amount each party receives is proportional to the number of elements the buyer learns about their inputs.  Smart-PSI is  mainly based on symmetric key primitives, modified F-PSI as well as a game theory based approach. The latter approach is leveraged to create tension, betrayal and distrust    between the clients (who  want to collude with the buyer to increase their shares) and buyer (who wants to pay less). This is the first time a game theory based approach utilised in a PSI protocol.



We develop \withRew in a modular fashion. Specifically, first, we propose the formal notion of ``PSI with Fair Compensation'' (\p) and devise the first construction, called ``\withFai'', that realises the notion. \p ensures that either all parties get the result or if the protocol aborts in an unfair manner (where only dishonest parties learn the result), then honest parties will receive financial compensation, i.e., adversaries are penalised. Next, we enhance \p to the notion of ``PSI with Fair Compensation and Reward'' (\ep) and develop \withRew that realises \ep. The latter notion ensures that honest parties (a) are rewarded regardless of whether all parties are honest, or a set of them aborts in an unfair manner and (b) are compensated in the case of an unfair abort. We formally prove the two PSIs using the simulation-based model. To devise efficient PSIs, we have developed a primitive, called ``unforgeable polynomial'' that might be of independent interest. 

A PSI, like Anesidora, that supports more than two parties and rewards set contributors can create opportunities for much richer analytics and incentivise parties to participate. It can be used (1) by an advertiser who wants to conduct advertisements targeted at certain customers by first finding their common shopping patterns distributed across different e-commerce companies' databases \cite{IonKNPSS0SY20}, (2) by a malware detection service that allows a party to send a query to a collection of malware databases held by independent antivirus companies to find out whether all of them consider a certain application as malware \cite{TamrakarLPEPA17}, or (3) by a bank, like ``WeBank'', that uses VFL and PSI to gather information about certain customers from various partners (e.g., national electronic invoice and other financial institutions) to improve its risk management of loans \cite{ChengLCY20}. In all these cases, the set contributors will be rewarded by such a PSI. 



We hope that our work initiates future research on developing reward mechanisms for participants of \emph{generic MPC}, as well. Such reward mechanisms have the potential to increase MPC's real-world adoption.  

\begin{paragraph}
%
{\textbf{Our Contributions Summary.}} In this work, we: (1) devise \withRew, the first PSI that lets participants receive a reward for contributing their set elements to the intersection, (2) develop \withFai, the first PSI that lets either all parties receive the result or if the protocol aborts in an unfair manner,  honest parties receive compensation, and (3) propose formal definitions of the above constructions.
%
\end{paragraph}


%We develop the PSI that rewards participants in a modular fashion; specifically, first, we propose (i) the notion of ``PSI with fair compensation'' (\p) and (ii) devise a construction, F-PSI, that realises the notion. \p ensures that either all clients get the result or if the protocol aborts in an unfair manner (where only dishonest parties learn the result), then honest parties will be financially compensated, i.e., adversaries are penalised. Then, we enhance \p to the notion of ``PSI with fair compensation and reward'' (\ep) and develop a construction that realises \ep. The latter notion ensures that honest parties are (1) rewarded regardless of whether all parties are honest or a set of them aborts in an unfair manner and (2) compensated in the case of an unfair abort. 

% !TEX root =main.tex


\vspace{-2.5mm}
\section{Related Work}\label{sec::related-work}


Since their introduction in \cite{DBLP:conf/eurocrypt/FreedmanNP04}, various PSIs have been designed. PSIs can be divided into \textit{traditional} and \textit{delegated} ones.  
%
In \textit{traditional} PSIs, data owners interactively compute the result using their local data. 
%
Recently, Raghuraman and Rindal \cite{RaghuramanR22} proposed two two-party PSIs, one secure against semi-honest/passive and the other against malicious/active adversaries. To date, these two protocols are the fastest two-party PSIs. They mainly rely on  Oblivious Key-Value Stores (OKVS) data structure and Vector Oblivious Linear Evaluation (VOLE). The protocols' computation cost is $O(c)$, where $c$ is  a set's cardinality.  They also impose $O(c\log c^{\st 2}+\kappa)$ and $O(c\cdot \kappa)$ communication costs in the semi-honest and malicious models respectively, 
where 
%$l$ is a set element's bit-size, and  
$\kappa$ is a security parameter.  
%
Also, researchers designed PSIs that  let multiple (i.e., more than two) parties efficiently compute the intersection. The multi-party PSIs in  \cite{DBLP:conf/scn/InbarOP18,DBLP:conf/ccs/KolesnikovMPRT17} are secure against  passive adversaries while those in \cite{Ben-EfraimNOP21,GhoshN19,ZhangLLJL19,DBLP:conf/ccs/KolesnikovMPRT17,NevoTY21} were designed to remain secure against  active ones. Abadi \et  \cite{AbadiMZ21} showed that the PSIs in  \cite{GhoshN19} are susceptible to several attacks.  To date, the  protocols  in   \cite{DBLP:conf/ccs/KolesnikovMPRT17} and  \cite{NevoTY21} are the most  efficient multi-party PSIs  designed to be  secure against passive and active  adversaries respectively. These protocols are secure even if  the majority of parties are corrupt.  
%
%The PSIs in  \cite{DBLP:conf/ccs/KolesnikovMPRT17,NevoTY21} to keep the overall costs low and avoid requiring all clients to interact with each other in all steps of the protocols, use a ``leader'' client which interacts with the rest of the clients. 
%
%The former relies on inexpensive symmetric key primitives such as  Programmable Pseudorandom Function (OPPRF) and Cuckoo Hashing, while the latter mainly uses OPPRF and OKVS. 

The overall computation and communication complexities of the PSI in  \cite{DBLP:conf/ccs/KolesnikovMPRT17} are  $O(c\cdot m^{\st 2}+c\cdot m )$ and $O(c\cdot m^{\st 2})$. Later, to achieve efficiency, Chandran \et \cite{ChandranD0OSS21} proposed a multi-party PSI that remains secure only if the minority of the parties is corrupt by a semi-honest adversary. The PSI in \cite{NevoTY21} has a parameter $t$ that determines how many parties can collude with each other and must be set before the protocol's execution, where $t\in [2, m)$.  
%
%The protocol divides the parties into three groups, clients: $A_{\st 1},..A_{\st m-t-1}$, leader: $A_{\st m-t}$, and servers: $A_{\st m-t+1},..A_{\st m}$. Each client needs to send a set of messages to every server and the leader which jointly compute the final result. Hence, 
%
Its computation and communication complexities are $O(c\cdot \kappa(m+t^{\st 2}-t(m+1)))$ and $O(c\cdot m\cdot \kappa)$ respectively.


Dong \et proposed a ``fair'' two-party PSIs \cite{DBLP:conf/dbsec/DongCCR13} that ensure either both parties receive the result or neither does, even if a malicious party aborts prematurely during the protocol's execution. It uses homomorphic encryption,  zero-knowledge proofs, and polynomial representation of sets. The protocol's  computation and communication complexities are $O(c^{\st 2})$ and $O(c)$  respectively. Since then, various fair two-party PSIs have been proposed, e.g.,  in \cite{DebnathD14,DebnathD16-,DebnathD16}. To date, the fair PSI in \cite{DebnathD16} has better complexity and performance compared to previous fair PSIs. It mainly uses  ElGamal encryption, verifiable encryption, and zero-knowledge proofs. The protocol's computation and communication cost is $O(c)$. But, its overall overhead is still high, as it relies on asymmetric key primitives, e.g.,  zero-knowledge proofs. So far, there exists no fair \emph{multi-party} PSI in the literature. Our \withFai is the first  fair multi-party PSI.%, which is also efficient.  


 \textit{Delegated} PSIs use  cloud computing  for computation and/or storage while preserving the privacy of  the computation inputs and outputs from the cloud. They can be divided further into protocols that support \textit{one-off} and \textit{repeated} delegation of PSI computation. The former like \cite{kamarascaling,kerschbaum12,c18} cannot reuse their outsourced encrypted data and require clients to re-encode their data locally for each computation. The most efficient such protocol is \cite{kamarascaling}, which has been designed for the two-party setting and its computation and communication complexity is $O(c)$.  In contrast, those protocols that support repeated PSI delegation let clients outsource the storage of their encrypted data to the cloud only once, and then execute an unlimited number of computations on the outsourced data. 
 %
 
The protocol in \cite{eopsi} is the first PSI that efficiently supports repeated delegation in the semi-honest model. It uses the polynomial representation of sets, pseudorandom function, and hash table. Its communication and computation complexities are $O(h\cdot d^{\st 2})$ and $O(h\cdot d)$ respectively, where $h$ is the total number of bins in the hash table, $d$ is a bin's capacity (often $d=100$), and $h\cdot d$ is linear with $c$.  
%
Recently, a multi-party PSI that supports repeated delegation and efficient \emph{updates} has been proposed in \cite{AbadiDMT22}. It is also in the semi-honest model and uses a pseudorandom function, hash table, and Bloom filters. It imposes $O(h\cdot d^{\st 2}\cdot m)$ and $O(h\cdot d\cdot m)$  computation and communication costs respectively, during the PSI computation. It also imposes $O(d^{\st 2})$  computation and communication overheads, during the update phase.  Its runtime, during the PSI computation, is up to two times faster than the  PSI in \cite{eopsi}. It is yet to be seen how a fair delegated (two/multi-party) PSI can be designed.










 
 
 
% !TEX root =main.tex

\vspace{-5mm}

\section{Preliminaries}


\vspace{-3mm}
% !TEX root =main.tex


\subsection{Notation Table}\label{sec::notation-table}
Table \ref{commu-breakdown-party} summarises the main notations used in the paper. 

\vspace{-3mm}

\begin{table}[!h]
\begin{scriptsize}
\begin{center}
\footnotesize{
\vspace{-6mm}
\caption{ \small{Notation Table}.}\label{commu-breakdown-party} 
\renewcommand{\arraystretch}{.9}
\scalebox{0.65}{
% 1st table
\begin{tabular}{|c|c|c|c|c|c|c|c|c|c|c|c|c|c|} 

%\hline 
%\multicolumn{1}{|c|}{\cellcolor{yellow!10}\scriptsize{Setting}}&\cellcolor{yellow!10} \scriptsize{Symbol}&\cellcolor{yellow!10} \scriptsize{Description}\\
%\hline 

\hline 
\multicolumn{1}{|c|}{\rotatebox[origin=c]{45}{\cellcolor{yellow!10}\scriptsize{ {Setting}}}}&\cellcolor{yellow!10} \scriptsize{Symbol}&\cellcolor{yellow!10} \scriptsize{Description}\\
\hline 

%{\rotatebox[origin=c]{90}{\scriptsize{ {Contracts}}}}

%%%%%%%%%. Generic %%%%%

\cellcolor{yellow!10}&\cellcolor{gray!20}\scriptsize$p$&\cellcolor{gray!20}\scriptsize \text{Large prime number}\\   
%
\cellcolor{yellow!10}&\cellcolor{white!20}\scriptsize{$\mathbb{F}_{\st p}$}&\cellcolor{white!20}\scriptsize \text{A finite field of prime order $p$}\\   
%
\cellcolor{yellow!10}&\cellcolor{gray!20}\scriptsize$\cl$&\cellcolor{gray!20}\scriptsize \text{Set of all clients, $\{ A_{\st 1},...,   A_{\st m},  D\}$ }\\   
%
\cellcolor{yellow!10}&\cellcolor{white!20}\scriptsize$D$&\cellcolor{white!20}\scriptsize \text{Dealer client}\\   
%
\cellcolor{yellow!10}&\cellcolor{gray!20}\scriptsize$A_{\st m}$&\cellcolor{gray!20}\scriptsize{Buyer client}\\ 
%
\cellcolor{yellow!10}&\cellcolor{white!20}\scriptsize$m$&\cellcolor{white!20}\scriptsize \text{Total number of clients (excluding $D$)}\\   
%
\cellcolor{yellow!10}&\cellcolor{gray!20}\scriptsize$\mathtt{H}$&\cellcolor{gray!20}\scriptsize \text{Hash function}\\ 
%
\cellcolor{yellow!10}&\cellcolor{white!20}\scriptsize$|S_{\st\cap}|$&\cellcolor{white!20}\scriptsize{Intersection size}\\ 
%
\cellcolor{yellow!10}&\cellcolor{gray!20}\scriptsize$\Smin$&\cellcolor{gray!20}\scriptsize{Smallest set's size}\\
%
\cellcolor{yellow!10}&\cellcolor{white!20}\scriptsize$\Smax$&\cellcolor{white!20}\scriptsize{Largest set's size}\\
%
\cellcolor{yellow!10}&\cellcolor{gray!20}\scriptsize$|$&\cellcolor{gray!20}\scriptsize{Divisible}\\
%
\cellcolor{yellow!10}&\cellcolor{white!20}\scriptsize$\setminus$&\cellcolor{white!20}\scriptsize{Set subtraction}\\
%
\cellcolor{yellow!10}&\cellcolor{gray!20}\scriptsize$c$&\cellcolor{gray!20}\scriptsize{Set's cardinality}\\ 
 %
 \cellcolor{yellow!10}&\cellcolor{white!20}\scriptsize$h$&\cellcolor{white!20}\scriptsize{Total number of bins in a hash table}\\ 
 %
 \cellcolor{yellow!10}&\cellcolor{gray!20}\scriptsize$d$&\cellcolor{gray!20}\scriptsize{A bin's capacity}\\ 
 %
\cellcolor{yellow!10}&\cellcolor{white!20}\scriptsize$\lambda$ &\cellcolor{white!20}\scriptsize Security parameter  \\  

%
\cellcolor{yellow!10}&\cellcolor{gray!20}\scriptsize \ole&\cellcolor{gray!20}\scriptsize{Oblivious Linear Evaluation}\\ 
%
\cellcolor{yellow!10}&\cellcolor{white!20}\scriptsize$\ole^{\st +}$&\cellcolor{white!20}\scriptsize{Advanced \ole}\\ 

\cellcolor{yellow!10}&\cellcolor{gray!20}\scriptsize$\comcom$&\cellcolor{gray!20}\scriptsize \text{Commitment algorithm of commitment}\\ 

\cellcolor{yellow!10}&\cellcolor{white!20}\scriptsize$\comver$&\cellcolor{white!20}\scriptsize \text{Verification algorithm of commitment}\\ 

\cellcolor{yellow!10}&\cellcolor{gray!20}\scriptsize$\mkgen$&\cellcolor{gray!20}\scriptsize \text{Tree construction algorithm of Merkle tree}\\ 

\cellcolor{yellow!10}&\cellcolor{white!20}\scriptsize$\mkprove$&\cellcolor{white!20}\scriptsize \text{Proof generation algorithm of Merkle tree}\\ 

\cellcolor{yellow!10}&\cellcolor{gray!20}\scriptsize$\mkver$&\cellcolor{gray!20}\scriptsize \text{Verification algorithm of Merkle tree}\\ 

\cellcolor{yellow!10}&\cellcolor{white!20}\scriptsize{\ct}&\cellcolor{white!20}\scriptsize \text{Coin tossing protocol}\\  

 \cellcolor{yellow!10}&\cellcolor{gray!20}\scriptsize{\vopr}&\cellcolor{gray!20}\scriptsize \text{Verifiable Oblivious Poly. Randomization}\\
 
 \cellcolor{yellow!10} &\cellcolor{white!20}\scriptsize{\zspa}&\cellcolor{white!20}\scriptsize \text{ Zero-sum Pseudorandom Values Agreement}\\
  
  \cellcolor{yellow!10}   &\cellcolor{gray!20}\scriptsize{\zspaa}&\cellcolor{gray!20}\scriptsize \text{\zspa with an External Auditor}\\

   \cellcolor{yellow!10}  &\cellcolor{white!20}\scriptsize{\p}&\cellcolor{white!20}\scriptsize \text{Multi-party PSI with Fair Compensation}\\
     
    \cellcolor{yellow!10}      &\cellcolor{gray!20}\scriptsize{\ep}&\cellcolor{gray!20}\scriptsize \text{Multi-party PSI with Fair Compensation and Reward}\\

  \cellcolor{yellow!10}   &\cellcolor{white!20}\scriptsize{\fpsi}&\cellcolor{white!20}\scriptsize \text{Protocol that realises \p}\\
     
     
  \cellcolor{yellow!10}   &\cellcolor{gray!20}\scriptsize{\epsi}&\cellcolor{gray!20}\scriptsize \text{
          Protocol that realises \ep}\\

\cellcolor{yellow!10}&\cellcolor{white!20}\scriptsize$\prf$ &\cellcolor{white!20}\scriptsize  Pseudorandom function \\ 


  \cellcolor{yellow!10}&\cellcolor{gray!20}\scriptsize$\prp$ &\cellcolor{gray!20}\scriptsize  Pseudorandom permutation \\ 

%
  \cellcolor{yellow!10}   &\cellcolor{white!20}\scriptsize{$gcd$}&\cellcolor{white!20}\scriptsize \text{Greatest common divisor}\\
%

\cellcolor{yellow!10}\multirow{-34}{*}{\rotatebox[origin=c]{90}{\cellcolor{yellow!10}\scriptsize{ {Generic}}}}
  \cellcolor{yellow!10}   &\cellcolor{gray!20}\scriptsize{$\negl$}&\cellcolor{gray!20}\scriptsize \text{Negligible function}\\
%
     \hline
%%%%%%%%%%%%
\end{tabular}
}
\scalebox{.668}{
\begin{tabular}{|c|c|c|c|c|c|c|c|c|c|c|c|c|c|} 
%%%%%%%%%%%
\hline 
\multicolumn{1}{|c|}{\rotatebox[origin=c]{45}{\cellcolor{yellow!10}\scriptsize{ {Setting}}}}&\cellcolor{yellow!10} \scriptsize{Symbol}&\cellcolor{yellow!10} \scriptsize{Description}\\
\hline 
     
\cellcolor{yellow!10}&\cellcolor{white!20}\scriptsize$\SCpc$&\cellcolor{white!20}\scriptsize \text{Prisoner's Contract}\\   

\cellcolor{yellow!10}&\cellcolor{gray!20}\scriptsize$\SCcc$&\cellcolor{gray!20}\scriptsize \text{Colluder’s Contract}\\   

\cellcolor{yellow!10}&\cellcolor{white!20}\scriptsize$\SCtc$&\cellcolor{white!20}\scriptsize \text{Traitor's Contract}\\   
%
\cellcolor{yellow!10}&\scriptsize  \cellcolor{gray!20}\scriptsize$\cc$&\cellcolor{gray!20}\scriptsize \text{Server’s cost for computing a task}\\   
%
\cellcolor{yellow!10}&\cellcolor{white!20}\scriptsize$\chc$&\cellcolor{white!20}\scriptsize \text{Auditor's cost for resolving disputes
}\\   
%
\cellcolor{yellow!10}&\scriptsize  \cellcolor{gray!20}\scriptsize$\dc$&\cellcolor{gray!20}\scriptsize \text{Deposit a server pays to get the job}\\  
%
\cellcolor{yellow!10}&\cellcolor{white!20}\scriptsize$\wc$&\cellcolor{white!20}\scriptsize \text{Amount a server receives for completing the task}\\  
%
\multirow{-8}{*}{\rotatebox[origin=c]{90}{\cellcolor{yellow!10}\scriptsize{ {Counter}}}}
%
\multirow{-8}{*}{\rotatebox[origin=c]{90}{\scriptsize{ {Collusion}}}}
%
\multirow{-8}{*}{\rotatebox[origin=c]{90}{\scriptsize{ {Contracts}}}}
&\cellcolor{gray!20}\scriptsize$(pk, sk)$&\cellcolor{gray!20}\scriptsize \text{\scf's auditor's public-private key pair}\\  
 %
\hline 
%%%% End of Counter Collusion Contacts %%%%%

%%%%%%%.      F-PSI      %%%%%%

%\qinit: Initiation predicate


\cellcolor{yellow!10}&\cellcolor{white!20}\scriptsize$\scf$&\cellcolor{white!20}\scriptsize {\fpsi's smart contract}\\   
%
\cellcolor{yellow!10}&\cellcolor{gray!20}\scriptsize$\bm\omega, \bm\omega',\bm\rho  $&\cellcolor{gray!20}\scriptsize {Random poly. of degree} $d$\\   
 %
\cellcolor{yellow!10}&\cellcolor{white!20}\scriptsize$\bm\gamma, \bm\delta$&\cellcolor{white!20}\scriptsize {Random poly. of degree} $d+1$\\
%
\cellcolor{yellow!10}&\cellcolor{gray!20}\scriptsize$\bm\nu^{\st{{(C)}}}$&\cellcolor{gray!20}\scriptsize {Blinded poly. sent by each $C$ to \scf}\\ 
%  
\cellcolor{yellow!10}&\cellcolor{white!20}\scriptsize$\bm\phi$&\cellcolor{white!20}\scriptsize {Blinded poly. encoding the intersection}\\   
%
\cellcolor{yellow!10}&\cellcolor{gray!20}\scriptsize$\bm\chi$&\cellcolor{gray!20}\scriptsize {Poly. sent to \scf to identify misbehaving parties}\\ 
%
\cellcolor{yellow!10}&\cellcolor{white!20}\scriptsize$\bar L$&\cellcolor{white!20}\scriptsize {List of identified misbehaving parties}\\ 
%

%%%%%%%%%%%%%
\cellcolor{yellow!10}&\cellcolor{gray!20}\scriptsize&\cellcolor{gray!20}\scriptsize {A portion of a party's deposit into \scf}\\   

\cellcolor{yellow!10}&\multirow{-2}{*}{\cellcolor{gray!20}\scriptsize$\yc$}&\cellcolor{gray!20}\scriptsize{transferred to honest clients if it misbehaves}\\ 
%%%%%%




\cellcolor{yellow!10}&\scriptsize$mk$&\scriptsize{Master key of \prf}\\ 
%
\cellcolor{yellow!10}&\cellcolor{gray!20}\scriptsize$\qinit$&\cellcolor{gray!20}\scriptsize{Initiation predicate}\\ 
%
\cellcolor{yellow!10}&\scriptsize$\qdel$&\scriptsize{Delivery predicate}\\ 
%
\cellcolor{yellow!10}&\cellcolor{gray!20}\scriptsize$\qUnFAbt$&\cellcolor{gray!20}\scriptsize{UnFair-Abort predicate}\\ 

\multirow{-14}{*}{\rotatebox[origin=c]{90}{\cellcolor{yellow!10}\scriptsize{ {\withFai (\fpsi)}}}}
%
\cellcolor{yellow!10}&\scriptsize$\qFAbt$&\scriptsize{Fair-Abort predicate}\\ 
%

\hline 
%%%%%   End of F-PSI  %%%%

%%%%%%%    E-PSI    %%%%%%

\cellcolor{yellow!10}&\cellcolor{gray!20}\scriptsize$\SCe$&\cellcolor{gray!20}\scriptsize {\epsi's smart contract} \\   
 %
\cellcolor{yellow!10}&\cellcolor{white!20}\scriptsize$\dc'$&\cellcolor{white!20}\scriptsize {Extractor's deposit} \\
  %   
\cellcolor{yellow!10}&\cellcolor{gray!20}\scriptsize$\yc'$&\cellcolor{gray!20}\scriptsize {Each client's deposit into \scf}\\   
%
\cellcolor{yellow!10}&\cellcolor{white!20}\scriptsize$\lc$&\cellcolor{white!20}\scriptsize {Reward a client earns for an intersection element}\\   
%
\cellcolor{yellow!10}&\cellcolor{gray!20}\scriptsize$\rc$&\cellcolor{gray!20}\scriptsize {Extractor's cost for extracting an intersection element}\\  
%
\cellcolor{yellow!10}&\cellcolor{white!20}\scriptsize$\fc$&\cellcolor{white!20}\scriptsize {Shorthand for $\lc(m-1)$}\\ 
%
\cellcolor{yellow!10}&\cellcolor{gray!20}&\cellcolor{gray!20}\scriptsize{Price a buyer pays for an intersection element}\\ 
%
\cellcolor{yellow!10}&\multirow{-2}{*}{\cellcolor{gray!20}\scriptsize$\vc$}&\cellcolor{gray!20}\scriptsize{$\vc=m\cdot \lc+2 \rc$}\\ 
%
\cellcolor{yellow!10}&\scriptsize$mk'$&\scriptsize{Another master key of \prf}\\ 
%

\cellcolor{yellow!10}&\cellcolor{gray!20}\scriptsize$ct_{\st mk}$&\cellcolor{gray!20}\scriptsize {Encryption of $mk$ under $pk$}\\   
%            
\cellcolor{yellow!10}&\scriptsize$\qdelwr$&\scriptsize{Delivery-with-Reward predicate}\\ 

\multirow{-12}{*}{\rotatebox[origin=c]{90}{\cellcolor{yellow!10}\scriptsize{ {\withRew (\epsi)}}}}
%
\cellcolor{yellow!10}&\cellcolor{gray!20}\scriptsize$\qUnFAbtwr$&\cellcolor{gray!20}\scriptsize{UnFair-Abort-with-Reward predicate}\\ 



\hline  



%%%%%%%%%%%

\end{tabular}\label{table:notation-table}}}
\end{center}
\end{scriptsize}
\vspace{-6mm}
\end{table}




%\subsection{Notations}

%Table \ref{table:notation-table}summarises the notations used in this paper. 

\vspace{-2mm}


% !TEX root =main.tex


 
  
\vspace{-2mm}
\subsection{Security Model}\label{sec::sec-model}
\vspace{-1mm}


The paper employs a simulation-based secure computation model \cite{DBLP:books/cu/Goldreich2004} to define and prove its protocols, addressing both active and passive adversaries. Definitions for both adversary types are outlined, assuming a static adversary and a secure off-chain communication channel between clients.


%In this paper, we use the simulation-based model of secure computation \cite{DBLP:books/cu/Goldreich2004} to define and prove our protocols. Since both types of active and passive adversaries are involved in our protocols, we outline definitions for both types.  We consider a static adversary and assume there is a secure (off-chain) channel between the clients.% and we consider a standard public blockchain, e.g., Ethereum.
%
 
  \vspace{-3mm}
  
 \subsubsection{Two-party Computation.} A two-party protocol $\Gamma$ problem is captured by specifying a random process that maps pairs of inputs to pairs of outputs, one for each party. Such process is referred to as a functionality  $f:\{0,1\}^{\st *}\times\{0,1\}^{\st *}\rightarrow\{0,1\}^{\st *}\times\{0,1\}^{\st *}$, where $f:=(f_{\st 1},f_{\st 2})$. For every input pair $(x,y)$, the output pair is a random variable $(f_{\st 1} (x,y), f_{\st 2} (x,y))$, such that the party with input $x$ obtains $f_{\st 1} (x,y)$ while the party with input $y$ receives $f_{\st 2} (x,y)$. When $f$ is deterministic, then $f_{\st 1} =f_{\st 2}$. If $f$ is asymmetric and only one party (say the first one) receives the result, $f$ is defined as $f:=(f_{\st 1}(x,y), \bot)$. 
 
  
  \vspace{-3mm}
 \subsubsection{Security in the Presence of Passive Adversaries.} 
 
%  In the passive adversarial model, the party corrupted by such an adversary correctly follows the protocol specification. Nonetheless, the adversary obtains the internal state of the corrupted party, including the transcript of all the messages received, and tries to use this to learn information that should remain private. 
  
In this setting, a protocol is secure if whatever can be computed by a party in the protocol can be computed using its input and output only. 
  %
%  In the simulation-based model, it is required that a party’s view in a protocol's 
% execution can be simulated given only its input and output. This implies that the parties learn nothing from the protocol's execution. 
 %
 
 \vspace{-1mm}
 
\begin{definition}
Let $f$ be the deterministic functionality which was defined above. Protocol $\Gamma$ security computes $f$ in the presence of a static  passive adversary if there exist polynomial time algorithms $(\mathsf {Sim}_{\st 1}, \mathsf {Sim}_{\st 2})$ such that: 
%
%\begin{center}
{\small{
$
  \{\mathsf {Sim}_{\st 1}(x, $ $f_{\st 1}(x,y))\}_{\st x,y}\\\stackrel{c}{\equiv} \{\mathsf{View}_{\st 1}^{\st \Gamma}(x,y) \}_{\st x,y}$ and 
  %
     $\{\mathsf{Sim}_{\st 2}(y, f_{\st 2}(x,y))\}_{\st x,y}\stackrel{c}{\equiv} \{\mathsf{View}_{\st 2}^{\st \Gamma}(x,y) \}_{\st x,y}
$}}
 % \end{center}
  %
  where party $i$’s view (during the execution of $\Gamma$) on input pair  $(x, y)$ is denoted by $\mathsf{View}_{\st i}^{\st \Gamma}(x,y)$ and equals $(w, r^{\st i}, m_{\st 1}^{\st i}, ..., m_{\st t}^{\st i})$, where $w\in\{x,y\}$ is the input of $i^{\st th}$ party, $r_{\st i}$ is the outcome of this party's internal random coin tosses, and $m_{\st j}^{\st i}$ represents the $j^{\st th}$ message this party receives.  %The output of the $i^{\st th}$ party during the execution of $\Gamma$ on $(x, y)$ is denoted by $\mathsf{Output}_{\st 1}^{\st \Gamma}(x,y)$ and can be generated from its own view of the execution.  The joint output of both parties is denoted by $\mathsf{Output}^{\st \Gamma}(x,y):=(\mathsf{Output}_{\st 1}^{\st \Gamma}(x,y), \mathsf{Output}_{\st 2}^{\st \Gamma}(x,y))$.
%
\end{definition}
 

  \vspace{-3mm}
  
 \subsubsection{Security in the Presence of Active Adversaries.}  In this setting, correctness is required beyond the possibility that a corrupted party may learn more than it should. To capture the threats,
a protocol's security is analysed by comparing what an adversary can do in the real protocol to what it can do in an ideal scenario. This is formalised by considering an ideal computation involving an incorruptible Trusted Third Party (TTP) to whom the parties send their inputs and receive the output of the ideal functionality. %Below, we describe the executions in the ideal and real models. 

%\
%
%\
%
%
% 
%First, we describe the execution in the ideal model. Let $P_{\st 1}$ and $P_{\st 2}$ be the parties participating in the
%protocol, $i\in \{0, 1\}$ be the index of the corrupted party, and $\mathcal A$ be a non-uniform
%probabilistic polynomial-time adversary. Also, let $z$ be an auxiliary input given to $\mathcal A$ while  $x$ and $y$ be the input of party $P_{\st 1}$ and $P_{\st 2}$  respectively.  The honest party, $P_{\st j}$, sends its received input to TTP.  The corrupted party $P_{\st i}$ may either abort (by replacing the input with a special abort message $\Lambda_{\st i}$),  send its received input or send some other input of the same length to TTP. This decision is made by the adversary and may depend on the input value of $P_{\st i}$ and $z$. If TTP receives $\Lambda_{\st i}$, then it sends $\Lambda_{\st i}$ to the honest party and the ideal execution terminates.  Upon obtaining an input pair $(x, y)$, TTP computes $f_{\st 1}(x, y)$ and $f_{\st 2}(x, y)$. It first sends $f_{\st i}(x, y)$ to  $P_{\st i}$ which replies with ``continue'' or $\Lambda_{\st i}$. In the former case, TTP sends  $f_{\st j}(x, y)$ to  $P_{\st j}$ and in the latter it sends $\Lambda_{\st i}$ to  $P_{\st j}$. The honest party always outputs the message that it obtained from TTP. A malicious party may output an arbitrary function of its initial inputs and the message it has obtained from TTP.  The ideal execution of $f$ on inputs $(x, y)$ and $z$ is denoted by $\mathsf{Ideal}^{\st f}_{\st\mathcal{A}(z), i}(x,y)$ and is defined as the output pair of the honest party and $\mathcal{A}$ from the above ideal execution.  In the real model, the real two-party protocol $\Gamma$ is executed
%without the involvement of TTP. In this setting, $\mathcal{A}$ sends all messages on
%behalf of the corrupted party and may follow an arbitrary strategy.
%The honest party follows the instructions of $\Gamma$. The real execution of $\Gamma$ is denoted by $\mathsf{Real}^{\st \Gamma}_{\st\mathcal{A}(z), i}(x,y)$, it is defined as the joint output of the parties engaging in the real execution of $\Gamma$ (on the inputs), in the presence of $\mathcal{A}$.
% 
% 
% Next, we define security. At a high level, the definition states that a secure protocol in the real model emulates the ideal model. This is formulated by stating that adversaries in the ideal model can simulate executions of the protocol in the real model. 
 
 \vspace{-2.5mm} 
  
  
\begin{definition}\label{def::MPC-active-adv}
Let $f$ be the two-party functionality defined above and $\Gamma$ be a two-party protocol that computes $f$.   Protocol $\Gamma$ securely computes $f$ with abort in the presence of static active adversaries if for every Probabilistic Polynomial Time (PPT) adversary $\mathcal{A}$ for the real model, there exists a non-uniform PPT adversary (or simulator) $\mathsf{Sim}$ for the ideal model, such that for every $i\in \{0,1\}$, it holds that: 
%
$
\{\mathsf {Ideal}^{\st f}_{\st \mathsf{Sim}(z), i}(x,y)\}_{\st x,y,z}\stackrel{c}{\equiv} \{\mathsf{Real}_{\st \mathcal{A}(z), i}^{\st \Gamma}(x,y) \}_{\st x,y,z},
$, 
%
 where the ideal execution of $f$ on inputs $(x,$ $ y)$ and $z$ is denoted by $\mathsf{Ideal}^{\st f}_{\st\mathcal{A}(z), i}(x,$ $y)$ and is defined as the output pair of the honest party and $\mathcal{A}$ from the ideal execution.   \end{definition}
 
 
 The real execution of $\Gamma$ is denoted by $\mathsf{Real}^{\st \Gamma}_{\st\mathcal{A}(z), i}(x,y)$, it is defined as the joint output of the parties engaging in the real execution of $\Gamma$, in the presence of $\mathcal{A}$.

  
%% !TEX root =main.tex


\vs 
\vs

\subsection{Security Model}\label{sec::sec-model}

In this paper, we use the simulation-based paradigm of secure computation \cite{DBLP:books/cu/Goldreich2004} to define and prove our protocols. Since both types of active and passive adversaries are involved in our protocols, we will outline definitions for both types (and refer readers to Appendix \ref{sec::sec-model-long} for more details).  We consider a static adversary, we assume there is an authenticated private (off-chain) channel between the clients and we consider a standard public blockchain, e.g., Ethereum.
%
 
 \vs
 \vs
 \subsubsection{Two-party Computation.} A two-party protocol $\Gamma$ problem is captured by specifying a random process that maps pairs of inputs to pairs of outputs, one for each party. Such process is referred to as a functionality denoted by  $f:\{0,1\}^{\st *}\times\{0,1\}^{\st *}\rightarrow\{0,1\}^{\st *}\times\{0,1\}^{\st *}$, where $f:=(f_{\st 1},f_{\st 2})$. For every input pair $(x,y)$, the output pair is a random variable $(f_{\st 1} (x,y), f_{\st 2} (x,y))$, such that the party with input $x$ obtains $f_{\st 1} (x,y)$ while the party with input $y$ receives $f_{\st 2} (x,y)$. When $f$ is deterministic, then $f_{\st 1} =f_{\st 2}$. In the setting where $f$ is asymmetric and only one party (say the first one) receives the result, $f$ is defined as $f:=(f_{\st 1}(x,y), \bot)$. 
 
 \vs
 \vs
 \subsubsection{Security in the Presence of Passive Adversaries.} 
 
%  In the passive adversarial model, the party corrupted by such an adversary correctly follows the protocol specification. Nonetheless, the adversary obtains the internal state of the corrupted party, including the transcript of all the messages received, and tries to use this to learn information that should remain private. 
  
In this setting, a protocol is secure if whatever can be computed by a party in the protocol can be computed using its input and output only. 
  %
%  In the simulation-based model, it is required that a party’s view in a protocol's 
% execution can be simulated given only its input and output. This implies that the parties learn nothing from the protocol's execution. 
 %
\begin{definition}
Let $f$ be the deterministic functionality defined above. Protocol $\Gamma$ security computes $f$ in the presence of a static  passive adversary if there exist polynomial-time algorithms $(\mathsf {Sim}_{\st 1}, \mathsf {Sim}_{\st 2})$ such that:
\end{definition}
%
\begin{center}
{\small{
$
  \{\mathsf {Sim}_{\st 1}(x,f_{\st 1}(x,y))\}_{\st x,y}\stackrel{c}{\equiv} \{\mathsf{View}_{\st 1}^{\st \Gamma}(x,y) \}_{\st x,y},
  %
    \{\mathsf{Sim}_{\st 2}(x,f_{\st 2}(x,y))\}_{\st x,y}\stackrel{c}{\equiv} \{\mathsf{View}_{\st 2}^{\st \Gamma}(x,y) \}_{\st x,y}
$}}
  \end{center}
  %
  where party $i$’s view (during the execution of $\Gamma$) on input pair  $(x, y)$ is denoted by $\mathsf{View}_{\st i}^{\st \Gamma}(x,y)$ and equals $(w, r^{\st i}, m_{\st 1}^{\st i}, ..., m_{\st t}^{\st i})$, where $w\in\{x,y\}$ is the input of $i^{\st th}$ party, $r_{\st i}$ is the outcome of this party's internal random coin tosses, and $m_{\st j}^{\st i}$ represents the $j^{\st th}$ message this party receives.  %The output of the $i^{\st th}$ party during the execution of $\Gamma$ on $(x, y)$ is denoted by $\mathsf{Output}_{\st 1}^{\st \Gamma}(x,y)$ and can be generated from its own view of the execution.  The joint output of both parties is denoted by $\mathsf{Output}^{\st \Gamma}(x,y):=(\mathsf{Output}_{\st 1}^{\st \Gamma}(x,y), \mathsf{Output}_{\st 2}^{\st \Gamma}(x,y))$.

\vs
\vs
  
 \subsubsection{Security in the Presence of Active Adversaries.}  In this adversarial model, correctness is required beyond the possibility that a corrupted party may learn more than it should. To capture the threats,
a protocol's security is analyzed by comparing what an adversary can do in the real protocol to what it can do in an ideal scenario. This is formalized by considering an ideal computation involving an incorruptible Trusted Third Party (TTP) to whom the parties send their inputs and receive the output of the ideal functionality. %Below, we describe the executions in the ideal and real models. 

%\
%
%\
%
%
% 
%First, we describe the execution in the ideal model. Let $P_{\st 1}$ and $P_{\st 2}$ be the parties participating in the
%protocol, $i\in \{0, 1\}$ be the index of the corrupted party, and $\mathcal A$ be a non-uniform
%probabilistic polynomial-time adversary. Also, let $z$ be an auxiliary input given to $\mathcal A$ while  $x$ and $y$ be the input of party $P_{\st 1}$ and $P_{\st 2}$  respectively.  The honest party, $P_{\st j}$, sends its received input to TTP.  The corrupted party $P_{\st i}$ may either abort (by replacing the input with a special abort message $\Lambda_{\st i}$),  send its received input or send some other input of the same length to TTP. This decision is made by the adversary and may depend on the input value of $P_{\st i}$ and $z$. If TTP receives $\Lambda_{\st i}$, then it sends $\Lambda_{\st i}$ to the honest party and the ideal execution terminates.  Upon obtaining an input pair $(x, y)$, TTP computes $f_{\st 1}(x, y)$ and $f_{\st 2}(x, y)$. It first sends $f_{\st i}(x, y)$ to  $P_{\st i}$ which replies with ``continue'' or $\Lambda_{\st i}$. In the former case, TTP sends  $f_{\st j}(x, y)$ to  $P_{\st j}$ and in the latter it sends $\Lambda_{\st i}$ to  $P_{\st j}$. The honest party always outputs the message that it obtained from TTP. A malicious party may output an arbitrary function of its initial inputs and the message it has obtained from TTP.  The ideal execution of $f$ on inputs $(x, y)$ and $z$ is denoted by $\mathsf{Ideal}^{\st f}_{\st\mathcal{A}(z), i}(x,y)$ and is defined as the output pair of the honest party and $\mathcal{A}$ from the above ideal execution.  In the real model, the real two-party protocol $\Gamma$ is executed
%without the involvement of TTP. In this setting, $\mathcal{A}$ sends all messages on
%behalf of the corrupted party and may follow an arbitrary strategy.
%The honest party follows the instructions of $\Gamma$. The real execution of $\Gamma$ is denoted by $\mathsf{Real}^{\st \Gamma}_{\st\mathcal{A}(z), i}(x,y)$, it is defined as the joint output of the parties engaging in the real execution of $\Gamma$ (on the inputs), in the presence of $\mathcal{A}$.
% 
% 
% Next, we define security. At a high level, the definition states that a secure protocol in the real model emulates the ideal model. This is formulated by stating that adversaries in the ideal model can simulate executions of the protocol in the real model. 
 
\begin{definition}\label{def::MPC-active-adv}
Let $f$ be the two-party functionality defined above and $\Gamma$ be a two-party protocol that computes $f$.   Protocol $\Gamma$ securely computes $f$ with abort in the presence of static active adversaries if for every non-uniform probabilistic polynomial time adversary $\mathcal{A}$ for the real model, there exists a non-uniform probabilistic polynomial-time adversary (or simulator) $\mathsf{Sim}$ for the ideal model, such that for every $i\in \{0,1\}$, it holds that: 
%
$
\{\mathsf {Ideal}^{\st f}_{\st \mathsf{Sim}(z), i}(x,y)\}_{\st x,y,z}\stackrel{c}{\equiv} \{\mathsf{Real}_{\st \mathcal{A}(z), i}^{\st \Gamma}(x,y) \}_{\st x,y,z}
$
%
\end{definition}
 
 where the ideal execution of $f$ on inputs $(x,$ $ y)$ and $z$ is denoted by $\mathsf{Ideal}^{\st f}_{\st\mathcal{A}(z), i}(x,$ $y)$ and is defined as the output pair of the honest party and $\mathcal{A}$ from the ideal execution. The real execution of $\Gamma$ is denoted by $\mathsf{Real}^{\st \Gamma}_{\st\mathcal{A}(z), i}(x,y)$, it is defined as the joint output of the parties engaging in the real execution of $\Gamma$ (on the inputs), in the presence of $\mathcal{A}$.
  
  


\vspace{-4mm}

\subsection{Smart Contracts}
\vspace{-1mm}

Cryptocurrencies, such as Bitcoin \cite{bitcoin} and Ethereum \cite{ethereum},  go beyond merely providing a decentralised currency; they also facilitate computations on transactions. These cryptocurrencies allow for a specific computational logic to be encoded in a computer program known as a \emph{``smart contract''}.  As of now, Ethereum stands as the foremost cryptocurrency framework that empowers users to define arbitrary smart contracts. Within this framework, contract code resides on the blockchain and is executed by all parties involved in maintaining the cryptocurrency. The correctness of program execution is ensured by the security of the underlying blockchain components. In the context of this work, \textbf{standard} public (Ethereum) smart contracts align with the requirements of our protocols.




%Cryptocurrencies, such as Bitcoin \cite{bitcoin} and Ethereum \cite{ethereum}, beyond offering a decentralised currency,  support computations on transactions. In this setting, often a certain computation logic is encoded in a computer program, called a \emph{``smart contract''}. To date, Ethereum is the most predominant cryptocurrency framework that enables users to define arbitrary smart contracts. In this framework,  contract code is stored on the blockchain and executed by all parties (i.e., miners) maintaining the cryptocurrency,  when the program inputs are provided by transactions. The program execution's correctness is guaranteed by the security of the underlying blockchain components. To prevent a denial-of-service attack, the framework requires a transaction creator to pay a  fee, called \emph{``gas''}, depending on the complexity of the contract running on it. 

\vspace{-4mm}

\subsection{Counter Collusion Smart Contracts}\label{Counter-Collusion-Smart-Contracts}
\vspace{-1mm}


To enable a party, such as a client, to efficiently delegate computation to multiple potentially colluding third parties, like servers, Dong   \et \cite{dong2017betrayal}  
introduced two primary smart contracts: the ``Prisoner's Contract'' ($\SCpc$) and the ``Traitor's Contract'' (\SCtc).  
%
$\SCpc$ jointly signed by both the client and the servers and is designed to incentivise accurate computation. This contract mandates that each server must submit a deposit before the computation is delegated. Furthermore, it is equipped with an external auditor that can be invoked to identify any misbehaving server when they provide dissimilar results. 

If a server behaves honestly, it is eligible to withdraw its deposit. However, if the auditor detects a cheating server, a portion of its deposit is transferred to the client. In the scenario where one server is honest while the other one cheats, the honest server receives a reward sourced from the deposit of the cheating server.

However, the dilemma, created by \SCpc between the two servers, can be resolved if they are able to establish an enforceable promise, such as through a ``Colluder's Contract'' (\SCcc). In this contract,  one party, referred to as the ``ringleader'', commits to paying a bribe to its counterpart if both parties engage in collusion and deliver an incorrect result to \SCpc. 
%
To counter \SCcc, Dong   \et proposed \SCtc, which provides incentives for a colluding server to expose the other server and report the collusion without facing penalties from \SCpc. In this study, we have made slight adjustments and employed these contracts. The relevant parameters for these contracts are outlined in \ref{table:notation-table}. For a comprehensive description of the parameters and contracts, we direct readers to Appendix \ref{appendix::Counter-Collusion-Contracts}. 




%\begin{itemize}
%\item[$\bullet$] $\bc$: the bribe paid by the ringleader of the collusion to the other
%server in the collusion agreement, in the Colluder’s contract.
%%
%\item[$\bullet$] $\cc$: a server’s cost for computing the task.
%%
%\item[$\bullet$] $\chc$: the fee paid to to invoke an auditor for recomputing a task and resolving
%disputes.
%%
%\item[$\bullet$] $\dc$: the deposit a server needs to pay to be eligible for getting the job.
%%
%\item[$\bullet$] $\tc$: the deposit the colluding parties need to pay in the collusion agreement, in the Colluder’s contract.
%%
%\item[$\bullet$] $\wc$: the amount that a server receives for completing the task.
%%
%\item[$\bullet$] $\wc \geq \cc$: the server would not accept underpaid jobs.
%%
%\item[$\bullet$] $\chc > 2\wc$: If it does not hold, then there would be no need to use the servers and the auditor would do the computation.
%%
%\item [$\bullet$] $(pk,sk)$: an asymmetric-key encryption's public-private key pair belonging to the auditor. 
%\end{itemize}
%\noindent The following relations need to hold when setting the contracts
%in order for the desirable equilibria to hold:
%%
%(i) $\dc>\cc+\chc$, (ii) $\bc<\cc$, and (iii) $\tc<\wc-\cc + 2\dc - \chc -\bc$.
%

%
\vspace{-3mm}
\subsection{Pseudorandom Function and Permutation}
\vspace{-1mm}

A pseudorandom function is a deterministic function that takes a key of length $\lambda$ and an input; and outputs a value  indistinguishable from that of  a truly random function.  In this paper, we use pseudorandom functions:   $\mathtt {PRF}: \{0,1\}^{\st \lambda}\times \{0,1\}^{\st *} \rightarrow  \mathbb{F}_{\st p}$, where $\log_{\st 2}(p)=\lambda$ is the security parameter. In practice, a pseudorandom function can be obtained from an efficient block cipher \cite{DBLP:books/crc/KatzLindell2007}. 
%
The definition of a pseudorandom permutation, $\mathtt {PRP}: \{0,1\}^{\st \lambda}\times \{0,1\}^{\st *} \rightarrow  \mathbb{F}_{\st p}$, is similar to that of a pseudorandom function, with a difference, it is required  $\PRP(k,.)$ to be indistinguishable from a uniform permutation, instead of a uniform function. %In cryptographic schemes that involve $\PRP$, sometimes honest parties may require to compute the inverse of pseudorandom permutation, i.e., $\mathtt {PRP}^{\st -1}(k, .)$, as well. In this case, it would require that $\PRP(k,.)$ be indistinguishable from a uniform permutation even if the distinguisher is additionally given oracle access to the inverse of the permutation. 




%\subsection{Random Extraction Beacon}
%\subsection{Commitment Scheme}

\vspace{-3mm}

% !TEX root =main.tex

\vspace{-1.5mm}

\subsection{Commitment Scheme}\label{subsec:short-commit}
\vspace{-1mm}

A commitment scheme involves a  \emph{sender} and a \emph{receiver}. It includes  two phases, \emph{commit} and  \emph{open}. In the \emph{commit} phase, the sender  commits to a message $x$ as $\mathtt{Com}(x,r)=\mathtt{Com}_{\scriptscriptstyle x}$, that involves a secret value,  $r$. In the \emph{open} phase, the sender sends the opening $\ddot{x}:=(x,r)$ to the receiver which verifies its correctness: $\mathtt{Ver}(\mathtt{Com}_{\scriptscriptstyle x},\ddot{x})\stackrel{\scriptscriptstyle ?}=1$ and accepts if the output is $1$. A commitment scheme must satisfy: (a) \textit{hiding}: it is infeasible for an adversary to learn any information about the committed  message $x$, until the commitment ${com}$ is opened, and (b) \textit{binding}: it is infeasible for a malicious sender to open a commitment ${com}$ to different values than that was  used in the commit phase. We refer readers to Appendix \ref{subsec:commit} for further details. 






\vspace{-4mm}

\subsection{Hash Tables}
\vspace{-1mm}

A hash table is an array of bins, each capable of containing a set of elements, and is paired with a hash function. To insert an element, we initially compute the element's hash and subsequently place the element into the bin corresponding to the computed hash value. In this paper, we ensure that the number of elements in each bin does not surpass a predefined capacity. By considering the maximum number of elements as $c$ and the maximum size of a bin as $d$, we can calculate the number of bins, denoted as $h$, through an analysis of hash tables using the the ``balls into the bins'' model  \cite{DBLP:conf/stoc/BerenbrinkCSV00}. In Appendix \ref{Preliminary-Hash-Table}, we provide an explanation of how the hash table parameters are configured.



%\subsection{Merkel Tree}

\vspace{-3mm}

% !TEX root =main.tex


\vspace{-.6mm}

\subsection{Merkle Tree}\label{sec::merkle-tree-short}
\vspace{-.7mm}


A Merkle tree is a data structure that facilitates a concise commitment to a set of values or blocks, involving two parties: a prover and a verifier. 
%
The Merkle tree scheme includes three algorithms; namely, $\mathtt{MT.genTree}$, $ \mathtt{MT.prove}$, and  $\mathtt{MT.verify}$. Briefly, the first algorithm constructs a Merkle tree on a set of blocks, the second generates a proof of a block's (or set of blocks') membership, and the third verifies the proof. Appendix \ref{sec::merkle-tree} provides more details. 

%
%
%
% The  Merkle tree scheme includes three algorithms $(\mkgen, \mkprove,$ $\mkver)$, defined as follows: 
%
%\begin{itemize}
%%
%\item[$\bullet$] The algorithm that constructs a Merkle tree, $\mkgen$, is run by $\mathcal{V}$. It takes  blocks, $u:=u_{\st 1},...,u_{\st n}$, as input. Then, it groups the blocks  in pairs. Next,   a collision-resistant hash function, $\mathtt{H}(.)$, is used to hash each pair. After that, the hash values are grouped in pairs and each pair is further hashed, and this process is repeated until only a single hash value, called ``root'', remains. This yields a  tree with the leaves corresponding to the input blocks and the root corresponding to the last remaining hash value. $\mathcal{V}$ sends the root to $\mathcal{P}$.
%%
%\item[$\bullet$] The proving algorithm, $\mkprove$, is run by $\mathcal{P}$. It takes a block index, $i$, and a tree as inputs. It outputs a vector proof, of  $\log_{\st 2}(n)$ elements. The proof asserts the membership of $i$-th block in the tree, and consists of all the sibling nodes on a path from the $i$-th block to the root of the Merkle tree (including $i$-th block). The proof is given to $\mathcal{V}$.
%%
%\item[$\bullet$] The verification algorithm, $\mkver$, is run by $\mathcal{V}$. It takes as an input $i$-th block, a proof, and the tree's root. It checks if the $i$-th block corresponds to the root. If the verification passes, it outputs $1$; otherwise, it outputs $0$.
%
%\end{itemize}
%
%The Merkle tree-based scheme has two properties: \emph{correctness} and \emph{security}. Informally, the correctness requires that if both parties run the algorithms correctly, then a proof is always accepted by  $\mathcal{V}$. The security requires that a computationally bounded malicious $\mathcal{P}$ cannot convince  $\mathcal{V}$ into accepting an incorrect proof, e.g., proof for a non-member block. The security relies on the assumption that it is computationally infeasible to find the hash function's collision. Usually, for the sake of simplicity, it is assumed that the number of blocks, $n$, is a power of $2$. The height of the tree, constructed on $m$ blocks, is $\log_{\st 2}(n)$. 

\vspace{-3mm}

\subsection{Polynomial Representation of Sets}\label{sec::poly-rep}
\vspace{-1mm}

The idea of using a polynomial to represent a set's elements was proposed by Freedman  \et in \cite{DBLP:conf/eurocrypt/FreedmanNP04}. Since then,   the idea has been widely used,  e.g., in \cite{GhoshS19,DBLP:conf/crypto/KissnerS05}. In this representation, set elements $S=\{s_{\st 1},...,s_{\st d}\}$ are defined over a finite field $\mathbb{F}_{\st p}$ and  set $S$ is represented as a polynomial of   form: $\mathbf{p}(x)=\prod\limits ^{\st {d}}_{\st i=1}(x-s_{\st i})$, where $\mathbf{p}(x) \in \mathbb{F}_{\st p}[X]$ and $\mathbb{F}_{\st p}[X]$ is a polynomial ring.  Often a   polynomial,  $\mathbf{p}(x)$, of degree $d$ is  represented in the ``coefficient form'' as follows:  $\mathbf{p}(x)=a_{\st 0}+a_{\st 1}\cdot x+...+ a_{\st d}\cdot x^{\st d}$. The form $\prod\limits ^{\st {d}}_{\st i=1}(x-s_{\st i})$ is a special case of the coefficient form. As shown in \cite{BonehGHWW13,DBLP:conf/crypto/KissnerS05}, for two sets $S^{\st (A)}$ and $S^{\st (B)}$ represented by polynomials $\mathbf{p}_{\st A}$ and $\mathbf{p}_{\st B}$ respectively, their product, which is polynomial $\mathbf{p}_{\st A}\cdot \mathbf{p}_{\st  B}$,  represents the set union, while their greatest common divisor, $gcd($$\mathbf{p}_{\st A}$$,\mathbf{p}_{\st B})$, represents the set intersection. For two polynomials $\mathbf{p}_{\st A}$ and $\mathbf{p}_{\st B}$ of degree $d$, and two random polynomials $\bm\gamma_{\st A}$ and  $\bm\gamma_{\st B}$ of degree $d$, it is proven in~\cite{BonehGHWW13,DBLP:conf/crypto/KissnerS05} that: $\bm\theta=\bm\gamma_{\st A}\cdot \mathbf{p}_{\st A}+\bm\gamma_{\st B}\cdot\mathbf{p}_{\st B}=\bm\mu\cdot gcd(\mathbf{p}_{\st A},\mathbf{p}_{\st B})$, where $\bm\mu$ is a uniformly random polynomial, and polynomial $\bm\theta$ contains only information about the elements in  $S^{\st (A)}\cap S^{\st (B)}$, and contains no information about other elements in $S^{\st (A)}$ or $S^{\st (B)}$.  

Given a polynomial $\bm\theta$ that encodes sets intersection, one can find the set elements in the intersection via one of the following approaches. First, via polynomial evaluation: the party who already has one of the original input sets, say  $\mathbf{p}_{\st A}$,  evaluates $\bm\theta$ at every element $s_{\st i}$ of $\mathbf{p}_{\st A}$ and considers $s_{\st i}$ in the intersection if $\mathbf{p}_{\st A}(s_{\st i})=0$. Second,   via polynomial root extraction:   the party who does not have one of the original input sets, extracts the roots of $\bm\theta$,  which contain  the roots of (i) random polynomial  $\bm\mu$ and (ii) the polynomial that represents the intersection, i.e., $gcd(\mathbf{p}_{\st A},\mathbf{p}_{\st B})$. In this approach, to distinguish errors (i.e., roots of $\bm\mu$) from the intersection, PSIs in \cite{eopsi,DBLP:conf/crypto/KissnerS05} use the \emph{``hash-based padding technique''}. In this technique, every element $u_{\st i}$ in the set universe $\mathcal{U}$, becomes $s_{\st i}=u_{\st i}||\mathtt{H}(u_{\st i})$, where $\mathtt{H}$ is a cryptographic hash function with a sufficiently large output size. Given a field's arbitrary element, $s \in \mathbb{F}_p$ and $\mathtt{H}$'s output size $|\mathtt{H}(.)|$, we can parse $s$ into $x_{\st 1}$ and $x_{\st 2}$, such that $s=x_{\st 1}||x_{\st 2}$ and  $|x_{\st 2}|=|\mathtt{H}(.)|$. In a  PSI that uses polynomial representation and this padding technique, after we extract each root of  $\bm\theta$, say $s$, we parse it into $(x_{\st 1}, x_{\st 2})$ and check $x_{\st 2}\stackrel{?}=\mathtt{H}(x_{\st 1})$.  If the equation holds, then we consider $s$ as an element of the intersection. 




%\TZ{What is meant by ``$\bm\theta$ contains only information about $S^{\st (A)}\cap S^{\st (B)}$"?}--> addressed.. 

%Polynomials can also be represented in the  ``point-value form''. In particular, a polynomial $\mathbf{p}(x)$ of degree $d$ can be represented as a set of $m$ ($m>d$) point-value pairs $\{(x_{\st 1},y_{\st 1}),...,$ $(x_{\st m},y_{\st m})\}$ such that all $x_{\st i}$ are distinct  non-zero points and $y_{\st i}=\mathbf{p}(x_{\st i})$ for all $i$, $1\le i\le m$. If  $x_{\st i}$  are fixed, then we can represent polynomials as a vector $\vv{\bm{y}}=[y_{\st 1}, ..., y_{\st m}]$. Polynomials in point-value form have  been used previously in PSIs~\cite{eopsi,opsi15,DBLP:conf/fc/AbadiTD16,Feather2020,GhoshS19,KolesnikovMPRT17}. A polynomial
%in this form can be converted into coefficient form via polynomial interpolation, e.g., using Lagrange interpolation~\cite{aho19}. Moreover,  one can add or multiply two polynomials,  in point-value form, by adding or multiplying their corresponding y-coordinates. In this case, the  polynomial interpolated from the result would be the two polynomials' addition or product. Often PSIs  that use this representation  assume that all $x_{\st i}$ are picked from $\mathbb{F} \setminus \mathcal{U}$.


\vspace{-3mm}

\subsection{Horner's Method}
\vspace{-1mm}

Horner's method \cite{DBLP:journals/ibmrd/Dorn62} allows for efficiently evaluating polynomials at a given point, e.g., $x_{\st 0}$. Specifically, given a polynomial of the form: $\bm\tau(x)= a_{\st 0}+a_{\st 1}\cdot x+a_{\st 2}\cdot x^{\st 2}+...+a_{\st d}\cdot x^{\st d}$ and a point: $x_{\st 0}$, one can efficiently evaluate the polynomial at $x_{\st 0}$ iteratively, in the following fashion: $\bm\tau(x_{\st 0})=a_{\st 0}+x_{\st 0}(a_{\st 1} + x_{\st 0}(a_{\st 2}+...+x_{\st 0}(a_{\st d-1}+x_{\st 0}\cdot a_{\st d})...)))$. Evaluating  a polynomial of degree $d$ naively requires  $d$ additions and $\frac{(d^{\st 2}+d)}{2}$ multiplications. However, using Horner's method the evaluation requires only $d$ additions and $2d$ multiplications. We use this method in this paper. 

\vspace{-3mm}

\subsection{Oblivious Linear Function Evaluation}\label{sec::OLE-plus}
\vspace{-1mm}

Oblivious Linear function Evaluation (\ole) is a two-party protocol that involves a sender and receiver. In \ole,  the sender  has two  inputs  $a, b\in \mathbb{F}_{\st p}$ and the receiver has a single input, $c \in \mathbb{F}_{p}$.  The protocol allows the receiver to learn only $s = a\cdot c + b \in \mathbb{F}_{\st p}$, while the sender learns nothing. Ghosh \textit{et al.} \cite{GhoshNN17} proposed an efficient \ole that has $O(1)$ overhead and involves mainly symmetric key operations. Later, in \cite{GhoshN19} an enhanced \ole, called $\ole^{\st +}$ was proposed. The latter ensures that the receiver cannot learn anything about the sender's inputs,  even if it sets its input to $0$. In this paper, we use $\ole^{\st +}$. We refer readers to Appendix \ref{apndx:F-OLE-plus}, for its construction.  %In this case, each party picks a random string, 

\vspace{-1mm}
% !TEX root =main.tex



\vspace{-3.6mm}



\subsection{Coin-Tossing Protocol}\label{sec::short-coin-tossing}
\vspace{-1.2mm}

A Coin-Tossing protocol, \ct, allows two mutually distrustful parties, say $A$ and $B$, to jointly generate a single random bit. Formally, \ct computes the functionality $\fct(in_{\st A}, in_{\st B})\rightarrow (out_{\st A}, out_{\st B})$, which takes $in_{\st A}$ and  $in_{\st B}$ as inputs of $A$ and $B$ respectively and outputs $out_{\st A}$ to $A$ and $out_{\st B}$ to $B$, where $out_{\st A}=out_{\st B}$. A basic security requirement of a \ct is that the resulting bit is (computationally) indistinguishable from a truly random bit. 
%
Two-party coin-tossing protocols can be generalised to \emph{multi-party} coin-tossing ones to generate a \emph{random string} (rather than a single bit). 
%
The overheads of multi-party coin-tossing protocols are often linear with the number of participants. In this paper, any secure multi-party \ct that generates a random string can be used. For the sake of simplicity, we allow a multi-party \fct to take $m$ inputs and output a single value, i.e., $\fct(in_{\st 1}, ..., in_{\st m})\rightarrow out$. We refer readers to Appendix \ref{sec::coin-tossing} for further details. 

 







%,  as an aborting party can be excluded from the next run of the protocol and the aborting party cannot learn partsets' intersection 













%% !TEX root =U-PSI.tex

\section{Priliminaries}

\subsection{Notations}
In the following, we  provide a basic and enhanced multiple clients PSI protocols. In the basic protocol, there are three types of parties involved in the protocols: $m>1$ authoriser clients: $\resizeT {\textit A}_{\resizeS {\textit  j}}$ who are not interested in the result, the result recipient client: $B$, and a non-colluding dealer: $D$. All parties are potentially semi-honest. In the first variant of the protocol the dealer does not have any set elements. Then we provide an enhanced protocol   that supports a distributed dealers such that the protocol remains secure even if  all but one dealers collude with client $B$ or  a subset of authoriser clients collude with each other and client $B$. Moreover, in the enhanced protocol, the dealers are actually clients who have  set elements as the computation's inputs.  Note that, in the following protocols, all  values and operations are defined over a finite field of prime order. Let each client $I\in \{\resizeT {\textit A}_{\resizeS {\textit  1}},...,\resizeT {\textit A}_{\resizeS {\textit  m}},B\}$, have a set $S^{\resizeS {\textit I} }=\{s^{\resizeS {\textit I} }_{\scriptscriptstyle 1},..., s^{\resizeS {\textit I} }_{\scriptscriptstyle d}\}$ and   $\vv{\bm{x}}=[x_{\scriptscriptstyle 1},..., x_{\scriptscriptstyle n}]$ be a  public vector of non-zero unique elements, where $n=2d+1$, and $d$ be the set cardinality's upper bound. Let $e$ be a flat price of learning one element of the set.   

\subsection{Shamir Secret Sharing}

\subsection{Oblivious Polynomial Evaluation and Oblivious Linear Function Evaluation} Oblivious polynomial evaluation: OPE, introduced by \cite{Naor:1999:OTP}, is a two-party protocol where a sender  inputs a polynomial, defined over a finite field, and a receiver inputs a single
point of the same finite field. At the end of the protocol, the sender receives nothing while the receiver  receives the polynomial evaluated on the point chosen by the receiver. Informally, the protocol is secure if the sender learns nothing on which point was chosen by the receiver and the receiver evaluates the polynomial on at most one point. OPE protocols can be broadly categorised in two groups: (a) computationally (or conditionally) secure, and (b) information theoretically (or unconditionally) secure. In this paper, we use OPE to evaluate a polynomial of \emph{degree one} and our focus is on the latter category as it best fits our purpose, e.g. due to its computational efficiency, (similar to our protocols) using a finite field over which are arithmetic operations are defined, etc. Information  theoretically OPE protocols can be further divided into three  classes: those that  (a) use an initializer that has an \emph{active role}, e.g. \cite{DBLP:conf/asiacrypt/ChangL01} (b) use an initializer that has a \emph{one-off role} and distributes  random parameters among receiver and sender in the setup phase and does not play any further role, e.g. \cite{DBLP:conf/acisp/HanaokaIMNOW04} and (c) are based on a distributed setting in which a set of servers implement the function of the sender, e.g. in \cite{DBLP:conf/icisc/CianciulloG18}. Our PSI protocol, can use either of these highly efficient OPE protocols, i.e.  \cite{DBLP:conf/acisp/HanaokaIMNOW04} or \cite{DBLP:conf/icisc/CianciulloG18}.  The scheme in \cite{DBLP:conf/acisp/HanaokaIMNOW04} is secure under the assumption that the initializer does not collude with the sender and receiver. Moreover, the scheme in \cite{DBLP:conf/icisc/CianciulloG18} remains secure as long as the sender does not collude with the servers, however if a threshold of the servers  collude with each other they cannot learn anything about the sender's or receiver's input and if a threshold of the servers collude with the receiver they cannot learn the sender's input. We note that the  use of the  OPE proposed in \cite{DBLP:conf/icisc/CianciulloG18}  in our protocol does not introduce any  additional servers, as  their roles are played by a subset of  clients in our protocol, i.e. assistant authorizer clients. For the sake of completeness, we provide both protocols \cite{DBLP:conf/acisp/HanaokaIMNOW04} in this section. 

%\begin{enumerate}
%\item There are three parties involved in this protocol: Sender: $S$, receiver $R$ and initializer $T$, where $S$'s input is $\beta(x)$ and $R$'s input is $b$.
%\item Setup: $T$ picks a random polynomial $\tau(x)$ of degree $d$ and a random value $v$. It sends $\tau(x)$ to $S$. Also, it sends  $v$ and $g=\tau(v)$ to $R$. 
%\item Computation: $R$ sends the value $l=b-v$ to $S$. $S$ then computes and sends to $R$ the polynomial $\delta (x)= \beta(l+x)+\tau(x)$. Then, $R$ computes $\delta(d)-g=\beta(b)=$
%\end{enumerate}





\begin{figure}[ht]
\setlength{\fboxsep}{2pt}
\begin{center}
\begin{boxedminipage}{11cm}
\small{

\

\noindent\textbf {Parties:} Sender: $\mathcal{S}$, receiver: $\mathcal{R}$ and initialiser: $\mathcal{T}$.

\noindent\textbf {Input:} $\mathcal{S}$ has a polynomial: $\beta(x)$  of degree at most $\mathsf{d}$. Also, $\mathcal{R}$ has a value:  $\mathsf{a}$.

\noindent\textbf {Output:} $\mathcal{R}$ obtains $\mathsf{b}=\beta(\mathsf{a})$ and $\mathcal{S}$ gets nothing.

\noindent\textbf {Setup:} 
\begin{enumerate}
\item Initializer $\mathcal{T}$ picks a random polynomial $\tau(x)$ of degree $\mathsf{d}$ and sends it to $\mathcal{S}$. 
\item Also, $\mathcal{T}$ picks a random value $\mathsf{v}$ and sends $\mathsf{v}$ and $\tau(\mathsf{v})$ to $\mathcal{R}$.
\end{enumerate}

\noindent\textbf {Computation:} 
\begin{enumerate}
\item $\mathcal{R}$ sends the value $\mathsf{f}=\mathsf{a}-\mathsf{v}$ to $\mathcal{S}$. 
\item Next, $\mathcal{S}$  computes polynomial $\delta(x)=\beta(\mathsf{f}+x)+\tau(x)$ and sends $\delta(x)$ to $\mathcal{R}$.
\item $\mathcal{R}$ computes $\delta(\mathsf{v})-\tau(\mathsf{v})=\mathsf{b}$
 \end{enumerate}
 \
}


\end{boxedminipage}
\end{center}
\caption{Efficient Information Theoretic OPE \cite{DBLP:conf/acisp/HanaokaIMNOW04}.} 
\label{fig:subroutines}
\end{figure}







\begin{figure}[ht]
\setlength{\fboxsep}{2pt}
\begin{center}
\begin{boxedminipage}{11cm}
\small{

\

\noindent\textbf {Parties:} Sender: $\mathcal{S}$, receiver: $\mathcal{R}$ and $\mathsf{k}$ servers: $\{\mathcal{S}_{\scriptscriptstyle {1}},...,\mathcal{S}_{\scriptscriptstyle \mathsf{k}}\}$.

\noindent\textbf {Input:} $\mathcal{S}$ has a polynomial: $\beta(x)=\mathsf{c}_{\scriptscriptstyle 0}+\mathsf{c}_{\scriptscriptstyle 1}x+...+\mathsf{c}_{\scriptscriptstyle \mathsf{d}}x^{\scriptscriptstyle \mathsf{d}}$  of degree at most $\mathsf{d}$. Also, $\mathcal{R}$ has a value:  $\mathsf{a}$.

\noindent\textbf {Output:} $\mathcal{R}$ obtains $\mathsf{b}=\beta(\mathsf{a})$ and $\mathcal{S}$ gets nothing.

\noindent\textbf {Setup:} 
\begin{enumerate}
\item $\mathcal{S}$ picks $\mathsf{d}$ random values: $\{\mathsf{r}_{\scriptscriptstyle 1},...,\mathsf{r}_{\scriptscriptstyle \mathsf{d}}\}$. Then, it computes the following values. $\forall i,1\leq i \leq \mathsf{d}: \mathsf{z}_{\scriptscriptstyle i}=\mathsf{r}_{\scriptscriptstyle i}\cdot \mathsf{c}_{\scriptscriptstyle i}$


\item $\mathcal{S}$  uses ($\mathsf{k},\mathsf{k}$) Shamir secret sharing scheme to split each coefficient $\mathsf{c}_{\scriptscriptstyle j}$,  into $\mathsf{k}$ shares.  In particular, for each $\mathsf{c}_{\scriptscriptstyle j} \in \{\mathsf{c}_{\scriptscriptstyle 0},...,\mathsf{c}_{\scriptscriptstyle \mathsf{d}}\}$, it:

\begin{enumerate}
\item picks a random polynomial $\tau_{\scriptscriptstyle j}(x)$ of degree at most $\mathsf{k}-1$, such that $\tau_{\scriptscriptstyle j}(0)=\mathsf{c}_{\scriptscriptstyle j}$. 
\item evaluates $\tau_{\scriptscriptstyle j}(x)$ at every elements of  vector $\{1,...,\mathsf{k}\}$.  At the end of this process, each $\mathsf{c}_{\scriptscriptstyle j}$ is split into $\{ \tau_{\scriptscriptstyle j}(1),..., \tau_{\scriptscriptstyle j}(\mathsf{k})\}$ shares. 



%$\forall w,\mathbf{1}\leq w \leq \mathbf{k}: \tau_{\scriptscriptstyle j}(w)= {\tau}_{\scriptscriptstyle j,w}$

\end{enumerate}

\item $\mathcal{S}$  splits each value $\mathsf{z}_{\scriptscriptstyle j}$ into $\mathsf{k}$ shares using ($\mathsf{k},\mathsf{k}$) Shamir secret sharing. At the end of this process, each $\mathsf{z}_{\scriptscriptstyle j}$ is split into $\{ \lambda_{\scriptscriptstyle j}(1),..., \lambda_{\scriptscriptstyle j}(\mathsf{k})\}$ shares, where $\lambda_{\scriptscriptstyle j}(x)$ is a random polynomial of degree at most $\mathsf{k}-1$, such that $\lambda_{\scriptscriptstyle j}(0)=\mathsf{z}_{\scriptscriptstyle j}$.

\item $\mathcal{S}$ privately sends to  $\mathcal{R}$ values: $\{\mathsf{r}_{\scriptscriptstyle 1},..., \mathsf{r}_{\scriptscriptstyle \mathsf{d}}\}$. Moreover, $\mathcal{S}$ sends  to each server: $\mathcal{S}_{\scriptscriptstyle  \mathsf{p}}$, two sets of shares (computed above): $\{\tau_{\scriptscriptstyle 0}( \mathsf{p}),..., \tau_{\scriptscriptstyle \mathsf{d}}( \mathsf{p})\}$ and $\{\lambda_{\scriptscriptstyle 1}( \mathsf{p}),..., \lambda_{\scriptscriptstyle \mathsf{d}}( \mathsf{p})\}$, where $1\leq \mathsf{p}\leq \mathsf{k}$. 



\end{enumerate}

\noindent\textbf {Computation:} 
\begin{enumerate}
\item  $\mathcal{R}$ broadcasts to all servers the following set: $\{\mathsf{e}_{\scriptscriptstyle 1 },...,\mathsf{e}_{\scriptscriptstyle \mathsf{d}}\}$ where $\mathsf{e}_{\scriptscriptstyle i}=\mathsf{a}^{\scriptscriptstyle i}-\mathsf{r}_{\scriptscriptstyle i}$


\item Each server: $\mathcal{S}_{\scriptscriptstyle  \mathsf{p}}$, computes the following value: $  \mathsf{u}_{\scriptscriptstyle \mathsf{p}}=\tau_{\scriptscriptstyle 0}( \mathsf{p})+ \sum\limits^{\scriptscriptstyle \mathsf{d}}_{\scriptscriptstyle i=1}(\tau_{\scriptscriptstyle i}( \mathsf{p})\cdot \mathsf{e}_{\scriptscriptstyle i}+\lambda_{\scriptscriptstyle i}( \mathsf{p}))$

\item $\mathcal{R}$, given pairs ($\mathsf{u}_{\scriptscriptstyle \mathsf{p}}, \mathsf{p}$), interpolates a polynomial $\phi(x)$, e.g.  using Lagrange interpolation, and considers the constant coefficient as the result, i.e. $\phi(0)=\mathsf{b}$.


 \end{enumerate}
 \
}


\end{boxedminipage}
\end{center}
\caption{Efficient Information Theoretic distributed OPE \cite{DBLP:conf/icisc/CianciulloG18}.} 
\label{fig:subroutines}
\end{figure}




\subsection{Pseudorandom Permutation} 


A pseudorandom permutation, $\mathtt{PRP}(,)$,









% !TEX root =U-PSI.tex


%\section{Security Definition}
%
%In this section, we provide the security definition of our protocol. There are two kinds of party involved in the protocol. Namely, (1) a set of clients $\{\resizeT {\textit A}_{\resizeS {\textit  1}},..., \resizeT {\textit A}_{\resizeS {\textit  m}}\}$ potentially malicious (i.e. active adversaries) and all may collude with each other, and (2) a non-colluding dealer: client $\resizeT {\textit D}$, potentially semi-honest (i.e. a passive adversary). In this work, we consider static adversary,  we assume there is an authenticated private (off-chain) channel between the clients and we consider a standard public blockchain, e.g. Ethereum.
% !TEX root =main.tex


\vs
\vs
\section{Definition of Multi-party PSI with Fair Compensation}\label{sec::F-PSI-model}%\label{Fair-PSI-Protocol}


\svs


In this section, we present the notion of multi-party PSI with Fair Compensation  (\p) which allows either all clients to get the result or the honest parties to be financially compensated if the protocol aborts in an unfair manner, where only dishonest parties learn the result.  


%We first provide the security model and assumptions used in \p. After that, we provide three subprotocols utilised by a construction, called \withFai (\fpsi), that realises \p; namely, \vopr, \zspa, and its extension \zspaa.  After that, we will give an overview of \fpsi followed by \fpsi's detailed description. 


 
 
 
 
 
 
 
% \subsection{The Model} In this section, we provide the security model of our protocol. In F-PSI, three types of parties are involved; namely, (1) a set of clients $\{A_{\st 1},...,A_{\st m}\}$ potentially \emph{malicious} (i.e., active adversaries) and all but one may collude with each other, (2) a non- colluding dealer, client D, potentially semi-honest (i.e., a passive adversary), and (3) an auditor $Aud$ potentially semi-honest, where all parties except the auditor have input set. For simplicity, we assume that given an address one can determine whether it belongs to an auditor. The basic functionality that a multi-party PSI  computes is defined as $f^{\st\text{PSI}} (S_{\st 1},..., S_{\st m+1})\rightarrow\underbrace{(S_{\st\cap},..., S_{\st\cap})}_{\st m+1}$, where $S_{\st\cap}=S_{\st 1} \cap, ..., \cap S_{\st m+1}$.  To formally define the \emph{fair PSI with compensations}, and add the fairness guarantee, we equip the above PSI functionality with the following predicate triple,  $Q:=(Q^{\st \text{Init}}, Q^{\st \text{Del}}, Q^{\st \text{Abt}})$  that are invoked before functionality $f^{\st \text{PSI}}$ is executed. These predicates were initially proposed in \cite{KiayiasZZ16}; nevertheless, below we will provide more formal accurate definition of them. First, we present a high level description of  these predicates. Predicate $Q^{\st \text{Init}}$ specifies under which condition the protocol should start executing (i.e., when all set owners have enough deposit), $Q^{\st \text{Del}}$ determines the situation where parties receive their output (i.e., when  honest parties receive their deposit back) while $Q^{\st \text{Abt}}$ specifies under which circumstance the simulator can force parties to abort (i.e., when an honest party receives its deposit back plus a predefined amount of compensation). Intuitively, by requiring a fair PSI protocol to implement such a wrapped version of $f^{\st\text{PSI}}$ that includes $Q$, we will ensure that an honest set owner might only abort if $Q^{\st \text{Abt}}$ returns  $1$, and might output a valid value if $Q^{\st \text{Del}}$ returns $1$. Now, we formally define each of these predicates.  
 
 
 
  %\subsection{The Model}\label{sec::F-PSI-model}
  
  
In a  $\mathcal{PSI}^{\st \mathcal{MFC}}$, three types of parties are involved; namely, (1) a set of clients $\{A_{\st 1},...,A_{\st m}\}$ potentially active adversaries and all but one may collude with each other, (2) a non-colluding dealer, $D$, potentially passive adversary and has an input set, and (3) an auditor \aud potentially active adversary, where all parties except \aud have input set. For simplicity, we assume that given an address,  one can determine whether it belongs to \aud. 
% 
The basic functionality that a multi-party PSI computes is defined as $f^{\st\text{PSI}} (S_{\st 1},..., S_{\st m+1})\rightarrow\underbrace{(S_{\st\cap},..., S_{\st\cap})}_{\st m+1}$, where $S_{\st\cap}= S_{\st 1} \cap S_{\st 2}, ...,\cap\  S_{\st m+1}$.  To formally define a \p, we equip $f^{\st\text{PSI}}$ with four predicates,  $Q:=(\qinit, \qdel, \qUnFAbt, \qFAbt)$, which ensure that certain financial conditions are met. 
   %  that are invoked after the functionality $f^{\st \text{PSI}}$ is executed. 
   We borrow three of these predicates (i.e., $\qinit, \qdel, \qUnFAbt$) from \cite{KiayiasZZ16}; nevertheless, we will (i) introduce an additional predicate  \qFAbt and (ii) provide more formal accurate definitions of these predicates. 
   
Predicate \qinit specifies under which condition a protocol that realises \p should start executing, i.e., when all set owners have enough deposit. Predicate \qdel determines in which situation parties receive their output, i.e., when honest parties receive their deposit back. Predicate \qUnFAbt specifies under which condition the simulator can force parties to abort if the adversary learns the output,  i.e., when an honest party receives its deposit back plus a predefined amount of compensation. Predicate \qFAbt specifies under which condition the simulator can force parties to abort if the adversary receives no output, i.e., when honest parties receive their deposits back. 
%
%We observed that the latter predicate should have been defined in the generic framework in \cite{KiayiasZZ16} too; as the framework should have also captured the cases where an adversary may abort without learning any output after the onset of the protocol.  
%
Intuitively, by requiring any protocol that realises \p to implement a wrapped version of $f^{\st\text{PSI}}$ that includes $Q$, we will ensure that an honest set owner only aborts in an unfair manner if \qUnFAbt returns  $1$, it only aborts in a fair manner if \qFAbt returns  $1$, and outputs a valid value if \qdel returns $1$. Now, we formally define each of these predicates.  
 

 \vs
 
 \begin{definition}
 %
  [\qinit: Initiation predicate] Let $\mathcal{G}$ be a stable ledger, $adr_{\st sc}$ be smart contract $sc$'s address, $Adr$ be a set of $m+1$ distinct addresses, and $\xc$ be a fixed amount of coins. Then, predicate $\qinit(\mathcal{G}, adr_{\st sc}, m+1, Adr, \xc)$ returns $1$ if every address in $Adr$ has at least $\xc$ coins in $sc$; otherwise, it returns $0$. 
 %
 \end{definition}

 
 \vs
 \vs
    \begin{definition}  [\qdel:
    %
    Delivery predicate] Let $pram:=(\mathcal{G}, adr_{\st sc}, \xc)$ be the parameters defined above, and   $adr_{\st i}\in Adr$ be the address of an honest party. 
    %
%    Let also $G$ be a compensation function that takes as input  two parameters $(deps, m')$, where $deps$ is the amount of coins  that all $m+1$ parties  deposit; it returns the amount of compensation each honest party must receive, i.e., $G(deps, m')\rightarrow c'$. 
    %
    Then, predicate $\qdel(pram, adr_{\st i})$ returns $1$ if $adr_{\st i}$ has sent $\xc$ amount to $sc$ and received  $\xc$ amount from it; thus,  its balance in $sc$ is $0$. Otherwise, it returns $0$. 
 %
  \end{definition}
 
 
 \vs
 \vs
 
   \begin{definition}  [\qUnFAbt: UnFair-Abort predicate]
   %
 Let $pram:=(\mathcal{G}, adr_{\st sc}, \xc)$ be the parameters defined above, and $Adr'\subset Adr$ be a set containing honest parties' addresses, $m' = |Adr'|$,  and   $adr_{\st i}\in Adr'$. Let also $G$ be a compensation function that takes as input  three parameters $(\depsc, adr_{\st i}, m')$, where $\depsc$ is the amount of coins  that all $m+1$ parties  deposit. It returns the amount of compensation each honest party must receive, i.e., $G(\depsc, ard_{\st i}, m')\rightarrow \xci$. Then, predicate $Q^{\st \text{UnF-Abt}}$ is defined as $\qUnFAbt(pram, G, \depsc, m', adr_{\st i})\rightarrow (a,b)$, where $a=1$ if $adr_{\st i}$ is an honest party's address and $adr_{\st i}$ has sent $\xc$ amount to $sc$ and received  $\xc+\xci$  from it, and $b=1$ if $adr_{\st i}$ is \aud's address and $adr_{\st i}$ received $\xci$  from $sc$. Otherwise, $a=b=0$. 
  %
  \end{definition}
  
  \vs
  \vs
  
\begin{definition}  [\qFAbt: Fair-Abort predicate]
   %
 Let $pram:=(\mathcal{G}, adr_{\st sc}, \xc)$ be the parameters defined above, and $Adr'\subset Adr$ be a set containing honest parties' addresses, $m' = |Adr'|$,     $adr_{\st i}\in Adr'$, and  $adr_{\st j}$ be \aud's address. Let $G$ be the compensation function, defined above and let $G(deps, ard_{\st j}, m')\rightarrow \xc_{\st j}$ be the compensation that the auditor must receive.  Then, predicate $\qFAbt(pram, G, \depsc, m', adr_{\st i},$ $ adr_{\st j})$ returns $1$, if $adr_{\st i}$ (s.t. $adr_{\st i}\neq adr_{\st j}$) has sent $\xc$ amount to $sc$ and received  $\xc$  from it, and $adr_{\st j}$ received $\xc_{\st j}$  from $sc$. Otherwise, it returns $0$. 
  %
 \end{definition}
  
  
  
 
 %Next, we present a formal definition of \p. %Note that we have upgraded the simulation-based definition of secure computation (i.e., Definition \ref{def::MPC-active-adv}) to define the security requirements of \p, by incorporating the above predicates into the definition. 
 
 \vs
 \vs
 
\begin{definition}[\p]\label{def::PSI-Q-fair}
Let $f^{\st \text{PSI}}$ be the multi-party PSI functionality defined above. We say  protocol $\Gamma$ realises  $f^{\st \text{PSI}}$ with $Q$-fairness in the presence of $m-1$ static active-adversary clients (i.e., $A_{\st j}$s) or a static passive dealer $D$ or passive auditor $Aud$, if for every non-uniform probabilistic polynomial time adversary $\mathcal{A}$ for the real model, there exists a non-uniform probabilistic polynomial-time adversary (or simulator) $\mathsf{Sim}$ for the ideal model, such that for every $I\in \{A_{\st 1},...,A_{\st m}, D, Aud\}$, it holds that: 
%
\begin{equation*}
\{\mathsf {Ideal}^{\st \mathcal{W}(f^{\st \text{PSI}},Q)}_{\st \mathsf{Sim}(z), I}(S_{\st 1},..., S_{\st m+1})\}_{\st S_{\st 1},..., S_{\st m+1},z}\stackrel{c}{\equiv} \{\mathsf{Real}_{\st \mathcal{A}(z), I}^{\st \Gamma}(S_{\st 1},..., S_{\st m+1}) \}_{\st S_{\st 1},..., S_{\st m+1},z}
\end{equation*}
where  $z$ is an auxiliary input given to $\mathcal{A}$ and  $\mathcal{W}(f^{\st \text{PSI}},Q)$ is a functionality that wraps $f^{\st \text{PSI}}$ with predicates $Q:=(\qinit, \qdel, \qUnFAbt, \qFAbt)$. 
  \end{definition}
 
%   \begin{definition}  [$Q^{\st \text{Del}}$:
%   %
%    Delivery predicate] Let $pram:=(\mathcal{G}, adr_{\st sc}, c)$ be the parameters defined above, and   $adr_{\st i}\in Adr$ be the address of an honest party. Let also $G$ be a compensation function that takes as input  two parameters $(deps, m')$, where $deps$ is the amount of coins  that all $m+1$ parties  deposit; it returns the amount of compensation each honest party must receive, i.e., $G(deps, m')\rightarrow c'$. Then, predicate $Q^{\st \text{Del}}(pram, G, deps, m', adr_{\st i})$ returns $1$ if $adr_{\st i}$ has sent $c$ amount to $sc$ and received  $c+c'$  from it. Otherwise, it returns $0$. 
% %
%  \end{definition}
 
 
 
%  \begin{definition}  [$Q^{\st \text{Abt}}$: Abort predicate]
% Let $pram:=(\mathcal{G}, adr_{\st sc}, c)$ be the parameters defined above, and $Adr'\subset Adr$ be a set containing honest parties' addresses, $m' = |Adr'|$,  and   $adr_{\st i}\in Adr'$. Let also $G$ be a compensation function that takes as input  three parameters $(deps, adr_{\st i}, m')$, where $deps$ is the amount of coins  that all $m+1$ parties  deposit, $adr_{\st i}$ is an hones party's address, and $m' = |Adr'|$; it returns the amount of compensation each honest party must receive, i.e., $G(deps, ard_{\st i}, m')\rightarrow c_{\st i}$. Then, predicate $Q^{\st \text{Abt}}(pram, G, deps, m', adr_{\st i})$ returns $1$ if $adr_{\st i}$ has sent $c$ amount to $sc$ and received  $c+c'$  from it. Otherwise, it returns $0$. 
%  
%  \end{definition}
 
 %Recall, the standard simulation-based model (presented in Section \ref{}) can adequately capture the security definition of secure multi-party computation and accordingly regular PSI; however, it is not 
 

 

  
% !TEX root =main.tex


\vspace{-3mm}



\section{Other Subroutines Used in \withFai}\label{sec::subroutines}
\vspace{-1mm}

In this section, we present three subroutines and a primitive that we developed and are used in the instantiation of \p, i.e., \withFai. 


\vspace{-3mm}
\subsection{Verifiable Oblivious Polynomial Randomisation (\vopr)}\label{sec::vopr}
\vspace{-1mm}

%In this section, we present ``Verifiable Oblivious Polynomial Randomisation'' (VOPR) protocol. 

In the \vopr, two parties are involved, (i) a sender which is potentially a passive adversary and (ii) a receiver that is potentially an active adversary. The protocol allows the receiver with input polynomial $\bm\beta$ (of degree $e'$) and the sender with input random polynomials $\bm\psi$ (of degree $e$) and  $\bm{\alpha}$ (of degree $e+e'$)   to compute: $\bm\theta=\bm\psi\cdot \bm\beta+\bm\alpha$, such that (a) the receiver learns only $\bm\theta$ and nothing about the sender's input even if it sets $\bm \beta=0$, (b) the sender learns nothing, and (c) the receiver's misbehaviour is detected in the protocol. Thus, the functionality that  \vopr computes is defined as $f^{\st {\vopr}}( (\bm\psi, \bm{\alpha}), \bm\beta)\rightarrow(\bot, \bm\psi\cdot \bm\beta+\bm\alpha)$. 
%
We will use {\vopr} in \withFai for two main reasons:  (a) to let a party re-randomise its counterparty's polynomial (representing its set) and (b) to impose a MAC-like structure to the randomised polynomial; such a structure will allow a verifier to detect if \vopr's output has been modified. 

Now, we outline how we design \vopr without using any (expensive) zero-knowledge proofs.\footnote{Previously, Ghosh \textit{et al.}  \cite{GhoshN19} designed a protocol called Oblivious Polynomial Addition (OPA) to meet similar security requirements that we laid out above. But, as shown in \cite{AbadiMZ21}, OPA  is susceptible to several serious attacks. } In the setup phase, both parties represent their input polynomials in the regular coefficient form; therefore, the sender's polynomials are defined as $\bm\psi=\sum\limits^{\st e}_{\st i=0}g_{\st i}\cdot x^{\st i}$ and  $\bm\alpha=\sum\limits^{\st e+e'}_{\st j=0}a_{\st j}\cdot x^{\st j}$ and the receiver's polynomial is defined as $\bm\beta=\sum\limits^{\st e'}_{\st i=0}b_{\st i}\cdot x^{\st i}$, where $b_{\st i}\neq 0$. However, the sender computes each coefficient $a_{\st j}$ (of polynomial $\bm \alpha$) as follows,  $a_{\st j}=\sum\limits^{\substack{\st k=e'\\ \st t=e}}_{\st t,k=0} a_{\st t,k}$,  where  $t+k=j$ and each $a_{\st t,k}$ is a random value. For instance, if $e=4$ and $e'=3$, then $a_{\st 3}=a_{\st \st 0,3}+a_{\st 3,0}+a_{\st 1,2}+a_{\st 2,1}$. Shortly, we explain why polynomial $\bm\alpha$ is constructed this way. 



In the computation phase,  to compute polynomial $\bm\theta$, the two parties interactively multiply and add the related coefficients in a secure way using $\ole^{\st +}$ (presented in Section \ref{sec::OLE-plus}). Specifically,
%
%For simplicity, let $i=0$. 
%
for every $j$  (where $0\leq j\leq e'$) the sender sends $g_{\st i}$ and $a_{\st i,j}$ to an instance of  $\ole^{\st +}$, while the receiver sends $b_{\st j}$ to the same instance,  which returns $c_{\st i,j}=g_{\st i}\cdot b_{\st j}+ a_{\st i,j}$ to the receiver. This process is repeated for every $i$, where $0 \leq i \leq e$. Then, the receiver uses $c_{\st i,j}$ values to construct the resulting polynomial, $\bm\theta=\bm\psi\cdot \bm\beta+\bm\alpha$.  


The reason that the sender imposes the above structure to (the coefficients of)  $\bm\alpha$ in the setup, is to let the parties securely compute $\bm\theta$ via  $\ole^{\st +}$. Specifically, by imposing this structure (1) the sender  can blind each product $g_{\st i}\cdot b_{\st j}$  with  random value $a_{\st i,j}$ which is a component of $\bm\alpha$'s coefficient and (2) the receiver can construct a result polynomial of the form $\bm\theta=\bm\psi\cdot \bm\beta+\bm\alpha$. 


To check the result's correctness, the sender picks and sends a random value $z$ to the receiver which computes  $\bm\theta(z)$ and $\bm\beta(z)$ and sends these two values  to the sender. The sender computes  $\bm\psi(z)$ and $\bm\alpha(z)$ and then checks if equation  $\bm\theta({ z})=\bm\psi({ z})\cdot \bm\beta({ z})+\bm\alpha({ z})$ holds. It accepts the result if the check passes.   

Figure \ref{fig:VOPR} describes \vopr in detail. Note, \vopr requires the sender to insert non-zero coefficients, i.e., $b_{\st i}\neq 0$ for all $i,0 \leq i \leq e'$. If the   sender inserts a zero-coefficient, then it will learn only a random value (due to  $\ole^{\st +}$), accordingly it cannot pass \vopr's verification phase. However, such a requirement will not affect Justitia's correctness, as we will discuss in Section \ref{Fair-PSI-Protocol} and Appendix \ref{sec::error-prob}.  

\vspace{-2mm}

% !TEX root =main.tex



%%%%%%%%
\begin{figure}[!htb]%[!htbp]
\setlength{\fboxsep}{.8pt}
\begin{center}
\scalebox{.85}{
    \begin{tcolorbox}[enhanced,width=5.5in, 
    drop fuzzy shadow southwest,
    colframe=black,colback=white]
%%%%%%%%

\svs
\begin{enumerate}[leftmargin=1mm]
\small{
\item[$\bullet$] \textit{Input.}
\begin{enumerate}[leftmargin=3mm]
\item[$\bullet$]  \textit{Public Parameters}: upper bound on input polynomials' degree: $e$ and $e'$. %, where $e\geq e'$.
%\item[$\bullet$]  \textit{Sender}: picks an upper bound on input polynomials's degree:  $d$ and $d'$. It sends them to the receiver.
\item[$\bullet$]  \textit{Sender Input}:  random polynomials: $\bm\psi=\sum\limits^{\st e}_{\st i=0}g_{\st i}\cdot x^{\st i}$ and  $\bm\alpha=\sum\limits^{\st e+e'}_{\st j=0}a_{\st j}\cdot x^{\st j}$, where $g_{\st i}\stackrel{\st \$}\leftarrow \mathbb{F}_p$.  Each $a_{\st j}$ has the  form: $a_{\st j}=\sum\limits^{\substack{\st k=e'\\ \st t=e}}_{\st t,k=0} a_{\st t,k}$,  such that $t+k=j$ and $a_{\st t,k}\stackrel{\st \$}\leftarrow \mathbb{F}_p$.

\item[$\bullet$] \textit{Receiver Input}:  polynomial $\bm\beta=\bm\beta_{\st 1}\cdot \bm\beta_{\st 2}=\sum\limits^{\st e'}_{\st i=0}b_{\st i}\cdot x^{\st i}$, where $\bm\beta_{\st 1}$ is a random polynomial of degree $1$ and $\bm\beta_{\st 2}$ is an arbitrary polynomial of degree $e'-1$.


\end{enumerate}
\item[$\bullet$] \textit{Output.} The receiver gets $\bm\theta=\bm\psi\cdot \bm\beta+\bm\alpha$.
\item \textbf{Computation:}

\begin{enumerate} [leftmargin=3mm]

\item Sender and receiver together for every $j$, $0\leq j\leq e'$,  invoke $e+1$ instances of $\ole^{\st +}$. In particular, $\forall j, 0\leq j\leq e'$: sender sends $g_{\st i}$ and $a_{\st i,j}$ while the receiver sends $b_{\st j}$ to $\ole^{\st +}$ that returns: $c_{\st i,j}=g_{\st i}\cdot b_{\st j}+ a_{\st i,j}$ to the receiver ($\forall i, 0\leq i\leq e$). 


 \item The receiver sums component-wise values $c_{\st i,j}$  that results in polynomial:
 %
 \vspace{-2mm}
 %
  $$\bm\theta=\bm\psi\cdot \bm\beta+\bm\alpha=\sum\limits^{\substack{\st i=e\\ \st j=e'}}_{\st i,j=0}c_{\st i, j}\cdot x^{\st i+j}$$ 
 %
  \vspace{-2mm}
  %
 




% \item The receiver sums component-wise values $c_{\st i,j}$  that results polynomial $\bm\theta=\bm\psi\cdot \bm\beta+\bm\alpha=\sum\limits^{\st e+e'}_{\st j=0}c_{\st j}\cdot x^{\st j}$, where  each $c_{\st j}$ has   form: $c_{\st j}=\sum\limits^{\substack{\st k=e'\\ \st t=e}}_{\st  t,k=0} c_{\st t,k}$, such that $ t+k=j$.
%\item Sender: $\forall j, 1\leq j\leq 2d+1$, computes $a_{\st j}=a(x_{\st j})$ and $r_{\st j}=r(x_{\st j})$. Then, it  inserts $(a_{\st j}, r_{\st j})$ into  $\mathcal{F}_{\st OLE^{\st +}}$
%\item\label{computing-receiver} Receiver:  $\forall j, 1 \leq j\leq 2d+1$, computes $b_{\st j}=b(x_{\st j})$. Then, it  inserts $b_{\st j}$ into  $\mathcal{F}_{\st OLE^{\st +}}$ and receives $s_{\st j}=a_{\st j}+b_{\st j}\cdot r_{\st j}$. It interpolates a polynomial $s(x)$ using pairs $s_{\st j},x_{\st j}$. 
\end{enumerate}
\vs
\item \label{Verification} \textbf{Verification:}
\begin{enumerate}[leftmargin=3mm]

\item \label{picking-random-x}Sender: picks a random value  $z$ and sends it to the receiver. 


\item\label{receiver-OLE-invocation} Receiver: sends $\theta_{\st z}=\bm\theta(z)$ and $\beta_{\st z}=\bm\beta(z)$ to the sender.

\item\label{receiver-OLE-invocation} Sender:  computes $\psi_{\st z}=\bm\psi(z)$ and $\alpha_{\st z}=\bm\alpha(z)$ and checks   if equation  $\theta_{\st z}=\psi{\st z}\cdot \beta_{\st z}+\alpha_{\st z}$ holds. If the equation holds, it concludes that the computation was performed correctly. Otherwise, it aborts. 
%
\vs
\end{enumerate}
}
 \end{enumerate}
 \end{tcolorbox}
 }
\end{center}
\vs
\vs
\caption{Verifiable Oblivious Polynomial Randomization ({\vopr}) } 
\label{fig:VOPR}
\end{figure}
 %%%%%%%

%\vspace{-1mm}
\begin{theorem}\label{theorem::VOPR}
%
Let $f^{\st \vopr}$ be the functionality defined above. If the enhanced \ole (i.e., $\ole^{\st +}$) is secure against malicious (or active) adversaries, then the  Verifiable Oblivious Polynomial Randomisation (\vopr), presented in Figure \ref{fig:VOPR}, securely computes $f^{\st \vopr}$ in the presence of (i) semi-honest sender and honest receiver or (ii) malicious receiver and honest sender. 
%
\end{theorem}

\vspace{-1mm}
We refer readers to Appendix \ref{sec::proof-of-vopr} for the proof of Theorem \ref{theorem::VOPR}. 


% !TEX root =main.tex

\subsection{Zero-sum Pseudorandom Values Agreement Protocol (\zspa)}

The \zspa  allows $m$ parties (the majority of which is potentially malicious) to efficiently agree on (a set of vectors, where each $i$-th vector has) $m$ pseudorandom values such that their sum equals zero. At a high level, the parties first sign a smart contract, register their accounts/addresses in it, and then run a  coin-tossing protocol \ct to agree on a key: $k$.  Next, one of the parties generates $m-1$ pseudorandom values $z_{\scriptscriptstyle i, j}$ (where $1\leq j\leq m-1$) using key $k$ and $\mathtt{PRF}$. It sets the last value as the additive inverse of the sum of the values generated, i.e. $z_{\scriptscriptstyle i, m}=-\sum\limits^{\scriptscriptstyle m-1}_{\scriptscriptstyle j=1}z_{\scriptscriptstyle i, j}$ (similar to the standard XOR-based secret sharing \cite{Schneier0078909}). 
%
%Next, it commits to each value, where it uses $k_{\scriptscriptstyle 2}$ to generate the randomness of each commitment. 
%
Then, it constructs a Merkel tree on top of the pseudorandom values and stores only the tree's root $g$ and the key's hash value $q$ in the smart contract.  Then, each party (using the key) locally checks if the values (on the contract) have been constructed correctly; if so, then it sends a (signed) ``approved" message to the contract which only accepts messages from registered parties. Hence, the functionality that \zspa computes is defined as $f^{\st \zspa}\underbrace{(\bot,..., \bot)}_{\st m}\rightarrow \underbrace{((k, g, q),..., (k, g,q))}_{\st m}$, where $g$ is the Markle tree's root built on the pseudorandom values $z_{\st i, j}$, $q$ is the hash value of the key used to generate the pseudorandom values, and $m\geq 2$. Figure \ref{fig:ZSPA} presents \zspa in detail.  


Briefly, \zspa will be used in \withFai to allow clients $\{A_{\st 1},...,A_{\st m}\}$ to provably agree on a set of pseudorandom values, where each set represents a pseudorandom polynomial (as the elements of the set are considered the polynomial's coefficients). Due to \zspa's property, the sum of these polynomials is zero.  Each of these polynomials will be used by a client to blind/encrypt the messages it sends to the smart contract, to protect the privacy of the plaintext message (from \aud, D, and the public). To compute the sum of the plaintext messages, one can easily sum the blinded messages, which removes the blinding polynomials. 

\input{ZSPA-protocol}

\begin{theorem}\label{theorem::ZSPA-comp-correctness}
Let $f^{\st \zspa}$ be the functionality defined above. If \ct is secure against a malicious adversary and the correctness of $\mathtt{PRF}$, $\mathtt{H}$, and Merkle tree holds, then \zspa,  in Figure \ref{fig:ZSPA}, securely computes $f^{\st \zspa}$ in the presence of $m-1 $ malicious  adversaries. 
\end{theorem}


\begin{proof}
For the sake of simplicity, we will assume the sender, which generates the result, sends the result directly to the rest of the parties, i.e., receivers, instead of sending it to a smart contract. We first consider the case in which the sender is corrupt. 

\

\noindent\textbf{Case 1: Corrupt sender.}  Let $\mathsf{Sim}^{\st \zspa}_{\st S}$ be the simulator using a subroutine adversary, $\mathcal{A}_{\st S}$. $\mathsf{Sim}^{\st \zspa}_{\st S}$ works as follows. 
%
\begin{enumerate}
%
\item simulates  \ct  and receives the output value $k$ from $f_{\st \ct}$, as we are in $f_{\st \ct}$-hybrid model.
%
\item sends $k$ to TTP and receives back from it $m$ pairs, where each pair is of the form $( g,  q)$. 
%
\item sends $ k$ to $\mathcal{A}_{\st S}$ and receives back from it $m$ pairs  where each pair is of the form $( g',  q')$. 
%
\item checks whether the following equations hold (for each pair): $ g= g' \hspace{2mm} \wedge  \hspace{2mm}  q= q'$. If the two equations do not hold, then it aborts (i.e., sends abort signal $\Lambda$ to the receiver) and proceeds to the next step.
%
\item outputs whatever $\mathcal{A}_{\st S}$ outputs.
%
 \end{enumerate}
 
 We first focus on the adversary’s output. In the real model, the only messages that the adversary receives are those messages it receives as the result of the ideal call to $f_{\st \ct}$. These messages have identical distribution to the distribution of the messages in the ideal model, as the \ct is secure. Now, we move on to the receiver’s output. We will show that the output distributions of the honest receiver in the ideal and real models are computationally indistinguishable. In the real model,  each element of pair $(g, p)$ is the output of a deterministic function on the output of $f_{\st \ct}$. We know the output of $f_{\st \ct}$ in the real and ideal models have an identical distribution, and so do the evaluations of deterministic functions (i.e., Merkle tree, $\mathtt{H}$, and $\mathtt{PRF}$) on them, as long as these three functions' correctness holds. Therefore, each pair $(g,q)$ in the real model has an identical distribution to pair $(g,  q)$ in the ideal model.  For the same reasons, the honest receiver in the real model aborts with the same probability as  $\mathsf{Sim}^{\st \zspa}_{\st S}$ does in the ideal model.  We conclude that the distributions of the joint outputs of the adversary and honest receiver in the real and ideal models are  (computationally) indistinguishable. 

\


\noindent\textbf{Case 2: Corrupt receiver.}   Let $\mathsf{Sim}^{\st \zspa}_{\st R}$ be the simulator that uses subroutine adversary $\mathcal{A}_{\st R}$. $\mathsf{Sim}^{\st \zspa}_{\st R}$ works as follows. 

\begin{enumerate}
%
\item simulates   \ct  and receives the output value $ k$ from $f_{\st \ct}$.
%
\item sends $ k$ to TTP and receives back $m$ pairs of the form $( g,  q)$ from TTP. 
%
\item sends $( k,  g,  q)$ to $\mathcal{A}_{\st R}$ and outputs whatever  $\mathcal{A}_{\st R}$ outputs. 
%
 \end{enumerate}
 
 
In the real model, the adversary receives two sets of messages, the first set includes the transcripts (including $ k$) it receives when it makes an ideal call to $f_{\st \ct}$ and the second set includes pair $(g, q)$. As we already discussed above (because we are in the  $f_{\st \ct}$-hybrid model) the distributions of the messages it receives from $f_{\st \ct}$ in the real and ideal models are identical. Moreover, the distribution of $f_{\st \ct}$'s output (i.e., $\bar k$ and $k$) in both models is identical; therefore, the honest sender's output distribution in both models is identical. As we already discussed,  the evaluations of deterministic functions (i.e., Merkle tree, $\mathtt{H}$, and $\mathtt{PRF}$) on $f_{\st \ct}$'s outputs have an identical distribution. Therefore, each pair $(g, q)$ in the real model has an identical distribution to the pair $(g, q)$ in the ideal model.  Hence, the distribution of the joint outputs of the adversary and honest receiver in the real and ideal models is indistinguishable.
%
  \hfill\(\Box\)\end{proof}

In addition to the security guarantee (i.e., computation's correctness against malicious sender or receiver) stated by Theorem \ref{theorem::ZSPA-comp-correctness}, \zspa offers  (a) privacy against the public, and (b)  non-refutability. Informally, privacy here means that given the state of the contract (i.e., $g$ and  $q$), an external party cannot learn any information about any of the pseudorandom values,  $z_{\scriptscriptstyle j}$; while non-refutability means that if a party sends ``approved" then in future cannot deny the knowledge of the values whose representation is stored in the contract. %Furthermore, indistinguishability means that every $z_{\scriptscriptstyle j}$ ($1\leq j \leq m$) should be indistinguishable from a truly random value. 




\begin{theorem}
If  $\mathtt{H}$ is preimage resistance, $\mathtt{PRF}$ is secure, the signature scheme used in the smart contract is secure (i.e., existentially unforgeable under chosen message attacks), and the blockchain is secure (i.e., offers persistence and liveness properties \cite{GarayKL15}) then \zspa offers (i) privacy against the public and (ii) non-refutability. 
\end{theorem}
 
 

\begin{proof}
First, we focus on privacy. Since key $k$, for $\mathtt{PRF}$, has been picked uniformly at random and $\mathtt{H}$ is preimage resistance, the probability that given $g$ the adversary can find $k$ is negligible in the security parameter, i.e., $\negl(\lambda)$. Furthermore, because $\mathtt{PRF}$ is secure (i.e., its outputs are indistinguishable from random values) and  $\mathtt{H}$ is preimage resistance, given the Merkle tree's root $g$, the probability that the adversary can find a leaf node, which is the output of $\mathtt{PRF}$, is $\negl(\lambda)$ too. 

Now we move on the non-refutability. Due to the persistency property of the blockchain, once a transaction/message goes more than $v$ blocks deep into the blockchain of one honest player (where $v$ is a security parameter), then it will be included in every honest player's blockchain with overwhelming probability, and it will be assigned a permanent
position in the blockchain (so it will not be modified with an overwhelming probability). Also, due to the liveness property,   all transactions originating from honest parties will eventually end up at a depth of more than $v$ blocks in an honest player's blockchain; therefore, the adversary cannot
perform a selective denial of service attack against honest account holders.  Moreover, due to the security of the digital signature (i.e., existentially unforgeable under chosen message attacks), one cannot deny sending the messages it sent to the blockchain and smart contract. 
%
\hfill\(\Box\)
\end{proof}



%
%
%\begin{theorem}
%If  $\mathtt{H}$ is preimage resistance, $\mathtt{PRF}$ is secure, the signature scheme used in the smart contract is secure (i.e., existentially unforgeable under chosen message attacks), and the blockchain is secure (i.e., offers liveness property and the hash power of the adversary is lower than those of honest miners) then \zspa offers (i) privacy against the public and (ii) non-refutability. 
%\end{theorem}
% 
% 
%
%\begin{proof}
%First, we focus on privacy. Since key $k$, for $\mathtt{PRF}$, has been picked uniformly at random and $\mathtt{H}$ is preimage resistance, the probability that given $g$ the adversary can find $k$ is negligible in the security parameter, i.e., $\negl(\lambda)$. Furthermore, because $\mathtt{PRF}$ is secure (i.e., its outputs are indistinguishable from random values) and  $\mathtt{H}$ is preimage resistance, given the Merkle tree's root $g$, the probability that the adversary can find a leaf node, which is the output of $\mathtt{PRF}$, is $\negl(\lambda)$ too. 
%  \hfill\(\Box\)\end{proof}




%Informally, there are four main security requirements that ZSPA must meet: (a) privacy, (b)  non-refutability, (c) indistinguishability, and (d) result correctness. Privacy here means given the state of the contract, an external party cannot learn any information about any of the (pseudorandom) values:  $z_{\scriptscriptstyle j}$; while non-refutability means that if a party sends ``approved" then in future cannot deny the knowledge of the values whose representation is stored in the contract. Furthermore, indistinguishability means that every $z_{\scriptscriptstyle j}$ ($1\leq j \leq m$) should be indistinguishable from a truly random value and result correctness means that a malicious result generator cannot convince other parties to accept an invalid final result, i.e., the root constructed on the invalid leaf node(s). In Figure \ref{fig:ZSPA}, we provide ZSPA that efficiently generates $b$ vectors where each vector elements is sum to zero. 






%\begin{figure}[ht]
%\setlength{\fboxsep}{0.7pt}
%\begin{center}
%\begin{boxedminipage}{12.3cm}

%
%\begin{figure}[ht]%[!htbp]
%\setlength{\fboxsep}{1pt}
%\begin{center}
%    \begin{tcolorbox}[enhanced,width=5.5in, 
%    drop fuzzy shadow southwest,
%    colframe=black,colback=white]
%
%
%\small{
%
%\begin{enumerate}
%\item[$\bullet$]  \textit{Parties.} $\{\resizeT {\textit A}_{\resizeS {\textit  1}},..., \resizeT {\textit A}_{\resizeS {\textit  m}}\}$
%\item[$\bullet$]  \textit{Input.}  $m$: the total number of participants and a deployed smart contract's address. 
%\item[$\bullet$] \textit{Output.}  $k$: a secret key that generates $b+1$ vectors $[z_{\scriptscriptstyle 0,1},...,z_{\scriptscriptstyle 1,m}],...,[z_{\scriptscriptstyle b,1},...,z_{\scriptscriptstyle b,m}]$ of pseudorandom values, $h$: hash of the key,  $g$: a Merkle tree's root, and a vector of signed messages. 
%
%
%%, such that the sum of each vector's elements equals zero: $\sum\limits^{\scriptscriptstyle m}_{\scriptscriptstyle j=1}z_{\scriptscriptstyle i,j}=0$. 
%
%
%\item All participants run a coin tossing protocol to agree on a key $k$  of $\mathtt{PRF}$.
%\item\label{ZSPA:val-gen} One of the parties:  
%\begin{enumerate}
%
%\item for every $i$ (where $0\leq i \leq b$), generates $m$ pseudorandom values as follows. 
%%
% $$\forall j, 1\leq j \leq m-1: z_{\scriptscriptstyle i,j}=\mathtt{PRF}(k,i||j), \hspace{5mm} z_{\scriptscriptstyle i,m}=-\sum\limits^{\scriptscriptstyle m-1}_{\scriptscriptstyle j=1}z_{\scriptscriptstyle i,j}$$
%%
%\item   constructs a Merkel tree on top of all pseudorandom values,  $\mathtt{MT.genTree}(z_{\scriptscriptstyle 0,1},...,z_{\scriptscriptstyle b,m})\rightarrow g$. 
%
%\item  sends the Merkel tree's root: $g$,   and the key's hash: $q=\mathtt {H}(k)$ to the smart contract. 
%
%\end{enumerate}
%
%\item\label{ZSPA:verify} The rest of parties (given $k_{\scriptscriptstyle 1}, k_{\scriptscriptstyle 2}$) check if, all $z_{\scriptscriptstyle i,j}$ values, the root $g$ and key's hash have been correctly generated (by redoing  step \ref{ZSPA:val-gen}). If the checks pass, each party sends a singed ``approved'' message to the  contract. Otherwise, it aborts. 
%
%
% \end{enumerate}
%}
% \end{tcolorbox}
%\end{center}
%\caption{Zero-sum Pseudorandom Values Agreement (ZSPA) Protocol} 
%\label{fig:ZSPA}
%\end{figure}
%




%%%%%%%%%%%%%%%%%%%%%%%%%%%%%%%%%%%%%%%
%\begin{figure}[ht]
%\setlength{\fboxsep}{0.7pt}
%\begin{center}
%\begin{boxedminipage}{12.3cm}
%
%\small{
%
%\begin{enumerate}
%\item[$\bullet$]  \textit{Parties:} $\{\resizeT {\textit A}_{\resizeS {\textit  1}},..., \resizeT {\textit A}_{\resizeS {\textit  m}}\}$
%\item[$\bullet$]  \textit{Public Parameters and Functions:} A pseudorandom function: $\mathtt{PRF}$, a deployed smart contract, and total number of participants: $m$. 
%\item[$\bullet$] \textit{Output}:  All parties agree on $b+1$ vectors $[z_{\scriptscriptstyle 0,1},...,z_{\scriptscriptstyle 1,m}],...,[z_{\scriptscriptstyle b,1},...,z_{\scriptscriptstyle b,m}]$, of pseudorandom values, such that the sum of each vector's elements equals zero: $\sum\limits^{\scriptscriptstyle m}_{\scriptscriptstyle j=1}z_{\scriptscriptstyle i,j}=0$
%
%
%\item All participants run a coin tossing protocol to agree on two keys $k_{\scriptscriptstyle 1}$ and $k_{\scriptscriptstyle 2}$ of $\mathtt{PRF}$.
%\item\label{ZSPA:val-gen} One of the parties:  
%\begin{enumerate}
%
%\item For every $i$, computes $m$ pseudorandom values: $\forall j, 1\leq j \leq m-1: z_{\scriptscriptstyle i,j}=\mathtt{PRF}(k_{\scriptscriptstyle 1},i||j)$ and sets $z_{\scriptscriptstyle i,m}=-\sum\limits^{\scriptscriptstyle m-1}_{\scriptscriptstyle j=1}z_{\scriptscriptstyle i,j}$, where $0\leq i \leq b$
%
%\item   commits to every $z_{\scriptscriptstyle i,j}$  as follows: $\mathtt{a}_{\scriptscriptstyle i,j}=\mathtt{Com}(z_{\scriptscriptstyle i,j}, q_{\scriptscriptstyle i,j})$, where the randomness of the commitment is computed as: $ q_{\scriptscriptstyle i,j}=\mathtt{PRF}(k_{\scriptscriptstyle 2},i||j)$ and  $1\leq j \leq m$.
%
%\item   constructs a Merkel tree on top of the committed values: $\mathtt{MT}(\mathtt{a}_{\scriptscriptstyle 0,1},...,\mathtt{a}_{\scriptscriptstyle b,m})\rightarrow g$ 
%
%\item  sends the Merkel tree's root: $g$,   and the keys' hashes: $\mathtt {H}(k_{\scriptscriptstyle 1})$ and $ \mathtt {H}(k_{\scriptscriptstyle 2})$, to the contract. 
%
%\end{enumerate}
%
%\item\label{ZSPA:verify} The rest of parties (given $k_{\scriptscriptstyle 1}, k_{\scriptscriptstyle 2}$) check if, all $z_{\scriptscriptstyle i,j}$ values, the root $g$ and keys' hashes have been correctly generated (by redoing  step \ref{ZSPA:val-gen}). If passed, each party sends a singed ``approved'' message to the  contract. Otherwise, it aborts. 
%
%
% \end{enumerate}
%}
%\end{boxedminipage}
%\end{center}
%\caption{Zero-sum Pseudorandom Values Agreement ($\mathtt{ZSPA}$) Protocol} 
%\label{fig:ZSPA}
%\end{figure}




%% !TEX root =main.tex




\begin{figure}[ht]%[!htbp]
\setlength{\fboxsep}{1pt}
\begin{center}
\scalebox{.85}{
    \begin{tcolorbox}[enhanced,width=5.5in, 
    drop fuzzy shadow southwest,
    colframe=black,colback=white]


{\small{

%\underline{$\mathtt{Audit}( \vv{{k}},  q, \bm\zeta, \bar d, g, \vv v)\rightarrow (L, \vv{{\mu}})$}
\begin{enumerate}
%\item[$\bullet$] Parties: clients: $\{  {   A}_{    {    1}},...,   {   A}_{    {    m}}\}$, the dealer and  an Arbiter.


\item[$\bullet$]    {Parties.} A set of clients $\{ A_{\st 1},...,  A_{\st m}\}$ and an external auditor, \aud. 

\item[$\bullet$]    {Input.}  $m$: the total number of participants (excluding the auditor), $\bm\zeta$: a random polynomial of degree $1$, $b$: the total number of vectors, and $adr$: a deployed smart contract's address. Let $b'=b-1$.





%\item[$\bullet$]   {Input.} $\vv{{k}}=[k_{\st 1},..., k_{\st m}]$,    $q$: a  hash value, $\bm\zeta$: a random polynoimal of degree $1$, $\bar d$: a polynoimal's degree,   $g$: a root of Merkle tree, and $\vv v$: binary vector of size $m$. 


\item[$\bullet$]  {Output of  each} $  A_{\st j}$.   $k$: a secret key that generates $b$ vectors $[z_{\scriptscriptstyle 0,1},...,z_{\scriptscriptstyle 0,m}],...,[z_{\scriptscriptstyle b',1},...,z_{\scriptscriptstyle b', m}]$ of pseudorandom values, $h$: hash of the key,  $g$: a Merkle tree's root, and a vector of signed messages. 



\item[$\bullet$]    {Output of \aud.} $L$: a list of misbehaving parties' indices, and  $\vv{{\mu}}$: a vector of random polynomials.
%
\item\label{ZSPA::ZSPA-invocation} {\textbf{\zspa invocation.}  $\zspa(\bot,..., \bot)\rightarrow \Big((k, g, q),..., (k, g,q )\Big)$}. 

All parties in $\{A_{\st 1},...,  A_{\st m}\}$ call the same instance of \zspa, which results in  $(k, g, q), ..., (k, g, q)$. 
%

\item\label{ZSPA-A::Auditor-computation}  {\textbf{Auditor computation.} $\mathtt{Audit}( \vv{{k}},  q, \bm\zeta, b, g)\rightarrow (L, \vv{{\mu}})$}. 

\aud\ takes the below steps. Note,  each $k_{\st j}\in \vv{{k}}$ is given by  $  A_{\st j}$. An honest party's input, $k_{\st j}$,  equals $k$, where $1\leq j \leq m$. 


\begin{enumerate}
%
\item runs the checks in the verification phase (i.e., Phase \ref{ZSPA:verify}) of \zspa for every $j$, i.e., $\mathtt{Verify}(k_{\st j}, g, q, m)\rightarrow (a_{\st j}, s)$.
\item appends $j$ to $L$, if any checks fails, i.e., if $a_{\st j}=0$. In this case, it skips the next two steps for the current $j$. 



%
%
%\item  Checks whether equation $\mathtt{H}(k_{\st j})=q$ holds  for every $j$, $1\leq j \leq m$.   
%%
%\begin{itemize}
%%
%\item[$\bullet$] if any $j$-th check fails,  it adds $j$ to $L$.
%%
%\item[$\bullet$]  if $L$ contains all $j\in[1,m]$, it returns $L$ and aborts. 
%%
%\end{itemize}
%%
%\item\label{zero-sum-arbiter-verification} Verifies the Merkle tree's root, $g$, by checking if the tree (corresponding to  $g$) has been correctly constructed on the correct leaf nodes. In particular, it takes the following steps. 
%
%\begin{enumerate}
%
%\item regenerates the tree's leaf nodes (similar to step \ref{ZSPA:val-gen} in Fig. \ref{fig:ZSPA}) as follows. Let $k$ be a key that passed the above check.  For every $i$ (where $0\leq i \leq \bar d$), it recomputes $m$ pseudorandom values: 
%%
%$$\forall j, 1\leq j \leq m-1: z_{\st i,j}=\mathtt{PRF}(k,i||j), \hspace{4mm} z_{\st i,m}=-\sum\limits^{\st m-1}_{\st j=1}z_{\st i,j}$$
%%
%\item   constructs a Merkel tree on top of all pseudorandom values generated in the previous step, i.e., $\mathtt{MT.genTree}(z_{\st 0,1},...,z_{\st \bar d,m})\rightarrow g'$. 
%%
%\item checks if $g=g'$. If the equation does not hold, then it adds to $L$ every index $j$ whose value in $\vv v$ is $1$, i.e., $\vv v[j]=1$; in this case, it returns $L$ and aborts.
%%
%\end{enumerate}
%

\item\label{ZSPA-A::gen-z} For every $i$ (where $0\leq i \leq b'$), it recomputes $m$ pseudorandom values: 
%
$\forall j, 1\leq j \leq m-1: z_{\st i,j}=\mathtt{PRF}(k,i||j), \hspace{4mm} z_{\st i,m}=-\sum\limits^{\st m-1}_{\st j=1}z_{\st i,j}$.
%
 \item generates polynomial $\bm\mu^{\st (j)}$ as follows: 
  %
   $\bm\mu^{\st (j)} = \bm\zeta\cdot \bm\xi^{\st (j)}-\bm\tau^{\st (j)}$, 
   %
    where $\bm\xi^{\st (j)}$ is a random polynomial of degree $b'-1$ and $\bm\tau^{\st (j)}=\sum\limits^{\st b'}_{\st i=0}z_{\st i,j}\cdot x^{\st i}$. By the end of this step, a vector $\vv{{\mu}}$ containing at most $m$ polynomials is generated. 
%
 \item returns   list $L$ and $\vv{{\mu}}$.
 
\end{enumerate}
 \end{enumerate}
}}
 \end{tcolorbox}
 }
\end{center}
\caption{\zspa with an external auditor (\zspaa)} 
\label{fig:arbiter}
\end{figure}



%%%%%%%%%%%%%%%%%%%%%%%%%%%%%%%%%%%%%%%%%%%%%%
%\begin{figure}[ht]%[!htbp]
%\setlength{\fboxsep}{1pt}
%\begin{center}
%    \begin{tcolorbox}[enhanced,width=5.5in, 
%    drop fuzzy shadow southwest,
%    colframe=black,colback=white]
%
%
%{\small{
%
%\underline{$\mathtt{Audit}( \vv{{k}},  q, \bm\zeta, \bar d, g, \vv v)\rightarrow (L, \vv{{\mu}})$}
%\begin{enumerate}
%%\item[$\bullet$] Parties: clients: $\{  {   A}_{    {    1}},...,   {   A}_{    {    m}}\}$, the dealer and  an Arbiter.
%\item[$\bullet$]   {Input.} $\vv{{k}}=[k_{\st 1},...,k_{\st m}]$,    $q$: a  hash value, $\bm\zeta$: a random polynoimal of degree $1$, $\bar d$: a polynoimal's degree,   $g$: a root of Merkle tree, and $\vv v$: binary vector of size $m$. 
%
%
%\item[$\bullet$]    {Output.} A list of rejected values' indices: $L$, a vector of random polynomials: $\vv{{\mu}}$.
%%
%\item  Checks whether equation $\mathtt{H}(k_{\st j})=q$ holds  for every $j$, $1\leq j \leq m$.   
%%
%\begin{itemize}
%%
%\item[$\bullet$] if any $j$-th check fails,  it adds $j$ to $L$.
%%
%\item[$\bullet$]  if $L$ contains all $j\in[1,m]$, it returns $L$ and aborts. 
%%
%\end{itemize}
%%
%\item\label{zero-sum-arbiter-verification} Verifies the Merkle tree's root, $g$, by checking if the tree (corresponding to  $g$) has been correctly constructed on the correct leaf nodes. In particular, it takes the following steps. 
%
%\begin{enumerate}
%
%\item regenerates the tree's leaf nodes (similar to step \ref{ZSPA:val-gen} in Fig. \ref{fig:ZSPA}) as follows. Let $k$ be a key that passed the above check.  For every $i$ (where $0\leq i \leq \bar d$), it recomputes $m$ pseudorandom values: 
%%
%$$\forall j, 1\leq j \leq m-1: z_{\st i,j}=\mathtt{PRF}(k,i||j), \hspace{4mm} z_{\st i,m}=-\sum\limits^{\st m-1}_{\st j=1}z_{\st i,j}$$
%%
%\item   constructs a Merkel tree on top of all pseudorandom values generated in the previous step, i.e., $\mathtt{MT.genTree}(z_{\st 0,1},...,z_{\st \bar d,m})\rightarrow g'$. 
%%
%\item checks if $g=g'$. If the equation does not hold, then it adds to $L$ every index $j$ whose value in $\vv v$ is $1$, i.e., $\vv v[j]=1$; in this case, it returns $L$ and aborts.
%%
%\end{enumerate}
%%
% \item Generates polynomial $\bm\mu^{\st (j)}$, for every $j$ such that $j\in[1,m]$ and $j \notin L$,  as follows:
%  %
%   $$\bm\mu^{\st (j)} = \bm\zeta\cdot \bm\xi^{\st (j)}-\bm\tau^{\st (j)}$$
%   %
%    where $\bm\xi^{\st (j)}$ is a random polynomial of degree $\bar d-1$ and $\bm\tau^{\st (j)}=\sum\limits^{\st \bar d}_{\st i=0}z_{\st i,j}\cdot x^{\st i}$. By the end of this step, a vector $\vv{{\mu}}$ containing at most $m$ polynomials is generated. 
%%
% \item Returns   list $L$ and $\vv{{\mu}}$.
% 
%
% \end{enumerate}
%}}
% \end{tcolorbox}
%\end{center}
%\caption{$\text{Audit}$ Algorithm} 
%\label{fig:arbiter}
%\end{figure}


% !TEX root =main.tex




\vs




\subsection{\zspa's Extension: \zspa with an External Auditor (\zspaa)}


In this section, we present an extension of \zspa, called \zspaa which lets a (trusted) third-party auditor, \aud, help identify misbehaving clients in the \zspa and generate a vector of random polynomials. Informally, \zspaa requires that misbehaving parties are always detected, except with a negligible probability. \aud of this protocol will be invoked by \withFai when \withFai's smart contract detects that a combination of the messages sent by the clients is not well-formed. Later, in \withFai's proof, we will show that even a \emph{semi-honest} \aud who observes all messages that clients send to \withFai's smart contracts, cannot learn anything about their set elements. We present \zspaa in Figure \ref{fig:arbiter}. 


\vs


\input{Arbiter-algorithm}




\begin{theorem}\label{theorem::ZSPA-A}
If \zspa is secure, $\mathtt{H}$ is second-preimage resistant, and the correctness of $\mathtt{PRF}$, $\mathtt{H}$, and Merkle tree holds,  then \zspaa securely computes $f^{\st \zspaa}$ in the presence of $m-1 $ malicious adversaries.% or (ii) a semi-honest auditor. 
\end{theorem}

\svs

We refer readers to Appendix \ref{sec::proof-of-zspaa} for the proof of Theorem \ref{theorem::ZSPA-A}. 

%As we stated previously, the ZSPA-A protocol will be invoked as a subroutine in the fair PSI protocol. As part of proving Theorem \ref{theorem::ZSPA-A}, we would like to show that the semi-honest auditor's view can be simulated (so it cannot learn the parties' set elements), even if it has access to those transcripts of the fair PSI protocol sent to the smart contract; because such an approach offers a stronger security guarantee than proving the ZSPA-A protocol in isolation.  Therefore, we will present the proof of Theorem \ref{theorem::ZSPA-A} after we present the fair PSI protocol. 





%\begin{theorem}\label{theorem::ZSPA-comp-correctness}
%If the coin-tossing protocol is secure against a malicious adversary, then the ZSPA protocol,  in Figure \ref{fig:ZSPA}, securely computes $f^{\st \text {ZSPA}}$ in the presence of a malicious adversary. 
%\end{theorem}


%\begin{figure}%[ht]
%\setlength{\fboxsep}{0.7pt}
%\begin{center}
%\begin{boxedminipage}{12.3cm}
%
%\small{
%
%\begin{enumerate}
%\item[$\bullet$] Parties: clients: $\{\resizeT {\textit A}_{\resizeS {\textit  1}},..., \resizeT {\textit A}_{\resizeS {\textit  m}}\}$, the dealer and  an Arbiter.
%\item[$\bullet$] Input: Empty malicious clients list: $L$ and a deployed smart contract's address. 
%\item[$\bullet$] Output: Misbehaving clients list: $L$
%\item Every client sends to the Arbiter  two keys: $k_{\scriptscriptstyle 1}, k_{\scriptscriptstyle 2}$, used to generate the zero-sum values and their commitments. 
%%
%\item  The Arbiter checks if the clients  provided correct keys, by ensuring that the keys' hashes matches the ones stored in the contract. It appends the IDs of those  provided inconsistent keys to $L$. If all clients provided inconsistent keys it aborts. Otherwise, it proceed to the next step where it uses correct keys: $k_{\scriptscriptstyle 1}, k_{\scriptscriptstyle 2}$. 
%%
%\item\label{zero-sum-arbiter-verification} The Arbiter (given correct keys) regenerate the  zero-sum values $z_{\scriptscriptstyle i, j}$ and verify the correctness of their commitments and the Merkel tree root contracted on top of the commitments, i.e. takes the same step as step \ref{ZSPA:verify} in Fig \ref{fig:ZSPA}.   It aborts if any of the   checks is rejected, and appends to $L$ the IDs of the clients which sent the ``approved'' message to the contract. 
%%
% \item The Arbiter for each client $\resizeT {\textit C}$, who provided correct keys,  generates polynomial $\bm\mu^{\resizeS {\textit {(C)}}}$, for each bin, as follows:
%  %
%   $$\bm\mu^{\resizeS {\textit {(C)}}} = \bm\zeta\cdot \bm\xi^{\resizeS {\textit {(C)}}}-\bm\tau^{\resizeS {\textit {(C)}}}$$
%   %
%    where $\bm\xi^{\resizeS {\textit {(C)}}}$ is a random polynomial of degree $3d+1$ and $\bm\tau^{\resizeS {\textit {(C)}}}=\sum\limits^{\st 3d+2}_{\st i=0}z_{\st i,c}\cdot x^{\st i}$. By the end of this step, a vector $\vv{\bm{\mu}}$ containing polynomial $\bm\mu^{\resizeS {\textit {(C)}}}$ for every bin of client $\resizeT {\textit C}$ that is not in list $L$. 
%    %
%     \item returns   list $L$ and $\vv{\bm{\mu}}$.
 
%
%
% \item The dealer, for each client $\resizeT {\textit C}\in \{\resizeT {\textit A}_{\resizeS {\textit  1}},..., \resizeT {\textit A}_{\resizeS {\textit  m}}\}$,  sends to the Arbiter a blind polynomial of the form: $\bm\zeta\cdot \bm\eta^{\resizeS {\textit {(D,C)}}}-(\bm\gamma^{\resizeS {\textit {(D,C)}}}+\bm\delta^{\resizeS {\textit {(D,C)}}})$, where $\bm\eta^{\resizeS {\textit {(D,C)}}}$ is a fresh random degree $3d+1$ polynomial. The blind polynomial will allow the arbiter to obliviously verify the correctness of the message each client sent to the  contract. 
% 
% \item The Arbiter for each client $\resizeT {\textit C}$ who provided correct keys: 
% 
% \begin{enumerate}
% \item adds together the blind polynomial above and the blind polynomial $\bm\nu^{\resizeS {\textit {(C)}}}$ the client sent to the contract (in step \ref{blindPoly-C-sends-to-contract} in the PSI protocol). Then, it removes the client's zero-sum pseudorandom values from the result. In particular, it computes:    
%\begin{equation*}
%\begin{split}
% \bm\iota^{\resizeS {\textit {(C)}}}&=\bm\zeta\cdot \bm\eta^{\resizeS {\textit {(D,C)}}}-(\bm\gamma^{\resizeS {\textit {(D,C)}}}+\bm\delta^{\resizeS {\textit {(D,C)}}})+\bm\nu^{\resizeS {\textit {(C)}}}-\sum\limits^{\scriptscriptstyle 3d+1}_{\scriptscriptstyle i=0}z_{\scriptscriptstyle i,c}\cdot x^{\scriptscriptstyle i} \\ &=\bm\zeta\cdot(\bm\eta^{\resizeS {\textit {(D,C)}}} + \bm\omega^{\resizeS {\textit {(D,C)}}}\cdot \bm\omega^{\resizeS {\textit {(C,D)}}}\cdot \bm\pi^{\resizeS {\textit {(C)}}}+\bm\rho^{\resizeS {\textit {(D,C)}}}\cdot \bm\rho^{\resizeS {\textit {(C,D)}}}\cdot \bm\pi^{\resizeS {\textit {(D)}}})
% \end{split}
%\end{equation*}
%  \item checks if $\bm\zeta$ can divide $\bm\iota^{\resizeS {\textit {(C)}}}$. If can not, it appends the client's ID to $L$.
%  \end{enumerate}
  %$deg(\eta^{\resizeS {\textit {D,C}}})=3d+1$
% \end{enumerate}
%}
%\end{boxedminipage}
%\end{center}
%\caption{$\mathtt{Arbiter}$ Protocol} 
%\label{fig:arbiter}
%\end{figure}









  
% !TEX root =main.tex




\vs 




\subsection{Unforgeable Polynomials}


In this section, we introduce the notion of ``unforgeable polynomials''. Informally, an unforgeable polynomial has a secret factor. To ensure that an unforgeable polynomial has not been tampered with, a verifier can check whether the polynomial is divisible by the secret factor. 


To turn an arbitrary polynomial $\bm\pi$ of degree $d$ into an unforgeable polynomial $\bm\theta$, one can (i) pick three secret random polynomials $(\bm\zeta, \bm\omega, \bm \gamma)$ and (ii) compute $\bm\theta=\bm\zeta\cdot \bm\omega\cdot\bm \pi + \bm \gamma \bmod p$, where  $deg(\bm\zeta)= 1, deg(\bm\omega)=d,$ and   $deg(\bm\gamma)= 2d+1$. 
%
To verify whether $\bm\theta$ has been tampered with, a verifier (given $\bm\theta, \bm \gamma$, and $\bm\zeta$) can check if $\bm\theta-\bm \gamma$ is divisible by $\bm\zeta$. The security of \emph{unforgeable polynomial} states that an adversary (who does not know the three secret random polynomials) cannot tamper with an unforgeable polynomial without being detected, except with a negligible probability, in the security parameter. Below, we formally state it. 




\begin{theorem}[Unforgeable Polynomial]\label{proof::unforgeable-poly}
%Let polynomials $\zeta$ and $\gamma$ be two secret uniformly random polynomials (i.e., $\zeta, \gamma\stackrel{\st\$}\leftarrow \mathbb F_{\st p}[x]$),   $GCD(\zeta, \gamma)=1$, polynomial $\pi$ be an arbitrary polynomial,   $deg(\zeta)= 1, deg(\gamma)= d+1$,  $deg(\pi)=d$, and $p$ be a $\lambda$-bit prime number. Also, let polynomial $\theta$ be defined as  $\theta=\zeta\cdot \pi+ \gamma \bmod p$. Given $(\theta,\pi)$, the probability that a PPT adversary (which does not know $\zeta$ and $\gamma$) can forge $\theta$ to an arbitrary polynomial $\theta'$ such that  $\theta'\neq \theta$, $deg(\theta')\leq poly(\lambda)$, and $\zeta$ divides $\theta'-\gamma$ is negligible in the security parameter, i.e.,
%
Let polynomials $\bm\zeta$, $\bm\omega$, and $\bm\gamma$ be three secret uniformly random polynomials (i.e., $\bm\zeta,\bm\omega, \bm\gamma\stackrel{\st\$}\leftarrow \mathbb F_{\st p}[x]$),   $GCD(\bm\zeta, \bm\gamma)=1$, polynomial $\bm\pi$ be an arbitrary polynomial,   $deg(\bm\zeta)= 1, deg(\bm\omega)=d,  deg(\bm\gamma)= 2d+1$,  $deg(\bm\pi)=d$, and $p$ be a $\lambda$-bit prime number. Also, let polynomial $\bm\theta$ be defined as  $\bm\theta=\bm\zeta\cdot \bm\omega\cdot\bm \pi+\bm \gamma \bmod p$. Given $(\bm\theta,\bm\pi)$, the probability that an adversary (which does not know $\bm\zeta, \bm\omega$, and $\bm\gamma$) can forge $\bm\theta$ to an arbitrary polynomial $\bm\delta$ such that  $\bm\delta\neq \bm\theta$, $deg(\bm\delta)= const(\lambda)$, and $\bm\zeta$ divides $\bm\delta-\bm\gamma$ is negligible in the security parameter $\lambda$, i.e., 
%
$Pr[ \bm\zeta \ | \ (\bm\delta-\bm\gamma) ]\leq \negl(\lambda)$.
%
\end{theorem}

\vs
\svs
\begin{proof}

Let $\bm\tau=\bm\delta-\bm\gamma$ and $\bm\zeta=a\cdot x+b$. Since $\bm\gamma$ is a random polynomial of degree $2d+1$ and unknown to the adversary, given $(\bm\theta, \bm\pi)$,  the adversary cannot learn anything about the factor $\bm\zeta$; as from its point of view every polynomial of degree $1$ in $\mathbb{F}_{\st p}[X]$ is equally likely to be $\bm\zeta$. Moreover,  polynomial $\bm\tau$ has at most $Max\big(deg(\bm\delta), 2d+1\big)$ irreducible non-constant factors.  For $\bm\zeta $ to divide $\bm\tau$,  one of the factors of $\bm\tau$ must be equal to $\bm\zeta$. We  also know that $\bm\zeta$ has been picked uniformly at random (i.e., $a,b
\stackrel{\st \$}\leftarrow \mathbb F_{\st p}$) and by definition $GCD(\bm\zeta, \bm\gamma)=1$. Thus, the probability that $\bm\zeta $ divides $\bm\tau$ is negligible in the security parameter, $\lambda$. Specifically, $Pr[ \bm\zeta \ | \ (\bm\delta-\bm\gamma)]\leq \frac{Max\big(deg(\bm\delta), 2d+1\big)} {2^{\st 2\lambda}}=\negl(\lambda)$. 
\hfill\(\Box\)\end{proof} 

%$Max\big(deg(\theta'), d+1\big)$
 An interesting feature of an unforgeable polynomial is that the verifier can perform the check without needing to know the original polynomial $\bm\pi$. Another feature of the unforgeable polynomial is that it supports \emph{linear combination} and accordingly \emph{batch verification}. Specifically, to turn $n$ arbitrary polynomials $[\bm\pi_{\st 1},..., \bm\pi_{\st n}]$ into unforgeable polynomials, one can construct  $\bm\theta_{\st i}=\bm\zeta\cdot \bm\omega_{\st i}\cdot \bm\pi_{\st i}+ \bm\gamma_{\st i} \bmod p$, where $\forall i, 1\leq i\leq n$.  
 
 

 
To check whether all polynomials $[\bm\theta_{\st 1},..., \bm\theta_{\st n}]$ are intact, a verifier can (i) compute their sum $\bm \chi=\sum\limits_{\st i=1}^{\st n}\bm\theta_{\st i}$ and (ii) check whether $\bm \chi- \sum\limits_{\st i=1}^{\st n}\bm\gamma_{\st i} $ is divisible by $\bm \zeta$.  Informally, the security of \emph{unforgeable polynomials' linear combination} states that an adversary (who does not know the three secret random polynomials for each $\bm\theta_{\st i}$) cannot tamper with any subset of the unforgeable polynomials without being detected, except with a negligible probability. We formally state it, below. 
 
 

%%%%%%%%%%%%%%%%%%%%%%%%%%%%%%%

\begin{theorem}[Unforgeable Polynomials' Linear Combination]\label{Unforgeable-Polynomials-Linear-Combination}
%
 Let polynomial $\bm\zeta$ be a secret polynomial picked uniformly at random; also, let   $\vv{\bm\omega}=[\bm\omega_{\st 1},..., \bm\omega_{\st n}]$ and $\vv{\bm\gamma}=[\bm\gamma_{\st 1},..., \bm\gamma_{\st n}]$ be two vectors of secret uniformly random polynomials (i.e., ${\bm\zeta}, \bm\omega_{\st i}, \bm\gamma_{\st i} \stackrel{\st\$}\leftarrow \mathbb F_{\st p}[x]$), $GCD(\bm\zeta,  \bm\gamma_{\st i})=1$,  $\vv{\bm\pi}=[\bm\pi_{\st 1},..., \bm\pi_{\st n}]$ be a vector of arbitrary polynomials,   $deg(\bm\zeta)= 1, deg(\bm\omega_{\st i})=d,  deg(\bm\gamma_{\st i})= 2d+1$,  $deg(\bm\pi_{\st i})=d$,  $p$ be a $\lambda$-bit prime number, and $1\leq i \leq n$. Moreover, let polynomial $\bm\theta_{\st i}$ be defined as  $\bm\theta_{\st i}=\bm\zeta\cdot \bm\omega_{\st i}\cdot \bm\pi_{\st i}+ \bm\gamma_{\st i} \bmod p$, and $\vv{\bm\theta} = [\bm\theta_{\st 1},..., \bm\theta_{\st n}]$.  Given $(\vv{\bm\theta}, \vv{\bm\pi})$, the probability that an adversary (which does not know $\bm\zeta, \vv{\bm\omega}$, and $\vv{\bm\gamma}$) can forge $t$ polynomials, without loss of generality, say $\bm\theta_{\st 1},..., \bm\theta_{\st t} \in \vv{\bm\theta}$ to arbitrary polynomials $\bm\delta_{\st 1},..., \bm\delta_{\st t}$ such that   $\sum\limits_{\st j=1}^{\st t}\bm\delta_{\st j}\neq \sum\limits_{\st j=1}^{\st t}\bm\theta_{\st j}$, $deg(\bm\delta_{\st j})= const(\lambda)$, and $\bm\zeta$ divides $(\sum\limits_{\st j=1}^{\st t}\bm\delta_{\st j} + \sum\limits_{\st j=t+1}^{\st n}\bm\theta_{\st j} - \sum\limits_{\st j=1}^{\st n}\bm\gamma_{\st j} )$ is negligible in the security parameter $\lambda$, i.e.,  
%
$Pr[ \bm\zeta \ | \ (\sum\limits_{\st j=1}^{\st t}\bm\delta_{\st j} + \sum\limits_{\st j=t+1}^{\st n}\bm\theta_{\st j} - \sum\limits_{\st j=1}^{\st n}\bm\gamma_{\st j} ) ]\leq \negl(\lambda)$.
%
\end{theorem}

\vs
\svs

%%%%
\begin{proof}  
This proof is a generalisation of that of Theorem \ref{proof::unforgeable-poly}.  
Let $\bm\tau_{\st j}=\bm\delta_{\st j}-\bm\gamma_{\st j}$ and $\bm\zeta=a\cdot x+b$. Since  every $\bm\gamma_{\st j}$ is a random polynomial of degree $2d+1$ and unknown to the adversary, given $(\vv{\bm\theta}, \vv{\bm\pi})$,  the adversary cannot learn anything about the factor $\bm\zeta$. Each polynomial $\bm\tau_{\st j}$ has at most $Max\big(deg(\bm\delta_{\st j}), 2d+1\big)$ irreducible non-constant factors. 
%
%In order for $\bm\zeta$ to divide polynomial $\sum\limits_{\st j=1}^{\st t}\bm\delta_{\st j} + \sum\limits_{\st j=t+1}^{\st n}\bm\theta_{\st j} - \sum\limits_{\st j=1}^{\st n}\bm\gamma_{\st j}$  one of the factors of every $\bm\tau_{\st j}$ needs to equal $\bm\zeta$, where $1 \leq j \leq t$. 
%
We  know that $\bm\zeta$ has been picked uniformly at random (i.e., $a,b
\stackrel{\st \$}\leftarrow \mathbb F_{\st p}$), by definition $GCD(\bm\zeta, \bm\gamma_{\st j})=1$, and $\bm\zeta$  does divide every $\bm\theta_{\st j}$. Therefore, the probability that $\bm\zeta$ divides $\sum\limits_{\st j=1}^{\st t}\bm\delta_{\st j} + \sum\limits_{\st j=t+1}^{\st n}\bm\theta_{\st j} - \sum\limits_{\st j=1}^{\st n}\bm\gamma_{\st j}$ equals the probability that $\bm\zeta$ equals to one of the factors of  every $\bm\tau_{\st j}$, that is negligible in the security parameter. Concretely,
%
$$Pr[ \bm\zeta \ | \ (\sum\limits_{\st j=1}^{\st t}\bm\delta_{\st j} + \sum\limits_{\st j=t+1}^{\st n}\bm\theta_{\st j} - \sum\limits_{\st j=1}^{\st n}\bm\gamma_{\st j} ) ]\leq  \frac{\prod \limits^{\st t}_{\st j=1}Max\big(deg(\bm\delta_{\st j}), 2d+1\big)} {2^{\st 2\lambda t}}=\negl(\lambda)$$
%
\hfill\(\Box\)
\end{proof} 

\svs

Briefly, in \withFai, we will use unforgeable polynomials (and their linear combinations) to allow a smart contract to efficiently check whether the polynomials that the clients send to it are intact, i.e., they are \vopr's outputs.











\vs
\vs

\section{\withFai: A Concrete Construction of \p}


\vs

\subsection{Main Challenges to Overcome}
\svs

 We need to address several key challenges, to design an efficient scheme that realises \p. Below, we outline these challenges.
 
 \vs
 
 
 \subsubsection{Keeping Overall Complexities Low.}
 
 In general, in multi-party PSIs, each client needs to send messages to the rest of the clients and/or engage in secure computation with them, e.g., in \cite{DBLP:conf/scn/InbarOP18,DBLP:conf/ccs/KolesnikovMPRT17}, which yields communication and/or computation quadratic with the number of clients. To address this challenge, we  (a) let one of the clients as a dealer interact with the rest,\footnote{This approach has similarity with the non-secure PSIs in \cite{GhoshN19}.} and   (b) use a smart contract, to which most messages are sent and also performs lightweight computations. These approaches will keep the communication and computation linear with the number of clients (and sets' cardinality). 
 
 

 
% \subsubsection{Securely Randomising Input Polynomials.}  In multi-party PSIs that rely on the polynomial representation, it is essential that an input polynomial of a client be randomised by another client \cite{AbadiMZ21}. To do that securely and efficiently, we require the dealer and each client together to engage in an instance of \vopr, which we developed in Section \ref{sec::subroutines}. 
 
 \vs
 \vs
 
 \subsubsection{Preserving the Privacy of Outgoing Messages.} Although the use of public smart contracts will help keep  complexities low, it introduces another challenge; namely, if clients do not protect the privacy of the messages they send to the smart contracts, then other clients (e.g., dealer) and non-participants of PSI (i.e., the public) can learn the clients' set elements and/or the intersection. Because standard smart contracts do not automatically preserve messages' privacy. To efficiently protect the privacy of each client's messages (sent to the contracts) from the dealer, we require the clients (except the dealer) to engage in \zspaa which lets each of them generate a pseudorandom polynomial with which it can blind its message. To protect the privacy of the intersection from the public, we require all clients to run a coin-tossing protocol to agree on a blinding polynomial, with which the final result that encodes the intersection on the smart contract will be blinded.  
 
 
 \vs
 \vs
 
 \subsubsection{Ensuring the Correctness of Subroutine Protocols' Outputs.} 
 
 In general, any MPC that must remain secure against an active adversary is equipped with a verification mechanism that ensures an adversary is detected if it deviates from the protocol and affects messages' integrity, during the protocol's execution. This is the case for the subroutine protocols that we use, i.e., \vopr and \zspaa. Nevertheless, this type of verification itself is not always sufficient. Because in certain cases, the output of an MPC protocol may be fed as input to another MPC and we need to ensure that the \emph{actual/intact} output of the first MPC is fed to the second one. This is the case in our PSI's subroutines as well. To address this challenge, we use unforgeable polynomials; specifically, the output of \vopr is an unforgeable polynomial (that encodes the actual output); if the adversary tampers with the \vopr's output and uses it later, then a verifier can detect it. We will have the same integrity guarantee for the output of \zspaa for free. Because (i) \vopr is called before \zspaa, and (ii) if clients use intact outputs of \zspaa, then the final result (i.e., the sum of all clients' messages) will not contain any output of \zspaa, as they would cancel out each other. Thus, by checking the correctness of the final result, one can ensure the correctness of the outputs of \vopr and \zspaa, in one go. 
 
 \vs
 \vs
  
\subsection{Description of \withFai (\fpsi)}\label{Fair-PSI-Protocol}
%This section presents \fpsi, a protocol that realises \p. 

\subsubsection{An overview.} At a high level, \withFai (\fpsi) works as follows. First, each client encodes its set elements into a polynomial. All clients sign a smart contract and deposit a predefined amount of coins into it.  Next,  one of the clients as a dealer, $D$, randomises the rest of the clients' polynomials and imposes a certain structure to their polynomials. The clients also randomise $D$'s polynomials. The randomised polynomials reveal nothing about the clients' original polynomials representing their set elements. Then, all clients send their randomised polynomials to the smart contract.  The contract combines all polynomials and checks whether the resulting polynomial still has the structure imposed by $D$. If the contract concludes that the resulting polynomial does not have the structure, then it invokes an auditor, \aud, to identify misbehaving clients and penalise them. Nevertheless, if the resulting polynomial has the structure, then the contract outputs an encoded polynomial and refunds the clients' deposits. In this case, all clients can use the resulting polynomial (output by the contract) to locally find the intersection.  





%%%%%%%%%%%%


One of the novelties of \fpsi is a lightweight verification mechanism which allows a smart contract to efficiently verify the correctness of the clients' messages without being able to learn/reveal the clients' set elements. To achieve this, $D$ randomises each client's polynomials and constructs unforgeable polynomials on the randomised polynomials (in one go). If any client modifies an unforgeable polynomial that it receives and sends the modified polynomial to the smart contract,  then the smart contract would detect it, by checking whether the sum of all clients' (unforgeable) polynomials is divisible by a certain polynomial of degree $1$.  The verification is lightweight because: (i) it does not use any public key cryptography (often computationally expensive), (ii) it needs to perform only polynomial division, and (iii) it can perform batch verification, i.e., it sums all clients randomised polynomials and then checks the result's correctness.


%%%%%%%%%%%%




%
%One of the novelties of F-PS is a lightweight verification mechanism which allows a smart contract to efficiently verify the correctness of the clients' messages without being able to learn/reveal the clients' set elements. To achieve this, the dealer during randomising other clients' polynomials, imposes a MAC-like structure on the randomised polynomials, such that if a client (who receives its randomised polynomial) tampers with it, then the smart contract would detect it. To do the verification, the smart contract needs to only check whether the sum of all clients' randomised polynomials is divisible by a polynomial of degree $1$.  The verification is lightweight because: (i) it does not rely on any public key cryptography (i.e., zero-knowledge proofs), (ii) it needs to perform only polynomial division, and (iii) it can perform batch verification, i.e., instead of individually checking each client's randomised polynomial, it sums all clients randomised polynomials (related to a hash table's bin) and then checks the result's correctness.


% his own
%outsourced dataset and having any knowledge of the other client’s dataset 
%
%
% mainly stems from our observation (stated  in Theorem \ref{proof::unforgeable-poly}) which leads to an  efficient verification mechanism carried out by the contract. 


Now, we describe \fpsi in more detail. First, all clients sign and deploy a  smart contract, \scf. Each of them put a certain amount of deposit into it. Then, they together run \ct to agree on a key, $mk$, that will be used to generate a set of blinding polynomials to hide the final result from the public. Next, each client locally maps its set elements to a hash table and represents the content of each hash table's bin as a polynomial, $\bm\pi$. After that, for each bin, the following steps are taken.  All clients, except $D$, engage in \zspaa to agree on a set of pseudorandom blinding factors such that their sum is zero.  %The clients will use these polynomials to hide from $D$ the polynomials that they will send to \scf. 

Then, $D$ randomises and constructs an unforgeable polynomial on each client's polynomial, $\bm\pi$. To do that, $D$ and every other client engage in \vopr that returns to the client a polynomial. $D$ and every other client invoke \vopr again to randomise $D$'s polynomial. \vopr returns to the client another unforgeable polynomial. Note that the output of \vopr reveals nothing about any client's original polynomial $\bm\pi$, as this polynomial has been blinded with another random polynomial by $D$, during the execution of \vopr. Each client sums the two polynomials,  blinds the result (using the output of  \zspaa), and sends it to \scf. 



After all of the clients send their input polynomials to \scf, $D$ sends to \scf a \emph{switching polynomial} that will allow \scf to obliviously switch the secret blinding polynomials used by $D$ (during the execution of \vopr) to blind each client's original polynomial $\bm\pi$  to another blinding polynomial that all clients can generate themselves, by using key $mk$.  The switching polynomial is constructed in a way that does not affect the verification of unforgeable polynomials. 




Next, $D$ sends to \scf a secret polynomial, $\bm\zeta$, that will let \scf check unforgeable polynomials' correctness. \scf\ sums all clients' polynomials and checks if $\bm\zeta$ can divide the sum. \scf\ accepts the clients' inputs if the polynomial divides the sum; otherwise, it invokes \aud to identify misbehaving parties.  In this case, all honest parties' deposit is refunded and the deposit of misbehaving parties is distributed among the honest ones as well. If all clients behave honestly,  then each client can locally find the intersection. To do that, it uses $mk$ to locally remove the blinding polynomial from the sum (that the contract generated), then evaluates the unblinded polynomial at each of its set elements and considers an element in the intersection when the evaluation equals zero. Figure \ref{fig:parties-interactions-in-Jus}, in Appendix \ref{sec::Workflow-of-withFai}, outlines the interaction between parties.
%The efficiency of the verification in our protocol  mainly stems from our  observation that if an adversary who know only $xx$ modified the polynomial of the form $xx$ then $\zeta$ will not divide result polynomial after unblinding will not divide with a high probability.  


\vs

\subsubsection{Detailed Description of \fpsi.} Below, we elaborate on how \fpsi exactly works (see Table \ref{table:notation-table} for description of the main notations used). 

\vs

\begin{enumerate}[leftmargin=4mm]

%\item[$\bullet$] \textbf{Input:} a pseudorandom function: $\mathtt{PRF}$, a hash table's parameters (i.e., the  total number of bins: $h$ and a bin's capacity: $d$), and clients' sets: $S^{\st (I)}$, where $I\in \bar{P}$.

%\item[$\bullet$] \textbf{Output:}  for every bin of the hash table, it outputs a polynomial: $\phi$, whose roots are  encrypted sets elements (of the bin) in the intersection.
\item\label{gen-FPSI-cont} All clients in $\cl=\{ A_{\st 1},...,   A_{\st m},  D\}$ sign a smart contract: \scf and deploy it to a blockchain. All clients get the deployed contract's address. Also, all clients engage in \ct to agree on a secret master key, $mk$.

\item \label{encode-encrypt} Each client in $\cl$  builds a  hash table,  $\mathtt{HT}$, and inserts the set elements into  it, i.e.,  $\forall i: \mathtt{H}( s_{\st i})={indx}$, then $ s_{\st i}\rightarrow \mathtt{HT}_{\st indx}$. It pads every bin with random dummy elements to $d$ elements (if needed). Then,  for every bin, it constructs a polynomial whose roots are the bin's content: $\bm\pi=\prod\limits^{\st d}_{\st i=1} (x-s'_{\st i})$, where $s'_{\st i}$ is either $s_{\st i}$ or a random value. 
%
\item \label{ZSPA} Every client $ C$ in $\cl\setminus D$, for every bin, agree on $b=3d+3$ vectors of pseudorandom blinding factors: $z_{\st i,j}$, such that the sum of each vector elements is zero, i.e., $\sum\limits^{\st m}_{\st j=1}z_{\st i,j}=0$, where $0\leq i\leq b-1$. To do that, they participate in step \ref{ZSPA::ZSPA-invocation} of \zspaa. By the end of this step, for each bin, they agree on a secret key $k$ (that will be used to generate the zero-sum values) as well as two values stored in $\mathcal{SC}_{\fpsi}$, i.e., $q$: the key's hash value and $g$: a Merkle tree's root. After time $t_{\st 1}$,  $D$ ensures that all other clients have agreed on the vectors (i.e., all provided ``approved''  to the contract). If the check fails, it halts. 
%
\item\label{F-PSI::each-client-deposit} Each client in $\cl$ deposits $\yc+\chc$ amount to $\mathcal{SC}_{\fpsi}$. After time $t_{\st 2}$, every client ensures that in total $(\yc+\chc)\cdot (m+1)$ amount has been deposited. Otherwise, it halts and the clients' deposit is refunded. 







\item  $D$ picks a  random polynomial $\bm\zeta \stackrel{\st\$}\leftarrow \mathbb{F}_{\st p}[X]$ of degree $1$, for each bin.  
It, for each client $C$, allocates to each bin two random polynomials: $\bm\omega^{\st(D,C)}, \bm\rho^{\st (D,C)}\stackrel{\st\$}\leftarrow \mathbb{F}_{\st p}[X]$ of degree $d$, and  two  random polynomials: $\bm\gamma^{\st (D,C)}, \bm\delta^{\st (D,C)} \leftarrow \mathbb{F}_{\st p}[X]$ of degree $3d+1$. Also, each client $C$, for each bin, picks two  random polynomials: $\bm\omega^{\st (C,D)}, \bm\rho^{\st (C,D)}\stackrel{\st\$}\leftarrow \mathbb{F}_{\st p}[X]$ of degree $d$. %It also evaluates each polynomial at every element of $\vv{\bm{x}}$ that results in  $\omega^{  {  {D,C}}}_{\st i}$ and $\rho^{  {  {D,C}}}_{\st i}$.




\item\label{e-psi::D-randomises}  $D$ randomises other clients' polynomials. To do so, for every bin, it invokes an instance of {\vopr} (presented in Fig. \ref{fig:VOPR}) with  each client $  C$; where  $D$ sends $\bm\zeta \cdot \bm\omega^{\st  {  {(D,C)}}}$ and $\bm\gamma^{\st  {  {(D,C)}}}$, while client $ C$ sends $\bm\omega^{\st  {  {(C,D)}}}\cdot \bm\pi^{\st  {  {(C)}}}$ to {\vopr}. Each client $C$, for every bin, receives a blind polynomial of the following form: 
%
$$\bm\theta^{  {  {(C)}}}_{\st 1}=\bm\zeta \cdot \bm\omega^{\st  {  {(D,C)}}}\cdot \bm\omega^{\st  {  {(C,D)}}}\cdot \bm\pi^{\st  {  {(C)}}}+\bm\gamma^{\st  {  {(D,C)}}}$$
%
 from {\vopr}. If any party aborts, the deposit would be refunded to all parties.

\item\label{e-psi::C-randomises} Each client $    {  C}$ randomises  $ {D}$'s polynomial. To do that, each client $    {  C}$, for each bin,  invokes an instance of {\vopr} with   $ {D}$,    where each client $    {  C}$  sends $\bm\rho^{\st  {  {(C,D)}}}$, while  ${D}$  sends $\bm\zeta\cdot\bm \rho^{\st  {  {(D,C)}}}\cdot \bm\pi^{  {  {(D)}}}$ and $\bm\delta^{\st  {  {(D,C)}}}$ to {\vopr}. Every client   $    {  C}$, for each bin,  receives a blind polynomial of the following form: 
%
$$\bm\theta^{  {  {(C)}}}_{\st 2}=\bm\zeta \cdot \bm\rho^{\st  {  {(D,C)}}}\cdot \bm\rho^{\st  {  {(C,D)}}}\cdot \bm\pi^{\st  {  {(D)}}}+\bm\delta^{\st  {  {(D,C)}}}$$
 from {\vopr}. If any party aborts, the deposit would be refunded to all parties.


\item\label{blindPoly-C-sends-to-contract} Each client $ C$, for every bin, masks the sum of polynomials $\bm\theta^{\st  {  {(C)}}}_{\st 1}$ and $\bm\theta^{\st  {  {(C)}}}_{\st 2}$  using the blinding factors: $z_{\st i,c}$, generated in step \ref{ZSPA}. Specifically, it computes the following blind polynomial (for every bin):  
%
$$\bm\nu^{ \st {  {(C)}} }= \bm\theta^{ \st {  {(C)}}}_{\st 1}+\bm\theta^{\st  {  {(C)}}}_{\st 2}+\bm \tau^{\st  {  {(C)}} }$$

where $\bm\tau^{\st  {  {(C)}}}=\sum\limits^{\st 3d+2}_{\st i=0}z_{\st i,c}\cdot x^{\st i}$. Next, it sends  all $\bm\nu^{\st  {  {(C)}} }$ to $\mathcal{SC}_{\fpsi}$. If any party aborts, the deposit would be refunded to all parties.


%\item Client $    {  D}$ ensures all clients have sent their inputs to $\mathcal{SC}_{  {   {F-PSI}}}$. In the case where $m'$ parties do not provide their inputs, client $    {  D}$ aborts. In this case, the rest (including the dealer) get their deposit back. Also,  the deposit of the parties who did not send  inputs would be evenly distributed among the rest. The total amount each party above receives is: $y+\frac{m'\cdot y}{m-m'}$




%%
%\item Client $    {  D}$ and each client $    {  C}$ collaboratively, for each bin, generate a polynomial that will be used to (obliviously) check if  $    {  C}$ misbehaved during the computation of each $\bm\nu^{  {  {(C)}} }$. To do so, for every bin, client $    {  D}$ invokes an instance of $\mathtt{VOPR}$ with  each client $    {  C}$, where  client $    {  D}$ sends: $\bm\zeta$, while client $    {  C}$ sends $\bm\xi^{  {  {(C)}}}$ and $-\bm\tau^{  {  {(C)}}}$   to $\mathtt{VOPR}$, where $\bm\xi^{  {  {(C)}}}$ is a random polynomial of degree $3d+1$. Client $    {  D}$ for each  $    {  C}$'s bin recives the following polynomial: 
%%
%$$\bm\mu^{  {  {(D,C)}}} = \bm\zeta\cdot \bm\xi^{  {  {(C)}}}-\bm\tau^{  {  {(C)}}}$$
%%




\item\label{f-psi::D-gen-random-poly} ${D}$ ensures all clients sent their inputs to $\mathcal{SC}_{\fpsi}$. If the check fails, it halts and the deposit would be refunded to all parties. It allocates a fresh pseudorandom polynomial $\bm\gamma'$ of degree $3d$, to each bin. To do so, it uses $mk$ to derive a key for each bin: $k_{\st  { {indx}}}=\mathtt{PRF}(mk, {    {   {indx}}})$ and then uses the derived key to generate $3d+1$ pseudorandom coefficients $g_{\st  { {j,indx}}}=\mathtt{PRF}(k_{\st  { {indx}}}, j)$ where $ 0\leq j \leq 3d$. Also, for each bin, it allocates a fresh random polynomial:  $\bm\omega'^{\st  {  {(D)}}}$ of degree $d$. 

\item\label{f-psi::D-gen-switching-poly}  $ {D}$,  for every bin, computes a polynomial of the form:  
%
$$\bm\nu^{\st  {  {(D)}}}=\bm\zeta \cdot  \bm\omega'^{\st  {  {(D)}}}\cdot \bm\pi^{\st  {  {(D)}} }-\sum\limits^{\st  {   A}_{\st  {   m}}}_{\st   {  {C }= }   {   A}_{\st  {  1}}}(\bm\gamma^{\st  {  {(D,C)}}} + \bm\delta^{\st  {  {(D,C)}}}) + \bm\zeta \cdot \bm\gamma'$$ 
It sends to $\mathcal{SC}_{\fpsi}$  polynomials $\bm\nu^{\st  {  {(D)}}}$ and $\bm\zeta$, for each bin.

 \item\label{compute-res-poly}  $\mathcal{SC}_{\fpsi}$ takes the following steps:
 \begin{enumerate}[leftmargin=1mm]
 \item for every bin, sums all related polynomials  provided by all clients in $\bar{P}$:
 %
 \begin{equation*}
\begin{split}
\hspace{-9mm} \bm\phi&= \bm\nu^{\st  {  {(D)}} }+\sum\limits^{\st  {   A}_{\st  {   m}}}_{\st   {  {C }= }   {   A}_{\st  {  1}}}\bm\nu^{\st  {  {(C)}} }\\
 &= \bm\zeta\cdot \bigg(\bm\omega'^{\st  {  {(D)}}}\cdot \bm\pi^{\st  {  {(D)}} } +\sum\limits^{\st  {   A}_{  {   m}}}_{\st  {  {C }= }   {   A}_{\st  {  1}}}(\bm\omega^{\st  {  {(D,C)}}} \cdot \bm\omega^{\st  {  {(C,D)}}}\cdot \bm\pi^{\st  {  (C)}}) +\bm\pi^{\st  {  {(D)}}}\cdot\sum\limits^{\st  {   A}_{ \st {   m}}}_{\st  {C= }   {   A}_{\st  {  1}}}(\bm\rho^{\st  {  {(D,C)}}} \cdot \bm\rho^{\st  {  {(C,D)}}}) + \bm\gamma'\bigg)
  \end{split}
\end{equation*}
% \item\label{F-PSI:detect-misbehaving-party} ensures that, for every bin, $\bm\zeta$ divides $\bm\phi$. Otherwise, it aborts and Arbiter protocol (presented in Fig. \ref{fig:arbiter}) is invoked to find misbehaving parties.
 
  \item\label{F-PSI:detect-misbehaving-party} checks whether, for every bin, $\bm\zeta$ divides $\bm\phi$. If the check passes, it sets $Flag=True$. Otherwise, it sets $Flag=False$. 
  
   %aborts and Arbiter protocol (presented in Fig. \ref{fig:arbiter}) is invoked to find misbehaving parties.
 
 
% \item if the check passes (i.e., $Flag=True$), each party gets back its deposit (i.e., $y$ amount).
 \end{enumerate}
 
\item\label{F-PSI::flag-is-true} If the check passes (i.e., $Flag=True$), then the following steps are taken:

\begin{enumerate}[leftmargin=2mm]
 \item $\mathcal{SC}_{\fpsi}$ sends back each party's deposit, i.e., $\yc+\chc$ amount.
 
  \item each client (given $\bm\zeta$ and $mk$) finds the elements in the intersection as follows. 
  \begin{enumerate}
  \item derives a bin's pseudorandom polynomial, $\bm\gamma'$, from $mk$. 
  
  \item removes the blinding polynomial from each bin's polynomial: 
  %
  $$\bm\phi'=\bm\phi-\bm\zeta\cdot \bm\gamma'$$ 
  
  \item\label{F-PSI::find-intersection} evaluates each bin's unblinded polynomial at every element $s_{\st i}$ belonging to that bin and considers the element in the intersection if the evaluation is zero: i.e., $\bm\phi'(s_{\st i})=0$.
 
 \end{enumerate}
 
 
 \end{enumerate}
 
 \item\label{F-PSI::flag-is-false} If the check does not pass (i.e., $Flag=False$), the following steps are taken.
 
 

 
 \begin{enumerate}[leftmargin=2mm]
 

 \item\label{auditor}  \aud asks every  ${  C}$ to send to it the  $\mathtt{PRF}$'s key (generated in step \ref{ZSPA}), for every bin. It inserts the keys to $\vv k$.  It generates a list $\bar L$ initially empty. Then, for every bin,  \aud takes step \ref{ZSPA-A::Auditor-computation} of \zspaa, i.e., invokes  $\mathtt{Audit}( \vv{{k}},  q, \bm\zeta, 3d+3, g)\rightarrow (L, \vv{{\mu}})$.  Every time it invokes $\mathtt{Audit}$, it appends the elements of returned $L$ to $\bar L$.  \aud for each bin sends  $ \vv{{\mu}}$ to $\mathcal{SC}_{\fpsi}$. It also sends  to $\mathcal{SC}_{\fpsi}$ the list $\bar L$ of all misbehaving clients detected so far.
 

 
 \item to  help identify further  misbehaving clients, $D$ takes the following steps,  for each bin of client $    {  C}$ whose ID is not in $\bar L$.   
 \begin{enumerate}
 \item\label{gen-unmaking-poly} computes polynomial $\bm\chi^{  {  {(D, C)}}}$ as follows. 
 %
 $$\bm\chi^{ \st {  {(D, C)}}}=\bm\zeta\cdot \bm\eta^{ \st {  {(D,C)}}}-(\bm\gamma^{ \st {  {(D,C)}}}+\bm\delta^{ \st {  {(D,C)}}})$$
 
 %+\bm\mu^{  {  {(D, C)}}}
 
  where $\bm\eta^{ \st {  {(D,C)}}}$ is a fresh random polynomial of degree $3d+1$. 
  
  \item\label{send-unmaking-poly} sends  polynomial $\bm\chi^{ \st {  {(D, C)}}}$ to  $\mathcal{SC}_{\fpsi}$. 
  

 \end{enumerate}
  Note, if $\bar L$ contains all clients' IDs, then $D$ does not need to take the above steps \ref{gen-unmaking-poly} and \ref{send-unmaking-poly}. 
 %%%%%%%%%%%%%%%%%%%%%%
 
 \item  $\mathcal{SC}_{\fpsi}$,   takes the following steps to check if the client misbehaved,  for each bin of client $    {  C}$ whose ID is not in $\bar L$.
 
 %for each client $    {  C}$'s bin, takes the following steps to check if the client misbehaved.
 
  \begin{enumerate}
  
 \item computes  polynomial $\bm\iota^{\st  {  {(C)}}}$ as follows: 
 %
  \begin{equation*}
\begin{split}
 \bm\iota^{ \st {  {(C)}}}&=\bm\chi^{\st  {  {(D, C)}}}+\bm\nu^{\st  {  {(C)}}} +\bm\mu^{ \st {  {(C)}}} \\ 
 &=\bm\zeta\cdot(\bm\eta^{ \st {  {(D,C)}}} + \bm\omega^{ \st {  {(D,C)}}}\cdot \bm\omega^{ \st {  {(C,D)}}}\cdot \bm\pi^{ \st {  {(C)}}}+\bm\rho^{ \st {  {(D,C)}}}\cdot \bm\rho^{ \st {  {(C,D)}}}\cdot \bm\pi^{ \st {  {(D)}}}+\bm\xi^{ \st {  {(C)}}})
 \end{split}
\end{equation*}

 where $\bm\mu^{ \st {  {(C)}}} \in \vv{\mu}$ generated and sent to $\mathcal{SC}_{\fpsi}$  by \aud in step \ref{auditor}.   
  \item checks if $\bm\zeta$  divides $\bm\iota^{ \st {  {(C)}}}$. If the check fails, it appends the client's ID to  a list $ L'$.
  %
  \end{enumerate}
   If $\bar L$ contains all clients' IDs, then $\mathcal{SC}_{\fpsi}$ does not take the above two steps. 

 %
   \item  $\mathcal{SC}_{\fpsi}$  refunds the honest parties' deposit. Also, it retrieves the total amount of  $\chc$ from the deposit of dishonest clients (i.e., those clients whose IDs are in $\bar L$ or $L'$) and sends it to \aud.  It also splits the remaining deposit of the misbehaving parties among the honest ones. Thus, each honest client  receives $\yc+\chc+\frac{m'\cdot (\yc+\chc)-\chc}{m-m'}$ amount in total, where $m'$ is the total number of misbehaving parties.
 
 
%  \item  $\mathcal{SC}_{  {   {F-PSI}}}$  refunds the honest parties' deposit and splits the deposit of the misbehaving parties (i.e., those clients whose IDs are in $\bar L$ or $L'$)  among the honest ones. Thus, each honest party would receive $y+\frac{m'\cdot y}{m-m'}$ amount in total, where $m'$ is the total number of misbehaving parties.
 %%%%%%%%%%%%%%%%%%%%%
  \end{enumerate}
  
% \item If  $Flag=False$,  then $\mathcal{SC}_{  {   {F-PSI}}}$,  for each client $    {  C}$'s bin, takes the following steps:
% 
%  \begin{enumerate}
%  
% \item computes the following polynomial: 
% 
%  \begin{equation*}
%\begin{split}
% \bm\iota^{  {  {(C)}}}&=\bm\chi^{  {  {(D, C)}}}+\bm\nu^{  {  {(C)}}} \\ 
% &=\bm\zeta\cdot(\bm\eta^{  {  {(D,C)}}} + \bm\omega^{  {  {(D,C)}}}\cdot \bm\omega^{  {  {(C,D)}}}\cdot \bm\pi^{  {  {(C)}}}+\bm\rho^{  {  {(D,C)}}}\cdot \bm\rho^{  {  {(C,D)}}}\cdot \bm\pi^{  {  {(D)}}}+\bm\xi^{  {  {(C)}}})
% \end{split}
%\end{equation*}
% 
%  \item checks if $\bm\zeta$  divides $\bm\iota^{  {  {(C)}}}$. If does not, it appends the client's ID to a  list, $L$.
%  
%  \end{enumerate}
 
% \begin{equation*}
%\begin{split}
% \bm\iota^{  {  {(C)}}}&=\bm\zeta\cdot \bm\eta^{  {  {(D,C)}}}-(\bm\gamma^{  {  {(D,C)}}}+\bm\delta^{  {  {(D,C)}}})+\bm\nu^{  {  {(C)}}}-\sum\limits^{\st 3d+1}_{\st i=0}z_{\st i,c}\cdot x^{\st i} \\ &=\bm\zeta\cdot(\bm\eta^{  {  {(D,C)}}} + \bm\omega^{  {  {(D,C)}}}\cdot \bm\omega^{  {  {(C,D)}}}\cdot \bm\pi^{  {  {(C)}}}+\bm\rho^{  {  {(D,C)}}}\cdot \bm\rho^{  {  {(C,D)}}}\cdot \bm\pi^{  {  {(D)}}})
% \end{split}
%\end{equation*}
%  \item checks if $\bm\zeta$ can divide $\bm\iota^{  {  {(C)}}}$. If can not, it appends the client's ID to $L$.
% 
 

 
 
 
% \item Each client (given $\bm\zeta$ and $k_{\st 1}$), finds the elements in the intersection as follows. First, it derives a bin's pseudorandom polynomial: $\bm\gamma'$ from $k_{\st 1}$.  Next, it removes the blinding polynomial from each bin's polynomial: $\bm\phi'=\bm\phi-\bm\zeta\cdot \bm\gamma'$. Then, it evaluates each bin's unblinded polynomial at every  element belonging to that bin and considers the element in the intersection if the evaluation is zero: i.e. $\bm\phi'(s^{  {  {(I)}}}_{\st i})=0$
 
  \end{enumerate}
 
% the result: $cccc$ by locally evaluating the result polynomial: $\phi(x)$, at every  encrypted element, $e^{  {  {(I)}}}_{\st i}$, it has and considering the elements in the intersection if the following equation holds.  $\forall i, 1\leq i\leq d: \phi(e^{  {  {(I)}} }_{\st i})-\zeta(e^{  {  {(I)}}}_{\st i})\cdot \gamma'(e^{  {  {(I)}}}_{\st i})=0$.
 
 
 
%\begin{remark} After the Arbiter detects  misbehaving parties,  in step \ref{F-PSI:detect-misbehaving-party},  it sends their ID's to $\mathcal{SC}_{  {   {F-PSI}}}$ which refunds the honest parties' deposit and splits the misbehaving parties' deposit among the honest ones. Thus, each honest party would receive: $y+\frac{m'\cdot y}{m-m'}$ amount in total, where $m'$ is the total number of misbehaving parties.
% \end{remark}
 
 

%\begin{remark}
One may be tempted to replace $\withFai$ with a scheme in which all clients send their encrypted sets to a server (potentially semi-honest and plays \aud's role) which computes the result in a privacy-preserving manner.  We highlight that the main difference is that in this (hypothetical) scheme the server is \emph{always involved};  whereas, in our protocol, \aud remains offline as long as the clients behave honestly and it is invoked only when the contract detects misbehaviours.  
%\end{remark}
 
 
 Next, we present a theorem that formally states the security of \fpsi. We refer readers to Appendix \ref{sec::F-PSI-proof} for the proof of this theorem. 
 
 \vs
 
 \begin{theorem}\label{theorem::F-PSI-security}
Let polynomials $\bm\zeta$, $\bm\omega$, and $\bm\gamma$ be three secret uniformly random polynomials. If  $\bm\theta=\bm\zeta\cdot \bm\omega\cdot\bm \pi+\bm \gamma \bmod p$ is an unforgeable polynomial (w.r.t. Theorem \ref{proof::unforgeable-poly}), \zspaa, \vopr,  $\mathtt{PRF}$, and smart contracts are secure, then \fpsi securely realises  $f^{\st \text{PSI}}$ with $Q$-fairness (w.r.t. Definition \ref{def::PSI-Q-fair}) in the presence of $m-1$  active-adversary clients (i.e., $A_{\st j}$s) or a passive dealer client, passive auditor, or passive public. 
 \end{theorem}
 







 
 


% !TEX root =main.tex


%\section{Sub Protocols}


\vs

\section{Definition of Multi-party PSI with Fair Compensation and Reward}

\svs

In this section, we upgrade \p to ``multi-party PSI with Fair Compensation and Reward'' (\ep), which (in addition to offering the features of \p) allows honest clients who contribute their set to receive a reward by a buyer who initiates the PSI computation and is interested in the result.


%In this section, we present the notion of ``Earn while You Reveal PSI'' (\ep), which allows honest clients who contribute their set to get paid by a buyer who initiates the PSI computation and is interested in the result. 

%In this section, we present an efficient  PSI that allows honest parties who contribute their set to get paid by a buyer who initiates the PSI computation and is interested in the result. 

%\subsection{The Model}

%In this section, we provide the security model of our smart-PSI protocol. There are two kind of parties involved in the protocol. Namely, (1) a set of clients $\{A_{\st 1},...,A_{\st m}\}$ potentially \emph{rational} (i.e. an adversary that picks the best strategy to maximise its profit) and all may collude with each other, and (2) a non- colluding dealer: client $D$, potentially semi-honest (i.e. a passive adversary).  Similar to F-PSI, we consider static adversary, we assume there is an authenticated private (off-chain) channel between the clients and we consider a standard public blockchain.
%  


In \ep, there are (1) a set of clients $\{A_{\st 1},...,A_{\st m}\}$ a subset of which is potentially active adversaries and may collude with each other, (2) a non-colluding dealer, $D$, potentially semi-honest, and (3) an auditor $Aud$ potentially semi-honest, where all clients (except \aud) have input set. Furthermore,  in \ep  there are two ``extractor'' clients, say $A_{\st 1}$ and $A_{\st 2}$, where $(A_{\st 1},A_{\st 2})\in \{A_{\st 1},...,A_{\st m}\}$. They volunteer to extract the (encoded) elements of the intersection and send them to a public bulletin board, i.e., a smart contract. In return, they will be paid. 
%
%We assume these two extractors are corrupted by an active adversary during interacting with other parties (and clients) until they collaborate with the rest of the clients to compute the intersection; 
%
We assume these extractors act rationally when they want to conduct the paid task of extracting the intersection and reporting it to the smart contract, so they can maximise their profit.\footnote{Thus, similar to any $A_{\st i}$ in \p, these extractors might be corrupted by an active adversary during the PSI computation.} For simplicity, we let client $A_{\st m}$ be the buyer, i.e., the party which initiates the PSI computation and is interested in the result. 


 The formal definition of \ep is built upon the definition of \p (presented in Section \ref{sec::F-PSI-model}); nevertheless, in \ep, we ensure that honest non-buyer clients receive a \emph{reward} for participating in the protocol and revealing a portion of their inputs deduced from the result. We:  (i)  upgrade the predicate \qdel to  \qdelwr to ensure that when honest clients receive the result, then an honest non-buyer client receives its deposit back plus a reward and a buyer client receives its deposit back minus the paid reward, and (ii) upgrade the predicate  \qUnFAbt to \qUnFAbtwr to ensure when an adversary aborts but learns the result, then an honest party receives its deposit back plus a predefined amount of compensation plus a reward.  The other two predicates (i.e., \qinit and \qFAbt) remain unchanged. Given the above changes, we denote the four predicates as $\bar Q:=(\qinit,  \qdelwr, \qUnFAbtwr, \qFAbt)$. Below, we present the formal definition of predicates \qdelwr and \qUnFAbtwr. 
 
 



    \begin{definition}  [\qdelwr:
    %
    Delivery-with-Reward predicate] Let $\mathcal{G}$ be a stable ledger, $adr_{\st sc}$ be smart contract $sc$'s address, $adr_{\st i}\in Adr$ be the address of an honest party, $\xc$ be a fixed amount of coins, and $pram:=(\mathcal{G}, adr_{\st sc}, \xc)$. Let $R$ be a reward function that takes as input the computation result: $res$, a party's address: $adr_{\st i}$, a reward a party should receive for each unit of revealed information:  $\lc$, and input size: $inSize$.  Then $R$ is defined as follows, if $adr_{\st i}$ belongs to a non-buyer, then it returns the total amount that $adr_{\st i}$ should be rewarded and if $adr_{\st i}$ belongs to a buyer client, then it returns the reward's leftover that the buyer can collect, i.e., $R(res, adr_{\st i}, \lc, inSize)\rightarrow \rewci$.    Then, the delivery with reward predicate $\qdelwr(pram,  adr_{\st i}, res, \lc, inSize)$ returns $1$ if $adr_{\st i}$ has sent $\xc$ amount to $sc$ and received at least $\xc+\rewci$ amount from it. Else, it returns $0$. 
%
  \end{definition}


\vs


   \begin{definition}  [\qUnFAbtwr: UnFair-Abort-with-Reward predicate]
   %
 Let $pram:=(\mathcal{G}, adr_{\st sc}, \xc)$ be the parameters defined above, and $Adr'\subset Adr$ be a set containing honest parties' addresses, $m' = |Adr'|$,  and   $adr_{\st i}\in Adr'$. Let also $G$ be a compensation function that takes as input  three parameters $(\depsc, adr_{\st i}, m')$, where $\depsc$ is the amount of coins that all $m+1$ parties deposit, $adr_{\st i}$ is an honest party's address, and $m' = |Adr'|$; it returns the amount of compensation each honest party must receive, i.e., $G(\depsc, ard_{\st i}, m')\rightarrow \xci$. Let $R$ be the reward function defined above, i.e., $R(res, adr_{\st i}, \lc, inSize)\rightarrow \rewci$, and let $\hat {pram}:=(res, \lc, inSize)$.  Then, predicate \qUnFAbtwr is defined as $\qUnFAbtwr(pram, \hat {pram}, G, R, \depsc, m', adr_{\st i})\rightarrow (a,b)$, where $a=1$ if $adr_{\st i}$ is an honest party's address which has sent $\xc$ amount to $sc$ and received  $\xc+\xci+\rewci$  from it, and $b=1$ if $adr_{\st i}$ is an auditor's address which received $\xci$  from $sc$. Otherwise, $a=b=0$. 
  %
  \end{definition}

 \svs
 
Next, we present the formal definition of multi-party PSI with Fair Compensation and Reward, \ep. 


%%%%%%%%




\begin{definition}[\ep]\label{def::PSI-Q-fair-reward}
Let $f^{\st \text{PSI}}$ be the multi-party PSI functionality defined in Section \ref{sec::F-PSI-model}. We say  protocol $\Gamma$ realises  $f^{\st \text{PSI}}$ with $\bar Q$-fairness-and-reward in the presence of $m-3$ static active-adversary clients $A_{\st j}$s and $two$ rational clients $A_{\st i}s$ or a static passive dealer  $D$ or passive auditor $Aud$, if for every non-uniform probabilistic polynomial time adversary $\mathcal{A}$ for the real model, there exists a non-uniform probabilistic polynomial-time adversary (or simulator) $\mathsf{Sim}$ for the ideal model, such that for every $I\in \{A_{\st 1},...,A_{\st m}, D, Aud\}$, it holds that: 
%
\begin{equation*}
\{\mathsf {Ideal}^{\st \mathcal{W}(f^{\st \text{PSI}}, \bar Q)}_{\st \mathsf{Sim}(z), I}(S_{\st 1},..., S_{\st m+1})\}_{\st S_{\st 1},..., S_{\st m+1},z}\stackrel{c}{\equiv} \{\mathsf{Real}_{\st \mathcal{A}(z), I}^{\st \Gamma}(S_{\st 1},..., S_{\st m+1}) \}_{\st S_{\st 1},..., S_{\st m+1},z}
\end{equation*}
where  $z$ is an auxiliary input given to $\mathcal{A}$ and  $\mathcal{W}(f^{\st \text{PSI}}, \bar Q)$ is a functionality that wraps $f^{\st \text{PSI}}$ with predicates $\bar Q:=(\qinit,  \qdelwr, \qUnFAbtwr, \qFAbt)$. 
  \end{definition}



%%%%%%%%

\vs


\section{\withRew: A Concrete Construction of \ep}



\subsection{Main Challenges to Overcome}



\subsubsection{Rewarding Clients Proportionate to the Intersection Cardinality.}
In PSIs, the main private information about the clients which is revealed to a result recipient is the private set elements that the clients have in common. Thus, honest clients must receive a reward proportionate to the intersection cardinality, from a buyer. To receive the reward, the clients need to reach a consensus on the intersection cardinality. The naive way to do that is to let every client find the intersection and declare it to the smart contract. Under the assumption that the majority of clients are honest, then the smart contract can reward the honest result recipient (from the buyer's deposit). Nevertheless, the honest majority assumption is strong in the context of multi-party PSI. Moreover, this approach requires all clients to extract the intersection, which would increase the overall costs.  Some clients may not even be interested in or available to do so. This task could also be conducted by a single entity, such as the dealer; but this approach would introduce a single point of failure and all clients have to depend on this entity.  
%
To address these challenges, we allow any two clients to become extractors.  Each of them finds and sends to the contract the (encrypted) elements in the intersection. It is paid by the contract if the contract concludes that it is honest. This allows us to avoid (i) the honest majority assumption, (ii) requiring all clients to find the intersection, and (iii) relying on a single trusted/semi-honest party to complete the task. 


\vs


\subsubsection{Dealing with Extractors' Collusion.}
%
Using two extractors itself introduces another challenge; namely, they may collude with each other (and with the buyer) to provide a consistent but incorrect result, e.g., both may declare that only $s_{\st 1}$ is in the intersection while 
 the actual intersection contains $100$ set elements, including $s_{\st 1}$.  This behaviour will not be detected by a verifier unless the verifier always conducts the delegated task itself too, which would defeat the purpose of delegation. To efficiently address this issue, we use the counter-collusion smart contracts (outlined in Section \ref{Counter-Collusion-Smart-Contracts}) which creates distrust between the two extractors and incentivises them to act honestly. 

%\subsubsection{Preserving the Intersection Privacy.} As stated above, the extractors are required to prove to a smart contract that they know 


\vs


\subsection{Description of \withRew (\epsi)}



\subsubsection{An Overview.} To construct  \epsi, we mainly use \fpsi, deterministic encryption, ``double-layered'' commitments, the hash-based padding technique (from Section \ref{sec::poly-rep}), and the counter-collusion smart contracts. % (described in Section \ref{Counter-Collusion-Smart-Contracts}). 
%
At a high level, \epsi works as follows. First, all clients run step \ref{gen-FPSI-cont} of \fpsi to agree on a set of parameters and \fpsi's smart contract.  They deploy another smart contract, say $\SCe$. They also agree on a secret key, $mk'$. Next, the buyer places a certain deposit into $\SCe$. This deposit will be distributed among honest clients as a reward. 
%
%All clients check the buyer's deposit and proceed to the next step if they agree with the deposit amount.  
%
The extractors and $D$ deploy one of the counter-collusion smart contracts, i.e., \SCpc. These three parties deposit a certain amount on this contract.  Each honest extractor will receive a portion of $D$'s deposit for carrying out its task honestly and each dishonest extractor will lose a portion of its deposit for acting maliciously. 
%
Then, each client encrypts its set elements (under $mk'$ using deterministic encryption) and then represents the encrypted elements as a polynomial. The reason each client encrypts its set elements is to ensure that the privacy of the plaintext elements in the intersection will be preserved from the public. 



%Then, each client represents the encryption of its set elements as a polynomial. 





%
Next, the extractors commit to the encryption of their set elements and publish the commitments. 
%
All clients (including $D$) take the rest of the steps in \fpsi using their input polynomials. This results in a blinded polynomial,  whose correctness is checked by \fpsi's smart contact. 

If  \fpsi's smart contact approves the result's correctness, then all parties receive the money that they deposited in \fpsi's contract. In this case, each extractor finds the set elements in the intersection. Each extractor proves to $\SCe$ that the encryptions of the elements in the intersection are among the commitments that the extractor previously published. 
%
If $\SCe$ accepts both extractors' proofs, then it pays each client (except the buyer) a reward, where the reward is taken from the buyer's deposit. The extractors receive their deposits back and are paid for carrying out the task honestly. Nevertheless, if $\SCe$ does not accept one of the extractors' proofs (or one extractor betrays the other), then it invokes the auditor in the counter-collusion contracts to identify the misbehaving extractor.  Then, $\SCe$ pays each honest client (except the buyer) a reward, taken from the misbehaving extractor. $\SCe$ also refunds the buyer's deposit.
%

If  \fpsi's smart contact does not approve the result's correctness and \aud identified misbehaving clients, then honest clients will receive (1) their deposit back from \fpsi's contract, and (2)  compensation and reward, taken from misbehaving clients. Moreover, the buyer and extractors receive their deposit back from $\SCe$. Figure \ref{fig:parties-interactions-in-ANE}, in Appendix \ref{sec::Workflow-of-withRew}, outlines the interaction between parties. 






\vs



\subsubsection{Detailed Description of \epsi.} Next, we describe the protocol in more detail (Table \ref{table:notation-table} summarises the main notations used). 



%At a high level, two clients volunteer to be come result extractors.
%
%In the case of betray, an arbiter (potentially semi-honest) is invoked who can verify the claim of a betrayer and distribute the buyer payment among honest clients. In order for the arbiter to do so without having access to any clients input, the arbiter is required to find the intersection that requires it to find the roots of result polynomials and  be able to distinguish the error roots from actual encrypted set element. To this happen, we slightly modify the fair PSI protocol. In particular, in step \ref{encode-encrypt}, each client  $I\in \textbf{P}$ maps the elements of its set $S^{\st { {(I)}}}:\{ s^{\st { {(I)}}}_{\st 1},..., s^{\st { {(I)}}}_{\st d}\}$ to random values by encrypting them as follows. $\forall i, 1\leq i\leq d: e^{\st { {(I)}}}_{\st i}=\mathtt{PRP}(mk_{\st 2}, s^{\st { {(I)}}}_{\st i})$. Then, it encodes its encrypted set element as $ {e}^{\st { {(I)}}}_{\st i} =e^{\st { {(I)}}}_{\st i} || \mathtt{H}(e^{\st { {(I)}}}_{\st i})$.  After that, it constructs a hash table  $\mathtt{HT}^{\st { {(I)}}}$ and inserts the encrypted elements into the table. $\forall i: \mathtt{H}(  {e}_{\st i}^{\st { {(I)}}})={ {  {indx}}}$, then $  {e}^{\st { {(I)}}}_{\st i}\rightarrow \mathtt{HT}^{\st { {(I)}}}_{\st {  {indx}}}$. It pads every bin with random elements to $d$ elements (if needed). Then,  for every bin, it constructs a polynomial whose roots are  the bin's content: $\pi^{\st { {(I)}}}=\prod\limits^{\st c}_{\st i=1} (x-e'_{\st i})$, where $e'_{\st i}$ is either $ {e}^{\st {  {(I)} }}_{\st i}$, or a dummy value. 





\begin{enumerate}[leftmargin=5mm]

%\item All clients in $\textbf{P}$ sign a smart contract $\mathcal{SC}$ and deploy it to a blockchain. Then, the buyer, $ { A}_{\st {  m}}$, deposits $v$ amounts in the contract.

\item\label{e-psi::call-F-PSI-stepOne}  All clients in $\cl=\{ A_{\st 1},...,   A_{\st m},  D\}$ together run step \ref{gen-FPSI-cont} of \fpsi (in Section \ref{Fair-PSI-Protocol}) to deploy \fpsi's contract $\mathcal{SC}_{\fpsi}$ and agree on a  master key, $mk$. 

\item\label{e-psi::deploy-SC-E-PSI} All clients in $\cl$  deploy a new smart contract, $\SCe$. The address of $\SCe$ is given to all clients. 

\item The buyer, client $ { A}_{\st {  m}}$, before time $t_{\st 1}$ deposits $\Smin\cdot \vc$  amount to $\SCe$. 
\item\label{e-PSI::buyer-deposit} All clients after  time $t_{\st 2}>t_{\st 1}$ ensure that the buyer has deposited $\Smin\cdot \vc$ amount on $\SCe$. Otherwise, they abort.



\item\label{e-PSI::extractor-deposit} $D$ signs \SCpc with the extractors. $\SCe$ transfers $\Smin\cdot \rc$ amount (from the buyer deposit) to \SCpc for each extractor. This is the maximum amount to be paid to an honest extractor for honestly declaring the elements of the intersection. %Each honest extractor will be paid $\hat w'= r\cdot |{ { {S}}}_{\st\cap}|$, where 
%
Each extractor  deposits $\dc'=\dc+\Smin\cdot \fc$ amount in \SCpc at time $t_{\st 3}$. At time $t_{\st 4}$ all clients ensure that the extractors deposited enough coins; otherwise, they withdraw their deposit and abort. 

%
\item\label{e-psi::commit-to-mk} $D$ encrypts $mk$ under the public key of the dispute resolver (in \SCpc); let $ct_{\st mk}$ be the ciphertext.  It also generates a commitment of $mk$ as follows: $z'=\mathtt{PRF}(mk, 0),\ com_{\st mk}=\comcom(mk, z')$. It stores $ct_{\st mk}$  and ${com}_{\st mk}$ in $\SCe$. 


\item\label{e-psi::gen-mk-prime} All clients in $\cl$ engage in \ct to agree on another key, $mk'$.
%
\item\label{Smart-PSI:encode-elem} Each client  in $\cl$ maps the elements of its set $S:\{ s_{\st 1},..., s_{\st c}\}$ to random values by encrypting them as: $\forall i, 1\leq i\leq c: e_{\st i}=\mathtt{PRP}(mk', s_{\st i})$. 
%
Then, it encodes its encrypted set element as $\bar{e}_{\st i} =e_{\st i} || \mathtt{H}(e_{\st i})$.  
%
After that, it constructs a hash table  $\mathtt{HT}$ and inserts the encoded elements into the table. $\forall i: \mathtt{H}( \bar{e}_{\st i})={ {  {j}}}$, then $\bar{e}_{\st i}\rightarrow \mathtt{HT}_{\st {  {j}}}$. It pads every bin with random dummy elements to $d$ elements (if needed). Then,  for every bin, it builds a polynomial whose roots are the bin's content: $\bm\pi^{\st { {(I)}}}=\prod\limits^{\st d}_{\st i=1} (x-e'_{\st i})$, where $e'_{\st i}$ is either $\bar{e}_{\st i}$, or a dummy value. 




\item\label{merkel-tree-cons} Every extractor in $\{A_{\st 1}, A_{\st 2}\}$: 

\begin{enumerate}[leftmargin=2.5mm]
%
%\item for each bin, derives a pseudorandom polynomial: $\gamma'_{\st {  {j}}}$, using key $mk$.
%
%\item for each bin, evaluates $\gamma'_{\st {  {j}}}$ at the encode set elements of that bin: $\gamma'_{\st {  {j,i}}}=\gamma'_{\st {  {j}}}(\bar{e}^{\st { {(I)}}}_{\st i})$.
\item\label{smart-PSI::commit-to-bin} for each $j$-th bin, commits to the bin's elements: $com_{\st{i,j}}=\comcom(e'_{\st i}, q_{\st i})$, where $q_{\st i}$ is a fresh randomness  used for the commitment and $e'_{\st i}$ is either $\bar{e}_{\st i}$, or a dummy value of the bin. %Thus, if the bin contains paddings, it  commits to the paddings too. 




%\item commits to every  encrypted element.  $\forall i, 1\leq i\leq d: \mathtt{a}^{\st { {(I)}}}_{\st i}=\mathtt{Com}(e^{\st { {(I)}}}_{\st i}, q^{\st { {(I)}}}_{\st i})$, where $q^{\st { {(I)}}}_{\st i}$ is a fresh randomness  used for the commitment.
\item  constructs a Merkel tree on all committed values as follows: \\$\mkgen(com_{\st 1,1},...,com_{\st d,h})\rightarrow g$. %Let $\mathtt{MT}^{\st g}$ be a Merkel tree with a root node $g^{\st I}$. 
\item stores the Merkel tree's root $g$ on $\SCe$.
\end{enumerate}


\item\label{e-psi::invoke-remainer-F-PSI} All clients in $\cl$   run steps \ref{ZSPA}--\ref{compute-res-poly} of \fpsi, where each client now deposits (in the $\mathcal{SC}_{\fpsi}$) $\yc'$ amount where $\yc'>\Smin\cdot \vc+{\chc}$. Recall, at the end of step \ref{compute-res-poly}  of \fpsi for each $j$-th bin (i) a random polynomial $\bm\zeta$ has been registered in $\mathcal{SC}_{\fpsi}$, (ii) a polynomial $\bm\phi$ (blinded by a random polynomial $\bm\gamma'$) has been extracted by $\mathcal{SC}_{\fpsi}$, and (iii) $\mathcal{SC}_{\fpsi}$  has checked this polynomial's  correctness. If the latter check:

\begin{itemize}[leftmargin=2mm]
\item[$\bullet$] passes (i.e., $Flag=True$): all parties run step \ref{F-PSI::flag-is-true} of \fpsi (with a minor difference, see Section \ref{sec::Discussion-Anesidora}).  In this case, each party receives $\yc'$ amount it deposited in $\mathcal{SC}_{\fpsi}$. They proceed to step \ref{smart-PSI::extractors} below.
\item[$\bullet$]  fails (i.e., $Flag=False$): all parties run step \ref{F-PSI::flag-is-false}  of \fpsi. In this case,
(as in \fpsi) \aud is paid $\chc$ amount, and each honest party receives back its deposit, i.e., $\yc'$ amount. Also,  from the misbehaving parties' deposit  $\frac{m'\cdot \yc'-\chc}{m-m'}$ amount is sent to each honest client,  to reward and compensate the client $\Smin\cdot \lc$ and $\frac{m'\cdot \yc'-\chc}{m-m'}- \Smin\cdot \lc$ amounts respectively, where $m'$ is the total number of misbehaving parties.  Moreover, $\SCe$ returns to the buyer its deposit (i.e., $\Smin\cdot \vc$ amount paid to $\SCe$), and returns to each extractor its deposit, i.e., $\dc'$ amount paid to \SCpc. Then, the protocol halts. 
\end{itemize}

\item\label{smart-PSI::extractors} Every extractor client: 
\begin{enumerate}[leftmargin=2.5mm]

\item finds the elements in the intersection. To do so, it first encodes each of its set elements to get $\bar e_{\st i}$, as explained in step \ref{Smart-PSI:encode-elem}.  
%i.e.,  it first computes $e^{\st { {(I)}}}_{\st i}=\mathtt{PRP}(mk',s^{\st { {(I)}}}_{\st i})$ and then computes $\bar{e}^{\st { {(I)}}}_{\st i} =e^{\st { {(I)}}}_{\st i} || \mathtt{H}(e^{\st { {(I)}}}_{\st i})$.
%
% and then encodes it:   (i.e. $ {e}^{\st { {(I)}}}_{\st i} =e^{\st { {(I)}}}_{\st i} || \mathtt{H}(e^{\st { {(I)}}}_{\st i})$). 
 %
 Then, it determines to which bin the encrypted value belongs, i.e., ${ {  {j}}}=\mathtt{H}( \bar{e}_{\st i})$. Next, it evaluates the resulting polynomial (for that bin) at the encrypted element. It considers the element in the intersection if the evaluation is zero, i.e., $\bm\phi( \bar{e}_{\st i})-\bm\zeta( \bar{e}_{\st i})\cdot \bm\gamma'( \bar{e}_{\st i})=0$. If the extractor is a traitor, by this point it should have signed \SCtc with $ { D}$ and provided all the inputs (e.g., correct result) to \SCtc. 

\item \label{extractor-proves} proves that every element in the intersection is among the elements it has committed to. Specifically, for each element in the intersection, say $\bar{e}_{\st i}$, it sends to $\SCe$: 



\begin{enumerate}[leftmargin=3.5mm]
%
%\item [$\bullet$]  the opening of commitment $\mathtt{a}'$, i.e., pair $\ddot {x}:=(mk, z')$. This is done only once for all elements in the intersection.  
%
\item [$\bullet$]  commitment $com_{\st i,j}$ (for $\bar{e}_{\st i}$) and its  opening ${\hat x}':=(\bar{e}_{\st i},  q_{\st i})$. 


%
%\item [$\bullet$] the element's commitment: $\mathtt{a}^{\st { {(I)}}}_{\st {  {j,i}}}=\mathtt{Com}(\gamma'_{\st {  {j,i}}}, q^{\st { {(I)}}}_{\st i})$, where $q^{\st { {{(I)}}} }_{\st i}$ was generated in step \ref{merkel-tree-cons}.   
%
%\item[$\bullet$]  $ \bar{e}^{\st { {{(I)}}} }_{\st i}$ and it  commitment's opening:  $\mathtt{m}^{\st { {{(I)}}} }_{\st i}=(\gamma'_{\st {  {j,i}}}, q^{\st { {{(I)}}} }_{\st i})$. 

%
\item[$\bullet$]   proof $h_{\st i}$ asserting $com_{\st i,j}$ is a leaf node of   a Merkel tree with  root $g$. 

%\item[$\bullet$] the index of the bin to which $ \bar{e}^{\st { {{(I)}}} }_{\st i}$ belongs, i.e., ${ {  {j}}}=\mathtt{H}(\bar {e}^{\st { {{(I)}}} }_{\st i})$. 
 \end{enumerate}
\item sends the opening of commitment $com_{\st mk}$, i.e., pair $\hat {x}:=(mk, z')$, to $\SCe$. This is done only once for all elements in the intersection.  

 \end{enumerate}
\item\label{e-psi::SC-verification} Contract $\SCe$:
%
\begin{enumerate}[leftmargin=2.4mm]

\item\label{e-psi::SC-verification--derive-mk}  verifies the opening of the commitment for $mk$, i.e., $\comver(com_{\st mk},\hat{x})=1$. If accepted, then it generates the bin's index to which $ \bar{e}_{\st i}$ belongs, i.e., ${ {  {j}}}=\mathtt{H}(\bar {e}_{\st i})$. It  uses $mk$ to derive the pseudorandom polynomial $\bm\gamma'$ for $j$-th bin. 


 \item\label{e-psi::SC-verification--check-three-vals} checks whether (i) the opening of commitment is valid,  (ii) the Merkle tree proof is valid, and (iii) the encrypted element is the resulting polynomial's root. Specifically, it ensures that the following relation holds: 
 %
$$\Bigg(\comver(com_{\st i,j}, \hat{x}')=1\Bigg)  \wedge  \Bigg(\mkver(h_{\st i},g)=1\Bigg) \wedge  \Bigg(\bm\phi( \bar{e}_{\st i})-\bm\zeta( \bar{e}_{\st i})\cdot \bm\gamma'( \bar{e}_{\st i})=0\Bigg)$$
%
%$$\mathtt{Ver_{\st com}}(\mathtt{a}^{\st { {{(I)}}} }_{\st {  {j,i}}},\mathtt{m}^{\st { {{(I)}}} }_{\st i})=1\ \ \ \ \wedge \ \ \ \ \mathtt{Ver_{\st MT}}(\mathtt{h}^{\st { {{(I)}}} }_{\st i},g^{\st { {{(I)}}} })=1 \ \ \ \ \wedge \ \ \ \  \phi( \bar{e}^{\st { {{(I)}}} }_{\st i})-\zeta( \bar{e}^{\st { {{(I)}}} }_{\st i})\cdot \gamma'_{\st {  {j,i}}}=0$$
%
%\item if all proofs of both extractors are valid and both extractors provide identical elements of the intersections (for each bin),  for each valid proof, it takes $m\cdot l$ coins from the buyer's deposit (in $\mathcal{SC}_{\st {  {EXT}}}$) and distributes it among all clients, except the buyer. 
%
\end{enumerate}

% !TEX root =main.tex




\item The parties are paid as follows. 

\begin{itemize}
%
\item[$\bullet$]  if the extractors' proofs are valid, they provided identical elements of the intersections (for each bin), and there is no traitor, then $\mathcal{SC}_{\epsi}$:
\begin{enumerate}
%
 \item takes $|S_{\st\cap}|\cdot m\cdot \lc$ amount from the buyer's deposit (in $\mathcal{SC}_{\epsi}$) and distributes it among all clients, excluding the buyer. 
 %
 \item calls \SCpc which returns the extractors' deposit (i.e., $\dc'$ amount each) and pays each extractor $|S_{\st\cap}|\cdot \rc$ amount, for doing their job correctly. 
  %
 \item checks if $|{ { {S}}}_{\scriptscriptstyle\cap}|<\Smin$. If the check passes, then it returns $(\Smin-|S_{\scriptscriptstyle\cap}|)\cdot \vc$ amount  to the buyer.
 %
 \end{enumerate}
% 
\item[$\bullet$] if both extractors failed to deliver any result, then $\mathcal{SC}_{\epsi}$:
%
\begin{enumerate}
%
\item refunds the buyer, by sending $\Smin\cdot \vc$ amount (deposited in $\mathcal{SC}_{\epsi}$) back to the buyer. 
%
\item retrieves each extractor's deposit from \SCpc and distributes it among the rest of the clients (excluding the buyer and extractors).  
%
 \end{enumerate}
 %
 \item[$\bullet$]\label{smart-PSI-inconsistency} Otherwise (e.g., if some proofs are invalid, if an extractor's result is inconsistent with the other extractor's result, or there is a traitor), $\mathcal{SC}_{\epsi}$ invokes (steps 8.c and 9 of) \SCpc and its auditor to identify the misbehaving extractor, with the help of $ct_{\st mk}$ after decrypting it. $\mathcal{SC}_{\epsi}$ asks \SCpc to pay the auditor the total amount of $\chc$ taken from the deposit of the extractor(s) who provided incorrect result to $\mathcal{SC}_{\epsi}$. Moreover,
%

\begin{enumerate}
%
\item if \underline{both extractors cheated}:
%
\begin{enumerate}[leftmargin=2mm]
%
\item\label{both-cheated-no-traitor} if there \underline{is no traitor}, then $\mathcal{SC}_{\epsi}$ refunds the buyer, by sending $\Smin\cdot \vc$ amount (deposited in $\mathcal{SC}_{\epsi}$) back to the buyer. It also distributes $2\cdot \dc'- \chc$ amount (taken from the extractors' deposit in \SCpc) among the rest of  the clients  (excluding the buyer and extractors). %The Prisoner's contract pays its dispute resolver $ch$ amount. 
%
%
\item if there \underline{is a traitor}, then:
%%%%%
\begin{enumerate}
%
%
\item\label{both-cheated-honest-traitor} if the traitor delivered a \underline{correct result} in \SCtc, $\mathcal{SC}_{\epsi}$ retrieves $\dc'-\dc$ amount from the other dishonest extractor's deposit (in \SCpc) and distributes it among the rest of the clients (excluding the buyer and dishonest extractor). It asks \SCpc to send $|S_{\st\cap}|\cdot \rc+\dc'+\dc-\chc$ amount to the traitor (via \SCtc). % and $\hat{ch}$ amount to the dispute resolver.
%
\SCtc refunds the traitor's deposit, i.e., $\chc$ amount. It refunds the buyer, by sending $\Smin\cdot \vc-|S_{\st\cap}|\cdot \rc$ amount (deposited in $\mathcal{SC}_{\epsi}$) to it.


 %Otherwise (i.e., if it delivered an incorrect result in the Traitor's contract), the Traitor's contract refunds the traitor's deposit (i.e., $ch$ amount).
%
\item if the traitor delivered an \underline{incorrect result} in \SCtc, then $\mathcal{SC}_{\epsi}$ pays the buyer and the rest of the clients in the same way it does in step \ref{both-cheated-no-traitor}. 
%
%distributes $2\cdot \hat d- \hat{ch}$ amount (taken from the extractors deposit in the Prisoner's contract) among the rest of the clients (except the buyer and extractors). 
%
%The Prisoner's contract pays its dispute resolver $\hat{ch}$ amount.
%
\SCtc refunds the traitor,  $\chc$ amount.  %It refunds the buyer, by sending ${\resizeT {\textit {S}}}_{\resizeS {\textit  min}}\cdot v$ amount (deposited in $\mathcal{SC}_{\resizeS {\textit  {EXT}}}$) back to the buyer.


%
\end{enumerate}
%%%%%
\end{enumerate}
%
\item if \underline{one of the extractors cheated}: 
%
\begin{enumerate}[leftmargin=2mm]
%
\item if there \underline{is no traitor},  $\mathcal{SC}_{\epsi}$ calls \SCpc that (a) returns the honest extractor's deposit ($\dc'$ amount), (b) pays this extractor $|S_{\st\cap}|\cdot \rc$ amount, for doing its job honestly, and (c) pays this extractor $ \dc-\chc$ amount taken from the dishonest extractor's deposit. 
%
%and (d) pays its dispute resolver $\hat {ch}$ amount taken from the dishonest extractor's deposit. 
%
%Also, $\mathcal{SC}_{\resizeS {\textit  {EXT}}}$ sends ${\resizeT {\textit {S}}}_{\resizeS {\textit  min}}\cdot v-|S_{\st\cap}|\cdot r$ amount (deposited in $\mathcal{SC}_{\resizeS {\textit  {EXT}}}$) back to the buyer.
%
 $\mathcal{SC}_{\epsi}$ pays the buyer and the rest of the clients in the same way it does in step \ref{both-cheated-honest-traitor}. 


% It retreives  $\hat d - \hat c - \hat{ch}$ amount from the dishonest extractor's deposit (in the Prisoner’s contract) and distributes it among the rest of clients (except the buyer and dishonest extractor). %
%
\item if there \underline{is a traitor}
%
%%%%%


\begin{enumerate}
%
\item\label{one-cheated-exists-traitor-honest-traitor}  if the traitor delivered a \underline{correct result} in \SCtc (but  cheated in $\mathcal{SC}_{\epsi}$),  $\mathcal{SC}_{\epsi}$ calls \SCpc that (a) returns the other honest extractor's deposit ($\dc'$ amount), (b) pays the honest extractor $|S_{\st\cap}|\cdot \rc$ amount taken from the buyer's deposit, for doing its job honestly,  (c) pays the honest extractor $\dc- \chc$ amount taken from the traitor's deposit,  
%
%(d) pays its dispute resolver $\hat{ch}$ amount taken from the traitor extractor's deposit (deposited in $\mathcal{SC}_{\resizeS {\textit  {EXT}}}$), 
%
 (d)
 pays to the traitor $|S_{\st\cap}|\cdot \rc$ amount taken from the buyer’s deposit (via the \SCtc), and (e) refunds the traitor $\dc'-\dc$ amount taken from its own deposit.  \SCtc refunds the traitor's deposit, $\chc$ amount.  $\mathcal{SC}_{\epsi}$ takes $|S_{\st\cap}|\cdot m\cdot \lc$ amount from the buyer's deposit (in $\mathcal{SC}_{\epsi}$) and distributes it among all clients, excluding the buyer. If $|{ { {S}}}_{\scriptscriptstyle\cap}|<\Smin$,   $\mathcal{SC}_{\epsi}$ returns $(\Smin-|{ { {S}}}_{\scriptscriptstyle\cap}|)\cdot \vc$ amount (from $\mathcal{SC}_{\epsi}$)  to the buyer. 
 
 
%
\item  if the traitor delivered an \underline{incorrect result} in \SCtc (and it cheated in $\mathcal{SC}_{\epsi}$), then $\mathcal{SC}_{\epsi}$ pays the honest extractor in the same manner as it did in step \ref{one-cheated-exists-traitor-honest-traitor}.  
%
%calls Prisoner's contract that (a) returns the other honest extractor's deposit (i.e., $\hat d$ amount), (b) pays the honest extractor $|S_{\st\cap}|\cdot r$ amount, for doing its job honestly, and (c) pays the honest extractor $\hat c$ amount taken from the traitor extractor's deposit. 
%
%, and  (d) pays its dispute resolver $\hat{ch}$ amount taken from the traitor extractor's deposit (deposited in $\mathcal{SC}_{\resizeS {\textit  {EXT}}}$). 
%
\SCtc refunds the traitor's deposit, i.e., $\chc$ amount. 
%
%$\mathcal{SC}_{\resizeS {\textit  {EXT}}}$ takes ${\resizeT {\textit {S}}}_{\resizeS {\textit  min}} \cdot f$ amount from the traitor's deposit (in $\mathcal{SC}_{\resizeS {\textit  {EXT}}}$) and distributes it among all clients, except the buyer and traitor. 
%
%Also, $\mathcal{SC}_{\resizeS {\textit  {EXT}}}$ sends ${\resizeT {\textit {S}}}_{\resizeS {\textit  min}}\cdot v-|S_{\st\cap}|\cdot r$ amount (deposited in $\mathcal{SC}_{\resizeS {\textit  {EXT}}}$) back to the buyer. 
%
Also, $\mathcal{SC}_{\epsi}$ pays the buyer and the rest of the clients in the same way it does in step \ref{both-cheated-honest-traitor}.


%
\end{enumerate}


\end{enumerate}

\end{enumerate}
\end{itemize}



%
%\begin{enumerate}
%
%\item fully refunds the buyer, as before.
%%
%\item calls Prisoner's contract that (a) returns the honest extractor's deposit (i.e., $\hat d$ amount), (b) pays this extractor $|S_{\st\cap}|\cdot r$ amount, for doing its job honestly,  (c) pays this extractor $\hat c$ amount taken from the dishonest extractor's deposit, and (d) pays its dispute resolver $ch$ amount taken from the dishonest extractor's deposit.
%%
%\item retrieves $\hat d-\hat c- ch$ amount from dishonest extractor's deposit (in the Prisoner's contract) and distributes it among the rest of the clients (except the buyer and dishonest extractor). 
%
%\end{enumerate}
%

 
 
 %%%%%%%%%%%. without Traitor
% \item Otherwise (e.g., if some proofs are invalid or if an extractor's result is inconsistent with the other extractor's result), $\mathcal{SC}_{\resizeS {\textit  {EXT}}}$ invokes (steps 8.c and 9 of) the Prisoner's contract to identify the misbehaving extractor. 
%%
%
%\begin{itemize}
%%
%\item[$\bullet$] if both extractors cheated, then $\mathcal{SC}_{\resizeS {\textit  {EXT}}}$:
%%
%\begin{enumerate}
%%
%\item refunds the buyer, by sending ${\resizeT {\textit {S}}}_{\resizeS {\textit  min}}\cdot v$ amount (deposited in $\mathcal{SC}_{\resizeS {\textit  {EXT}}}$) back to the buyer.
%%
%\item retrieves $\hat d-ch$ amount from each extractor's deposit (in the Prisoner's contract) and distributes it among the rest of the clients (except the buyer and extractors). The Prisoner's contract pays its dispute resolver $ch$ amount for each dishonest extractor. 
%%
%\end{enumerate}
%%
%\item[$\bullet$] if one of the extractors cheated, then $\mathcal{SC}_{\resizeS {\textit  {EXT}}}$:
%%
%\begin{enumerate}
%%
%\item fully refunds the buyer, as before.
%%
%\item calls Prisoner's contract that (a) returns the honest extractor's deposit (i.e., $\hat d$ amount), (b) pays this extractor $|S_{\st\cap}|\cdot r$ amount, for doing its job honestly,  (c) pays this extractor $\hat c$ amount taken from the dishonest extractor's deposit, and (d) pays its dispute resolver $ch$ amount taken from the dishonest extractor's deposit.
%%
%\item retrieves $\hat d-\hat c- ch$ amount from dishonest extractor's deposit (in the Prisoner's contract) and distributes it among the rest of the clients (except the buyer and dishonest extractor). 
%%
%\end{enumerate}
%%
%\end{itemize}








%\item Contract $\mathcal{SC}_{\st {  {EXT}}}$ after time $t_{\st 3}$ checks if $|{ { {S}}}_{\st\cap}|<{ { {S}}}_{\st {  min}}$. In this case, it returns $({ { {S}}}_{\st {  min}}-|{ { {S}}}_{\st\cap}|)\cdot v$ amount  to the buyer.

\end{enumerate}

\vs
\vs

 \begin{theorem}\label{theorem::E-PSI-security}
If  $\mathtt{PRP}$, $\mathtt{PRF}$, the commitment scheme, smart contracts, the Merkle tree scheme, \fpsi and the counter-collusion contracts are secure and the public key encryption is semantically secure,  then  \epsi realises  $f^{\st \text{PSI}}$ with $\bar Q$-fairness-and-reward (w.r.t. Definition \ref{def::PSI-Q-fair-reward}) in the presence of $m-3$ static active-adversary clients $A_{\st j}$s and $two$ rational clients $A_{\st i}s$ or a static passive dealer $D$ or passive auditor $Aud$, or passive public which sees the intersection cardinality.
 \end{theorem}

\svs
We refer readers to Appendices \ref{sec::E-PSI-proof} and \ref{sec::Discussion-Anesidora} for the proof of Theorem \ref{theorem::E-PSI-security} and several remarks on the \epsi respectively. 
%
%\begin{remark}










% !TEX root =main.tex


\vspace{-5mm}

\section{Evaluation}\label{sec::valuation}

\vspace{-2mm}

In this section, we analyse the asymptotic costs of \epsi. We also compare its costs and features with the fastest two and multiple parties PSIs in \cite{AbadiDMT22,DBLP:conf/ccs/KolesnikovMPRT17,NevoTY21,RaghuramanR22}) and with the fair PSIs in \cite{DebnathD16,DBLP:conf/dbsec/DongCCR13}. Tables \ref{table::Asymptotic-Cost} and \ref{table::comparisonTable} summarise the result of the cost analysis and the comparison respectively. 

\vspace{-4.5mm}

% !TEX root =main.tex




\vs
\vs
 \begin{table}[!htb]

\caption{ {\small{Asymptotic costs of different parties in \epsi. In the table, $h$ is the total number of bins, $d$ is a bin's capacity (i.e., $d=100$), $m$ is the total number of clients (excluding $D$), $|S|$ is a set cardinality, and $\bar\xi$ is \ole's security parameter.
%
}}} \label{table::Asymptotic-Cost} 
% \vspace{-3mm}
\begin{center}
\scalebox{.78}{
\renewcommand{\arraystretch}{1}
\begin{tabular}{|c|c|c|c|c|} 

   %\hline
        \cline{1-3}  
   %
{\scriptsize {Party}}&{\scriptsize {Computation Cost}}&{\scriptsize {Communication Cost}}\\
     \cline{1-3}  
%&\scriptsize$e=1$&\scriptsize$e>1$\\
\hline

    %SO-PoR 1st row
\scriptsize Client $A_{\st  3},...,    A_{\st   m}$& \cellcolor{gray!50}   \scriptsize$O\Big(h\cdot d(m+d)+|S|(\frac{d^{\st 2}+d}{2})\Big)$& \cellcolor{gray!50}  \scriptsize$O\Big(h\cdot d^{\st 2}\cdot \bar\xi\Big)$\\
 %  { }
     \cline{1-3}  
     %SO-PoR 2nd row
\scriptsize Dealer $D$&   \cellcolor{gray!20}\scriptsize$O\Big(h\cdot m(d^{\st 2}+d)+|S|(\frac{d^{\st 2}+d}{2})\Big)$ &  \cellcolor{gray!20}\scriptsize$O\Big(h\cdot d^{\st 2}\cdot \bar\xi\cdot m\Big)$\\
      \cline{1-3}   
      

       %[3] 1st row 
       
   \scriptsize   {Auditor $\aud$ }& \cellcolor{gray!50}\scriptsize$O\Big(h\cdot m\cdot d\Big)$&  \cellcolor{gray!50}\scriptsize$O\Big(h\cdot d\Big)$\\      
            \cline{1-3} 

 % \scriptsize \ \ \ \ \ \ \ \ --------------&&\\
 \scriptsize{Extractor} $A_{\st  1},    A_{\st   2}$& \cellcolor{gray!20}\scriptsize$O\Big(h\cdot d(m+d)+|S|(\frac{d^{\st 2}+d}{2})\Big)$& \cellcolor{gray!20}\scriptsize$O\Big(|S_{\scriptscriptstyle\cap}|\cdot \log_{\st 2}|S|\Big)$\\
     \cline{2-3}
%{\scriptsize Auditor $\mathcal{D}_{\st n}$}&    \cellcolor{gray!20}\scriptsize$O(\sum\limits_{\st i=e}^{\st n}\frac{n!}{i!(n- i)!})$&    \cellcolor{gray!20}\scriptsize$ O(\sum\limits_{\st i=e}^{\st n}\frac{n!}{i!(n- i)!})$\\
     \cline{1-3}  
     
 \scriptsize Smart contract $\mathcal{SC}_{\epsi}$\ \&\ $\mathcal{SC}_{\fpsi}$& \cellcolor{gray!50}\scriptsize $O\Big( |S_{\st \cap}|(d+ \log_{\st 2} |S|)+h\cdot m\cdot d\Big)$& \cellcolor{gray!50}\scriptsize ---\\
 
   \hline
   
   \hline
   
    
     \scriptsize Overal Complexity & \cellcolor{gray!20}\scriptsize $O\Big(h\cdot d^{2}\cdot m \Big)$& \cellcolor{gray!20}\scriptsize {$O\Big(h\cdot d^{\st 2}\cdot \bar\xi\cdot m\Big)$}\\
     
      \cline{1-3}  

\end{tabular} 
} 
\end{center}
\end{table}






%!TEX root = main.tex


\vs


\begin{table} 


\vs
\vs

\caption{ \small{Comparison of the asymptotic complexities and features of state-of-the-art PSIs. In the table, $t$ is a parameter that determines the maximum number of colluding parties, $\kappa$ is a security parameter, and $c$ is a set cardinality.}}  \label{table::comparisonTable} 
\renewcommand{\arraystretch}{.9}
\begin{center}
\scalebox{.78}{
\begin{tabular}{|c|c|c|c|c|c|c|c|c|c|} 
\hline

%\multicolumn{3}{c|}

%\multirow{2}{*} {\scriptsize {Schemes}} &{\scriptsize {Computation}}& \scriptsize{Communication}&{\scriptsize{Fairness}}&{ \scriptsize Rewarding}& {\scriptsize{ Sym-key based}}& {\scriptsize{Multi-party}}&\scriptsize Active Adversary \\
%\hline

\multirow{2}{*} {\scriptsize {Schemes}} &\multicolumn{2}{c|}{\scriptsize Asymptotic Cost}&\multicolumn{5}{c|}{\scriptsize{Features}} \\

\cline{2-8}

& \scriptsize{Computation}&\scriptsize{Communication}&{\scriptsize{Fairness}}&{ \scriptsize Rewarding}& {\scriptsize{ Sym-key based}}& {\scriptsize{Multi-party}}&\scriptsize Active Adversary\\




\hline 

%&\scriptsize {Modular expo.}&\cellcolor{gray!20}\scriptsize {$0$}&\cellcolor{gray!20}\scriptsize$5$&\cellcolor{gray!20}\scriptsize$12$\\


\scriptsize  \scriptsize{ \cite{AbadiDMT22}}&\cellcolor{gray!20}\scriptsize{$O( h\cdot d^{\st 2}\cdot m)$}&\cellcolor{gray!20}\scriptsize$O(h\cdot d\cdot m)$&\cellcolor{gray!20}\scriptsize\textcolor{red}{$\times$}&\cellcolor{gray!20}\scriptsize\textcolor{red}{$\times$}&\cellcolor{gray!20}\scriptsize\textcolor{blue}\checkmark  &\cellcolor{gray!20}\scriptsize\textcolor{blue}\checkmark&\cellcolor{gray!20}\scriptsize\textcolor{red}{$\times$} \\


\hline 


\scriptsize \cite{DebnathD16}&\cellcolor{gray!50}\scriptsize{$O(c)$}&\cellcolor{gray!50}\scriptsize{$O(c)$}&\cellcolor{gray!50}\scriptsize\textcolor{blue}\checkmark&\cellcolor{gray!50}\scriptsize\textcolor{red}{$\times$}&\cellcolor{gray!50}\scriptsize\textcolor{red}{$\times$} &\cellcolor{gray!50}\scriptsize\textcolor{red}{$\times$}&\cellcolor{gray!50}\scriptsize\textcolor{blue}\checkmark \\ 




\hline

\scriptsize {\cite{DBLP:conf/dbsec/DongCCR13}}&\cellcolor{gray!20}\scriptsize{$O(c^{\st 2}$)}&\cellcolor{gray!20}\scriptsize$O(c)$&\cellcolor{gray!20}\scriptsize\textcolor{blue}\checkmark&\cellcolor{gray!20}\scriptsize\textcolor{red}{$\times$}  &\cellcolor{gray!20}\scriptsize\textcolor{red}{$\times$} &\cellcolor{gray!20}\scriptsize\textcolor{red}{$\times$}&\cellcolor{gray!20} \scriptsize\textcolor{blue}\checkmark\\ 

\hline
\scriptsize \cite{DBLP:conf/ccs/KolesnikovMPRT17}   &\cellcolor{gray!50}\scriptsize{$O(c\cdot m^{\st 2}+c\cdot m )$}&\cellcolor{gray!50}\scriptsize$O(c\cdot m^{\st 2})$&\cellcolor{gray!50}\scriptsize\textcolor{red}{$\times$}&\cellcolor{gray!50}\scriptsize\textcolor{red}{$\times$}  &\cellcolor{gray!50}\scriptsize\textcolor{blue}\checkmark &\cellcolor{gray!50}\scriptsize\textcolor{blue}\checkmark&\cellcolor{gray!50}\scriptsize\textcolor{red}{$\times$}\\ 

\hline


\scriptsize \cite{NevoTY21}&\cellcolor{gray!20}\scriptsize{$O(c\cdot \kappa(m+t^{\st 2}-t(m+1)))$}&\cellcolor{gray!20}\scriptsize{$O(c\cdot m\cdot \kappa)$}&\cellcolor{gray!20}\scriptsize{\textcolor{red}{$\times$}}&\cellcolor{gray!20}\scriptsize\textcolor{red}{$\times$}&\cellcolor{gray!20}\scriptsize\textcolor{blue}\checkmark  &\cellcolor{gray!20}\scriptsize\textcolor{blue}\checkmark&\cellcolor{gray!20}\scriptsize\textcolor{blue}\checkmark\\ 

\hline


\scriptsize \cite{RaghuramanR22}&\cellcolor{gray!50}\scriptsize{$O(c)$}&\cellcolor{gray!50}\scriptsize{$O(c\cdot \kappa)$}&\cellcolor{gray!50}\scriptsize{\textcolor{red}{$\times$}}&\cellcolor{gray!50}\scriptsize\textcolor{red}{$\times$} &\cellcolor{gray!50}\scriptsize\textcolor{blue}\checkmark &\cellcolor{gray!50}\scriptsize{\textcolor{red}{$\times$}} &\cellcolor{gray!50}\scriptsize\textcolor{blue}\checkmark\\ 

\hline



{\scriptsize \textbf{Ours:} \epsi}&\cellcolor{gray!20}\scriptsize{$O (h\cdot d^{2}\cdot m)$}&\cellcolor{gray!20}\scriptsize$O (h\cdot d^{\st 2}\cdot \bar\xi\cdot m )$&\cellcolor{gray!20}\scriptsize\textcolor{blue}\checkmark&\cellcolor{gray!20}\scriptsize \textcolor{blue}\checkmark&\cellcolor{gray!20}\scriptsize\textcolor{blue}\checkmark &\cellcolor{gray!20}\scriptsize\textcolor{blue}\checkmark&\cellcolor{gray!20}\scriptsize\textcolor{blue}\checkmark \\

\hline 
%}
\end{tabular}
%}
%\renewcommand{\arraystretch}{1}
%\end{footnotesize}
}
\end{center}
%}
\end{table}







\subsection{Computation Cost}


\subsubsection{Client's and Dealer's Costs.}

In step \ref{e-psi::call-F-PSI-stepOne}, the cost of each client (including dealer $D$) is $O(m)$ and mainly involves an invocation of \ct. 
% 
In steps \ref{e-psi::deploy-SC-E-PSI}--\ref{e-PSI::extractor-deposit}, the clients' cost is negligible as it involves deploying smart contracts and reading from them. 
%
In step \ref{e-psi::gen-mk-prime}, the clients' cost is  $O(m)$, as they need to invoke an instance of \ct. 
%
In step \ref{Smart-PSI:encode-elem}, each client invokes \prp and $\mathtt{H}$ linear with its set's cardinality. In the same step, it also constructs $h$ polynomials, where the construction of each polynomial involves $d$  modular multiplications and additions. Thus, its complexity in this step is $O(h\cdot d)$. As shown in \cite{AbadiDMT22},  $O(h\cdot d)=O(|S|)$ and  $d=100$ for all set sizes. 
%


In step \ref{e-psi::invoke-remainer-F-PSI}, each client   $A_{\st  1},...,    A_{\st   m}$ (excluding $D$): (i) invokes an instance of \zspaa which involves $O(h\cdot m)$ invocation of \ct, $3h\cdot m (d+1)$ invocation of \prf, $3h\cdot m (d+1)$ addition, and $O(h\cdot m\cdot d)$ invocation of $\mathtt{H}$ (in step \ref{ZSPA} of subroutine \fpsi), (ii) invokes $2h$ instances of \vopr, where each \vopr invocation involves $2d(1+d)$ invocations of $\ole^{\st +}$, multiplications, and additions  (in steps \ref{e-psi::D-randomises} and \ref{e-psi::C-randomises} of \fpsi), and (iii) performs $h(3d+2)$ modular addition (in step \ref{blindPoly-C-sends-to-contract} of  \fpsi).  
 %
 Also, if $Flag=True$, each client (including $D$) invokes $\prf$ $h (3d+1)$ times, and performs $h (3d+1)$ additions, and performs polynomial evaluations linear with $|S|$, where each evaluation involves  $O(d)$  additions and $O(d)$ multiplications.% (in step \ref{F-PSI::flag-is-true} of \fpsi). 
 
 Step \ref{e-psi::commit-to-mk} involves only $D$ whose cost in this step is constant, as it involves invoking a public key encryption, \prf,  and commitment only once. Furthermore, $D$:  (a) invokes $2h\cdot m$ instances of \vopr  (in steps \ref{e-psi::D-randomises} and \ref{e-psi::C-randomises} of \fpsi), (b) invokes $\prf$ $h(3d+1)$ times (in step \ref{f-psi::D-gen-random-poly} of \fpsi), and (c) performs $h(d^{\st 2}+1)$ multiplications and $3h\cdot m\cdot d$ additions (in step \ref{f-psi::D-gen-switching-poly} of \fpsi). If $Flag=False$, then $D$ performs $O(h\cdot m\cdot d)$ multiplications and additions (in step \ref{F-PSI::flag-is-false} of \fpsi).  

\vspace{-2.7mm}
 \subsubsection{Auditor's Cost.}
 
If $Flag=False$, then \aud invokes $\prf$ $3h\cdot m(d+1)$ times  and  invokes $\mathtt{H}$ $O(h\cdot m\cdot d)$ times (in step \ref{F-PSI::flag-is-false} of \fpsi). 


\vspace{-2.7mm}
 \subsubsection{Extractor's Cost.}
 
 
 In step \ref{merkel-tree-cons}, each extractor invokes the commitment scheme linear with the number of its set cardinality $|S|$ and constructs a Merkle tree on top of the commitments. %Therefore, its complexity is $O(|S|)$.  
%
In step \ref{smart-PSI::extractors}, each extractor invokes $\mathtt{H}$ linear with its set cardinality $|S|$; it also performs polynomial evaluations linear with $|S|$. 

\vspace{-2.7mm}
 \subsubsection{Smart Contracts' Cost.}

In step \ref{e-psi::invoke-remainer-F-PSI}, the subroutine smart contract $\mathcal{SC}_{\fpsi}$ performs $h\cdot m(3d+1)$ additions and $h$ polynomial divisions,  where each division includes dividing a polynomial of degree $3d+1$ by a polynomial of degree $1$ (in step \ref{compute-res-poly} of \fpsi). In step \ref{e-psi::SC-verification--derive-mk}, $\mathcal{SC}_{\epsi}$ invokes the commitment's verification algorithm $\comver$ once,  $\mathtt{H}$ at most $|S_{\st \cap}|$ times, and $\prf$ $|S_{\st \cap}| (3d+1)$ times. In step \ref{e-psi::SC-verification--check-three-vals}, $\mathcal{SC}_{\epsi}$ invokes  $\comver$ at most $|S_{\st \cap}|$ times, and calls $\mathtt{H}$ $O(|S_{\st \cap}|\cdot \log_{\st 2} |S|)$ times. In the same step, it  performs polynomial evaluation linear with  $|S_{\st \cap}|$. Thus, its overall complexity is $O( |S_{\st \cap}|(d+ \log_{\st 2} |S|))$.
%



%In the same step, the subroutine smart contract $\mathcal{SC}_{\fpsi}$ performs $h\cdot m(3d+1)$ additions and $h$ polynomial divisions,  where each division includes dividing a polynomial of degree $3d+1$ by a polynomial of degree $1$ (in step \ref{compute-res-poly} of \fpsi). 





%Moreover, if $Flag=True$, then each client invokes $\prf$ $h (3d+1)$ times, and performs $h (3d+1)$ additions, and performs polynomial evaluations linear with its set cardinality, where each evaluation involves  $O(d)$  additions and $O(\frac{d^{\st 2}+d}{2})$ multiplications (in step \ref{F-PSI::flag-is-true} of \fpsi). If $Flag=False$, then (a) \aud invokes $\prf$ $3h\cdot m(d+1)$ times  and  invokes $\mathtt{H}$ $O(h\cdot m\cdot d)$ times, and (b) $D$ performs $O(h\cdot m\cdot d)$ multiplications and additions (in step \ref{F-PSI::flag-is-false} of \fpsi). 




%

  
  % (where each evaluation involves  $O(d)$  additions and $O(\frac{d^{\st 2}+d}{2})$ multiplications). 



%In step \ref{e-psi::SC-verification}, $\mathcal{SC}_{\epsi}$ invokes the commitment's verification algorithm, the hash function, at most linear with the intersection cardinality $|S_{\st \cap}|$, invokes $\prf$ $3d+1$ times, invokes the hash function $O(\log_{\st 2} (d\cdot m))$ times, and performs polynomial evaluations linear with the smallest set cardinality.% (where each evaluation involves  $O(d)$  additions and $O(\frac{d^{\st 2}+d}{2})$ multiplications). 

 %\scf  performs $h\cdot m(3d+1)$ modular additions and $h$ polynomial divisions (in step \ref{compute-res-poly} of F-PSI). 


\vspace{-3mm}
\subsection{Communication Cost}

\vspace{-.5mm}

In steps  \ref{e-psi::call-F-PSI-stepOne} and \ref{e-psi::gen-mk-prime}, the communication cost of the clients is dominated by the cost of \ct which is $O(m)$. In steps \ref{e-psi::deploy-SC-E-PSI}--\ref{e-psi::commit-to-mk}, the clients' cost is negligible, as it involves sending a few transactions to the smart contracts, e.g., $\mathcal{SC}_{\fpsi}$, $\mathcal{SC}_{\epsi}$, and \SCpc. Step \ref{merkel-tree-cons} involves only extractors whose cost is $O(h)$ as each of them only sends to $\mathcal{SC}_{\epsi}$  a single value for each bin. In step \ref{e-psi::invoke-remainer-F-PSI}, the clients' cost is dominated by \vopr's cost; specifically, each pair of clients and $D$ invokes \vopr $O(d^{\st 2})$ times for each bin; therefore, the cost of each client (excluding $D$) is $O(h\cdot d^{\st 2}\cdot \bar\xi)$ while the cost of $D$ is $O(h\cdot d^{\st 2}\cdot \bar\xi\cdot m)$, where $\bar\xi$ is the subroutine \ole's security parameter. 
%
Step \ref{smart-PSI::extractors}  involves only the extractors, where each extractor's cost is dominated by the size of the Merkle tree's proof it sends to $\mathcal{SC}_{\epsi}$, i.e., $O(|S_{\scriptscriptstyle\cap}|\cdot \log_{\st 2}|S|)$, where $|S|$ is the extractor's set cardinality. 
%
In step \ref{F-PSI::flag-is-false}, \aud sends $h$ polynomials of degree $3d+1$ to $\mathcal{SC}_{\fpsi}$; so, its complexity is $O(h\cdot d)$. 
%
The rest of the steps impose negligible communication costs. 

\vspace{-2mm}
\subsection{Comparison}
Below we show that \epsi offers various features that the state-of-the-art PSIs do not offer simultaneously while keeping its overall overheads similar to the efficient PSIs.  

\vspace{-2mm}
\subsubsection{Computation Complexity.} The  computation complexity  of \epsi is similar to that of PSI in \cite{AbadiDMT22}, but is better than the multiparty PSI's complexity in \cite{DBLP:conf/ccs/KolesnikovMPRT17} as  the latter's complexity is quadratic with the number of parties, i.e., $O(|S|\cdot d\cdot m)$ versus $O(|S|\cdot m^{\st 2}+|S|\cdot m )$. Also, \epsi's complexity  is better than the complexity of the PSI in  \cite{NevoTY21}  that is quadratic with parameter $t$. Similar to the two-party PSIs in \cite{DebnathD16,RaghuramanR22}, \epsi's complexity is linear with $|S|$.  The two-party PSI in \cite{DBLP:conf/dbsec/DongCCR13} imposes a higher computation overhead than \epsi does, as its complexity is quadratic with sets' cardinality. Hence, the complexity of \epsi is: (i) linear with the set cardinality, similar to the above schemes except the one in \cite{DBLP:conf/dbsec/DongCCR13} and (ii) linear with the total number of parties, similar to  the above multi-party schemes, except the one in \cite{DBLP:conf/ccs/KolesnikovMPRT17}. 
%
%Hence, the computation complexity of \epsi is linear with the set cardinality and the number of parties, similar to the above schemes except for the ones in \cite{DBLP:conf/ccs/KolesnikovMPRT17,DBLP:conf/dbsec/DongCCR13} whose complexities are quadratic with the set cardinality or the number of parties. 

\vspace{-2mm}
\subsubsection{Communication Complexity.}  \epsi's communication complexity is slightly higher than the complexity of the PSI in \cite{AbadiDMT22}, by a factor of $d\cdot \bar\xi$. However, it is better than the  PSI's complexity in \cite{DBLP:conf/ccs/KolesnikovMPRT17} as the latter has a complexity quadratic with the number of parties. \epsi's complexity is slightly higher than the one in \cite{NevoTY21}, by a factor of $d$. Similar to the two-party PSIs in  \cite{DebnathD16,RaghuramanR22,DBLP:conf/dbsec/DongCCR13}, \epsi's complexity is linear with $c$. 
%
Therefore, the communication complexity of \epsi is linear with the set cardinality and number of parties, similar to the above schemes except for the one in \cite{DBLP:conf/ccs/KolesnikovMPRT17} whose complexity is quadratic with the number of parties. 

\vspace{-2mm}
\subsubsection{Features.} \epsi is the only scheme that offers all the five features, i.e., supports fairness, rewards participants, is based on symmetric key primitives, supports multi-party, and is secure against active adversaries. After \epsi is the scheme in \cite{NevoTY21} which offers three of the above features. The rest of the schemes support only two of the above features. For the sake of fair comparison, we highlight that our \epsi and  \fpsi are the only PSIs that use smart contracts (that require additional but standard blockchain-related assumptions), whereas the rest of the above protocols do not use smart contracts. 










%\input{Related-work}













