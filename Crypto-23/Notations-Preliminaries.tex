% !TEX root =main.tex


\vs

\section{Notations and Preliminaries}

\vs
\subsection{Notations}

Table \ref{table:notation-table} summarises the main notations used in this paper. 

\vs


% !TEX root =main.tex


\subsection{Notation Table}\label{sec::notation-table}
Table \ref{commu-breakdown-party} summarises the main notations used in the paper. 

\vspace{-3mm}

\begin{table}[!h]
\begin{scriptsize}
\begin{center}
\footnotesize{
\vspace{-6mm}
\caption{ \small{Notation Table}.}\label{commu-breakdown-party} 
\renewcommand{\arraystretch}{.9}
\scalebox{0.65}{
% 1st table
\begin{tabular}{|c|c|c|c|c|c|c|c|c|c|c|c|c|c|} 

%\hline 
%\multicolumn{1}{|c|}{\cellcolor{yellow!10}\scriptsize{Setting}}&\cellcolor{yellow!10} \scriptsize{Symbol}&\cellcolor{yellow!10} \scriptsize{Description}\\
%\hline 

\hline 
\multicolumn{1}{|c|}{\rotatebox[origin=c]{45}{\cellcolor{yellow!10}\scriptsize{ {Setting}}}}&\cellcolor{yellow!10} \scriptsize{Symbol}&\cellcolor{yellow!10} \scriptsize{Description}\\
\hline 

%{\rotatebox[origin=c]{90}{\scriptsize{ {Contracts}}}}

%%%%%%%%%. Generic %%%%%

\cellcolor{yellow!10}&\cellcolor{gray!20}\scriptsize$p$&\cellcolor{gray!20}\scriptsize \text{Large prime number}\\   
%
\cellcolor{yellow!10}&\cellcolor{white!20}\scriptsize{$\mathbb{F}_{\st p}$}&\cellcolor{white!20}\scriptsize \text{A finite field of prime order $p$}\\   
%
\cellcolor{yellow!10}&\cellcolor{gray!20}\scriptsize$\cl$&\cellcolor{gray!20}\scriptsize \text{Set of all clients, $\{ A_{\st 1},...,   A_{\st m},  D\}$ }\\   
%
\cellcolor{yellow!10}&\cellcolor{white!20}\scriptsize$D$&\cellcolor{white!20}\scriptsize \text{Dealer client}\\   
%
\cellcolor{yellow!10}&\cellcolor{gray!20}\scriptsize$A_{\st m}$&\cellcolor{gray!20}\scriptsize{Buyer client}\\ 
%
\cellcolor{yellow!10}&\cellcolor{white!20}\scriptsize$m$&\cellcolor{white!20}\scriptsize \text{Total number of clients (excluding $D$)}\\   
%
\cellcolor{yellow!10}&\cellcolor{gray!20}\scriptsize$\mathtt{H}$&\cellcolor{gray!20}\scriptsize \text{Hash function}\\ 
%
\cellcolor{yellow!10}&\cellcolor{white!20}\scriptsize$|S_{\st\cap}|$&\cellcolor{white!20}\scriptsize{Intersection size}\\ 
%
\cellcolor{yellow!10}&\cellcolor{gray!20}\scriptsize$\Smin$&\cellcolor{gray!20}\scriptsize{Smallest set's size}\\
%
\cellcolor{yellow!10}&\cellcolor{white!20}\scriptsize$\Smax$&\cellcolor{white!20}\scriptsize{Largest set's size}\\
%
\cellcolor{yellow!10}&\cellcolor{gray!20}\scriptsize$|$&\cellcolor{gray!20}\scriptsize{Divisible}\\
%
\cellcolor{yellow!10}&\cellcolor{white!20}\scriptsize$\setminus$&\cellcolor{white!20}\scriptsize{Set subtraction}\\
%
\cellcolor{yellow!10}&\cellcolor{gray!20}\scriptsize$c$&\cellcolor{gray!20}\scriptsize{Set's cardinality}\\ 
 %
 \cellcolor{yellow!10}&\cellcolor{white!20}\scriptsize$h$&\cellcolor{white!20}\scriptsize{Total number of bins in a hash table}\\ 
 %
 \cellcolor{yellow!10}&\cellcolor{gray!20}\scriptsize$d$&\cellcolor{gray!20}\scriptsize{A bin's capacity}\\ 
 %
\cellcolor{yellow!10}&\cellcolor{white!20}\scriptsize$\lambda$ &\cellcolor{white!20}\scriptsize Security parameter  \\  

%
\cellcolor{yellow!10}&\cellcolor{gray!20}\scriptsize \ole&\cellcolor{gray!20}\scriptsize{Oblivious Linear Evaluation}\\ 
%
\cellcolor{yellow!10}&\cellcolor{white!20}\scriptsize$\ole^{\st +}$&\cellcolor{white!20}\scriptsize{Advanced \ole}\\ 

\cellcolor{yellow!10}&\cellcolor{gray!20}\scriptsize$\comcom$&\cellcolor{gray!20}\scriptsize \text{Commitment algorithm of commitment}\\ 

\cellcolor{yellow!10}&\cellcolor{white!20}\scriptsize$\comver$&\cellcolor{white!20}\scriptsize \text{Verification algorithm of commitment}\\ 

\cellcolor{yellow!10}&\cellcolor{gray!20}\scriptsize$\mkgen$&\cellcolor{gray!20}\scriptsize \text{Tree construction algorithm of Merkle tree}\\ 

\cellcolor{yellow!10}&\cellcolor{white!20}\scriptsize$\mkprove$&\cellcolor{white!20}\scriptsize \text{Proof generation algorithm of Merkle tree}\\ 

\cellcolor{yellow!10}&\cellcolor{gray!20}\scriptsize$\mkver$&\cellcolor{gray!20}\scriptsize \text{Verification algorithm of Merkle tree}\\ 

\cellcolor{yellow!10}&\cellcolor{white!20}\scriptsize{\ct}&\cellcolor{white!20}\scriptsize \text{Coin tossing protocol}\\  

 \cellcolor{yellow!10}&\cellcolor{gray!20}\scriptsize{\vopr}&\cellcolor{gray!20}\scriptsize \text{Verifiable Oblivious Poly. Randomization}\\
 
 \cellcolor{yellow!10} &\cellcolor{white!20}\scriptsize{\zspa}&\cellcolor{white!20}\scriptsize \text{ Zero-sum Pseudorandom Values Agreement}\\
  
  \cellcolor{yellow!10}   &\cellcolor{gray!20}\scriptsize{\zspaa}&\cellcolor{gray!20}\scriptsize \text{\zspa with an External Auditor}\\

   \cellcolor{yellow!10}  &\cellcolor{white!20}\scriptsize{\p}&\cellcolor{white!20}\scriptsize \text{Multi-party PSI with Fair Compensation}\\
     
    \cellcolor{yellow!10}      &\cellcolor{gray!20}\scriptsize{\ep}&\cellcolor{gray!20}\scriptsize \text{Multi-party PSI with Fair Compensation and Reward}\\

  \cellcolor{yellow!10}   &\cellcolor{white!20}\scriptsize{\fpsi}&\cellcolor{white!20}\scriptsize \text{Protocol that realises \p}\\
     
     
  \cellcolor{yellow!10}   &\cellcolor{gray!20}\scriptsize{\epsi}&\cellcolor{gray!20}\scriptsize \text{
          Protocol that realises \ep}\\

\cellcolor{yellow!10}&\cellcolor{white!20}\scriptsize$\prf$ &\cellcolor{white!20}\scriptsize  Pseudorandom function \\ 


  \cellcolor{yellow!10}&\cellcolor{gray!20}\scriptsize$\prp$ &\cellcolor{gray!20}\scriptsize  Pseudorandom permutation \\ 

%
  \cellcolor{yellow!10}   &\cellcolor{white!20}\scriptsize{$gcd$}&\cellcolor{white!20}\scriptsize \text{Greatest common divisor}\\
%

\cellcolor{yellow!10}\multirow{-34}{*}{\rotatebox[origin=c]{90}{\cellcolor{yellow!10}\scriptsize{ {Generic}}}}
  \cellcolor{yellow!10}   &\cellcolor{gray!20}\scriptsize{$\negl$}&\cellcolor{gray!20}\scriptsize \text{Negligible function}\\
%
     \hline
%%%%%%%%%%%%
\end{tabular}
}
\scalebox{.668}{
\begin{tabular}{|c|c|c|c|c|c|c|c|c|c|c|c|c|c|} 
%%%%%%%%%%%
\hline 
\multicolumn{1}{|c|}{\rotatebox[origin=c]{45}{\cellcolor{yellow!10}\scriptsize{ {Setting}}}}&\cellcolor{yellow!10} \scriptsize{Symbol}&\cellcolor{yellow!10} \scriptsize{Description}\\
\hline 
     
\cellcolor{yellow!10}&\cellcolor{white!20}\scriptsize$\SCpc$&\cellcolor{white!20}\scriptsize \text{Prisoner's Contract}\\   

\cellcolor{yellow!10}&\cellcolor{gray!20}\scriptsize$\SCcc$&\cellcolor{gray!20}\scriptsize \text{Colluder’s Contract}\\   

\cellcolor{yellow!10}&\cellcolor{white!20}\scriptsize$\SCtc$&\cellcolor{white!20}\scriptsize \text{Traitor's Contract}\\   
%
\cellcolor{yellow!10}&\scriptsize  \cellcolor{gray!20}\scriptsize$\cc$&\cellcolor{gray!20}\scriptsize \text{Server’s cost for computing a task}\\   
%
\cellcolor{yellow!10}&\cellcolor{white!20}\scriptsize$\chc$&\cellcolor{white!20}\scriptsize \text{Auditor's cost for resolving disputes
}\\   
%
\cellcolor{yellow!10}&\scriptsize  \cellcolor{gray!20}\scriptsize$\dc$&\cellcolor{gray!20}\scriptsize \text{Deposit a server pays to get the job}\\  
%
\cellcolor{yellow!10}&\cellcolor{white!20}\scriptsize$\wc$&\cellcolor{white!20}\scriptsize \text{Amount a server receives for completing the task}\\  
%
\multirow{-8}{*}{\rotatebox[origin=c]{90}{\cellcolor{yellow!10}\scriptsize{ {Counter}}}}
%
\multirow{-8}{*}{\rotatebox[origin=c]{90}{\scriptsize{ {Collusion}}}}
%
\multirow{-8}{*}{\rotatebox[origin=c]{90}{\scriptsize{ {Contracts}}}}
&\cellcolor{gray!20}\scriptsize$(pk, sk)$&\cellcolor{gray!20}\scriptsize \text{\scf's auditor's public-private key pair}\\  
 %
\hline 
%%%% End of Counter Collusion Contacts %%%%%

%%%%%%%.      F-PSI      %%%%%%

%\qinit: Initiation predicate


\cellcolor{yellow!10}&\cellcolor{white!20}\scriptsize$\scf$&\cellcolor{white!20}\scriptsize {\fpsi's smart contract}\\   
%
\cellcolor{yellow!10}&\cellcolor{gray!20}\scriptsize$\bm\omega, \bm\omega',\bm\rho  $&\cellcolor{gray!20}\scriptsize {Random poly. of degree} $d$\\   
 %
\cellcolor{yellow!10}&\cellcolor{white!20}\scriptsize$\bm\gamma, \bm\delta$&\cellcolor{white!20}\scriptsize {Random poly. of degree} $d+1$\\
%
\cellcolor{yellow!10}&\cellcolor{gray!20}\scriptsize$\bm\nu^{\st{{(C)}}}$&\cellcolor{gray!20}\scriptsize {Blinded poly. sent by each $C$ to \scf}\\ 
%  
\cellcolor{yellow!10}&\cellcolor{white!20}\scriptsize$\bm\phi$&\cellcolor{white!20}\scriptsize {Blinded poly. encoding the intersection}\\   
%
\cellcolor{yellow!10}&\cellcolor{gray!20}\scriptsize$\bm\chi$&\cellcolor{gray!20}\scriptsize {Poly. sent to \scf to identify misbehaving parties}\\ 
%
\cellcolor{yellow!10}&\cellcolor{white!20}\scriptsize$\bar L$&\cellcolor{white!20}\scriptsize {List of identified misbehaving parties}\\ 
%

%%%%%%%%%%%%%
\cellcolor{yellow!10}&\cellcolor{gray!20}\scriptsize&\cellcolor{gray!20}\scriptsize {A portion of a party's deposit into \scf}\\   

\cellcolor{yellow!10}&\multirow{-2}{*}{\cellcolor{gray!20}\scriptsize$\yc$}&\cellcolor{gray!20}\scriptsize{transferred to honest clients if it misbehaves}\\ 
%%%%%%




\cellcolor{yellow!10}&\scriptsize$mk$&\scriptsize{Master key of \prf}\\ 
%
\cellcolor{yellow!10}&\cellcolor{gray!20}\scriptsize$\qinit$&\cellcolor{gray!20}\scriptsize{Initiation predicate}\\ 
%
\cellcolor{yellow!10}&\scriptsize$\qdel$&\scriptsize{Delivery predicate}\\ 
%
\cellcolor{yellow!10}&\cellcolor{gray!20}\scriptsize$\qUnFAbt$&\cellcolor{gray!20}\scriptsize{UnFair-Abort predicate}\\ 

\multirow{-14}{*}{\rotatebox[origin=c]{90}{\cellcolor{yellow!10}\scriptsize{ {\withFai (\fpsi)}}}}
%
\cellcolor{yellow!10}&\scriptsize$\qFAbt$&\scriptsize{Fair-Abort predicate}\\ 
%

\hline 
%%%%%   End of F-PSI  %%%%

%%%%%%%    E-PSI    %%%%%%

\cellcolor{yellow!10}&\cellcolor{gray!20}\scriptsize$\SCe$&\cellcolor{gray!20}\scriptsize {\epsi's smart contract} \\   
 %
\cellcolor{yellow!10}&\cellcolor{white!20}\scriptsize$\dc'$&\cellcolor{white!20}\scriptsize {Extractor's deposit} \\
  %   
\cellcolor{yellow!10}&\cellcolor{gray!20}\scriptsize$\yc'$&\cellcolor{gray!20}\scriptsize {Each client's deposit into \scf}\\   
%
\cellcolor{yellow!10}&\cellcolor{white!20}\scriptsize$\lc$&\cellcolor{white!20}\scriptsize {Reward a client earns for an intersection element}\\   
%
\cellcolor{yellow!10}&\cellcolor{gray!20}\scriptsize$\rc$&\cellcolor{gray!20}\scriptsize {Extractor's cost for extracting an intersection element}\\  
%
\cellcolor{yellow!10}&\cellcolor{white!20}\scriptsize$\fc$&\cellcolor{white!20}\scriptsize {Shorthand for $\lc(m-1)$}\\ 
%
\cellcolor{yellow!10}&\cellcolor{gray!20}&\cellcolor{gray!20}\scriptsize{Price a buyer pays for an intersection element}\\ 
%
\cellcolor{yellow!10}&\multirow{-2}{*}{\cellcolor{gray!20}\scriptsize$\vc$}&\cellcolor{gray!20}\scriptsize{$\vc=m\cdot \lc+2 \rc$}\\ 
%
\cellcolor{yellow!10}&\scriptsize$mk'$&\scriptsize{Another master key of \prf}\\ 
%

\cellcolor{yellow!10}&\cellcolor{gray!20}\scriptsize$ct_{\st mk}$&\cellcolor{gray!20}\scriptsize {Encryption of $mk$ under $pk$}\\   
%            
\cellcolor{yellow!10}&\scriptsize$\qdelwr$&\scriptsize{Delivery-with-Reward predicate}\\ 

\multirow{-12}{*}{\rotatebox[origin=c]{90}{\cellcolor{yellow!10}\scriptsize{ {\withRew (\epsi)}}}}
%
\cellcolor{yellow!10}&\cellcolor{gray!20}\scriptsize$\qUnFAbtwr$&\cellcolor{gray!20}\scriptsize{UnFair-Abort-with-Reward predicate}\\ 



\hline  



%%%%%%%%%%%

\end{tabular}\label{table:notation-table}}}
\end{center}
\end{scriptsize}
\vspace{-6mm}
\end{table}





\vs


% !TEX root =main.tex


\vs 
\vs

\subsection{Security Model}\label{sec::sec-model}

In this paper, we use the simulation-based paradigm of secure computation \cite{DBLP:books/cu/Goldreich2004} to define and prove our protocols. Since both types of active and passive adversaries are involved in our protocols, we will outline definitions for both types (and refer readers to Appendix \ref{sec::sec-model-long} for more details).  We consider a static adversary, we assume there is an authenticated private (off-chain) channel between the clients and we consider a standard public blockchain, e.g., Ethereum.
%
 
 \vs
 \vs
 \subsubsection{Two-party Computation.} A two-party protocol $\Gamma$ problem is captured by specifying a random process that maps pairs of inputs to pairs of outputs, one for each party. Such process is referred to as a functionality denoted by  $f:\{0,1\}^{\st *}\times\{0,1\}^{\st *}\rightarrow\{0,1\}^{\st *}\times\{0,1\}^{\st *}$, where $f:=(f_{\st 1},f_{\st 2})$. For every input pair $(x,y)$, the output pair is a random variable $(f_{\st 1} (x,y), f_{\st 2} (x,y))$, such that the party with input $x$ obtains $f_{\st 1} (x,y)$ while the party with input $y$ receives $f_{\st 2} (x,y)$. When $f$ is deterministic, then $f_{\st 1} =f_{\st 2}$. In the setting where $f$ is asymmetric and only one party (say the first one) receives the result, $f$ is defined as $f:=(f_{\st 1}(x,y), \bot)$. 
 
 \vs
 \vs
 \subsubsection{Security in the Presence of Passive Adversaries.} 
 
%  In the passive adversarial model, the party corrupted by such an adversary correctly follows the protocol specification. Nonetheless, the adversary obtains the internal state of the corrupted party, including the transcript of all the messages received, and tries to use this to learn information that should remain private. 
  
In this setting, a protocol is secure if whatever can be computed by a party in the protocol can be computed using its input and output only. 
  %
%  In the simulation-based model, it is required that a party’s view in a protocol's 
% execution can be simulated given only its input and output. This implies that the parties learn nothing from the protocol's execution. 
 %
\begin{definition}
Let $f$ be the deterministic functionality defined above. Protocol $\Gamma$ security computes $f$ in the presence of a static  passive adversary if there exist polynomial-time algorithms $(\mathsf {Sim}_{\st 1}, \mathsf {Sim}_{\st 2})$ such that:
\end{definition}
%
\begin{center}
{\small{
$
  \{\mathsf {Sim}_{\st 1}(x,f_{\st 1}(x,y))\}_{\st x,y}\stackrel{c}{\equiv} \{\mathsf{View}_{\st 1}^{\st \Gamma}(x,y) \}_{\st x,y},
  %
    \{\mathsf{Sim}_{\st 2}(x,f_{\st 2}(x,y))\}_{\st x,y}\stackrel{c}{\equiv} \{\mathsf{View}_{\st 2}^{\st \Gamma}(x,y) \}_{\st x,y}
$}}
  \end{center}
  %
  where party $i$’s view (during the execution of $\Gamma$) on input pair  $(x, y)$ is denoted by $\mathsf{View}_{\st i}^{\st \Gamma}(x,y)$ and equals $(w, r^{\st i}, m_{\st 1}^{\st i}, ..., m_{\st t}^{\st i})$, where $w\in\{x,y\}$ is the input of $i^{\st th}$ party, $r_{\st i}$ is the outcome of this party's internal random coin tosses, and $m_{\st j}^{\st i}$ represents the $j^{\st th}$ message this party receives.  %The output of the $i^{\st th}$ party during the execution of $\Gamma$ on $(x, y)$ is denoted by $\mathsf{Output}_{\st 1}^{\st \Gamma}(x,y)$ and can be generated from its own view of the execution.  The joint output of both parties is denoted by $\mathsf{Output}^{\st \Gamma}(x,y):=(\mathsf{Output}_{\st 1}^{\st \Gamma}(x,y), \mathsf{Output}_{\st 2}^{\st \Gamma}(x,y))$.

\vs
\vs
  
 \subsubsection{Security in the Presence of Active Adversaries.}  In this adversarial model, correctness is required beyond the possibility that a corrupted party may learn more than it should. To capture the threats,
a protocol's security is analyzed by comparing what an adversary can do in the real protocol to what it can do in an ideal scenario. This is formalized by considering an ideal computation involving an incorruptible Trusted Third Party (TTP) to whom the parties send their inputs and receive the output of the ideal functionality. %Below, we describe the executions in the ideal and real models. 

%\
%
%\
%
%
% 
%First, we describe the execution in the ideal model. Let $P_{\st 1}$ and $P_{\st 2}$ be the parties participating in the
%protocol, $i\in \{0, 1\}$ be the index of the corrupted party, and $\mathcal A$ be a non-uniform
%probabilistic polynomial-time adversary. Also, let $z$ be an auxiliary input given to $\mathcal A$ while  $x$ and $y$ be the input of party $P_{\st 1}$ and $P_{\st 2}$  respectively.  The honest party, $P_{\st j}$, sends its received input to TTP.  The corrupted party $P_{\st i}$ may either abort (by replacing the input with a special abort message $\Lambda_{\st i}$),  send its received input or send some other input of the same length to TTP. This decision is made by the adversary and may depend on the input value of $P_{\st i}$ and $z$. If TTP receives $\Lambda_{\st i}$, then it sends $\Lambda_{\st i}$ to the honest party and the ideal execution terminates.  Upon obtaining an input pair $(x, y)$, TTP computes $f_{\st 1}(x, y)$ and $f_{\st 2}(x, y)$. It first sends $f_{\st i}(x, y)$ to  $P_{\st i}$ which replies with ``continue'' or $\Lambda_{\st i}$. In the former case, TTP sends  $f_{\st j}(x, y)$ to  $P_{\st j}$ and in the latter it sends $\Lambda_{\st i}$ to  $P_{\st j}$. The honest party always outputs the message that it obtained from TTP. A malicious party may output an arbitrary function of its initial inputs and the message it has obtained from TTP.  The ideal execution of $f$ on inputs $(x, y)$ and $z$ is denoted by $\mathsf{Ideal}^{\st f}_{\st\mathcal{A}(z), i}(x,y)$ and is defined as the output pair of the honest party and $\mathcal{A}$ from the above ideal execution.  In the real model, the real two-party protocol $\Gamma$ is executed
%without the involvement of TTP. In this setting, $\mathcal{A}$ sends all messages on
%behalf of the corrupted party and may follow an arbitrary strategy.
%The honest party follows the instructions of $\Gamma$. The real execution of $\Gamma$ is denoted by $\mathsf{Real}^{\st \Gamma}_{\st\mathcal{A}(z), i}(x,y)$, it is defined as the joint output of the parties engaging in the real execution of $\Gamma$ (on the inputs), in the presence of $\mathcal{A}$.
% 
% 
% Next, we define security. At a high level, the definition states that a secure protocol in the real model emulates the ideal model. This is formulated by stating that adversaries in the ideal model can simulate executions of the protocol in the real model. 
 
\begin{definition}\label{def::MPC-active-adv}
Let $f$ be the two-party functionality defined above and $\Gamma$ be a two-party protocol that computes $f$.   Protocol $\Gamma$ securely computes $f$ with abort in the presence of static active adversaries if for every non-uniform probabilistic polynomial time adversary $\mathcal{A}$ for the real model, there exists a non-uniform probabilistic polynomial-time adversary (or simulator) $\mathsf{Sim}$ for the ideal model, such that for every $i\in \{0,1\}$, it holds that: 
%
$
\{\mathsf {Ideal}^{\st f}_{\st \mathsf{Sim}(z), i}(x,y)\}_{\st x,y,z}\stackrel{c}{\equiv} \{\mathsf{Real}_{\st \mathcal{A}(z), i}^{\st \Gamma}(x,y) \}_{\st x,y,z}
$
%
\end{definition}
 
 where the ideal execution of $f$ on inputs $(x,$ $ y)$ and $z$ is denoted by $\mathsf{Ideal}^{\st f}_{\st\mathcal{A}(z), i}(x,$ $y)$ and is defined as the output pair of the honest party and $\mathcal{A}$ from the ideal execution. The real execution of $\Gamma$ is denoted by $\mathsf{Real}^{\st \Gamma}_{\st\mathcal{A}(z), i}(x,y)$, it is defined as the joint output of the parties engaging in the real execution of $\Gamma$ (on the inputs), in the presence of $\mathcal{A}$.
  
  


\vs 


\subsection{Smart Contracts}
\svs

Cryptocurrencies, such as Bitcoin \cite{bitcoin} and Ethereum \cite{ethereum}, beyond offering a decentralised currency,  support computations on transactions. In this setting, a certain computation logic is encoded in a computer program, called a \emph{``smart contract''}. To date, Ethereum is the most predominant cryptocurrency framework that enables users to define arbitrary smart contracts. In this framework,  contract code is stored on the blockchain and executed by all parties (i.e., miners) maintaining the cryptocurrency. The program execution's correctness is guaranteed by the security of the underlying blockchain components. %To prevent a denial-of-service attack, the framework requires a transaction creator to pay a  fee, called \emph{``gas''}, depending on the complexity of the contract running on it. 

\vs


\subsection{Counter Collusion Smart Contracts}\label{Counter-Collusion-Smart-Contracts}



To let a client efficiently delegate a computation to a  couple of potentially colluding rational servers, Dong   \et \cite{dong2017betrayal} proposed two main smart contracts; namely, ``Prisoner's Contract'' ($\SCpc$) and ``Traitor's Contract'' (\SCtc).  
%
\SCpc is signed by the client and the servers. It tries to incentivize correct computation by using the following idea. It requires each server to pay a deposit before the computation is delegated and is equipped with an external auditor that is invoked to detect a misbehaving server only when the servers provide non-equal results. 



If a server behaves honestly, it can withdraw its deposit. But, if a cheating server is detected by the auditor, then (a portion) of its deposit is transferred to the client. If one of the servers is honest and the other one cheats, then the honest server receives a reward taken from the cheating server's deposit. However, the dilemma, created by \SCpc between the two servers, can be addressed if they can make an enforceable promise, e.g., via a ``Colluder's Contract'' (\SCcc),  in which one party, called ``ringleader'', would pay its counterparty a bribe if both provide an incorrect (but consistent) result. To counter \SCcc, Dong   \et proposed \SCtc, which incentivises a colluding server to betray the other one and report the collusion without being penalised by \SCpc. In this work, we slightly adjust and use these contracts. We state these contracts' main parameters in Table \ref{table:notation-table}. We refer readers to Appendix \ref{appendix::Counter-Collusion-Contracts} for the full description of the  contracts. 




%\begin{itemize}
%\item[$\bullet$] $\bc$: the bribe paid by the ringleader of the collusion to the other
%server in the collusion agreement, in the Colluder’s contract.
%%
%\item[$\bullet$] $\cc$: a server’s cost for computing the task.
%%
%\item[$\bullet$] $\chc$: the fee paid to to invoke an auditor for recomputing a task and resolving
%disputes.
%%
%\item[$\bullet$] $\dc$: the deposit a server needs to pay to be eligible for getting the job.
%%
%\item[$\bullet$] $\tc$: the deposit the colluding parties need to pay in the collusion agreement, in the Colluder’s contract.
%%
%\item[$\bullet$] $\wc$: the amount that a server receives for completing the task.
%%
%\item[$\bullet$] $\wc \geq \cc$: the server would not accept underpaid jobs.
%%
%\item[$\bullet$] $\chc > 2\wc$: If it does not hold, then there would be no need to use the servers and the auditor would do the computation.
%%
%\item [$\bullet$] $(pk,sk)$: an asymmetric-key encryption's public-private key pair belonging to the auditor. 
%\end{itemize}
%\noindent The following relations need to hold when setting the contracts
%in order for the desirable equilibria to hold:
%%
%(i) $\dc>\cc+\chc$, (ii) $\bc<\cc$, and (iii) $\tc<\wc-\cc + 2\dc - \chc -\bc$.
%

%

\vs
\vs
\subsection{Pseudorandom Function and Permutation}
\svs

Informally, a pseudorandom function is a deterministic function that takes a key of length $\lambda$ and an input; and outputs a value  indistinguishable from that of  a truly random function.  In this paper, we use pseudorandom functions:   $\mathtt {PRF}: \{0,1\}^{\st \lambda}\times \{0,1\}^{\st *} \rightarrow  \mathbb{F}_{\st p}$, where $|p|=\lambda$ is the security parameter \cite{DBLP:books/crc/KatzLindell2007}. 


The definition of a pseudorandom permutation, $\mathtt {PRP}: \{0,1\}^{\st \lambda}\times \{0,1\}^{\st *} \rightarrow  \mathbb{F}_{\st p}$, is very similar to that of a pseudorandom function, with a difference; namely, it is required the keyed function $\PRP(k,.)$ to be indistinguishable from a uniform permutation, instead of a uniform function. In cryptographic schemes that involve $\PRP$, sometimes honest parties may require to compute the inverse of pseudorandom permutation, i.e., $\mathtt {PRP}^{\st -1}(k, .)$, as well. In this case, it would require that $\PRP(k,.)$ be indistinguishable from a uniform permutation even if the distinguisher is additionally given oracle access to the inverse of the permutation. 


\vs

%\subsection{Random Extraction Beacon}
%\subsection{Commitment Scheme}
% !TEX root =main.tex



\vs

\subsection{Commitment Scheme}\label{subsec:commit}


 A commitment scheme involves a  \emph{sender} and a \emph{receiver}. It also  involves  two phases; namely, \emph{commit} and  \emph{open} \cite{DBLP:books/cu/Goldreich2004}. In the commit phase, the sender commits to a message: $x$ as $\comcom(x,r)=com$, that involves a secret value: $r\stackrel{\st\$}\leftarrow \{0,1\}^{\st\lambda}$. At the end of the commit phase,  the commitment ${com}$ is sent to the receiver. In the open phase, the sender sends the opening $\hat{x}:=(x, r)$ to the receiver who verifies its correctness: $\comver({com},\hat{x})\stackrel{\st ?}=1$ and accepts if the output is $1$.  A commitment scheme must satisfy two properties: (a) \textit{hiding}: it is infeasible for an adversary (i.e., the receiver) to learn any information about the committed  message $x$, until the commitment ${com}$ is opened, and (b) \textit{binding}: it is infeasible for an adversary (i.e., the sender) to open a commitment ${com}$ to different values $\hat{x}':=(x',r')$ than that was  used in the commit phase, i.e., infeasible to find  $\hat{x}'$, \textit{s.t.} $\comver({com},\hat{x})=\comver({com},\hat{x}')=1$, where $\hat{x}\neq \hat{x}'$.  %There exist efficient  commitment schemes both in (a) the standard model, e.g., Pedersen scheme \cite{Pedersen91}, and (b)  the random oracle model using the well-known hash-based scheme such that committing  is : $\mathtt{H}(x||r)={com}$ and $\comver({com},\hat{x})$ requires checking: $\mathtt{H}(x||r)\stackrel{\st ?}={com}$, where $\mathtt{H}:\{0,1\}^{\st *}\rightarrow \{0,1\}^{\st\lambda}$ is a collision-resistant hash function, i.e., the probability to find $x$ and $x'$ such that $\mathtt{H}(x)=\mathtt{H}(x')$ is negligible in the security parameter $\lambda$.

\vs

\subsection{Hash Tables}

\svs

A hash table is an array of  bins each of which can hold a set of elements. It comes with a hash function, $\mathtt{H}$. To insert an element, we first compute the element's hash,  and then store the element in the bin whose index is the element's hash. Given the maximum number of elements $c$ and the bin's maximum size $d$, we can determine the number of bins, $h$, by analysing hash tables under the balls into the bins model  \cite{DBLP:conf/stoc/BerenbrinkCSV00}. Appendix \ref{Preliminary-Hash-Table} explains how the hash table parameters are set.

\vs

%\subsection{Merkel Tree}
% !TEX root =main.tex


\vspace{-.6mm}

\subsection{Merkle Tree}\label{sec::merkle-tree-short}
\vspace{-.7mm}


A Merkle tree is a data structure that facilitates a concise commitment to a set of values or blocks, involving two parties: a prover and a verifier. 
%
The Merkle tree scheme includes three algorithms; namely, $\mathtt{MT.genTree}$, $ \mathtt{MT.prove}$, and  $\mathtt{MT.verify}$. Briefly, the first algorithm constructs a Merkle tree on a set of blocks, the second generates a proof of a block's (or set of blocks') membership, and the third verifies the proof. Appendix \ref{sec::merkle-tree} provides more details. 

%
%
%
% The  Merkle tree scheme includes three algorithms $(\mkgen, \mkprove,$ $\mkver)$, defined as follows: 
%
%\begin{itemize}
%%
%\item[$\bullet$] The algorithm that constructs a Merkle tree, $\mkgen$, is run by $\mathcal{V}$. It takes  blocks, $u:=u_{\st 1},...,u_{\st n}$, as input. Then, it groups the blocks  in pairs. Next,   a collision-resistant hash function, $\mathtt{H}(.)$, is used to hash each pair. After that, the hash values are grouped in pairs and each pair is further hashed, and this process is repeated until only a single hash value, called ``root'', remains. This yields a  tree with the leaves corresponding to the input blocks and the root corresponding to the last remaining hash value. $\mathcal{V}$ sends the root to $\mathcal{P}$.
%%
%\item[$\bullet$] The proving algorithm, $\mkprove$, is run by $\mathcal{P}$. It takes a block index, $i$, and a tree as inputs. It outputs a vector proof, of  $\log_{\st 2}(n)$ elements. The proof asserts the membership of $i$-th block in the tree, and consists of all the sibling nodes on a path from the $i$-th block to the root of the Merkle tree (including $i$-th block). The proof is given to $\mathcal{V}$.
%%
%\item[$\bullet$] The verification algorithm, $\mkver$, is run by $\mathcal{V}$. It takes as an input $i$-th block, a proof, and the tree's root. It checks if the $i$-th block corresponds to the root. If the verification passes, it outputs $1$; otherwise, it outputs $0$.
%
%\end{itemize}
%
%The Merkle tree-based scheme has two properties: \emph{correctness} and \emph{security}. Informally, the correctness requires that if both parties run the algorithms correctly, then a proof is always accepted by  $\mathcal{V}$. The security requires that a computationally bounded malicious $\mathcal{P}$ cannot convince  $\mathcal{V}$ into accepting an incorrect proof, e.g., proof for a non-member block. The security relies on the assumption that it is computationally infeasible to find the hash function's collision. Usually, for the sake of simplicity, it is assumed that the number of blocks, $n$, is a power of $2$. The height of the tree, constructed on $m$ blocks, is $\log_{\st 2}(n)$. 

\vs

\subsection{Polynomial Representation of Sets}\label{sec::poly-rep}
\svs

Using a polynomial to represent a set's elements was proposed by Freedman  \et in \cite{DBLP:conf/eurocrypt/FreedmanNP04}. In this representation, set elements $S=\{s_{\st 1},...,s_{\st d}\}$ are defined over  $\mathbb{F}_{\st p}$ and  set $S$ is represented as a polynomial of   form: $\mathbf{p}(x)=\prod\limits ^{\st {d}}_{\st i=1}(x-s_{\st i})$, where $\mathbf{p}(x) \in \mathbb{F}_{\st p}[X]$ and $\mathbb{F}_{\st p}[X]$ is a polynomial ring.  Often a   polynomial,  $\mathbf{p}(x)$, of degree $d$ is  represented in the ``coefficient form'' as follows:  $\mathbf{p}(x)=a_{\st 0}+a_{\st 1}\cdot x+...+ a_{\st d}\cdot x^{\st d}$. As shown in \cite{DBLP:conf/crypto/KissnerS05}, for two sets $S^{\st (A)}$ and $S^{\st (B)}$ represented by polynomials $\mathbf{p}_{\st A}$ and $\mathbf{p}_{\st B}$ respectively, their product: $\mathbf{p}_{\st A}\cdot \mathbf{p}_{\st  B}$  represents the set union, while their greatest common divisor: $gcd($$\mathbf{p}_{\st A}$$,\mathbf{p}_{\st B})$ represents the set intersection. For two polynomials $\mathbf{p}_{\st A}$ and $\mathbf{p}_{\st B}$ of degree $d$, and two random polynomials $\bm\gamma_{\st A}$ and  $\bm\gamma_{\st B}$ of degree $d$, it is proven in~\cite{DBLP:conf/crypto/KissnerS05} that: $\bm\theta=\bm\gamma_{\st A}\cdot \mathbf{p}_{\st A}+\bm\gamma_{\st B}\cdot\mathbf{p}_{\st B}=\bm\mu\cdot gcd(\mathbf{p}_{\st A},\mathbf{p}_{\st B})$, where $\bm\mu$ is a uniformly random polynomial, and polynomial $\bm\theta$ contains only information about the elements in  $S^{\st (A)}\cap S^{\st (B)}$, and contains no information about other elements in $S^{\st (A)}$ or $S^{\st (B)}$.  

Given a polynomial $\bm\theta$ that encodes sets intersection, one can find the set elements in the intersection via one of the following approaches. First, via polynomial evaluation: the party who already has one of the original input sets, say  $\mathbf{p}_{\st A}$,  evaluates $\bm\theta$ at every element $s_{\st i}$ of $\mathbf{p}_{\st A}$ and considers $s_{\st i}$ in the intersection if $\mathbf{p}_{\st A}(s_{\st i})=0$. Second,   via polynomial root extraction:   the party who does not have one of the original input sets, extracts the roots of $\bm\theta$,  which contain  the roots of (i) random polynomial  $\bm\mu$ and (ii) the polynomial that represents the intersection, i.e., $gcd(\mathbf{p}_{\st A},\mathbf{p}_{\st B})$. In this approach, to distinguish errors (i.e., roots of $\bm\mu$) from the intersection, PSIs in \cite{eopsi,DBLP:conf/crypto/KissnerS05} use the \emph{``hash-based padding technique''}. In this technique, every element $u_{\st i}$ in the set universe $\mathcal{U}$, becomes $s_{\st i}=u_{\st i}||\mathtt{H}(u_{\st i})$, where $\mathtt{H}$ is a cryptographic hash function with a sufficiently large output size. Given a field's arbitrary element, $s \in \mathbb{F}_p$ and $\mathtt{H}$'s output size $|\mathtt{H}(.)|$, we can parse $s$ into $x_{\st 1}$ and $x_{\st 2}$, such that $s=x_{\st 1}||x_{\st 2}$ and  $|x_{\st 2}|=|\mathtt{H}(.)|$. In a  PSI that uses polynomial representation and this padding technique, after we extract each root of  $\bm\theta$, say $s$, we parse it into $(x_{\st 1}, x_{\st 2})$ and check $x_{\st 2}\stackrel{?}=\mathtt{H}(x_{\st 1})$.  If the equation holds, then we consider $s$ as an element of the intersection. 




%\TZ{What is meant by ``$\bm\theta$ contains only information about $S^{\st (A)}\cap S^{\st (B)}$"?}--> addressed.. 

%Polynomials can also be represented in the  ``point-value form''. In particular, a polynomial $\mathbf{p}(x)$ of degree $d$ can be represented as a set of $m$ ($m>d$) point-value pairs $\{(x_{\st 1},y_{\st 1}),...,$ $(x_{\st m},y_{\st m})\}$ such that all $x_{\st i}$ are distinct  non-zero points and $y_{\st i}=\mathbf{p}(x_{\st i})$ for all $i$, $1\le i\le m$. If  $x_{\st i}$  are fixed, then we can represent polynomials as a vector $\vv{\bm{y}}=[y_{\st 1}, ..., y_{\st m}]$. Polynomials in point-value form have  been used previously in PSIs~\cite{eopsi,opsi15,DBLP:conf/fc/AbadiTD16,Feather2020,GhoshS19,KolesnikovMPRT17}. A polynomial
%in this form can be converted into coefficient form via polynomial interpolation, e.g., using Lagrange interpolation~\cite{aho19}. Moreover,  one can add or multiply two polynomials,  in point-value form, by adding or multiplying their corresponding y-coordinates. In this case, the  polynomial interpolated from the result would be the two polynomials' addition or product. Often PSIs  that use this representation  assume that all $x_{\st i}$ are picked from $\mathbb{F} \setminus \mathcal{U}$.



\vs


\subsection{Horner's Method}
\svs

Horner's method \cite{DBLP:journals/ibmrd/Dorn62} allows for efficiently evaluating polynomials at a given point. Specifically, given a polynomial of the form: $\bm\tau(x)= a_{\st 0}+a_{\st 1}\cdot x+a_{\st 2}\cdot x^{\st 2}+...+a_{\st n}\cdot x^{\st n}$ and a point: $x_{\st 0}$, one can efficiently evaluate $\bm\tau(x)$ at $x_{\st 0}$ iteratively, in the following fashion: $\bm\tau(x_{\st 0})=a_{\st 0}+x_{\st 0}(a_{\st 1} + x_{\st 0}(a_{\st 2}+...+x_{\st 0}(a_{\st n-1}+x_{\st 0}\cdot a_{\st n})...)))$. Evaluating  a polynomial of degree $n$ naively requires  $n$ additions and $\frac{(n^{\st 2}+n)}{2}$ multiplications. However, using Horner's method the evaluation requires only $n$ additions and $n$ multiplications. We use this method in this paper. 

\vs


\subsection{Oblivious Linear Function Evaluation}\label{sec::OLE-plus}

Oblivious Linear function Evaluation (\ole) is a two-party protocol that involves a sender and receiver. In \ole,  the sender  has two  inputs  $a, b\in \mathbb{F}_{\st p}$ and the receiver has a single input, $c \in \mathbb{F}_{p}$.  The protocol allows the receiver to learn only $s = a\cdot c + b \in \mathbb{F}_{\st p}$, while the sender learns nothing. Ghosh \textit{et al.} \cite{GhoshNN17} proposed an efficient \ole that has $O(1)$ overhead and involves mainly symmetric key operations. Later, in \cite{GhoshN19} an enhanced \ole, called $\ole^{\st +}$, was proposed. The latter ensures that the receiver cannot learn anything about the sender's inputs,  even if it sets its input to $0$. In this paper, we use $\ole^{\st +}$. We refer readers to Appendix \ref{apndx:F-OLE-plus}, for its construction.  %In this case, each party picks a random string, 


\vs

\subsection{Coin-Tossing Protocol}

A Coin-Tossing protocol, \ct, allows two mutually distrustful parties, say $A$ and $B$, to jointly generate a single random bit. Formally, \ct computes the functionality $\fct(in_{\st A}, in_{\st B})\rightarrow (out_{\st A}, out_{\st B})$, which takes $in_{\st A}$ and  $in_{\st B}$ as inputs of $A$ and $B$ respectively and outputs $out_{\st A}$ to $A$ and $out_{\st B}$ to $B$, where $out_{\st A}=out_{\st B}$. A basic security requirement of a \ct is that the resulting bit is indistinguishable from a truly random bit. Blum proposed a simple \ct in \cite{Blum82} that works as follows. Party $A$ picks a random bit $in_{\st A}\stackrel{\st \$}\leftarrow\{0,1\}$, commits to it and sends the commitment to $B$ which sends its choice of random input, $in_{\st B}\stackrel{\st \$}\leftarrow\{0,1\}$, to $A$ which sends the commitment opening (including $in_{\st A}$) to $B$, which checks if the commitment matches its opening. If so, each party computes the final random bit as $in_{\st A}\oplus in_{\st B}$.  

There also exist \emph{fair} coin-tossing protocols, e.g., in \cite{MoranNS09}, that ensure either both parties learn the result or nobody does. They can be generalised to \emph{multi-party} coin-tossing protocols to generate a \emph{random string}, e.g., see \cite{BeimelOO10,KiayiasRDO17}.
%
The  complexities of (fair) multi-party coin-tossing protocols are often linear with the number of participants. In this paper, any multi-party \ct that generates a random string can be used. For simplicity, we let a multi-party \fct take $m$ inputs and output a single value, i.e., $\fct(in_{\st 1}, ..., in_{\st m})\rightarrow out$. 

%,  as an aborting party can be excluded from the next run of the protocol and the aborting party cannot learn partsets' intersection 







