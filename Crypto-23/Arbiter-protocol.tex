% !TEX root =main.tex




\vs




\subsection{\zspa's Extension: \zspa with an External Auditor (\zspaa)}


In this section, we present an extension of \zspa, called \zspaa which lets a (trusted) third-party auditor, \aud, help identify misbehaving clients in the \zspa and generate a vector of random polynomials. Informally, \zspaa requires that misbehaving parties are always detected, except with a negligible probability. \aud of this protocol will be invoked by \withFai when \withFai's smart contract detects that a combination of the messages sent by the clients is not well-formed. Later, in \withFai's proof, we will show that even a \emph{semi-honest} \aud who observes all messages that clients send to \withFai's smart contracts, cannot learn anything about their set elements. We present \zspaa in Figure \ref{fig:arbiter}. 


\vs


% !TEX root =main.tex




\begin{figure}[ht]%[!htbp]
\setlength{\fboxsep}{1pt}
\begin{center}
\scalebox{.85}{
    \begin{tcolorbox}[enhanced,width=5.5in, 
    drop fuzzy shadow southwest,
    colframe=black,colback=white]


{\small{

%\underline{$\mathtt{Audit}( \vv{{k}},  q, \bm\zeta, \bar d, g, \vv v)\rightarrow (L, \vv{{\mu}})$}
\begin{enumerate}[leftmargin=1mm]
%\item[$\bullet$] Parties: clients: $\{  {   A}_{    {    1}},...,   {   A}_{    {    m}}\}$, the dealer and  an Arbiter.


\item[$\bullet$]    {Parties.} A set of clients $\{ A_{\st 1},...,  A_{\st m}\}$ and an external auditor, \aud. 

\item[$\bullet$]    {Input.}  $m$: the total number of participants (excluding the auditor), $\bm\zeta$: a random polynomial of degree $1$, $b$: the total number of vectors, and $adr$: a deployed smart contract's address. Let $b'=b-1$.





%\item[$\bullet$]   {Input.} $\vv{{k}}=[k_{\st 1},..., k_{\st m}]$,    $q$: a  hash value, $\bm\zeta$: a random polynoimal of degree $1$, $\bar d$: a polynoimal's degree,   $g$: a root of Merkle tree, and $\vv v$: binary vector of size $m$. 


\item[$\bullet$]  {Output of  each} $  A_{\st j}$.   $k$: a secret key that generates $b$ vectors $[z_{\scriptscriptstyle 0,1},...,z_{\scriptscriptstyle 0,m}],...,[z_{\scriptscriptstyle b',1},...,z_{\scriptscriptstyle b', m}]$ of pseudorandom values, $h$: hash of the key,  $g$: a Merkle tree's root, and a vector of signed messages. 



\item[$\bullet$]    {Output of \aud.} $L$: a list of misbehaving parties' indices, and  $\vv{{\mu}}$: a vector of random polynomials.
%
\item\label{ZSPA::ZSPA-invocation} {\textbf{\zspa invocation.}  $\zspa(\bot,..., \bot)\rightarrow \Big((k, g, q),..., (k, g,q )\Big)$}. 

All parties in $\{A_{\st 1},...,  A_{\st m}\}$ call the same instance of \zspa, which results in  $(k, g, q), ..., (k, g, q)$. 
%

\item\label{ZSPA-A::Auditor-computation}  {\textbf{Auditor computation.} $\mathtt{Audit}( \vv{{k}},  q, \bm\zeta, b, g)\rightarrow (L, \vv{{\mu}})$}. 

\aud\ takes the below steps. Note,  each $k_{\st j}\in \vv{{k}}$ is given by  $  A_{\st j}$. An honest party's input, $k_{\st j}$,  equals $k$, where $1\leq j \leq m$. 


\begin{enumerate}
%
\item runs the checks in the verification phase (i.e., Phase \ref{ZSPA:verify}) of \zspa for every $j$, i.e., $\mathtt{Verify}(k_{\st j}, g, q, m)\rightarrow (a_{\st j}, s)$.
\item appends $j$ to $L$, if any checks fails, i.e., if $a_{\st j}=0$. In this case, it skips the next two steps for the current $j$. 



%
%
%\item  Checks whether equation $\mathtt{H}(k_{\st j})=q$ holds  for every $j$, $1\leq j \leq m$.   
%%
%\begin{itemize}
%%
%\item[$\bullet$] if any $j$-th check fails,  it adds $j$ to $L$.
%%
%\item[$\bullet$]  if $L$ contains all $j\in[1,m]$, it returns $L$ and aborts. 
%%
%\end{itemize}
%%
%\item\label{zero-sum-arbiter-verification} Verifies the Merkle tree's root, $g$, by checking if the tree (corresponding to  $g$) has been correctly constructed on the correct leaf nodes. In particular, it takes the following steps. 
%
%\begin{enumerate}
%
%\item regenerates the tree's leaf nodes (similar to step \ref{ZSPA:val-gen} in Fig. \ref{fig:ZSPA}) as follows. Let $k$ be a key that passed the above check.  For every $i$ (where $0\leq i \leq \bar d$), it recomputes $m$ pseudorandom values: 
%%
%$$\forall j, 1\leq j \leq m-1: z_{\st i,j}=\mathtt{PRF}(k,i||j), \hspace{4mm} z_{\st i,m}=-\sum\limits^{\st m-1}_{\st j=1}z_{\st i,j}$$
%%
%\item   constructs a Merkel tree on top of all pseudorandom values generated in the previous step, i.e., $\mathtt{MT.genTree}(z_{\st 0,1},...,z_{\st \bar d,m})\rightarrow g'$. 
%%
%\item checks if $g=g'$. If the equation does not hold, then it adds to $L$ every index $j$ whose value in $\vv v$ is $1$, i.e., $\vv v[j]=1$; in this case, it returns $L$ and aborts.
%%
%\end{enumerate}
%

\item\label{ZSPA-A::gen-z} For every $i$ (where $0\leq i \leq b'$), it recomputes $m$ pseudorandom values: 
%
$\forall j, 1\leq j \leq m-1: z_{\st i,j}=\mathtt{PRF}(k,i||j), \hspace{4mm} z_{\st i,m}=-\sum\limits^{\st m-1}_{\st j=1}z_{\st i,j}$.
%
 \item generates polynomial $\bm\mu^{\st (j)}$ as follows: 
  %
   $\bm\mu^{\st (j)} = \bm\zeta\cdot \bm\xi^{\st (j)}-\bm\tau^{\st (j)}$, 
   %
    where $\bm\xi^{\st (j)}$ is a random polynomial of degree $b'-1$ and $\bm\tau^{\st (j)}=\sum\limits^{\st b'}_{\st i=0}z_{\st i,j}\cdot x^{\st i}$. By the end of this step, a vector $\vv{{\mu}}$ containing at most $m$ polynomials is generated. 
%
 \item returns   list $L$ and $\vv{{\mu}}$.
 
\end{enumerate}
 \end{enumerate}
}}
 \end{tcolorbox}
 }
\end{center}
\vs
\vs
\caption{\zspa with an external auditor (\zspaa)} 
\label{fig:arbiter}
\end{figure}



%%%%%%%%%%%%%%%%%%%%%%%%%%%%%%%%%%%%%%%%%%%%%%
%\begin{figure}[ht]%[!htbp]
%\setlength{\fboxsep}{1pt}
%\begin{center}
%    \begin{tcolorbox}[enhanced,width=5.5in, 
%    drop fuzzy shadow southwest,
%    colframe=black,colback=white]
%
%
%{\small{
%
%\underline{$\mathtt{Audit}( \vv{{k}},  q, \bm\zeta, \bar d, g, \vv v)\rightarrow (L, \vv{{\mu}})$}
%\begin{enumerate}
%%\item[$\bullet$] Parties: clients: $\{  {   A}_{    {    1}},...,   {   A}_{    {    m}}\}$, the dealer and  an Arbiter.
%\item[$\bullet$]   {Input.} $\vv{{k}}=[k_{\st 1},...,k_{\st m}]$,    $q$: a  hash value, $\bm\zeta$: a random polynoimal of degree $1$, $\bar d$: a polynoimal's degree,   $g$: a root of Merkle tree, and $\vv v$: binary vector of size $m$. 
%
%
%\item[$\bullet$]    {Output.} A list of rejected values' indices: $L$, a vector of random polynomials: $\vv{{\mu}}$.
%%
%\item  Checks whether equation $\mathtt{H}(k_{\st j})=q$ holds  for every $j$, $1\leq j \leq m$.   
%%
%\begin{itemize}
%%
%\item[$\bullet$] if any $j$-th check fails,  it adds $j$ to $L$.
%%
%\item[$\bullet$]  if $L$ contains all $j\in[1,m]$, it returns $L$ and aborts. 
%%
%\end{itemize}
%%
%\item\label{zero-sum-arbiter-verification} Verifies the Merkle tree's root, $g$, by checking if the tree (corresponding to  $g$) has been correctly constructed on the correct leaf nodes. In particular, it takes the following steps. 
%
%\begin{enumerate}
%
%\item regenerates the tree's leaf nodes (similar to step \ref{ZSPA:val-gen} in Fig. \ref{fig:ZSPA}) as follows. Let $k$ be a key that passed the above check.  For every $i$ (where $0\leq i \leq \bar d$), it recomputes $m$ pseudorandom values: 
%%
%$$\forall j, 1\leq j \leq m-1: z_{\st i,j}=\mathtt{PRF}(k,i||j), \hspace{4mm} z_{\st i,m}=-\sum\limits^{\st m-1}_{\st j=1}z_{\st i,j}$$
%%
%\item   constructs a Merkel tree on top of all pseudorandom values generated in the previous step, i.e., $\mathtt{MT.genTree}(z_{\st 0,1},...,z_{\st \bar d,m})\rightarrow g'$. 
%%
%\item checks if $g=g'$. If the equation does not hold, then it adds to $L$ every index $j$ whose value in $\vv v$ is $1$, i.e., $\vv v[j]=1$; in this case, it returns $L$ and aborts.
%%
%\end{enumerate}
%%
% \item Generates polynomial $\bm\mu^{\st (j)}$, for every $j$ such that $j\in[1,m]$ and $j \notin L$,  as follows:
%  %
%   $$\bm\mu^{\st (j)} = \bm\zeta\cdot \bm\xi^{\st (j)}-\bm\tau^{\st (j)}$$
%   %
%    where $\bm\xi^{\st (j)}$ is a random polynomial of degree $\bar d-1$ and $\bm\tau^{\st (j)}=\sum\limits^{\st \bar d}_{\st i=0}z_{\st i,j}\cdot x^{\st i}$. By the end of this step, a vector $\vv{{\mu}}$ containing at most $m$ polynomials is generated. 
%%
% \item Returns   list $L$ and $\vv{{\mu}}$.
% 
%
% \end{enumerate}
%}}
% \end{tcolorbox}
%\end{center}
%\caption{$\text{Audit}$ Algorithm} 
%\label{fig:arbiter}
%\end{figure}





\begin{theorem}\label{theorem::ZSPA-A}
If \zspa is secure, $\mathtt{H}$ is second-preimage resistant, and the correctness of $\mathtt{PRF}$, $\mathtt{H}$, and Merkle tree holds,  then \zspaa securely computes $f^{\st \zspaa}$ in the presence of $m-1 $ malicious adversaries.% or (ii) a semi-honest auditor. 
\end{theorem}

\svs

We refer readers to Appendix \ref{sec::proof-of-zspaa} for the proof of Theorem \ref{theorem::ZSPA-A}. 

%As we stated previously, the ZSPA-A protocol will be invoked as a subroutine in the fair PSI protocol. As part of proving Theorem \ref{theorem::ZSPA-A}, we would like to show that the semi-honest auditor's view can be simulated (so it cannot learn the parties' set elements), even if it has access to those transcripts of the fair PSI protocol sent to the smart contract; because such an approach offers a stronger security guarantee than proving the ZSPA-A protocol in isolation.  Therefore, we will present the proof of Theorem \ref{theorem::ZSPA-A} after we present the fair PSI protocol. 





%\begin{theorem}\label{theorem::ZSPA-comp-correctness}
%If the coin-tossing protocol is secure against a malicious adversary, then the ZSPA protocol,  in Figure \ref{fig:ZSPA}, securely computes $f^{\st \text {ZSPA}}$ in the presence of a malicious adversary. 
%\end{theorem}


%\begin{figure}%[ht]
%\setlength{\fboxsep}{0.7pt}
%\begin{center}
%\begin{boxedminipage}{12.3cm}
%
%\small{
%
%\begin{enumerate}
%\item[$\bullet$] Parties: clients: $\{\resizeT {\textit A}_{\resizeS {\textit  1}},..., \resizeT {\textit A}_{\resizeS {\textit  m}}\}$, the dealer and  an Arbiter.
%\item[$\bullet$] Input: Empty malicious clients list: $L$ and a deployed smart contract's address. 
%\item[$\bullet$] Output: Misbehaving clients list: $L$
%\item Every client sends to the Arbiter  two keys: $k_{\scriptscriptstyle 1}, k_{\scriptscriptstyle 2}$, used to generate the zero-sum values and their commitments. 
%%
%\item  The Arbiter checks if the clients  provided correct keys, by ensuring that the keys' hashes matches the ones stored in the contract. It appends the IDs of those  provided inconsistent keys to $L$. If all clients provided inconsistent keys it aborts. Otherwise, it proceed to the next step where it uses correct keys: $k_{\scriptscriptstyle 1}, k_{\scriptscriptstyle 2}$. 
%%
%\item\label{zero-sum-arbiter-verification} The Arbiter (given correct keys) regenerate the  zero-sum values $z_{\scriptscriptstyle i, j}$ and verify the correctness of their commitments and the Merkel tree root contracted on top of the commitments, i.e. takes the same step as step \ref{ZSPA:verify} in Fig \ref{fig:ZSPA}.   It aborts if any of the   checks is rejected, and appends to $L$ the IDs of the clients which sent the ``approved'' message to the contract. 
%%
% \item The Arbiter for each client $\resizeT {\textit C}$, who provided correct keys,  generates polynomial $\bm\mu^{\resizeS {\textit {(C)}}}$, for each bin, as follows:
%  %
%   $$\bm\mu^{\resizeS {\textit {(C)}}} = \bm\zeta\cdot \bm\xi^{\resizeS {\textit {(C)}}}-\bm\tau^{\resizeS {\textit {(C)}}}$$
%   %
%    where $\bm\xi^{\resizeS {\textit {(C)}}}$ is a random polynomial of degree $3d+1$ and $\bm\tau^{\resizeS {\textit {(C)}}}=\sum\limits^{\st 3d+2}_{\st i=0}z_{\st i,c}\cdot x^{\st i}$. By the end of this step, a vector $\vv{\bm{\mu}}$ containing polynomial $\bm\mu^{\resizeS {\textit {(C)}}}$ for every bin of client $\resizeT {\textit C}$ that is not in list $L$. 
%    %
%     \item returns   list $L$ and $\vv{\bm{\mu}}$.
 
%
%
% \item The dealer, for each client $\resizeT {\textit C}\in \{\resizeT {\textit A}_{\resizeS {\textit  1}},..., \resizeT {\textit A}_{\resizeS {\textit  m}}\}$,  sends to the Arbiter a blind polynomial of the form: $\bm\zeta\cdot \bm\eta^{\resizeS {\textit {(D,C)}}}-(\bm\gamma^{\resizeS {\textit {(D,C)}}}+\bm\delta^{\resizeS {\textit {(D,C)}}})$, where $\bm\eta^{\resizeS {\textit {(D,C)}}}$ is a fresh random degree $3d+1$ polynomial. The blind polynomial will allow the arbiter to obliviously verify the correctness of the message each client sent to the  contract. 
% 
% \item The Arbiter for each client $\resizeT {\textit C}$ who provided correct keys: 
% 
% \begin{enumerate}
% \item adds together the blind polynomial above and the blind polynomial $\bm\nu^{\resizeS {\textit {(C)}}}$ the client sent to the contract (in step \ref{blindPoly-C-sends-to-contract} in the PSI protocol). Then, it removes the client's zero-sum pseudorandom values from the result. In particular, it computes:    
%\begin{equation*}
%\begin{split}
% \bm\iota^{\resizeS {\textit {(C)}}}&=\bm\zeta\cdot \bm\eta^{\resizeS {\textit {(D,C)}}}-(\bm\gamma^{\resizeS {\textit {(D,C)}}}+\bm\delta^{\resizeS {\textit {(D,C)}}})+\bm\nu^{\resizeS {\textit {(C)}}}-\sum\limits^{\scriptscriptstyle 3d+1}_{\scriptscriptstyle i=0}z_{\scriptscriptstyle i,c}\cdot x^{\scriptscriptstyle i} \\ &=\bm\zeta\cdot(\bm\eta^{\resizeS {\textit {(D,C)}}} + \bm\omega^{\resizeS {\textit {(D,C)}}}\cdot \bm\omega^{\resizeS {\textit {(C,D)}}}\cdot \bm\pi^{\resizeS {\textit {(C)}}}+\bm\rho^{\resizeS {\textit {(D,C)}}}\cdot \bm\rho^{\resizeS {\textit {(C,D)}}}\cdot \bm\pi^{\resizeS {\textit {(D)}}})
% \end{split}
%\end{equation*}
%  \item checks if $\bm\zeta$ can divide $\bm\iota^{\resizeS {\textit {(C)}}}$. If can not, it appends the client's ID to $L$.
%  \end{enumerate}
  %$deg(\eta^{\resizeS {\textit {D,C}}})=3d+1$
% \end{enumerate}
%}
%\end{boxedminipage}
%\end{center}
%\caption{$\mathtt{Arbiter}$ Protocol} 
%\label{fig:arbiter}
%\end{figure}








