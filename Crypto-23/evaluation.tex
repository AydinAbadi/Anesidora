% !TEX root =main.tex


\section{Evaluation}\label{sec::valuation}
In this section, we analyse the asymptotic costs of \epsi. We also compare its costs and features with the fastest two and multiple parties PSIs in \cite{AbadiDMT22,DBLP:conf/ccs/KolesnikovMPRT17,NevoTY21,RaghuramanR22}) and with the fair PSIs in \cite{DebnathD16,DBLP:conf/dbsec/DongCCR13}. Tables \ref{table::Asymptotic-Cost} and \ref{table::comparisonTable} summarise the result of the cost analysis and the comparison respectively. 


% !TEX root =main.tex




\vs
\vs
 \begin{table}[!htb]

\caption{ {\small{Asymptotic costs of different parties in \epsi. In the table, $h$ is the total number of bins, $d$ is a bin's capacity (i.e., $d=100$), $m$ is the total number of clients (excluding $D$), $|S|$ is a set cardinality, and $\bar\xi$ is \ole's security parameter.
%
}}} \label{table::Asymptotic-Cost} 
% \vspace{-3mm}
\begin{center}
\scalebox{.78}{
\renewcommand{\arraystretch}{1}
\begin{tabular}{|c|c|c|c|c|} 

   %\hline
        \cline{1-3}  
   %
{\scriptsize {Party}}&{\scriptsize {Computation Cost}}&{\scriptsize {Communication Cost}}\\
     \cline{1-3}  
%&\scriptsize$e=1$&\scriptsize$e>1$\\
\hline

    %SO-PoR 1st row
\scriptsize Client $A_{\st  3},...,    A_{\st   m}$& \cellcolor{gray!50}   \scriptsize$O\Big(h\cdot d(m+d)+|S|(\frac{d^{\st 2}+d}{2})\Big)$& \cellcolor{gray!50}  \scriptsize$O\Big(h\cdot d^{\st 2}\cdot \bar\xi\Big)$\\
 %  { }
     \cline{1-3}  
     %SO-PoR 2nd row
\scriptsize Dealer $D$&   \cellcolor{gray!20}\scriptsize$O\Big(h\cdot m(d^{\st 2}+d)+|S|(\frac{d^{\st 2}+d}{2})\Big)$ &  \cellcolor{gray!20}\scriptsize$O\Big(h\cdot d^{\st 2}\cdot \bar\xi\cdot m\Big)$\\
      \cline{1-3}   
      

       %[3] 1st row 
       
   \scriptsize   {Auditor $\aud$ }& \cellcolor{gray!50}\scriptsize$O\Big(h\cdot m\cdot d\Big)$&  \cellcolor{gray!50}\scriptsize$O\Big(h\cdot d\Big)$\\      
            \cline{1-3} 

 % \scriptsize \ \ \ \ \ \ \ \ --------------&&\\
 \scriptsize{Extractor} $A_{\st  1},    A_{\st   2}$& \cellcolor{gray!20}\scriptsize$O\Big(h\cdot d(m+d)+|S|(\frac{d^{\st 2}+d}{2})\Big)$& \cellcolor{gray!20}\scriptsize$O\Big(|S_{\scriptscriptstyle\cap}|\cdot \log_{\st 2}|S|\Big)$\\
     \cline{2-3}
%{\scriptsize Auditor $\mathcal{D}_{\st n}$}&    \cellcolor{gray!20}\scriptsize$O(\sum\limits_{\st i=e}^{\st n}\frac{n!}{i!(n- i)!})$&    \cellcolor{gray!20}\scriptsize$ O(\sum\limits_{\st i=e}^{\st n}\frac{n!}{i!(n- i)!})$\\
     \cline{1-3}  
     
 \scriptsize Smart contract $\mathcal{SC}_{\epsi}$\ \&\ $\mathcal{SC}_{\fpsi}$& \cellcolor{gray!50}\scriptsize $O\Big( |S_{\st \cap}|(d+ \log_{\st 2} |S|)+h\cdot m\cdot d\Big)$& \cellcolor{gray!50}\scriptsize ---\\
 
   \hline
   
   \hline
   
    
     \scriptsize Overal Complexity & \cellcolor{gray!20}\scriptsize $O\Big(h\cdot d^{2}\cdot m \Big)$& \cellcolor{gray!20}\scriptsize {$O\Big(h\cdot d^{\st 2}\cdot \bar\xi\cdot m\Big)$}\\
     
      \cline{1-3}  

\end{tabular} 
} 
\end{center}
\end{table}






%!TEX root = main.tex


\vs


\begin{table} 


\vs
\vs

\caption{ \small{Comparison of the asymptotic complexities and features of state-of-the-art PSIs. In the table, $t$ is a parameter that determines the maximum number of colluding parties, $\kappa$ is a security parameter, and $c$ is a set cardinality.}}  \label{table::comparisonTable} 
\renewcommand{\arraystretch}{.9}
\begin{center}
\scalebox{.78}{
\begin{tabular}{|c|c|c|c|c|c|c|c|c|c|} 
\hline

%\multicolumn{3}{c|}

%\multirow{2}{*} {\scriptsize {Schemes}} &{\scriptsize {Computation}}& \scriptsize{Communication}&{\scriptsize{Fairness}}&{ \scriptsize Rewarding}& {\scriptsize{ Sym-key based}}& {\scriptsize{Multi-party}}&\scriptsize Active Adversary \\
%\hline

\multirow{2}{*} {\scriptsize {Schemes}} &\multicolumn{2}{c|}{\scriptsize Asymptotic Cost}&\multicolumn{5}{c|}{\scriptsize{Features}} \\

\cline{2-8}

& \scriptsize{Computation}&\scriptsize{Communication}&{\scriptsize{Fairness}}&{ \scriptsize Rewarding}& {\scriptsize{ Sym-key based}}& {\scriptsize{Multi-party}}&\scriptsize Active Adversary\\




\hline 

%&\scriptsize {Modular expo.}&\cellcolor{gray!20}\scriptsize {$0$}&\cellcolor{gray!20}\scriptsize$5$&\cellcolor{gray!20}\scriptsize$12$\\


\scriptsize  \scriptsize{ \cite{AbadiDMT22}}&\cellcolor{gray!20}\scriptsize{$O( h\cdot d^{\st 2}\cdot m)$}&\cellcolor{gray!20}\scriptsize$O(h\cdot d\cdot m)$&\cellcolor{gray!20}\scriptsize\textcolor{red}{$\times$}&\cellcolor{gray!20}\scriptsize\textcolor{red}{$\times$}&\cellcolor{gray!20}\scriptsize\textcolor{blue}\checkmark  &\cellcolor{gray!20}\scriptsize\textcolor{blue}\checkmark&\cellcolor{gray!20}\scriptsize\textcolor{red}{$\times$} \\


\hline 


\scriptsize \cite{DebnathD16}&\cellcolor{gray!50}\scriptsize{$O(c)$}&\cellcolor{gray!50}\scriptsize{$O(c)$}&\cellcolor{gray!50}\scriptsize\textcolor{blue}\checkmark&\cellcolor{gray!50}\scriptsize\textcolor{red}{$\times$}&\cellcolor{gray!50}\scriptsize\textcolor{red}{$\times$} &\cellcolor{gray!50}\scriptsize\textcolor{red}{$\times$}&\cellcolor{gray!50}\scriptsize\textcolor{blue}\checkmark \\ 




\hline

\scriptsize {\cite{DBLP:conf/dbsec/DongCCR13}}&\cellcolor{gray!20}\scriptsize{$O(c^{\st 2}$)}&\cellcolor{gray!20}\scriptsize$O(c)$&\cellcolor{gray!20}\scriptsize\textcolor{blue}\checkmark&\cellcolor{gray!20}\scriptsize\textcolor{red}{$\times$}  &\cellcolor{gray!20}\scriptsize\textcolor{red}{$\times$} &\cellcolor{gray!20}\scriptsize\textcolor{red}{$\times$}&\cellcolor{gray!20} \scriptsize\textcolor{blue}\checkmark\\ 

\hline
\scriptsize \cite{DBLP:conf/ccs/KolesnikovMPRT17}   &\cellcolor{gray!50}\scriptsize{$O(c\cdot m^{\st 2}+c\cdot m )$}&\cellcolor{gray!50}\scriptsize$O(c\cdot m^{\st 2})$&\cellcolor{gray!50}\scriptsize\textcolor{red}{$\times$}&\cellcolor{gray!50}\scriptsize\textcolor{red}{$\times$}  &\cellcolor{gray!50}\scriptsize\textcolor{blue}\checkmark &\cellcolor{gray!50}\scriptsize\textcolor{blue}\checkmark&\cellcolor{gray!50}\scriptsize\textcolor{red}{$\times$}\\ 

\hline


\scriptsize \cite{NevoTY21}&\cellcolor{gray!20}\scriptsize{$O(c\cdot \kappa(m+t^{\st 2}-t(m+1)))$}&\cellcolor{gray!20}\scriptsize{$O(c\cdot m\cdot \kappa)$}&\cellcolor{gray!20}\scriptsize{\textcolor{red}{$\times$}}&\cellcolor{gray!20}\scriptsize\textcolor{red}{$\times$}&\cellcolor{gray!20}\scriptsize\textcolor{blue}\checkmark  &\cellcolor{gray!20}\scriptsize\textcolor{blue}\checkmark&\cellcolor{gray!20}\scriptsize\textcolor{blue}\checkmark\\ 

\hline


\scriptsize \cite{RaghuramanR22}&\cellcolor{gray!50}\scriptsize{$O(c)$}&\cellcolor{gray!50}\scriptsize{$O(c\cdot \kappa)$}&\cellcolor{gray!50}\scriptsize{\textcolor{red}{$\times$}}&\cellcolor{gray!50}\scriptsize\textcolor{red}{$\times$} &\cellcolor{gray!50}\scriptsize\textcolor{blue}\checkmark &\cellcolor{gray!50}\scriptsize{\textcolor{red}{$\times$}} &\cellcolor{gray!50}\scriptsize\textcolor{blue}\checkmark\\ 

\hline



{\scriptsize \textbf{Ours:} \epsi}&\cellcolor{gray!20}\scriptsize{$O (h\cdot d^{2}\cdot m)$}&\cellcolor{gray!20}\scriptsize$O (h\cdot d^{\st 2}\cdot \bar\xi\cdot m )$&\cellcolor{gray!20}\scriptsize\textcolor{blue}\checkmark&\cellcolor{gray!20}\scriptsize \textcolor{blue}\checkmark&\cellcolor{gray!20}\scriptsize\textcolor{blue}\checkmark &\cellcolor{gray!20}\scriptsize\textcolor{blue}\checkmark&\cellcolor{gray!20}\scriptsize\textcolor{blue}\checkmark \\

\hline 
%}
\end{tabular}
%}
%\renewcommand{\arraystretch}{1}
%\end{footnotesize}
}
\end{center}
%}
\end{table}







\subsection{Computation Cost}


In step \ref{e-psi::call-F-PSI-stepOne}, each client's cost is $O(m)$ and mainly involves an invocation of \ct. 
% 
In steps \ref{e-psi::deploy-SC-E-PSI}--\ref{e-PSI::extractor-deposit}, the clients' cost is negligible as it involves deploying smart contracts and reading from the deployed contracts. 
%
Step \ref{e-psi::commit-to-mk} involves only $D$ whose cost in this step is constant, as it involves invoking a public key encryption, \prf,  and commitment only once. In step \ref{e-psi::gen-mk-prime}, the clients' cost is  $O(m)$, as they need to invoke an instance of \ct. 
%
In step \ref{Smart-PSI:encode-elem}, each client invokes \prp and $\mathtt{H}$ linear with its set's cardinality. In the same step, it also constructs $h$ polynomials, where the construction of each polynomial involves $d$  modular multiplications and additions. Thus, its complexity in this step is $O(h\cdot d)$. It has been shown in \cite{AbadiDMT22} that $O(h\cdot d)=O(c)$ and  $d=100$ for all set sizes. 
%
In step \ref{merkel-tree-cons}, each extractor invokes the commitment scheme linear with the number of its set cardinality $|S|$ and constructs a Merkle tree on top of the commitments. Therefore, its complexity is $O(|S|)$. 


In step \ref{e-psi::invoke-remainer-F-PSI}, each client   $A_{\st  1},...,    A_{\st   m}$: (i) invokes an instance of \zspaa which involves $O(h\cdot m)$ invocation of \ct, $3h\cdot m (d+1)$ invocation of \prf, $3h\cdot m (d+1)$ addition, and $O(h\cdot m\cdot d)$ invocation of $\mathtt{H}$ (in step \ref{ZSPA} of subroutine \fpsi), (ii) invokes $2h$ instances of \vopr, where each \vopr invocation involves $2d(1+d)$ invocations of $\ole^{\st +}$, multiplications, and additions  (in steps \ref{e-psi::D-randomises} and \ref{e-psi::C-randomises} of \fpsi), and (iii) performs $h(3d+2)$ modular addition (in step \ref{blindPoly-C-sends-to-contract} of  \fpsi).  
 %
The dealer $D$:  (a) invokes $2h\cdot m$ instances of \vopr  (in steps \ref{e-psi::D-randomises} and \ref{e-psi::C-randomises} of \fpsi), (b) invokes $\prf$ $h(3d+1)$ times (in step \ref{f-psi::D-gen-random-poly} of \fpsi), and (c) performs $h(d^{\st 2}+1)$ multiplications and $3h\cdot m\cdot d$ additions (in step \ref{f-psi::D-gen-switching-poly} of \fpsi). 
% 
In the same step, the subroutine smart contract $\mathcal{SC}_{\fpsi}$ performs $h\cdot m(3d+1)$ additions and $h$ polynomial divisions,  where each division includes dividing a polynomial of degree $3d+1$ by a polynomial of degree $1$ (in step \ref{compute-res-poly} of \fpsi). 
%
Moreover, if $Flag=True$, then each client invokes $\prf$ $h (3d+1)$ times, and performs $h (3d+1)$ additions, and performs polynomial evaluations linear with its set cardinality, where each evaluation involves  $O(d)$  additions and $O(\frac{d^{\st 2}+d}{2})$ multiplications (in step \ref{F-PSI::flag-is-true} of \fpsi). If $Flag=False$, then (a) \aud invokes $\prf$ $3h\cdot m(d+1)$ times  and  invokes $\mathtt{H}$ $O(h\cdot m\cdot d)$ times, and (b) $D$ performs $O(h\cdot m\cdot d)$ multiplications and additions (in step \ref{F-PSI::flag-is-false} of \fpsi). 




In step \ref{smart-PSI::extractors}, each extractor invokes $\mathtt{H}$ linear with its set cardinality $|S|$; it also performs polynomial evaluations linear with $|S|$. 
%
In step \ref{e-psi::SC-verification--derive-mk}, $\mathcal{SC}_{\epsi}$ invokes the commitment's verification algorithm $\comver$ once,  $\mathtt{H}$ at most $|S_{\st \cap}|$ times, and $\prf$ $|S_{\st \cap}| (3d+1)$ times. In step \ref{e-psi::SC-verification--check-three-vals}, $\mathcal{SC}_{\epsi}$ invokes  $\comver$ at most $|S_{\st \cap}|$ times, and calls $\mathtt{H}$ $O(|S_{\st \cap}|\cdot \log_{\st 2} |S|)$ times. In the same step, it  performs polynomial evaluation $|S_{\st \cap}|$ times. Thus, its overall complexity is $O( |S_{\st \cap}|(d+ \log_{\st 2} |S|))$.
  
  % (where each evaluation involves  $O(d)$  additions and $O(\frac{d^{\st 2}+d}{2})$ multiplications). 



%In step \ref{e-psi::SC-verification}, $\mathcal{SC}_{\epsi}$ invokes the commitment's verification algorithm, the hash function, at most linear with the intersection cardinality $|S_{\st \cap}|$, invokes $\prf$ $3d+1$ times, invokes the hash function $O(\log_{\st 2} (d\cdot m))$ times, and performs polynomial evaluations linear with the smallest set cardinality.% (where each evaluation involves  $O(d)$  additions and $O(\frac{d^{\st 2}+d}{2})$ multiplications). 

 %\scf  performs $h\cdot m(3d+1)$ modular additions and $h$ polynomial divisions (in step \ref{compute-res-poly} of F-PSI). 



\subsection{Communication Cost}



In steps  \ref{e-psi::call-F-PSI-stepOne} and \ref{e-psi::gen-mk-prime}, the communication cost of the clients is dominated by the cost of \ct which is $O(m)$. In steps \ref{e-psi::deploy-SC-E-PSI}--\ref{e-psi::commit-to-mk}, the clients' cost is negligible, as it involves sending a few transactions to the smart contracts, e.g., $\mathcal{SC}_{\fpsi}$, $\mathcal{SC}_{\epsi}$, and \SCpc. Step \ref{merkel-tree-cons} involves only extractors whose cost is $O(h)$ as each of them only sends to $\mathcal{SC}_{\epsi}$  a single value for each bin. In step \ref{e-psi::invoke-remainer-F-PSI}, the clients' cost is dominated by \vopr's cost; specifically, each pair of client and $D$ invokes \vopr $O(d^{\st 2})$ times for each bin; therefore, the cost of each client (excluding $D$) is $O(h\cdot d^{\st 2}\cdot \bar\xi)$ while the cost of $D$ is $O(h\cdot d^{\st 2}\cdot \bar\xi\cdot m)$, where $\bar\xi$ is the subroutine \ole's security parameter. 
%
Step \ref{smart-PSI::extractors}  involves only the extractors, where each extractor's cost is dominated by the size of the Merkle tree's proof it sends to $\mathcal{SC}_{\epsi}$, i.e., $O(|S_{\scriptscriptstyle\cap}|\cdot \log_{\st 2}|S|)$, where $|S|$ is the extractor's set cardinality. 
%
In step \ref{F-PSI::flag-is-false}, \aud sends $h$ polynomials of degree $3d+1$ to $\mathcal{SC}_{\fpsi}$; thus, its complexity is $O(h\cdot d)$. 
%
The rest of the steps impose negligible communication costs. 


\subsection{Comparison}
Below we show that \epsi offers various features that the state-of-the-art PSIs do not offer simultaneously while keeping its overall overheads similar to the efficient PSIs.  

\subsubsection{Computation Complexity.} The  computation complexity  of \epsi is similar to that of PSI in \cite{AbadiDMT22}, but is better than the multiparty PSI's complexity in \cite{DBLP:conf/ccs/KolesnikovMPRT17} as  the latter's complexity is quadratic with the number of parties. Also, \epsi's complexity  is better than the complexity of the PSI in  \cite{NevoTY21}  that is quadratic with parameter $t$, i.e., the total number of parties that may collude. Similar to the two-party PSIs in \cite{DebnathD16,RaghuramanR22}, \epsi's complexity is linear with $c$.  The two-party PSI in \cite{DBLP:conf/dbsec/DongCCR13} imposes a higher computation overhead than \epsi does, as its complexity is quadratic with sets' cardinality. Hence, the complexity of \epsi is: (i) linear with the set cardinality, similar to the above schemes except the one in \cite{DBLP:conf/dbsec/DongCCR13} and (ii) linear with the total number of parties, similar to  the above multi-party schemes, except the one in \cite{DBLP:conf/ccs/KolesnikovMPRT17}. 
%
Hence, the computation complexity of \epsi is linear with the set cardinality and the number of parties, similar to the above schemes except for the ones in \cite{DBLP:conf/ccs/KolesnikovMPRT17,DBLP:conf/dbsec/DongCCR13} whose complexities are quadratic with the set cardinality or the number of parties. 


\subsubsection{Communication Complexity.}  \epsi's communication complexity is slightly higher than the complexity of the PSI in \cite{AbadiDMT22}, by a factor of $d\cdot \bar\xi$. However, it is better than the  PSI's complexity in \cite{DBLP:conf/ccs/KolesnikovMPRT17} as the latter has a complexity quadratic with the number of parties. \epsi's complexity is slightly higher than the one in \cite{NevoTY21}, by a factor of $d$. Similar to the two-party PSIs in  \cite{DebnathD16,RaghuramanR22,DBLP:conf/dbsec/DongCCR13}, \epsi's complexity is linear with $c$. 
%
Therefore, the communication complexity of \epsi is linear with the set cardinality and number of parties, similar to the above schemes except the one in \cite{DBLP:conf/ccs/KolesnikovMPRT17} whose complexity is quadratic with the number of parties. 


\subsubsection{Features.} \epsi is the only scheme that offers all the five features, i.e., supports fairness, rewards participants, is based on symmetric key primitives, supports multi-party, and is secure against active adversaries. After \epsi is the scheme in \cite{NevoTY21} which offers three of the above features. The rest of the schemes support only two of the above features. 








