% !TEX root =main.tex


\section{Counter Collusion Contracts}\label{appendix::Counter-Collusion-Contracts}

In this section, we present Prisoner’s Contract (\SCpc), Colluder’s Contract (\SCcc), Traitor’s Contract (\SCtc) originally proposed by Dong \textit{et al.} \cite{dong2017betrayal}. As we previously stated, we have slightly adjusted the contracts.  We will highlight the adjustments in blue. For the sake of completeness, below we restate the parameters used in these contracts and their relation. 


\begin{itemize}
\item[$\bullet$] $\bc$: the bribe paid by the ringleader of the collusion to the other
server in the collusion agreement, in \SCcc.
%
\item[$\bullet$] $\cc$: a server’s cost for computing the task.
%
\item[$\bullet$] $\chc$: the fee paid to invoke an auditor for recomputing a task and resolving
disputes.
%
\item[$\bullet$] $\dc$: the deposit a server needs to pay to be eligible for getting the job.
%
\item[$\bullet$] $\tc$: the deposit the colluding parties need to pay in the collusion agreement, in \SCcc.
%
\item[$\bullet$] $\wc$: the amount that a server receives for completing the task.
%
\item[$\bullet$] $\wc \geq \cc$: the server would not accept underpaid jobs.
%
\item[$\bullet$] $\chc > 2\wc$: If it does not hold, then there would be no need to use the servers and the auditor would do the computation.
%
\item [$\bullet$] $(pk,sk)$: an asymmetric-key encryption's public-private key pair belonging to the auditor. 
\end{itemize}
\noindent The following relations need to hold when setting the contracts
in order for the desirable equilibria to hold:
%
(i) $\dc>\cc+\chc$, (ii) $\bc<\cc$, and (iii) $\tc<\wc-\cc + 2\dc - \chc -\bc$.





% !TEX root =main.tex


\subsection{Prisoner's Contract (\SCpc)}

\SCpc has been designed for outsourcing a certain computation. It is signed by a client who delegates the computation and two other parties  (or servers)  who perform the computation.  This contract tries to incentivize correct computation by using the following idea. It requires each server to pay a deposit before the computation is delegated. If a server behaves honestly, then it can withdraw its deposit. However, if a server cheats (and is detected), its deposit is transferred to the client. When one of the servers is honest and the other one cheats, the honest server receives a reward. This reward is taken from the deposit of the cheating server.  Hence, the goal of \SCpc is to create a Prisoner’s dilemma between the two servers in the following sense. Although the servers may collude with each other (to cut costs and provide identical but incorrect computation results) which leads to a higher payoff than both behaving honestly,  there is an even higher payoff if one of the servers manages to persuade the other server to collude and provide an incorrect result while itself provides a correct result. In this setting, each server knows that collusion is not stable as its counterparty will always try to deviate from the collusion to increase its payoff.  If a server tries to convince its counterparty (without a credible and enforceable promise), then the latter party will consider it as a trap; consequently, collusion will not occur. Below, we restate the contract. %We have slightly adjusted the contract such that misbehaving servers deposit is transferred to another contract (denoted with $\mathcal{SC}_{\epsi}$) instead of transferring directly to the client. 

\begin{enumerate}
%
\item The contract is signed by three parties; namely, client $ {D}$ and two other parties $E_{\st 1}$ and $E_{\st 2}$. A third-party auditor will resolve any dispute between $ { D}$ and the servers.  \textcolor{blue}{The address of another contract, called $\mathcal{SC}_{\epsi}$, is hardcoded in this contract.} 
%
\item The servers agree to run computation $f$ on input $x$, both of which have been provided by $ { D}$. 
%
\item The parties agree on three deadlines $(T_{\st 1}, T_{\st 2}, T_{\st 3})$, where $T_{\st i+1} > T_{\st i}$.
%
\item $ { D}$ agrees to pay $\wc$ to each server for the correct and on-time computation. Therefore, $ { D}$ deposits $2\cdot \wc$ amount in the contract. \textcolor{blue}{This deposit is transferred from $\mathcal{SC}_{\epsi}$ to this contract}.
%
\item Each server deposits \textcolor{blue}{$\dc'=$ }$\dc$\textcolor{blue}{$\ +\ \Xc$} amount in the contract. 
%
\item The servers must pay the deposit before $T_{\st 1}$. If a server fails to meet the deadline, then the contract would refund the parties' deposit (if any) and terminates.  
%
\item The servers must deliver the computation's result before $T_{\st 2}$. 
%
\item  The following steps are taken when (i) both servers provided the computation's result or (ii) deadline $T_{\st 2}$ elapsed. 
%
\begin{enumerate}
%
\item if both servers failed to deliver the computation's result, then the contract transfers their deposits to \textcolor{blue}{$\mathcal{SC}_{\epsi}$}.
%
\item if both servers delivered the result, and the results are equal, then (after verifying the results) \textcolor{blue}{this contract} must pay the agreed
amount $\wc$ and refund the deposit $\dc'$ to each server. 
%
\item\label{prisoner-cont-raise-disp} otherwise, $ { D}$ raises a dispute with the auditor.  
%
\end{enumerate}
%
\item When a dispute is raised, the auditor (which is independent of \aud in \fpsi) re-generates the computation's result, \textcolor{blue}{by using algorithm $\mathtt{resComp}(.)$ described shortly in Appendix \ref{sec::auditor-res-Comp}.} Let $y_{\st t}, y_{\st 1},$ and $y_{\st 2}$ be the result computed by the auditor, $E_{\st 1}$, and $E_{\st 2}$ respectively. The auditor uses the following role to identify the cheating party.  
%
\begin{itemize}
%
\item[$\bullet$] if $E_{\st i}$ failed to deliver the result (i.e., $y_{\st i}$ is null), then it has cheated. 
%
\item[$\bullet$] if a result $y_{\st i}$ has been delivered before the deadline and $y_{\st i}\neq y_{\st t}$, then $E_{\st i}$ has cheated. 
%
\end{itemize}
%
\textcolor{blue}{The auditor sends its verdict to \SCpc.} 
%
\item Given the auditor's decision, the dispute is settled according to the following rules.
%
\begin{itemize}
%
\item[$\bullet$] if none of the servers cheated, then \textcolor{blue}{this contract} transfers to each server (i) $\wc$ amount for performing the computation and (ii) its deposit, i.e., $\dc'$ amount. The client also pays the auditor $\chc$ amount.  
%
\item[$\bullet$] if both servers cheated, then \textcolor{blue}{this contract (i) pays the auditor the total amount of $\chc$, taken from the servers' deposit, and (ii) transfers to  $\mathcal{SC}_{\epsi}$ the rest of the deposit, i.e., $2\cdot \dc'-\chc$ amount.} 
%
\item[$\bullet$] if one of the servers cheated, then \textcolor{blue}{this contract (i) pays the auditor the total amount of $\chc$, taken from the misbehaving server's deposit, (ii) transfers the honest server's deposit (i.e., $\dc'$ amount) back to this server,  (iii) transfers to the honest server $\wc+\dc-\chc$ amount (which covers its computation cost and the reward), and  (iv) transfers to $\mathcal{SC}_{\epsi}$ the rest of the misbehaving server's deposit, i.e., $\Xc$ amount.} The cheating server receives nothing.  
%
\end{itemize} 
%
\item After deadline $T_{\st 3}$, if $ { D}$ has neither paid nor raised a dispute, then \textcolor{blue}{this contract} pays $\wc$ to each server which delivered a result before deadline $T_{\st 2}$ and refunds each server its deposit, i.e., $\dc'$ amount. Any deposit left after that will be transferred to \textcolor{blue}{$\mathcal{SC}_{\epsi}$}. 
%
\end{enumerate}


Now, we explain why we have made the above changes to the original \SCpc of Dong \et \cite{dong2017betrayal}. In the original \SCpc (a) the client does not deposit any amount in this contract; instead, it directly sends its coins to a server (and auditor) according to the auditor's decision,  (b) the computation correctness is determined only within this contract (with the involvement of the auditor if required), and (c)  the auditor simply re-generates the computation's result given the computation's plaintext inputs.  Nevertheless, in \epsi, (1) \emph{all clients} need to deposit a certain amount in $\mathcal{SC}_{\epsi}$ and only the contracts must transfer the parties' deposit,  (2) $\mathcal{SC}_{\epsi}$ also needs to verify a part of the computation's correctness without the involvement of the auditor and accordingly distribute the parties deposit based on the verification's outcome,  and (3) the auditor must be able to re-generate the computation's result without being able to learn the computation's plaintext input, i.e., elements of the set. Therefore, we have included the address of $\mathcal{SC}_{\epsi}$ in \SCpc to let the parties' deposit move between the two contracts (if necessary) and have allowed \SCpc to distribute the parties' deposit; thus, the requirements in points (1) and (2) are met. To meet the requirement in point (3) above, we have included a new algorithm, called $\mathtt{resComp}(.)$, in \SCpc.  Shortly, we will provide more detail about this algorithm. Moreover, to make this contract compatible with \epsi, we increased the amount of each server's deposit by  $\Xc$. Nevertheless, this adjustment does not change the logic behind \SCpc's design and its analysis.  







\subsubsection{Auditor's Result-Computation Algorithm.}\label{sec::auditor-res-Comp}


In this work,  we use \SCpc to delegate the computation of intersection cardinality to two extractor clients, a.k.a. servers in the original \SCpc. In this setting, the contract's auditor is invoked when an inconsistency is detected in step \ref{smart-PSI-inconsistency} of \epsi. For the auditor to recompute the intersection cardinality, we have designed $\mathtt{resComp}(.)$ algorithm. The auditor uses this algorithm for every bin's index $indx$,  where $1\leq indx\leq h$ and $h$ is the hash table's length. We present this algorithm in Figure \ref{fig::resComp}.  The auditor collects the inputs of this algorithm as follows: (a)  reads random polynomial $\bm\zeta$, and blinded polynomial $\bm\phi$ from contract $\mathcal{SC}_{\fpsi}$, (b) reads the ciphertext if secret key $mk$ from $\mathcal{SC}_{\epsi}$, and (c) fetches public parameters $(des_{\mathtt{H}}, h)$ from the hash table's public description. 



Note that in the original  \SCpc of Dong \et \cite{dong2017betrayal}, the auditor is assumed to be fully trusted. However, in this work, we have relaxed such an assumption. We have designed \epsi and $\mathtt{resComp}(.)$ in such a way that even a semi-honest auditor cannot learn anything about the actual elements of the sets, as they have been encrypted under a key unknown to the auditor. 



% !TEX root =main.tex


%\vspace{-2mm}
\begin{figure}%[!htbp]
\setlength{\fboxsep}{1pt}
\begin{center}
\scalebox{.85}{
    \begin{tcolorbox}[enhanced,width=5.5in, 
    drop fuzzy shadow southwest,
    colframe=black,colback=white]
   % {\small{
    %\vspace{-2.5mm}
 \underline{$\mathtt{resComp}(\bm\zeta, \bm\phi, sk, ct_{\st mk}, indx, {des}_{\st \mathtt{H}})\rightarrow R$}\\
%
%\vspace{-2.2mm}
\begin{itemize}
%
\item \noindent\textit{Input}. $\bm\zeta$: a random polynomial of degree $1$, $\bm\phi$: a blinded polynomial of the form $\bm\zeta\cdot(\bm\epsilon + \bm\gamma')$ where $\bm\epsilon$ and $\bm\gamma'$ are arbitrary and  pseudorandom polynomials respectively, $deg(\bm\phi)-1=deg(\bm\gamma')$, $sk$: the auditor's secret key, $ct_{\st mk}$: ciphertext of $mk$ which is a  key of $\mathtt{PRF}$, $indx$: an input of $\mathtt{PRF}$, and   ${des}_{\st \mathtt{H}}$: a description of hash function $\mathtt{H}$.
%
\item\noindent\textit{Output}. $R$: a set containing valid roots of unblinded $\bm\phi$. 
%
\end{itemize}
%
\begin{enumerate}
%
\item decrypts the ciphertext $ct_{\st mk}$ under key $sk$. Let $mk$ be the result. 
%
\item unblinds polynomial $\bm\phi$, as follows:
%
\begin{enumerate}
%
\item re-generates pseudorandom polynomial $\bm\gamma'$ using key $mk$. Specifically, it uses $mk$ to derive a key: $k=\mathtt{PRF}(mk,  indx)$. Then, it uses the derived key to generate $3d+1$ pseudorandom coefficients, i.e.,  $ \forall j, 0\leq j \leq deg(\bm\phi)-1: g_{\st j}=\mathtt{PRF}(k, j)$. Next, it uses these coefficients to construct polynomial $\bm\gamma'$, i.e., $\bm\gamma'=\sum\limits_{\st j=0}^{\st deg(\bm\phi)-1} g_{\st j}\cdot x^{\st j}$.
%
\item removes the blinding factor from $\bm\phi$. Specifically, it computes polynomial $\bm\phi'$ of the following form $\bm\phi'= \bm\phi - \bm\zeta\cdot\bm\gamma'$. 
%
\end{enumerate}
%
\item extracts roots of polynomial $\bm\phi'$. 
\item finds valid roots, by (i) parsing each root $\bar{e}$ as $(e_{\st 1}, e_{\st 2})$ with the assistance of $des_{\st \mathtt{H}}$  and (ii) checking if $e_{\st 2}=\mathtt{H}(e_{\st 1})$. It considers a root valid, if this equation holds. 
%
\item returns set $R$ containing all valid roots.
%

%\vspace{-1mm}
\end{enumerate}
%}   }
\end{tcolorbox}
}
\end{center}
%\vspace{-3mm}
\caption{Auditor's result computation, $\mathtt{resComp}(.)$, algorithm} 
\label{fig::resComp}
\end{figure}





















% !TEX root =main.tex





\subsection{Colluder's Contract (\SCcc)}

Recall that \SCpc aimed at creating a dilemma between the two servers. However, this dilemma can be addressed if they can make an  enforceable promise.  This enforceable promise can be another smart contract, called Colluder's Contract (\SCcc).  This contract imposes additional rules that ultimately would affect the parties’ payoffs and would make collusion the most profitable strategy for the colluding parties. In \SCcc, the party who initiates the collusion would pay its counterpart a certain amount (or bribe) if both follow the collusion and provide an incorrect result of the computation to \SCpc.  Note, \SCcc requires both servers to send a fixed amount of deposit when signing the contract.  The party who deviates from collusion will be punished by losing the deposit. Below, we reiterate the description of \SCcc. 


\begin{enumerate}
%
\item The contract is signed between the server who initiates the collusion, called ringleader (LDR) and the other server called follower (FLR). 
%
\item The two agree on providing to \SCpc a different result $res'$ than a correct computation of $f$ on $x$ would yield, i.e., $res'\neq f(x)$. Parameter $res'$ is recorded in this contract. 
%
\item LDR and FLR deposit $\tc+\bc$ and $\tc$ amounts in this contract respectively. 
%
\item The above deposit must be paid before the result delivery deadline in \SCpc, i.e., before deadline $T_{\st 2}$. If this condition is not met, the parties' deposit in this contract is refunded and this contract is terminated. 
%
\item When \SCpc is finalised (i.e., all the results have been provided), the following steps are taken.
%
\begin{enumerate}
%
\item \underline{both follow the collusion}: if both LDR and FLR provided $res'$ to \SCpc, then $\tc$ and $\tc+\bc$ amounts are delivered to LDR and FLR respectively. Therefore, FLR receives its deposit plus the bribe, which is $\bc$ amount. 
%
\item \underline{only FLR deviates from the collusion}:  if LDR and FLR provide $res'$ and $res''\neq res'$ to \SCpc respectively, then $2\cdot \tc+\bc$ amount is transferred to LDR while nothing is sent to FLR. 
%
\item \underline{only LDR deviates from the collusion}: if LDR and FLR provide $res''\neq res'$ and $res'$ to \SCpc respectively, then $2\cdot \tc+\bc$ amount is sent to FLR while nothing is transferred to LDR.  
%
\item \underline{both deviate from the collusion}: if LDR and FLR deviate and provide any result other than $res'$ to \SCpc, then  $2\cdot \tc+\bc$ amount is sent to LDR and $\tc$ amount is sent to FLR. 
%
\end{enumerate}
%
\end{enumerate}

The amount of bribe a rational LDR is willing to pay is less than its computation cost (i.e., $\bc<\cc$). Otherwise, this collusion would not offer a higher payoff. We refer readers to \cite{dong2017betrayal} for further discussion. 
























% !TEX root =main.tex


\subsection{Traitor's Contract (\SCtc)}

\SCtc incentivises a colluding server (who has signed \SCcc) to betray its counterpart and report the collusion without being penalised by \SCpc.  The Traitor’s contract promises that the reporting server will not be
punished by \SCpc which makes it safe for the reporting server to follow the collusion strategy (of \SCcc), and get away from the punishment imposed by \SCpc. Below, we restate the description of \SCtc. 


\begin{enumerate}
%
\item This contract is signed among $ { D}$ and the traitor server (TRA) who reports the collusion. This contract is signed only if the parties have already signed \SCpc. 
%
\item  $ { D}$ signs this contract only with the first server who reports the collusion. 
%
\item The traitor TRA must also provide to this contract the result of the computation, i.e., $f(x)$. The result provided in this contract could be different than the one provided in \SCpc, e.g., when TRA has to follow \SCcc and provide an incorrect result to \SCpc. 
%
\item  $  D$ needs to pay  $\wc \ +\ $\textcolor{blue}{ $\dc' \ + $} $\dc-\chc$ amount to this contract. This amount equals the maximum amount TRA could lose in \SCpc plus the reward. \textcolor{blue}{This deposit will be transferred via $\mathcal{SC}_{\epsi}$ to this contract.}  TRA must deposit in this contract $\chc$ amount to cover the cost of resolving a potential dispute.  
%
\item  This contract should be signed before the deadline  $T_{\st 2}$  for the delivery of the computation result in \SCpc. If it is not signed on time, then this contract would be terminated and any deposit paid will be refunded.
%
\item It is required that the TRA provide the computation result to this contract before the above deadline $T_{\st 2}$.
%
\item If this contract is fully signed, then during the execution of \SCpc, $ { D}$ always raises a dispute, i.e., takes step \ref{prisoner-cont-raise-disp} in \SCpc.
%
\item After \SCpc is finalised (with the involvement of the auditor), the following steps are taken to pay the parties involved.   
%
\begin{enumerate}
%
\item  if none of the servers cheated in \SCpc (according to the auditor), then $\wc \ +\ $\textcolor{blue}{ $\dc' \ + $} $\dc-\chc$ amount is refunded to \textcolor{blue}{$\mathcal{SC}_{\epsi}$} and TRA's deposit (i.e., $\chc$ amount) is transferred to $ { D}$. Nothing is paid to TRA. 
%
\item if in \SCpc, the other server did not cheat and TRA  cheated; however, TRA provided a correct result in this contract, then \textcolor{blue}{ $\dc' \ + $} $\dc-\chc$ amount is transferred to  \textcolor{blue}{$\mathcal{SC}_{\epsi}$}. Also,    TRA gets its deposit back (i.e., $\chc$ amount) plus $\wc$ amount for providing a correct result to this contract. 
%
\item if in \SCpc, both servers cheated; however, TRA delivered a correct computation result to this contract, then TRA gets its deposit back (i.e., $\chc$ amount), it also receives $\wc \ +\ $\textcolor{blue}{ $\dc' \ + $} $\dc-\chc$ amount.   
%
\item otherwise,  $\wc \ +\ $\textcolor{blue}{ $\dc' \ + $} $\dc-\chc$ and $\chc$ amounts are transferred to  \textcolor{blue}{$\mathcal{SC}_{\epsi}$} and TRA respectively. 
%
\end{enumerate}
%
\item If TRA provided a result to this contract, and deadline $T_{\st 3}$ (in \SCpc) has passed, then all deposits, if any left, will be transferred to TRA. %this ensures that the deposit will not be locked in the contract forever. 
%
\end{enumerate}

TRA must take the following three steps to report collusion: (i) it waits until \SCcc is signed by the other server, (ii) it reports the collusion to $ { D}$ before signing \SCcc, and (iii) it signs \SCcc only after it signed \SCtc with $ { D}$. 

Note that the original analysis of \SCtc does not require \SCtc to remain secret. Therefore, in our  \epsi, parties TRA and $ { D}$ can sign this contract and   then store its address in $\mathcal{SC}_{\epsi}$. Alternatively, to keep this contract confidential, $ { D}$ can deploy a template \SCtc to the blockchain and store the commitment of the contract's address (instead of the plaintext address) in $\mathcal{SC}_{\epsi}$. When a traitor (TRA) wants to report collusion, it signs the deployed \SCtc with  $ { D}$ which provides the commitment opening to TRA. In this case, at the time when the deposit is distributed,   either $ { D}$ or TRA   provides the opening of the commitment to  $\mathcal{SC}_{\epsi}$ which checks whether it matches the commitment. If the check passes, then it distributes the deposit as before. 



































